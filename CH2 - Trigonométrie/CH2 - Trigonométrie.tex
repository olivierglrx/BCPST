\documentclass[a4paper, 11pt]{article}
\input{macro/package.tex}
\input{macro/environement}
% Header et footer

\pagestyle{fancy}
\fancyhead{}
\fancyfoot{}
\renewcommand{\headwidth}{\textwidth}
\renewcommand{\footrulewidth}{0.4pt}
\renewcommand{\headrulewidth}{0pt}
\renewcommand{\footruleskip}{5px}

\fancyfoot[R]{Olivier Glorieux}
%\fancyfoot[R]{Jules Glorieux}

\fancyfoot[C]{ Page \thepage }
\fancyfoot[L]{1BIOA - Lycée Chaptal}
%\fancyfoot[L]{MP*-Lycée Chaptal}
%\fancyfoot[L]{Famille Lapin}

\input{macro/newcommand.tex}
\geometry{hmargin=2.0cm, vmargin=3.5cm}

\author{Olivier Glorieux}
\usetikzlibrary{matrix,arrows,decorations.pathmorphing}

\begin{document}


\tableofcontents
\title{Chapitre 2 - Trigonométrie}




\section{R\'esolution des \'equations trigonom\'etriques}

%------------------------------------------------
%-------------------------------------------------
\subsection{R\'esolution des \'equations fondamentales: $\mathbf{\cos{(x)}=a,\ \sin{(x)}=a,\ \tan{(x)}=a}$}


\subsubsection{R\'esolution de $\cos{(x)}=a$}





\begin{prop}
Soit $a \in [-1,1]$. Il existe alors un unique angle $\theta$ dans $[0, \pi]$ tel que $ \cos(\theta)=a$\\
On note alors  $\theta = \arccos(a)$
\end{prop}
\warning  $\arccos (a)$ est par définition un nombre dans $[0,\pi]$.\\
\warning La fonction $\arccos$ n'est pas défini sur $\R$, mais seulement sur $[-1,1]$.


\vspace{0.5cm}
Valeurs particuli\`eres :\\
\begin{center}
\begin{tabular}{|c|c|c|c|c|c|c|c|c|c|}
\hline
\rule[-3mm]{0pt}{8mm}  $a$& $ -1$ & $-\sqrt{3}/2$ & $-\sqrt{2}/2$ & $-1/2$   &  $0$  & $1/2$ & $\sqrt{2}/2$&  $\sqrt{3}/2$&  $1$ \\ 
\hline
\rule[-3mm]{0pt}{8mm}  $\arccos{a}$ & $\pi$ & $5\pi/6$& $3\pi/4$  &$2\pi/3$   & $\pi/2$ &$\pi/3$&$\pi/2$&$\pi/6$&0 \\
\hline
\end{tabular}
\end{center}
\qquad



{\footnotesize
\begin{exo} R\'esoudre sur $\R$ les \'equations trigonom\'etriques suivantes: $\cos{(x)}=-2$, $\cos{(x)}=-1$, $\cos{(x)}=0$, $\ddp\cos{(x)}=\frac{1}{2}$, $\cos{(x)}=1$, $\ddp\cos{(x)}=\frac{\sqrt{2}}{2}$. 
\end{exo}}
\vspace{0.3cm}

$\cos(x) = \demi$. 
Sur $[0,\pi]$ il y a une unique solution, qui est par définitioin 
$x = \arccos (\demi) = \pi/3$

Sur $[-\pi, \pi]$, il y a $2$ solutions, $\pi/3$ et $-\pi/3$ par symmétrie de la fonction $\cos$. 

Sur $\R$ il y a une infinité de solutions données par:
$$x  \equiv \pi/3\,  [2\pi]$$
ou 
$$x  \equiv -\pi/3\,  [2\pi]$$


%



 Notons $\mathcal{S}$ l'ensemble des solutions de l'\'equation $\cos{(x)}=a$, $a\in\R$.
 
 \begin{center}
 \fbox{
$\begin{array}{ll}
\bullet \textmd{ Si } \ a>1\ \hbox{ou}\ a<-1, &\hspace{0.5cm} \mathcal{S}=\emptyset \phantom{\hspace{7cm}} \\
\bullet \textmd{ Si } \  a=1, &\hspace{0.5cm} \mathcal{S}=2\pi\Z \phantom{\hspace{7cm}} \\
\bullet \textmd{ Si } \  a=-1, &\hspace{0.5cm} \mathcal{S}=\pi+2\pi\Z\phantom{\hspace{7cm}}\\
\bullet \textmd{ Si } \  -1<a<1, & \hspace{0.5cm} \mathcal{S}=\pm \arccos(a) +2\pi\Z \phantom{\hspace{8cm}}
\end{array}$}
\end{center}



\begin{exemple}
R\'esoudre sur $\R$ puis sur $\lbrack 0,2\pi\lbrack$: $\cos{(3x)}=\ddp\demi$.
%\vspace*{7cm}
\end{exemple}
\begin{cor}
Notons $y= 3x$. On a vu que $y $  vérifié : 
$$y  \equiv \pi/3\,  [2\pi]$$
ou 
$$y  \equiv -\pi/3\,  [2\pi]$$

Soit  en revenant à la variable $x$: 
$$3x  \equiv \pi/3\,  [2\pi]$$
ou 
$$3x  \equiv -\pi/3\,  [2\pi]$$
Ce qui est équivalent à 
$$x  \equiv \pi/9\,  [2\pi/3]$$
ou 
$$x  \equiv -\pi/9\,  [2\pi/3].$$

On obtient ainsi les solutions sur $[0,2\pi[$:`
\begin{itemize}
\item $\pi/9$,
\item  $\pi/9+ 2\pi/3 =7\pi/9$,
\item  $\pi/9+ 4\pi/3 =13\pi/9$
\item  $\pi/9+ 6\pi/3 =19\pi/9>2\pi $ on est allé trop loin,$\pi/9+ 6\pi/3$ n'est pas solution sur $[0,2\pi[$.
\end{itemize}

On refait la même chose avec $-\pi/9$:
\begin{itemize}
\item $-\pi/9$,(qui n'est pas solution car n'est pas dans $[0,2\pi[$)
\item  $-\pi/9+ 2\pi/3 =5\pi/9$,
\item  $-\pi/9+ 4\pi/3 =11\pi/9$
\item  $-\pi/9+ 6\pi/3 =17\pi/9 $
\end{itemize}

Les solutions de  $\cos{(3x)}=\ddp\demi$ sur $[0,2\pi[$ sont:
$$\cS=\{ \pi/9, 7\pi/9, 13\pi/9,5\pi/9, 11\pi/9,17\pi/9\}$$






\end{cor}



{\footnotesize
\begin{exo} R\'esoudre sur $\R$ puis sur $\lbrack 0,2\pi\lbrack$: $\ddp \cos{(2x)}=-\frac{\sqrt{3}}{2}$ et $\ddp \cos{(x)}=\frac{7}{8}$ et repr\'esenter \`{a} chaque fois les solutions sur le cercle trigonom\'etrique.
\end{exo}}







\subsubsection{R\'esolution de $\sin{(x)}=a$}

\noindent 


\begin{prop}
Soit $a \in [-1,1]$. Il existe alors un unique angle $\theta$ dans $\left[ -\ddp \frac{\pi}{2}, \frac{\pi}{2} \right]$ tel que $\sin(\theta) =a$\\
On note alors $\theta=\arcsin(a)$.
\end{prop}
\warning Par définition $\arcsin(a) \in [-\pi/2, \pi/2]$.\\
\warning Le domaine de définition de $\arcsin$ est $[-1,1]$. 

\vspace{0.5cm}
Valeurs particuli\`eres :\\

\noindent \begin{tabular}{|c|c|c|c|c|c|c|c|c|c|}
\hline
\rule[-3mm]{0pt}{8mm}  $a$& $ -1$ & $-\sqrt{3}/2$ & $-\sqrt{2}/2$ & $-1/2$   &  $0$  & $1/2$ & $\sqrt{2}/2$&  $\sqrt{3}/2$&  $1$ \\ 
\hline
\rule[-3mm]{0pt}{8mm}  $\arcsin{a}$ & $-\pi/2$ &$-\pi/3$& $-\pi/4$  &$-\pi/6$   &$0$  &$\pi/6$&$\pi/4$&$\pi/3$&$\pi/2$ \\
\hline
\end{tabular}

\qquad


{\footnotesize
\begin{exo} R\'esoudre sur $\R$  les \'equations trigonom\'etriques suivantes: $\sin{(x)}=-6$, $\sin{(x)}=-1$, $\sin{(x)}=0$, $\ddp \sin{(x)}=\demi$, $\sin{(x)}=1$, $\ddp \sin{(x)}=\frac{\sqrt{3}}{2}$. 
\end{exo}}
\vspace{0.3cm}
Sur $[-\pi/2,\pi/2]$, l'équation $\sin{(x)}=\frac{\sqrt{3}}{2}$ a pour solution 
$$x= \arcsin(\frac{\sqrt{3}}{2})= \pi/3$$

Sur $[-\pi, \pi]$ les solutions sont 
$$x =\pi/3 \quad \text{ ou } \quad x= \pi-\pi/3 = 2\pi/3$$
Sur $\R$ les solutions sont 
$$\cS  =\{ \pi/3 +2k\pi , 2\pi/3 +2k\pi \, |\, k\in \Z\}$$


\begin{center}
 \fbox{
$\begin{array}{ll}
\bullet \textmd{ Si } \ a>1\ \hbox{ou}\ a<-1, &\hspace{0.5cm} \mathcal{S}=\emptyset \phantom{\hspace{7cm}} \\
\bullet \textmd{ Si } \  a=1, &\hspace{0.5cm} \mathcal{S}=\frac{\pi}{2}+2\pi\Z \phantom{\hspace{7cm}} \\
\bullet \textmd{ Si } \  a=-1, &\hspace{0.5cm} \mathcal{S}=-\frac{\pi}{2}++2\pi\Z\phantom{\hspace{7cm}}\\
\bullet \textmd{ Si } \  -1<a<1, & \hspace{0.5cm} \mathcal{S}=\pm \arcsin(a) +2\pi\Z \phantom{\hspace{8cm}} \cup \pi +\arcsin(a) +2\pi\Z 
\end{array}$}
\end{center}


\begin{exemple}
R\'esoudre sur $\R$ puis sur $\lbrack -\pi,\pi\lbrack$: $\sin{(3x)}= \ddp \frac{\sqrt{3}}{2}$.

\end{exemple}
\begin{cor}
On a vu que les solutions de $\sin(y) = \frac{\sqrt{3}}{2}$ étaient données par 
$$y \equiv \pi/3\, [2\pi] \quadou  y \equiv 2\pi/3\, [2\pi].$$
Donc 
$$3x \equiv \pi/3\, [2\pi] \quadou  3x \equiv 2\pi/3\, [2\pi].$$
Soit encore: 
$$x \equiv \pi/9\, [2\pi/3] \quadou  x \equiv 2\pi/9\, [2\pi/3].$$


\begin{itemize}
\item $\pi/9$
\item $\pi/9 +2\pi/3 = 7\pi/9$
  \item $\pi/9 -2\pi/3 = -5\pi/9$
    \item $\pi/9 -4\pi/3 = -11\pi/9<-\pi$ n'est donc pas solution sur $[-\pi, \pi[$. 
\end{itemize}
et  pour $2\pi/9$
\begin{itemize}
\item $2\pi/9$
\item $2\pi/9 +2\pi/3 = 8\pi/9$
  \item $2\pi/9 -2\pi/3 = -4\pi/9$
    \item $2\pi/9 -4\pi/3 = -10\pi/9<-\pi$ n'est donc pas solution sur $[-\pi, \pi[$. 
\end{itemize}

Les solutions sur $[-\pi , \pi[$ sont données alors par : 
$$\cS=\{ \pi/9,2\pi/9, 7\pi/9 ,-5\pi/,8\pi/9 ,-4\pi/9  \}$$ 


\end{cor}


{\footnotesize
\begin{exo} R\'esoudre sur $\R$ puis sur $\lbrack -\pi,\pi\lbrack$: $\ddp \sin{(4x)}=-\frac{1}{2}$ et $\ddp \sin{(x)}=\frac{1}{5}$ et repr\'esenter \`{a} chaque fois les solutions sur le cercle trigonom\'etrique.
\end{exo}}

\vspace{0.5cm}






%------------------------------------------------
\subsubsection{R\'esolution de $\tan{(x)}=a$}


\begin{prop}
Soit $a \in \R$. Il existe alors un unique angle $\theta$ dans $\left] -\ddp \frac{\pi}{2}, \frac{\pi}{2} \right[$ tel que $\tan(\theta) =a $\\
On note alors $\theta = \arctan(a)$
\end{prop}



\vspace{0.5cm}
Valeurs particuli\`eres :\\

\noindent \begin{tabular}{|c|c|c|c|c|c|c|c|}
\hline
\rule[-3mm]{0pt}{8mm}  $a$& $-\sqrt{3}$&$-1$ & $-\sqrt{3}/3$ & $ 0$ & $\sqrt{3}/3$ & $1$ & $\sqrt{3}$     \\
\hline
\rule[-3mm]{0pt}{8mm}  $\arctan{a}$ &-$\pi/3$  &$-\pi/4$ &$-\pi/6$&$0$   &$\pi/6$   &$\pi/4$  &$\pi/3$ \\
\hline
\end{tabular}

\qquad



{\footnotesize
\begin{exo} R\'esoudre sur $\R$ les \'equations trigonom\'etriques suivantes: $\tan{(x)}=-1$, $\tan{(x)}=0$, $\tan{(x)}=1$, $\tan{(x)}=-\sqrt{3}$, $\tan{(x)}=-\frac{1}{\sqrt{3}}$. 
\end{exo}}
\begin{cor}
Sur $]-\pi/2, \pi/2[$, $\tan{(x)}=-\frac{1}{\sqrt{3}}$ admet pour solution, 
$x=\arctan(-\frac{1}{\sqrt{3}}) =-\pi/6$. 

Sur $\R$ les solutions sont donc : 
$$\{ -\pi/6 +k\pi \, |\, k\in /Z\}$$
\end{cor}






Notons $\mathcal{S}$ l'ensemble des solutions de l'\'equation $\tan{(x)}=a$, $a\in\R$.\\
$$ S= \{ \arctan(x)+k\pi\, |\, k\in \Z\}. $$


\begin{exemple}
R\'esoudre sur $\R$ puis sur $\lbrack -\pi,\pi\lbrack$: $\ddp \tan{\left(\frac{x}{2}\right)}=-\frac{1}{\sqrt{3}}$.
\end{exemple}

\begin{cor}
On a vu que les solutions de $\tan(y)= -\frac{1}{\sqrt{3}}$ étaient 
$$y \equiv -\pi/6 \, [\pi]$$
Donc 
$$\frac{x}{2} \equiv \frac{-\pi}{6}\, [\pi],$$
Soit 
$$x \equiv \frac{-\pi}{3} \, [2\pi].$$
Sur $[-\pi, \pi[$ les solutions sont $\{ -\pi/3\}$. 


\end{cor}



{\footnotesize
\begin{exo} R\'esoudre sur $\R$ puis sur $\lbrack -\pi,\pi\lbrack$: $\tan{(3x)}=1$ et $\tan(x) = -2$ et repr\'esenter \`{a} chaque fois les solutions sur le cercle trigonom\'etrique.
\end{exo}}













\subsubsection{R\'esolution de $\cos{(x)}=\cos (y)$,  $\sin{(x)}=\sin(y)$,  $\tan{(x)}=\tan(y)$}

\noindent D'apr\`es les r\'esultats pr\'ec\'edents, on a :
%\begin{minipage}[c]{0.6\textwidth}
$$\cos x = \cos y \; \Leftrightarrow \; \left\{ \begin{array}{l}
x\equiv y \, [2\pi]\\
\textmd{ou}\\
x\equiv -y \, [2\pi]

\end{array}\right.$$
%\end{minipage}
\qquad

%\noindent\begin{minipage}[c]{0.6\textwidth}
$$\sin x = \sin y \; \Leftrightarrow \; \left\{ \begin{array}{l}
x\equiv y \, [2\pi]\\
\textmd{ou}\\
x\equiv \pi-y \, [2\pi]

\end{array}\right.$$
%\end{minipage}
\qquad


%\noindent\begin{minipage}[c]{0.6\textwidth}
$$\tan x = \tan y \; \Leftrightarrow \; 
x\equiv y \, [\pi]$$
%\end{minipage}
\qquad

Pour résoudre les autres équations du type $\cos(x)=\sin(y)$ on se ramène à une équation précédente par 
exemple en faisant $\cos(\frac{\pi}{2}-y) =\sin(y)$ et en résolvant : 
$$\cos(x) = \cos(\frac{\pi}{2}-y)$$

{\footnotesize
\begin{exo} R\'esoudre sur $\R$ puis sur $[0, 2\pi[$: $\ddp \cos\left(2x +\frac{\pi}{2}\right)=\cos\left(x-\frac{\pi}{4}\right)$ et $\tan \left(2x+\frac{\pi}{2}\right) = \tan x$ et repr\'esenter \`{a} chaque fois les solutions sur le cercle trigonom\'etrique.
\end{exo}}
\begin{cor}
Sur $\R$ l'équation a pour solution 
$$
\left\{ \begin{array}{l}
2x+\frac{\pi}{2}\equiv x-\frac{\pi}{4}\, [2\pi]\\
\textmd{ou}\\
2x+\frac{\pi}{2}\equiv -x+\frac{\pi}{4}\, [2\pi]
\end{array}\right.$$

$$
\left\{ \begin{array}{l}
2x+\frac{\pi}{2}\equiv x-\frac{\pi}{4}\, [2\pi]\\
\textmd{ou}\\
2x+\frac{\pi}{2}\equiv -x+\frac{\pi}{4}\, [2\pi]
\end{array}\right.$$


$$
\left\{ \begin{array}{l}
x\equiv -\frac{\pi}{2}-\frac{\pi}{4}\, [2\pi]\\
\textmd{ou}\\
3x\equiv -\frac{\pi}{2}+\frac{\pi}{4}\, [2\pi]
\end{array}\right.$$


$$
\left\{ \begin{array}{l}
x\equiv -\frac{3\pi}{4}\, [2\pi]\\
\textmd{ou}\\
3x\equiv -\frac{\pi}{4}\, [2\pi]
\end{array}\right.$$

$$
\left\{ \begin{array}{l}
x\equiv -\frac{3\pi}{4}\, [2\pi]\\
\textmd{ou}\\
x\equiv -\frac{\pi}{12}\, [2\pi/3]
\end{array}\right.$$



On cherche les solutions appartenant à $[0,2\pi[ $ :
\begin{itemize}
\item $-\frac{3\pi}{4} \notin[0,2\pi[  $ 
\item $-\frac{3\pi}{4} +2\pi = \frac{5\pi}{4}\in[0,2\pi[  $ 
\item $-\frac{\pi}{12}  \notin[0,2\pi[  $ 
\item $-\frac{\pi}{12} +\frac{2\pi}{3} = \frac{7\pi}{12} \in [0,2\pi[$
\item $-\frac{\pi}{12} +\frac{4\pi}{3} = \frac{15\pi}{12} \in [0,2\pi[$
\item $-\frac{\pi}{12} +\frac{6\pi}{3} = \frac{23\pi}{12} \in [0,2\pi[$
\end{itemize}


Les solutions dans $[0,2\pi[$ sont:
$$\cS  = \{ \frac{5\pi}{4} , \frac{7\pi}{12} , \frac{15\pi}{12} ,  \frac{23\pi}{12}\}.$$

\end{cor}









\vspace*{0.5cm}



\pagebreak
%----------------------------------

%------------------------------------------------
%-------------------------------------------------
\subsection{R\'esolution des autres \'equations}

%\setlength\fboxrule{1pt}
%\noindent \doublebox{
%\begin{minipage}[t]{0.95\textwidth}
%
%\phantom{\hspace{-0.2cm}} \dotfill
%\end{minipage}}
%\setlength\fboxrule{0.5pt}\\

M\'ethode g\'en\'erale:
Transformer l'expression pour se ramener \`{a} r\'esoudre des \'equations fondamentales.\\
%\noindent\begin{tabular}{||l||}
%\hline\hline
%M\'ethode g\'en\'erale:\\

%\hline\hline
%\end{tabular}\\

\vspace{0.4cm}
%------------------------------------------------
\subsubsection{\'Equation de type $a\cos{(x)}+b\sin{(x)}=c$, avec $(a,b) \in (\R^\star)^2$}

Le but est de factoriser l'expression pour faire appara\^itre un seul cosinus. Pour cela, on cherche $r\in \R^+,$ et $\phi \in [0,\pi[$ tels que  
$$a\cos{(x)}+b\sin{(x)}=r\cos(x-\phi)$$

On obtient : 

$$a\cos{(x)}+b\sin{(x)} = r \cos(x) \cos(\phi) + r\sin(x) \sin(\phi)$$
En identifiant on obtient : 

$$
\left\{ \begin{array}{l}
r\cos(\phi) = a\\
\textmd{et}\\
r\sin(\phi) = b
\end{array}\right.$$


Ainsi on doit avoir $r^2 =a^2+b^2$ et
$$
\left\{ \begin{array}{l}
\cos(\phi) = \frac{a}{\sqrt{a^2+b^2}}\\
\textmd{et}\\
\sin(\phi) = \frac{b}{\sqrt{a^2+b^2}}
\end{array}\right.$$


Une fois sous cette forme on résout
$$r\cos(x-\phi) = c$$


\begin{exemple}
Résoudre $\sqrt{3}\cos{(x)}+\sin{(x)}=1$.
\end{exemple}

\begin{cor}
On cherche $r>0$ et $\phi\in [0,2\pi[$  tel que   $\sqrt{3}\cos{(x)}+\sin{(x)} =r \cos(x-\phi)$
D'après les calculs précédents on sait que nécessairement 
$$r^2 =3+1= 4$$
donc $r=2$ car $r>0$. On a ensuite 

$$
\left\{ \begin{array}{l}
\cos(\phi) = \frac{\sqrt{3}}{2}\\
\textmd{et}\\
\sin(\phi) = \frac{1}{2}
\end{array}\right.$$
Donc $\phi= \frac{\pi}{6}$. Ainsi 
$$\sqrt{3}\cos{(x)}+\sin{(x)} =2 \cos(x-\frac{\pi}{6})$$
Donc l'équation $\sqrt{3}\cos{(x)}+\sin{(x)} =1$ équivaut à 
$$\cos(x-\frac{\pi}{6})=\frac{1}{2}$$
Donc 
$$
\left\{ \begin{array}{l}
x-\frac{\pi}{6}\equiv \frac{\pi}{3}\,  [2\pi]\\
\textmd{ou }\\
x-\frac{\pi}{6}\equiv -\frac{\pi}{3}\,  [2\pi]\\
\end{array}\right.$$

Soit 
$$
\left\{ \begin{array}{l}
x\equiv \frac{\pi}{2}\,  [2\pi]\\
\textmd{ou }\\
x\equiv \frac{\pi}{6}\,  [2\pi]\\
\end{array}\right.$$

\end{cor}

\begin{rem}
En physique, cette m\'ethode permet de d\'eterminer l'amplitude et la phase d'un signal d\'efini comme la somme de deux signaux.
\end{rem}


%
%
%
%\vspace{0.3cm}
%\setlength\fboxrule{1pt}
%\noindent \doublebox{
%\begin{minipage}[t]{0.8\textwidth}
%M\'ethode: r\'esoudre $a\cos{(x)}+b\sin{(x)}=c$.
%\begin{itemize}
%\item[$\bullet$] \dotfill 
%\item[$\bullet$] \dotfill 
%\end{itemize}
%\end{minipage}}
%\setlength\fboxrule{0.5pt}\\
%
%
%
%\begin{exemple}
%R\'esoudre sur $\R$ puis sur $\lbrack 0,\pi\lbrack$ : $\sqrt{3}\cos{(x)}+\sin{(x)}=\sqrt{2}$.
%\vspace*{5.5cm}
%\end{exemple}
%

{\footnotesize
\begin{exo} R\'esoudre sur $\R$ puis sur $\lbrack 0,\pi\lbrack$ les \'equations suivantes :  $\cos{(x)}-\sin{(x)}=\sqrt{\frac{3}{2}}$ et  $\cos{(2x)}+\sqrt{3}\sin{(2x)}=-\sqrt{2}$.
\end{exo}}
%{\footnotesize\begin{proof} 
%\vspace{3cm}
%\end{proof}}
\vspace{0.4cm}


\vspace{0.5cm}



%-----------------------------------------
\subsubsection{\'Equation o\`{u} le cosinus, sinus ou la tangente peuvent \^{e}tre prise comme variable}

 Il s'agit des \'equations o\`{u} l'on peut poser le changement de variable $X=\cos{(x)}$ ou $X=\sin{(x)}$ ou $X=\tan{(x)}$. 
On reconna\^{i}t ces \'equations lorsque l'on peut mettre l'\'equation \`{a} r\'esoudre sous la forme d'une \'equation ne comportant 
soit que des $\cos{(x)},\ \cos^2{(x)},\ \cos^3{(x)}\dots$, soit que des $\sin{(x)},\ \sin^2{(x)},\ \sin^3{(x)}\dots$, soit que des $\tan{(x)},\ \tan^2{(x)},\ \tan^3{(x)}\dots$.\\
%
%\setlength\fboxrule{1pt}
%\noindent \doublebox{
%\begin{minipage}[t]{0.8\textwidth}
%\begin{itemize}
%\item[$\bullet$] \dotfill 
%\item[$\bullet$] \dotfill 
%\item[$\bullet$] \dotfill 
%\end{itemize}
%\end{minipage}}
%\setlength\fboxrule{0.5pt}
%%

 \begin{tabular}{ll}
$\bullet$ Poser $X$ \'egal \`a $\cos{x}$ ou $\sin{x}$ ou $\tan{x}$.\\
$\bullet$ R\'esoudre l'\'equation en $X$ du second, troisi\`eme... degr\'e ainsi obtenue. \\
$\bullet$ Revenir ensuite \`{a} $x$ en r\'esolvant des \'equations fondamentales. \\
\end{tabular}

\begin{exemple}
R\'esoudre sur $\R$ puis sur $\lbrack 0,2\pi\lbrack$ l'\'equation : $2\cos^2 (x) + \cos (x) -1 = 0$. 

\end{exemple}
\begin{cor}
On pose $X=\cos(x)$ on obtient 
$$2X^2+X-1=0,$$
dont les solutions sont $X =-1$ et $X=\frac{1}{2}$. L'équation est équivalent à 
$$
\left\{ \begin{array}{l}
\cos(x) = -1\\
\textmd{ou }\\
\cos(x) = \frac{1}{2}
\end{array}\right.$$
Soit 
$$
\left\{ \begin{array}{l}
x \equiv  \pi\, [2\pi] \\
\textmd{ou }\\
x \equiv \frac{\pi}{3}, [2\pi] \\
\textmd{ou }\\
x \equiv -\frac{\pi}{3}, [2\pi] 
\end{array}\right.$$
Sur $[0,2\pi[$ les solutions sont donc :
$$\cS=\{ \pi, \frac{\pi}{3}, \frac{5\pi}{3}\}.$$

\end{cor}


{\footnotesize
\begin{exo} R\'esoudre sur $\R$ puis sur $\lbrack 0,2\pi\lbrack$ les \'equations suivantes: $4\cos^2{(x)}-2(\sqrt{3}+\sqrt{2})\cos{(x)}+\sqrt{6}=0$, $\tan^3{(x)}-\sqrt{3}\tan^2{(x)}-\tan{(x)}+\sqrt{3}=0$ et $2\sin^2{(x)}+5\cos{(x)}-4=0$.
\end{exo}}
\vspace{0.3cm}
%------------------------------------------------





%------------------------------------------------
\subsubsection{Autres types d'\'equation}

\noindent Lorsqu'on est dans aucun des cas pr\'ec\'edents, on utilise les formules trigonom\'etriques pour se ramener \`a une \'equation factoris\'ee dont chaque terme est une \'equation fondamentale.

\begin{exemple}
R\'esoudre sur $\R$ puis sur $\lbrack -\pi,\pi\lbrack$ : $1+\cos{(x)}+\cos{(2x)}+\cos{(3x)}=0$.

\end{exemple}

\begin{cor}
$$\cos(2x)=2\cos^2(x)-1$$

\begin{align*}
\cos(3x) &=\Re(e^{3ix})&\\
			&=\Re({e^{ix}}^3)\\
			&=\Re({(\cos(3x)+i\sin(x))}^3\\ 
			&=\cos^3(x)-3\cos(x) \sin^2(x)\\
			&=\cos^3(x)-3\cos(x) (1-\cos^2(x))\\
			&=4\cos^3(x)-3\cos(x)
\end{align*}


On a donc 
\begin{align*}
1+\cos{(x)}+\cos{(2x)}+\cos{(3x)}&= 1+\cos(x) + 2\cos^2(x)-1 + 4\cos^3(x)-3\cos(x)\\
													&= -2\cos(x) + 2\cos^2(x)+ 4\cos^3(x)
\end{align*}
L'équation ést donc équivalente à 
$$-\cos(x)+\cos^2(x)+2\cos^3(x)=0.$$
On fait le changement de variable $X=\cos(x)$ on obtient : 
$$-X+X^2+2X^3=0$$
Soit $$X(2X^2+X-1)=0$$
Les solutions sont donc $X=0$ ou $2X^2+X-1=0$. En revenant à la variable $x$ on obtient 
$\cos(x)=0$ si et seulement si $x\equiv \frac{\pi}{2}\, [\pi]$
Les solutions sont donc 
$$
\left\{ \begin{array}{l}
x \equiv  \pi\, [2\pi] \\
\textmd{ou }\\
x \equiv \frac{\pi}{3}, [2\pi] \\
\textmd{ou }\\
x \equiv -\frac{\pi}{3}, [2\pi] \\
\textmd{ou }\\
x\equiv \frac{\pi}{2}\, [\pi]
\end{array}\right.$$






\end{cor}

%\setlength\fboxrule{1pt}
%\noindent \doublebox{
%\begin{minipage}[t]{0.8\textwidth}
%\begin{itemize}
%\item[$\bullet$] \dotfill \phantom{\hspace{4cm}} 
%\item[$\bullet$] \dotfill 
%\item[$\bullet$] \dotfill  \phantom{\hspace{2cm}}  
%\end{itemize}
%\end{minipage}}
%\setlength\fboxrule{0.5pt}

%\noindent \begin{tabular}{||l||}
%\hline\hline
%$\bullet$ Rechercher le domaine de d\'efinition.\\
%$\bullet$ Utiliser les formules trigonom\'etriques pour se ramener \`a une \'equation factoris\'ee\\ dont chaque terme est de type \'equation fondamentale.\\
%$\bullet$ R\'esoudre alors s\'epar\'ement chaque \'equation fondamentale.\\
%\hline\hline
%\end{tabular}\\


{\footnotesize
\begin{exo} R\'esoudre sur $\R$ puis sur $\lbrack -\pi,\pi\lbrack$ : $\cos{(2x)}+\cos{(x)}=\sin{(2x)}+\sin{(x)}$.
\end{exo}}

\begin{cor}
On utilise les formules d'additivité de cosinus et sinus : 
$$\cos(2x)+\cos(x)= 2\cos(\frac{3x}{2})\cos(\frac{x}{2})$$
$$\sin(2x)+\sin(x)= 2\sin(\frac{3x}{2})\cos(\frac{x}{2})$$ 
On obtient donc l'équation: 
$$\cos(\frac{3x}{2})\cos(\frac{x}{2}) =\sin(\frac{3x}{2})\cos(\frac{x}{2})$$
soit 
$$\cos(\frac{x}{2})(\cos(\frac{3x}{2})-\sin(\frac{3x}{2}))=0$$
Ce qui équivaut 
$$
\left\{ \begin{array}{l}
\cos(\frac{x}{2})=0 \\
\textmd{ou }\\
(\cos(\frac{3x}{2})-\sin(\frac{3x}{2})=0 
\end{array}\right.$$

$$
\left\{ \begin{array}{l}
\frac{x}{2}\equiv \frac{\pi}{2}\, [\pi] \\
\textmd{ou }\\
\cos(\frac{3x}{2})=\cos(\frac{\pi}{2}-\frac{3x}{2})
\end{array}\right.$$

$$
\left\{ \begin{array}{l}
x\equiv \pi\, [2\pi] \\
\textmd{ou }\\
\frac{3x}{2}=(\frac{\pi}{2}-\frac{3x}{2}) \, [2\pi]\\
\textmd{ou }\\
\frac{3x}{2}=(-\frac{\pi}{2}+\frac{3x}{2}) \, [2\pi]
\end{array}\right.$$

\end{cor}


%\begin{cor}[2]
%On utilise les formules d'additivité de cosinus et sinus : 
%$$\cos(2x)= 2\cos^2(x)-1$$
%$$\sin(2x)=2\sin(x)\cos(x)$$
%L'équation est donc équivalente à 
%$$2\cos^2(x)-1+\cos(x) =2\sin(x)\cos(x)+\sin(x)$$
%Soit en factorisant :
%$$(2\cos(x)+1)(\cos(x)-1) =\sin(x)(2\cos(x)+1)=0$$
%
%
%
%\end{cor}


%

%



%------------------------------------------------
%-------------------------------------------------
%-------------------------------------------------
%--------------------------------------------------
%------------------------------------------------
\section{R\'esolution des in\'equations trigonom\'etriques}

%------------------------------------------------
%-------------------------------------------------
\subsection{R\'esolution des in\'equations fondamentales}


\noindent Les in\'equations fondamentales sont les in\'equations de type $\cos{x}\leq a$, $\sin{x}\geq a$, $\tan{x}<a$.  \\

\setlength\fboxrule{1pt}
\noindent \doublebox{
\begin{minipage}[t]{0.55\textwidth}
On r\'esout GRAPHIQUEMENT sur le cercle trigonom\'etrique.
\end{minipage}}
\setlength\fboxrule{0.5pt}

\begin{rem}
\warning\, \, Ne jamais r\'esoudre une in\'equation sans passer par le cercle trigonom\'etrique.
\end{rem}



\begin{exemple}
R\'esoudre  sur $\lbrack 0,2\pi\lbrack$ puis sur $\R$l'in\'equation : $\cos{(x)}<\demi$.
\end{exemple}
\begin{cor}
Sur $[0,2\pi]$ les solutions sont $$]\frac{\pi}{3}, 2\pi-\frac{\pi}{3}[=]\frac{\pi}{3}, \frac{5\pi}{3}[$$
Sur $\R$ les solutions sont 
$$\ddp \bigcup_{k\in \Z} ]\frac{\pi}{3}+2k\pi, \frac{5\pi}{3}+2k\pi[$$

\end{cor}



{\footnotesize
\begin{exo} R\'esoudre sur $\R$ puis sur $\lbrack 0,2\pi\lbrack$ les in\'equations suivantes:  $\ddp \sin{(x)}\geq \frac{\sqrt{3}}{2}$, $\ddp \cos{(2x)}>-\frac{\sqrt{3}}{2}$, $\ddp \sin{(3x)}<\frac{\sqrt{2}}{2}$, $\tan{(x)}\geq -1$ et $\ddp -1<\tan{(x)}<\frac{1}{\sqrt{3}}$.
\end{exo}}


%------------------------------------------------
%-------------------------------------------------
\subsection{R\'esolution des autres in\'equations}


\subsubsection{In\'equation de type $a\cos{(x)}+b\sin{(x)}$}

\begin{exemple}
R\'esoudre sur $\R$ puis sur $\lbrack 0,\pi\lbrack$ : $\sqrt{3}\cos{(x)}-\sin{(x)}>\sqrt{2}$.

\end{exemple}
\begin{cor}
On va mettre $\sqrt{3}\cos{(x)}-\sin{(x)}$ sous la forme $r\cos(x-\phi)$ pour $r>0$ et $\phi\in [-\pi,\pi]$. 

On a vu que $r^2= 3+1=4$, ce qui donne $r=2$. 

Donc $\cos(\phi) =\frac{\sqrt{3}}{2}$ et $\sin(\phi) = \frac{-1}{2} $ 
D'où $\phi=\frac{-\pi}{6}$.

Donc l'équation devient $$2 \cos(x+\frac{\pi}{6}) >\sqrt{2}.$$
Soit 
$$\cos(x+\frac{\pi}{6}) >\frac{\sqrt{2}}{2}$$
Donc  sur $]-\pi, \pi]$:
$$-\frac{\pi}{4}< x+\frac{\pi}{6} <\frac{\pi}{4}$$
Soit 
$$-\frac{\pi}{4}-\frac{\pi}{6} < x <\frac{\pi}{4}-\frac{\pi}{6} $$
$$-\frac{5\pi}{12} < x <\frac{\pi}{12} $$
Donc sur $\R$ les solutions sont données par:
$$\cS=\bigcup_{k\in \Z} ]-\frac{5\pi}{12}+2k\pi , \frac{\pi}{12} +2k\pi[$$


\end{cor}


{\footnotesize
\begin{exo} R\'esoudre sur $\R$ puis sur $\lbrack 0,\pi\lbrack$ les in\'equations suivantes: $\ddp \cos{(x)}-\sin{(x)}\geq \sqrt{\frac{3}{2}}$ et \noindent $\ddp \cos{(2x)}+\sqrt{3}\sin{(2x)}\leq -\sqrt{2}$.
\end{exo}}

%\vspace*{0.5cm}
%

%------------------------------------------------
\subsubsection{In\'equation o\`{u} le cosinus, sinus ou la tangente peuvent \^{e}tre pris comme variable}


\begin{exemple}
R\'esoudre sur $\R$ puis sur $\lbrack 0,2\pi\lbrack$ l'in\'equation : $2\cos^2 (x) + \cos (x) > 1$. 

\end{exemple}
\begin{cor}
$2X^2+X-1>0 $ a pour solution 
$$X\in ]-\infty , -1[ \cup ]\frac{1}{2},\infty[$$
Donc l'équation $2\cos^2 (x) + \cos (x) > 1$  est équivalente à 
$$\cos(x) >\frac{1}{2}$$

\end{cor}


{\footnotesize
\begin{exo} R\'esoudre sur $\R$ puis sur $\lbrack 0,2\pi\lbrack$ les in\'equations suivantes: $4\cos^2{(x)}-2(\sqrt{3}+\sqrt{2})\cos{(x)}+\sqrt{6}<0$, $\tan^3{(x)}-\sqrt{3}\tan^2{(x)}-\tan{(x)}+\sqrt{3}\geq 0$ et $2\sin^2{(x)}+5\cos{(x)}-4\leq 0$.
\end{exo}}
\vspace{0.3cm}

%%------------------------------------------------
%{Autres types d'in\'equation}\\
%
%
%{\footnotesize
%\begin{exo} R\'esoudre sur $\lbrack 0,\pi\rbrack$ les in\'equations suivantes: 
%%$1+\cos{(x)}+\cos{(2x)}+\cos{(3x)} < 0$, 
%$\sin{(x)}+\sin{(2x)}+\sin{(3x)}>0$ et $\cos{(3x)}+\cos{(x)}-\sin{(3x)}-\sin{(x)}>0$.
%\end{exo}}
%
%
%{\footnotesize
%\begin{exo} R\'esoudre dans $\R$ l'in\'equation: $2\sin{(x)}-1<\sqrt{1-4\cos^2{(x)}}$.
%\end{exo}}




\section{Etude des fonctions trigonométriques}

\subsection{Reduction du domaine d'étude}

\begin{defi}
Soit $f$ une fonction num\'erique de domaine de d\'efinition $\mathcal{D}_f$.
\begin{itemize}
 \item[$\bullet$]
On dit que \textbf{$f$ est paire} si
\begin{itemize}
\item[$\star$] $\forall x \in D_f$ , $-x\in D_f$ 

et 

\item[$\star$] $\forall x \in D_f$, $f(-x)=f(x)$.
\end{itemize} 
Graphiquement, la courbe repr\'esentative de $f$ est symmétrique par rapport à l'axe des ordonnées. 
\item[$\bullet$] On dit que \textbf{$f$ est impaire} si
\begin{itemize}
\item[$\star$] $\forall x \in D_f$ , $-x\in D_f$ 

et 

\item[$\star$] $\forall x \in D_f$, $f(-x)=-f(x)$.
\end{itemize}   
Graphiquement, la courbe repr\'esentative de $f$ est symmétrique par rapport à l'originie. 
\end{itemize}
\end{defi}


\begin{rem}
Lorsqu'une fonction $f$ est paire ou impaire, il suffit de l'\'etudier et de la tracer sur $\mathcal{D}_f\cap\R_+$
puis d'effectuer la sym\'etrie indiqu\'ee ci-dessus pour obtenir la repr\'esentation graphique compl\`{e}te.
\end{rem}

%\vspace{0.2cm}
\begin{exemples} 
\begin{itemize}
\item[$\bullet$] Exemples de fonctions paires: $x\mapsto x^2$, $x\mapsto \cos(x)$, $x\mapsto \frac{e^{x}+e^{-x}}{2}$
\item[$\bullet$] Exemples de fonctions impaires: $x\mapsto x^3$, $x\mapsto \sin(x)$, $x\mapsto \frac{e^{x}-e^{-x}}{2}$
\end{itemize}
\end{exemples}

\vspace{0.2cm}

{\footnotesize
\begin{exercice}
Que dire de l'opposée, l'inverse d'une fonction paire (resp.  impaire) ?

Que dire de la somme de deux fonctions paires (resp. impaires) ?

Que dire de la somme d'une fonction paire et d'une fonction impaire ? 


Que dire de la composée de deux fonctions paires (resp impaires)  ? 

Que dire de la composée  d'une fonction paire et d'une fonction impaire ? 

Montrer que toute fonction définie sur $\R$ s'écrit comme la somme d'une fonction paire et d'une fonction impaire. 
% Soit $f:\R\rightarrow \R$ une fonction  Compl\'eter et d\'emontrer les propri\'et\'es suivantes:
%\begin{enumerate}
%\item L'oppos\'ee d'une fonction paire (resp impaire) est \dotfill (resp \dotfill).\phantom{\hspace{4cm}}
%\item La compos\'ee d'une fonction paire et d'une fonction impaire est \dotfill. \phantom{\hspace{7cm} }
%\end{enumerate}
\end{exercice}}

\begin{rem}
    Il peut exister d'autres symétries : eg $f(a-x)= f(x)$ pour un $a\in \R$ et tout $x\in \R$. Dans ce cas la courbe de $f$ sera symétrique par rapport  à la droite d'équation $x=\frac{a}{2}$. On peut alors diminuer le domaine d'étude de "moitié" : en considérant seuelement la partie $]-\infty, \frac{a}{2}]$

    Exemple : 
    $f(x)=\cos(2x)$ alors $f(\pi-x)=f(x)$ donc la courbe représentative de $f$ est symmétrique par rapport à la droite d'équation $x=\frac{\pi}{2}$.
\end{rem}



\begin{defi}
Soit $f$ une fonction num\'erique de domaine de d\'efinition $\mathcal{D}_f$.\\
 On dit que $f$ est p\'eriodique de p\'eriode $T>0$ si
\begin{itemize}
\item[$\star$] $ x\in D_f\equivaut x+T\in D_f$

et

\item[$\star$] $\forall x\in D_f$, $f(x+T)=f(x)$
\end{itemize}  
Graphiquement, la courbe repr\'esentative de $f$ est invariante par la translation de vecteur $T\vec{i}$ 
\end{defi}
}

\begin{rem} Si $f$ est p\'eriodique de p\'eriode $T$, alors pour tout $k \in \Z$, on a $f(x+kT)=f(x)$\\
Il suffit de l'\'etudier et de la tracer sur un intervalle de longueur $T$ puis d'effectuer la translation de vecteur $T\vec{i}$ pour obtenir la repr\'esentation graphique compl\`{e}te.
\end{rem}

%\vspace{0.2cm}

\begin{exemples} Exemples de fonctions p\'eriodiques: $x\mapsto \cos(x)$, $x\mapsto \sin(x)$, $x\mapsto \floor{x}-x$.
\end{exemples}

\vspace{0.1cm}

{\footnotesize
\begin{exercice} Soit $f:\R\rightarrow \R$ une fonction. D\'emontrer les propri\'et\'es suivantes:
\begin{enumerate}
\item Si $f$ est une fonction p\'eriodique de p\'eriode $T$ et $g:\ \R\rightarrow \R$, alors $g\circ f$ est $T$ p\'eriodique.
\item Si $g$ est une fonction p\'eriodique de p\'eriode $T$ et $(a,b)\in \R^{\star}\times \R$, alors $h:\ x\mapsto g(ax+b)$ est p\'eriodique de p\'eriode $\ddp\frac{T}{|a|}$.\\
Application : donner la p\'eriode des fonctions $x \mapsto \cos(2x)$, $x \mapsto \sin\left(\frac{\pi x}{4}\right)$ et $x \mapsto \tan(3x)$.
\end{enumerate}
\end{exercice}}



%------------------------------------------------


\end{document}