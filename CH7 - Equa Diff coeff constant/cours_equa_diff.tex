\documentclass[a4paper, 11pt]{article}
\input{macro/package.tex}
\input{macro/environement}
% Header et footer

\pagestyle{fancy}
\fancyhead{}
\fancyfoot{}
\renewcommand{\headwidth}{\textwidth}
\renewcommand{\footrulewidth}{0.4pt}
\renewcommand{\headrulewidth}{0pt}
\renewcommand{\footruleskip}{5px}

\fancyfoot[R]{Olivier Glorieux}
%\fancyfoot[R]{Jules Glorieux}

\fancyfoot[C]{ Page \thepage }
\fancyfoot[L]{1BIOA - Lycée Chaptal}
%\fancyfoot[L]{MP*-Lycée Chaptal}
%\fancyfoot[L]{Famille Lapin}

\input{macro/newcommand.tex}
\geometry{hmargin=2.0cm, vmargin=2.5cm}




\begin{document}
% \tableofcontents
\title{CH7 - Équations différentielles à coefficients constants}


\noindent Une \'equation diff\'erentielle est une \'equation dont la variable est une fonction $f$, et qui met en jeu $f$ et ses d\'eriv\'ees. Les \'equations diff\'erentielles interviennent dans de nombreux domaines physiques et biologiques pour \'etudier des ph\'enom\`enes qui \'evoluent dans le temps, comme par exemple des r\'eactions chimiques ou la croissance d'une population.

\begin{exemple}
  En dynamique des populations, le mod\`ele de Malthus d\'ecrit l'\'evolution d'une population plac\'ee dans des conditions id\'eales (nourriture et place illimit\'ee) gr\^ace \`a une \'equation diff\'erentielle. Le nombre d'individus au temps $t$ est not\'e $N(t)$, et la fonction $N$ ob\'eit \`a l'\'equation diff\'erentielle
  $$N'(t) = \lambda N(t),$$
  ce qui signifie que la croissance de la population est proportionnelle au nombre d'individus.
\end{exemple}


\noindent Les \'equations diff\'erentielles sont omnipr\'esentes en sciences. D\`es qu'il s'agit d'\'etudier les variations au cours du temps d'une quantit\'e, que ce soit en physique, en chimie, en biologie ou m\^eme en \'economie, on est amen\'e \`a mod\'eliser le ph\'enom\`ene \`a l'aide d'\'equations diff\'erentielles. Vous en avez d\'ej\`a (ou vous allez en) rencontr\'ees en \'electricit\'e, en m\'ecanique, en cin\'etique...

%------------------------------------------------
%-------------------------------------------------
%debut
%--------------------------------------------------
%------------------------------------------------
%----------------------------------------------------
%-----------------------------------------------------
%-------------------------------------------------------
\section{\'Equations diff\'erentielles lin\'eaires du premier ordre}

\noindent Dans toute cette section, on consid\`{e}re $I$ un intervalle de $\R$, et deux fonctions $a,b: I\rightarrow \R$ continues sur $I$.


  %----------------------------------------------------
  %-----------------------------------------------------
  %\subsection{D\'efinitions et notations}

  {\noindent

    \begin{defi}  \textbf{\'Equations diff\'erentielles lin\'eaires du premier ordre:}
      \begin{itemize}
        \item[$\bullet$] On appelle \'equation diff\'erentielle lin\'eaire du premier ordre sous forme r\'esoluble toute \'equation de la\vsec\\
              forme: $y'(x)+a(x)y(x)=b(x) (1) $
        \item[$\bullet$] Lorsque $b$ est la fonction nulle, on dit que c'est une équation homogéne. \vsec\\
              \noindent On appelle \'equation homog\`ene associ\'ee \`a (1) l'\'equation: \\
              \vsec
        \item[$\bullet$] Une \'equation diff\'erentielle lin\'eaire est dite \`a coefficients constants lorsque la fonction $a$ est constante.
      \end{itemize}
    \end{defi}

  }
\vsec\vsec

{\noindent

  \begin{defi}  \textbf{Solution d'une \'equation diff\'erentielle lin\'eaire du premier ordre:}\\
    \noindent On appelle solution de l'\'equation diff\'erentielle lin\'eaire (1) toute fonction: \vsec
    \begin{itemize}
      \item[$\bullet$] $f$ dérivable sur $\R$ \vsec
      \item[$\bullet$] Pour tout $x\in\R$,  $f'(x) +a(x)f(x)=b(x)$\vsec
    \end{itemize}
  \end{defi}

}

Dans ce chapitre on se contentera de regarder le cas où la fonction $x\mapsto a(x)$ est constante. On cherche donc à résoudre les équations de la forme
$$y'(x)+ay(x)=b(x)  $$
où $a\in\R$ et $b$ est une fonction de réelle.
\subsection{Résolution de l'équation homogène associée}
\begin{exercice}
  R\'esolution de $y^{\prime}(x)+2y(x)=0$.
\end{exercice}


%\newpage
\vspace*{\fill}
\begin{theorem}
  Soit $a\in\R$
  \noindent Les solutions de l'\'equation diff\'erentielle homog\`{e}ne associ\'ee : $y^{\prime}(x)+ay(x)=0$ sont les fonctions:
  $$S_h=\hspace{5cm}$$
\end{theorem}




%----------------------------------------------------
\newpage
%-----------------------------------------------------
\subsection{Solution particuli\`ere  avec second membre}

\begin{exercice}
  \begin{enumerate}
    \item Trouver une solution $f_p$ de  $y^{\prime}(x)+2y(x)=3x+1$.
    \item Vérifier que $f(x)=f_p(x)+e^{-2x}$ est aussi solution de l'équation précédente.
  \end{enumerate}
\end{exercice}







\newpage
Lorsque $a$ est constante et que le second membre $b(x)$ est de type polynomial et/ou exponentiel la recherche d'une solution particuli\`ere est simple et rapide en appliquant les m\'ethodes suivantes:\vsec

\noindent
\begin{tabular}{|l|l|}
  \hline
  \rule[-5mm]{0pt}{10mm} \textbf{Expression de $b(x)$ et condition \'eventuelle} & \textbf{Forme de la solution particuli\`ere $y_p(x)$}      \\
  \hline
  \rule[-5mm]{0pt}{10mm}  Polyn\^ome de degr\'e $n$                              &                                                            \\
  \hline
  \rule[-5mm]{0pt}{10mm} $P(x)e^{mx}\qquad$ $m\in\bC$, $m\not=-a$                & $Q(x)e^{mx}$ avec $\deg P=\deg Q$, $Q$ \`a d\'eterminer    \\
  \rule[-5mm]{0pt}{10mm} $P(x)e^{-a x}$                                          & $ xQ(x) e^{mx}$ avec $\deg P=\deg Q$, $Q$ \`a d\'eterminer \\
  \hline
\end{tabular}

\vsec


\begin{prop}
  Soient $ b_1, b_2: I\rightarrow \R$ des fonctions continues sur $I$ et soient $(\lambda_1,\lambda_2)\in\R^2$. Si $y_1$ et $y_2$ sont respectivement des solutions de
  $$y^{\prime}(x)+ay(x)=b_1(x)\qquad \hbox{et}\qquad y^{\prime}(x)+ay(x)=b_2(x)$$
  alors:\vsec\\
  $\lambda_1y_1+\lambda_2y_2$ est solution de $$y'+a(x)y = \lambda_1b_1(x)+\lambda_2b_2(x)$$
\end{prop}



\begin{theorem}
  La solution g\'en\'erale de $y'(x)+ay(x)=b(x)$ est la somme de la solution g\'en\'erale de l'\'equation diff\'erentielle homog\`ene associ\'ee  $y'(x)+ay(x)=0$ et d'une solution particuli\`ere de $y'(x)+ay(x)=b(x)$. Ainsi, si:
  \begin{itemize}
    \item[$\bullet$] on conna\^{i}t l'ensemble des solutions de l'\'equation homog\`ene associ\'ee (2) : $S_h=\{ x\mapsto C e^{-ax}\, |\, C\in \R\}$
    \item[$\bullet$] On note $y_p$ une solution particuli\`ere de $y'(x)+ay(x)=b(x)$.
  \end{itemize}
  \vsec
  Alors l'ensemble des solutions de $y'(x)+ay(x)=b(x)$ est :
  \conclusion{$S_h=\{ x\mapsto y_p(x) + C e^{-ax}\, |\, C\in \R\}$}
\end{theorem}



\begin{minipage}{0.9\textwidth}
  \begin{tcolorbox}[{colback=red!5!white,colframe=red!75!black}]
    \textbf{M\'ethode pour r\'esoudre une \'equation diff\'erentielle lin\'eaire du premier ordre}
    \begin{itemize}
      \item[$\bullet$] Commencer par dire que c'est une \'equation diff\'erentielle lin\'eaire du premier ordre.
      \item[$\bullet$] R\'esoudre l'\'equation diff\'erentielle homog\`ene associ\'ee.
      \item[$\bullet$] Recherche d'une solution particuli\`ere de l'\'equation avec second membre.
      \item[$\bullet$] Additionner les deux.
    \end{itemize}
  \end{tcolorbox}
\end{minipage}

\newpage
\begin{exercice}
  R\'esoudre l'\'equations diff\'erentielle suivante :

  $$y^{\prime}+2y=xe^{-2x}.$$

\end{exercice}

\newpage



%----------------------------------------------------
%-----------------------------------------------------
\subsection{\'Equation diff\'erentielle lin\'eaire du premier ordre avec condition initiale}


\begin{prop}
  Soit $x_0\in I$ et $y_0\in\R$.\\
  \noindent L'\'equation diff\'erentielle lin\'eaire du premier ordre $y^{\prime}(x)+ay(x)=b(x)$ admet
  une unique solution $y$ v\'erifiant $y(x_0)=y_0$.\\
  \noindent La condition $y(x_0)=y_0$ d\'etermine la constante.
\end{prop}
\begin{defi}
  Le système équation+condition initiale s'appelle \underline{Problème de Cauchy}
\end{defi}





\begin{minipage}{0.9\textwidth}
  \begin{tcolorbox}[colback=red!5!white,colframe=red!75!black]

    \textbf{M\'ethode pour trouver une solution v\'erifiant une condition initiale :}
    \begin{itemize}
      \item[$\bullet$] R\'esoudre l'\'equation diff\'erentielle avec la m\'ethode g\'en\'erale.
      \item[$\bullet$] Utiliser la condition $y(t_0) = y_0$ pour d\'eterminer la constante $C$.
    \end{itemize}

  \end{tcolorbox}
\end{minipage}

\begin{exercice}
  R\'esoudre l'\'equation diff\'erentielle $y^{\prime}(x)+2y(x)=x$ avec $y(1)=2$.
\end{exercice}


%----

\newpage
\section{\'Equations diff\'erentielles lin\'eaires du second ordre \`a coefficients constants}

\noindent On consid\`{e}re dans toute cette partie $I$ un intervalle de $\R$, $(a,b,c)$ trois constantes r\'eelles avec $a\not= 0$ et $f: I\rightarrow \bK$ continue (avec $\bK=\R$ ou $\bC$).



\begin{defi}
  \textbf{\'Equations diff\'erentielles lin\'eaires du second ordre \`a coefficients constants:}
  \begin{itemize}
    \item[$\bullet$] On appelle \'equation diff\'erentielle lin\'eaire du second ordre \`a coefficients constants toute \'equation de la forme :
          $$ay''(x)+by'(x)+cy(x)=f(x)$$

    \item[$\bullet$] Lorsque $f$ est la fonction nulle, on dit que c'est une \'equation homogène.
          \noindent On appelle \'equation homog\`ene associ\'ee \`a (1) l'\'equation:  $$ay''(x)+by'(x)+cy(x)=0$$
    \item[$\bullet$] On appelle \'equation caract\'eristique associ\'ee, l'\'equation: $$ $$
  \end{itemize}
\end{defi}


\begin{defi}
  \textbf{Solution d'une \'equation diff\'erentielle lin\'eaire du second ordre:}\\
  \noindent On appelle solution de l'\'equation diff\'erentielle lin\'eaire (1) toute fonction $u$
  \begin{itemize}
    \item[$\bullet$] Définie et dérivable deux fois sur $\R$
    \item[$\bullet$] Pour tout $x\in \R$ : $$au''(x) +bu'(x) +cu(x)=f(x)$$
  \end{itemize}
\end{defi}


%----------------------------------------------------
%-----------------------------------------------------
\begin{exercice}
  Trouver une solution de l'équation
  $$y''(x)+y(x)=0$$
\end{exercice}

%----------------------------------------------------
%-----------------------------------------------------
\newpage
\subsection{R\'esolution de l'\'equation lin\'eaire homog\`ene associ\'ee}

\noindent  {\noindent

  \begin{prop}
    \label{prop:ordre2_h}
    On note $\Delta$ le discriminant de l'\'equation caract\'eristique associ\'ee (3). Les solutions de l'\'equation diff\'erentielle homog\`ene associ\'ee (2) sont donn\'ees par:
    \begin{itemize}
      \item[$\bullet$] Si $\Delta>0$, l'\'equation caract\'eristique admet deux racines r\'elles distinctes $r_1$ et $r_2$ et
            $$S_h = \hspace{7cm} $$
      \item[$\bullet$] Si $\Delta=0$, l'\'equation caract\'eristique admet une racine double $r$ et\vsec
            $$S_h = \hspace{7cm} $$
      \item[$\bullet$] Si $\Delta<0$, l'\'equation caract\'eristique admet deux racines complexes conjugu\'ees $\alpha \pm i \omega$ et \vsec
            $$S_h = \hspace{7cm} $$
    \end{itemize}
    \vspace*{0cm}
  \end{prop}

}


\begin{rem}
  On notera l'analogie avec les suites lin\'eaires r\'ecurrentes d'ordre deux.
\end{rem}

\begin{exercice}
  R\'esoudre dans $\R$ les \'equation diff\'erentielle suivante

  $$y^{\prime\prime}(x)-2y^{\prime}(x)+y(x)=0$$
\end{exercice}



\newpage

%----------------------------------------------------
%-----------------------------------------------------
\subsection{Solution particuli\`ere  avec second membre}

\textbf{\large{Seconds membres particuliers}}\\
Le programme officiel indique que la forme des solutions particulières doit vous être donner. Si toutefois ce n'est pas le cas, voici un tableau qui r\'ecapitule sous quelle forme il faut chercher la solution particuli\`ere $y_p$ selon la forme du second membre $f(x)$ de l'\'equation diff\'erentielle \`a coefficients constants. N'apprenez surtout pas ce tableau par coeur. Soit vous comprenez que les solutions particulières sont "de la même forme" que le second membre, soit vous oubliez ça le plus vite possible et vous vous remettez à votre cours de SVT.


On rappelle ici qu'\`a toute \'equation diff\'erentielle du second ordre \`a coefficients constants: $ay^{\prime\prime}+by^{\prime}+cy=f(x)$, on lui associe l'\'equation caract\'eristique

\begin{equation}\label{eq caractéristique}\tag{EC}
  ax^2+bx+c=0
\end{equation}


\begin{minipage}[t]{0.98\textwidth}
  \hspace*{-1cm}
  \begin{tabular}{|l|l|}
    \hline
    \rule[-5mm]{0pt}{10mm} \textbf{Expression de $f(x)$ et condition \'eventuelle}                                            & \textbf{Forme de la solution particuli\`ere $y_p(x)$}                   \\
    \hline
    \rule[-5mm]{0pt}{10mm}  Polyn\^ome de degr\'e $n$ avec $c\not= 0$                                                         & Polyn\^ome de degr\'e $n$ \`a d\'eterminer                              \\
    \hline
    \rule[-5mm]{0pt}{10mm} $e^{mx}\qquad$ $m\in\bC$, $m$ n'est pas racine de (\ref{eq caractéristique})                       & $\alpha e^{mx}$, $\alpha$ \`a d\'eterminer                              \\
    \rule[-5mm]{0pt}{10mm}  $e^{mx}\qquad$ $m\in\bC$, $m$ est racine simple de (\ref{eq caractéristique})                     & $\alpha x e^{mx}$, $\alpha$ \`a d\'eterminer                            \\
    \rule[-5mm]{0pt}{10mm} $e^{mx}\qquad$ $m\in\bC$, $m$ est racine double de (\ref{eq caractéristique})                      & $\alpha x^2 e^{mx}$, $\alpha$ \`a d\'eterminer                          \\
    \hline
    \rule[-5mm]{0pt}{10mm} $P(x)e^{mx}\qquad$ $m\in\bC$, $m$ n'est pas racine de (\ref{eq caractéristique})                   & $Q(x)e^{mx}$ avec $\deg P=\deg Q$, $Q$ \`a d\'eterminer                 \\
    \rule[-5mm]{0pt}{10mm} $P(x)e^{mx}\qquad$ $m\in\bC$, $m$ est racine simple de (\ref{eq caractéristique})                  & $ xQ(x) e^{mx}$ avec $\deg P=\deg Q$, $Q$ \`a d\'eterminer              \\
    \rule[-5mm]{0pt}{10mm} $P(x)e^{mx}\qquad$ $m\in\bC$, $m$ est racine double de (\ref{eq caractéristique})                  & $ x^2Q(x) e^{mx}$ avec $\deg P=\deg Q$, $Q$ \`a d\'eterminer            \\
    \hline
    \rule[-5mm]{0pt}{10mm} $\cos{(wx)}$ ou $\sin{(wx)}\qquad$ $w\in\R^{\star}$, $iw$ pas racine de (\ref{eq caractéristique}) & $\alpha\cos{(wx)}+\beta\sin{(wx)}$, $\alpha,\beta$ \`a d\'eterminer     \\
    \rule[-5mm]{0pt}{10mm} $\cos{(wx)}$ ou $\sin{(wx)}\qquad$ $w\in\R^{\star}$, $iw$ racine de (\ref{eq caractéristique})     & $\alpha x\cos{(wx)}+\beta x\sin{(wx)}$, $\alpha,\beta$ \`a d\'eterminer \\
    \hline
  \end{tabular}
\end{minipage}
\vspace{0.5cm}


\textbf{\large{Principe de superposition}}



\begin{prop}
  Soient $(a,b,c)\in\R^3$ trois r\'eels avec $a\not= 0$, $f_1, f_2: I\rightarrow \bK$ deux fonctions continues sur $I$ et soient $(\lambda_1,\lambda_2)\in\R^2$. Si $y_1$ et $y_2$ sont respectivement des solutions de
  $$ay^{\prime\prime}+by^{\prime}+cy=f_1(x)\qquad \hbox{et}\qquad ay^{\prime\prime}+by^{\prime}+cy=f_2(x)$$
  alors $\lambda_1y_1+\lambda_2y_2$ est solution de $$ay''+by'+cy = \lambda_1f_1(x)+\lambda_2f_2(x)$$
\end{prop}

%



\begin{theorem}
  La solution g\'en\'erale de (1) est la somme de la solution g\'en\'erale de l'\'equation diff\'erentielle homog\`ene associ\'ee (2) et d'une solution particuli\`ere de (1). Ainsi, si:
  \begin{itemize}
    \item[$\bullet$] on conna\^{i}t l'ensemble des solutions de l'\'equation homog\`ene associ\'ee (2) : $$S_h=\{ t\mapsto C_1u_1(t)+ C_2u_2(t)\, |\,  (C_1,C_2)\in \R^2\}$$
    \item[$\bullet$] On note $y_p$ une solution particuli\`ere de (1)
  \end{itemize}
  \vsec
  Alors l'ensemble des solutions de (1) est : $$S_h=\{ t\mapsto y_p(t)  C_1u_1(t)+ C_2u_2(t)\, |\,  (C_1,C_2)\in \R^2\}$$
\end{theorem}

\begin{minipage}{0.9\textwidth}
  \begin{tcolorbox}[colback=red!5!white,colframe=red!75!black]

    \textbf{M\'ethode: R\'esoudre une \'equation diff\'erentielle lin\'eaire du second ordre \`a coefficients constants}
    \begin{itemize}
      \item[$\bullet$] Commencer par dire que c'est une \'equation diff\'erentielle lin\'eaire du second ordre \`a coefficients constants.
      \item[$\bullet$] R\'esoudre l'\'equation diff\'erentielle homog\`ene associ\'ee.
      \item[$\bullet$] Rechercher une solution particuli\`ere de l'\'equation avec second membre.
      \item[$\bullet$] Additionner les deux.
    \end{itemize}
  \end{tcolorbox}
\end{minipage}
\setlength\fboxrule{0.5pt}
\begin{exercice}
  R\'esoudre $y^{\prime\prime}+y= 9$ .
\end{exercice}
\newpage
%----------------------------------------------------
%-----------------------------------------------------
\subsection{\'Equation diff\'erentielle lin\'eaire du second ordre avec conditions initiales}

\noindent Il y a deux constantes \`a d\'eterminer, donc pour avoir unicit\'e de la solution, il faut, cette fois-ci, avoir deux conditions initales.\vsec

\begin{prop}
  Soit $x_0\in \R$ et $(y_0,y_1)\in\R^2$.\\
  \noindent L'\'equation diff\'erentielle lin\'eaire du second ordre \`a coefficients constants $$ay^{\prime\prime}(x)+by^{\prime}(x)+cy(x)=f(x)$$ admet une unique solution $y$ v\'erifiant
  $$\left\lbrace\begin{array}{lll}
      y(x_0)          & = & y_0\vsec \\
      y^{\prime}(x_0) & = & y_1.
    \end{array}\right.$$
\end{prop}


\begin{minipage}{0.9\textwidth}
  \begin{tcolorbox}[colback=red!5!white,colframe=red!75!black]
    \textbf{M\'ethode pour trouver une solution v\'erifiant une condition initiale :}
    \begin{itemize}
      \item[$\bullet$] R\'esoudre l'\'equation diff\'erentielle avec la m\'ethode g\'en\'erale.
      \item[$\bullet$] Utiliser les conditions $y(t_0) = y_0$ et $y^{\prime}(x_0)=y_1$ pour d\'eterminer les constantes $A$ et $B$.
    \end{itemize}

  \end{tcolorbox}
\end{minipage}

\begin{exercice}
  R\'esoudre $y^{\prime\prime}-2y^{\prime}-3y = 9x^2$ avec $y(0)=0$ et $y^{\prime}(0)=1$.
\end{exercice}

\end{document}