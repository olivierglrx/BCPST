\documentclass[a4paper, 11pt]{article}
\usepackage[utf8]{inputenc}
\usepackage{amssymb,amsmath,amsthm}
\usepackage{geometry}
\usepackage[T1]{fontenc}
\usepackage[french]{babel}
\usepackage{fontawesome}
\usepackage{pifont}
\usepackage{tcolorbox}
\usepackage{fancybox}
\usepackage{bbold}
\usepackage{tkz-tab}
\usepackage{tikz}
\usepackage{fancyhdr}
\usepackage{sectsty}
\usepackage[framemethod=TikZ]{mdframed}
\usepackage{stackengine}
\usepackage{scalerel}
\usepackage{xcolor}
\usepackage{hyperref}
\usepackage{listings}
\usepackage{enumitem}
\usepackage{stmaryrd} 
\usepackage{comment}


\hypersetup{
    colorlinks=true,
    urlcolor=blue,
    linkcolor=blue,
    breaklinks=true
}





\theoremstyle{definition}
\newtheorem{probleme}{Problème}
\theoremstyle{definition}


%%%%% box environement 
\newenvironment{fminipage}%
     {\begin{Sbox}\begin{minipage}}%
     {\end{minipage}\end{Sbox}\fbox{\TheSbox}}

\newenvironment{dboxminipage}%
     {\begin{Sbox}\begin{minipage}}%
     {\end{minipage}\end{Sbox}\doublebox{\TheSbox}}


%\fancyhead[R]{Chapitre 1 : Nombres}


\newenvironment{remarques}{ 
\paragraph{Remarques :}
	\begin{list}{$\bullet$}{}
}{
	\end{list}
}




\newtcolorbox{tcbdoublebox}[1][]{%
  sharp corners,
  colback=white,
  fontupper={\setlength{\parindent}{20pt}},
  #1
}







%Section
% \pretocmd{\section}{%
%   \ifnum\value{section}=0 \else\clearpage\fi
% }{}{}



\sectionfont{\normalfont\Large \bfseries \underline }
\subsectionfont{\normalfont\Large\itshape\underline}
\subsubsectionfont{\normalfont\large\itshape\underline}



%% Format théoreme, defintion, proposition.. 
\newmdtheoremenv[roundcorner = 5px,
leftmargin=15px,
rightmargin=30px,
innertopmargin=0px,
nobreak=true
]{theorem}{Théorème}

\newmdtheoremenv[roundcorner = 5px,
leftmargin=15px,
rightmargin=30px,
innertopmargin=0px,
]{theorem_break}[theorem]{Théorème}

\newmdtheoremenv[roundcorner = 5px,
leftmargin=15px,
rightmargin=30px,
innertopmargin=0px,
nobreak=true
]{corollaire}[theorem]{Corollaire}
\newcounter{defiCounter}
\usepackage{mdframed}
\newmdtheoremenv[%
roundcorner=5px,
innertopmargin=0px,
leftmargin=15px,
rightmargin=30px,
nobreak=true
]{defi}[defiCounter]{Définition}

\newmdtheoremenv[roundcorner = 5px,
leftmargin=15px,
rightmargin=30px,
innertopmargin=0px,
nobreak=true
]{prop}[theorem]{Proposition}

\newmdtheoremenv[roundcorner = 5px,
leftmargin=15px,
rightmargin=30px,
innertopmargin=0px,
]{prop_break}[theorem]{Proposition}

\newmdtheoremenv[roundcorner = 5px,
leftmargin=15px,
rightmargin=30px,
innertopmargin=0px,
nobreak=true
]{regles}[theorem]{Règles de calculs}


\newtheorem*{exemples}{Exemples}
\newtheorem{exemple}{Exemple}
\newtheorem*{rem}{Remarque}
\newtheorem*{rems}{Remarques}
% Warning sign

\newcommand\warning[1][4ex]{%
  \renewcommand\stacktype{L}%
  \scaleto{\stackon[1.3pt]{\color{red}$\triangle$}{\tiny\bfseries !}}{#1}%
}


\newtheorem{exo}{Exercice}
\newcounter{ExoCounter}
\newtheorem{exercice}[ExoCounter]{Exercice}

\newcounter{counterCorrection}
\newtheorem{correction}[counterCorrection]{\color{red}{Correction}}


\theoremstyle{definition}

%\newtheorem{prop}[theorem]{Proposition}
%\newtheorem{\defi}[1]{
%\begin{tcolorbox}[width=14cm]
%#1
%\end{tcolorbox}
%}


%--------------------------------------- 
% Document
%--------------------------------------- 






\lstset{numbers=left, numberstyle=\tiny, stepnumber=1, numbersep=5pt}




% Header et footer

\pagestyle{fancy}
\fancyhead{}
\fancyfoot{}
\renewcommand{\headwidth}{\textwidth}
\renewcommand{\footrulewidth}{0.4pt}
\renewcommand{\headrulewidth}{0pt}
\renewcommand{\footruleskip}{5px}

\fancyfoot[R]{Olivier Glorieux}
%\fancyfoot[R]{Jules Glorieux}

\fancyfoot[C]{ Page \thepage }
\fancyfoot[L]{1BIOA - Lycée Chaptal}
%\fancyfoot[L]{MP*-Lycée Chaptal}
%\fancyfoot[L]{Famille Lapin}



\newcommand{\Hyp}{\mathbb{H}}
\newcommand{\C}{\mathcal{C}}
\newcommand{\U}{\mathcal{U}}
\newcommand{\R}{\mathbb{R}}
\newcommand{\T}{\mathbb{T}}
\newcommand{\D}{\mathbb{D}}
\newcommand{\N}{\mathbb{N}}
\newcommand{\Z}{\mathbb{Z}}
\newcommand{\F}{\mathcal{F}}




\newcommand{\bA}{\mathbb{A}}
\newcommand{\bB}{\mathbb{B}}
\newcommand{\bC}{\mathbb{C}}
\newcommand{\bD}{\mathbb{D}}
\newcommand{\bE}{\mathbb{E}}
\newcommand{\bF}{\mathbb{F}}
\newcommand{\bG}{\mathbb{G}}
\newcommand{\bH}{\mathbb{H}}
\newcommand{\bI}{\mathbb{I}}
\newcommand{\bJ}{\mathbb{J}}
\newcommand{\bK}{\mathbb{K}}
\newcommand{\bL}{\mathbb{L}}
\newcommand{\bM}{\mathbb{M}}
\newcommand{\bN}{\mathbb{N}}
\newcommand{\bO}{\mathbb{O}}
\newcommand{\bP}{\mathbb{P}}
\newcommand{\bQ}{\mathbb{Q}}
\newcommand{\bR}{\mathbb{R}}
\newcommand{\bS}{\mathbb{S}}
\newcommand{\bT}{\mathbb{T}}
\newcommand{\bU}{\mathbb{U}}
\newcommand{\bV}{\mathbb{V}}
\newcommand{\bW}{\mathbb{W}}
\newcommand{\bX}{\mathbb{X}}
\newcommand{\bY}{\mathbb{Y}}
\newcommand{\bZ}{\mathbb{Z}}



\newcommand{\cA}{\mathcal{A}}
\newcommand{\cB}{\mathcal{B}}
\newcommand{\cC}{\mathcal{C}}
\newcommand{\cD}{\mathcal{D}}
\newcommand{\cE}{\mathcal{E}}
\newcommand{\cF}{\mathcal{F}}
\newcommand{\cG}{\mathcal{G}}
\newcommand{\cH}{\mathcal{H}}
\newcommand{\cI}{\mathcal{I}}
\newcommand{\cJ}{\mathcal{J}}
\newcommand{\cK}{\mathcal{K}}
\newcommand{\cL}{\mathcal{L}}
\newcommand{\cM}{\mathcal{M}}
\newcommand{\cN}{\mathcal{N}}
\newcommand{\cO}{\mathcal{O}}
\newcommand{\cP}{\mathcal{P}}
\newcommand{\cQ}{\mathcal{Q}}
\newcommand{\cR}{\mathcal{R}}
\newcommand{\cS}{\mathcal{S}}
\newcommand{\cT}{\mathcal{T}}
\newcommand{\cU}{\mathcal{U}}
\newcommand{\cV}{\mathcal{V}}
\newcommand{\cW}{\mathcal{W}}
\newcommand{\cX}{\mathcal{X}}
\newcommand{\cY}{\mathcal{Y}}
\newcommand{\cZ}{\mathcal{Z}}







\renewcommand{\phi}{\varphi}
\newcommand{\ddp}{\displaystyle}


\newcommand{\G}{\Gamma}
\newcommand{\g}{\gamma}

\newcommand{\tv}{\rightarrow}
\newcommand{\wt}{\widetilde}
\newcommand{\ssi}{\Leftrightarrow}

\newcommand{\floor}[1]{\left \lfloor #1\right \rfloor}
\newcommand{\rg}{ \mathrm{rg}}
\newcommand{\quadou}{ \quad \text{ ou } \quad}
\newcommand{\quadet}{ \quad \text{ et } \quad}
\newcommand\fillin[1][3cm]{\makebox[#1]{\dotfill}}
\newcommand\cadre[1]{[#1]}
\newcommand{\vsec}{\vspace{0.3cm}}

\DeclareMathOperator{\im}{Im}
\DeclareMathOperator{\cov}{Cov}
\DeclareMathOperator{\vect}{Vect}
\DeclareMathOperator{\Vect}{Vect}
\DeclareMathOperator{\card}{Card}
\DeclareMathOperator{\Card}{Card}
\DeclareMathOperator{\Id}{Id}
\DeclareMathOperator{\PSL}{PSL}
\DeclareMathOperator{\PGL}{PGL}
\DeclareMathOperator{\SL}{SL}
\DeclareMathOperator{\GL}{GL}
\DeclareMathOperator{\SO}{SO}
\DeclareMathOperator{\SU}{SU}
\DeclareMathOperator{\Sp}{Sp}


\DeclareMathOperator{\sh}{sh}
\DeclareMathOperator{\ch}{ch}
\DeclareMathOperator{\argch}{argch}
\DeclareMathOperator{\argsh}{argsh}
\DeclareMathOperator{\imag}{Im}
\DeclareMathOperator{\reel}{Re}



\renewcommand{\Re}{ \mathfrak{Re}}
\renewcommand{\Im}{ \mathfrak{Im}}
\renewcommand{\bar}[1]{ \overline{#1}}
\newcommand{\implique}{\Longrightarrow}
\newcommand{\equivaut}{\Longleftrightarrow}

\renewcommand{\fg}{\fg \,}
\newcommand{\intent}[1]{\llbracket #1\rrbracket }
\newcommand{\cor}[1]{{\color{red} Correction }#1}

\newcommand{\conclusion}[1]{\begin{center} \fbox{#1}\end{center}}


\renewcommand{\title}[1]{\begin{center}
    \begin{tcolorbox}[width=14cm]
    \begin{center}\huge{\textbf{#1 }}
    \end{center}
    \end{tcolorbox}
    \end{center}
    }

    % \renewcommand{\subtitle}[1]{\begin{center}
    % \begin{tcolorbox}[width=10cm]
    % \begin{center}\Large{\textbf{#1 }}
    % \end{center}
    % \end{tcolorbox}
    % \end{center}
    % }

\renewcommand{\thesection}{\Roman{section}} 
\renewcommand{\thesubsection}{\thesection.  \arabic{subsection}}
\renewcommand{\thesubsubsection}{\thesubsection. \alph{subsubsection}} 

\newcommand{\suiteu}{(u_n)_{n\in \N}}
\newcommand{\suitev}{(v_n)_{n\in \N}}
\newcommand{\suite}[1]{(#1_n)_{n\in \N}}
%\newcommand{\suite1}[1]{(#1_n)_{n\in \N}}
\newcommand{\suiteun}[1]{(#1_n)_{n\geq 1}}
\newcommand{\equivalent}[1]{\underset{#1}{\sim}}

\newcommand{\demi}{\frac{1}{2}}
\geometry{hmargin=2.0cm, vmargin=2.5cm}




\begin{document}
\tableofcontents

\title{ CH 8 :  Vocabulaire des Applications}
% debut
%------------------------------------------------


%------------------------------------------------
%-------------------------------------------------
%-------------------------------------------------
%--------------------------------------------------
%------------------------------------------------
\section{Applications d'un ensemble dans un autre}

%-----------------------------------------------------------
%----------------------------------------------------------
\subsection{D\'efinitions et exemples}





\begin{defi} Soient $E$ et $F$ deux ensembles.
	\begin{itemize}
		\item[$\bullet$] Une application (ou fonction) de $E$ dans $F$ est un proc\'ed\'e qui associe à tout élément de $E$ un unique élément de $F$.
		\item[$\bullet$] $E$ est appelé domaine de définition de $f$, ou ensemble de départ.
		\item[$\bullet$] $F$ est  l'ensemble d'arrivée de $f$
		\item[$\bullet$] $y=f(x)$ est l'image de $x$ par $f$.
		\item[$\bullet$] $x$ est  un antécédent de $y$ par $f$.
			%\item[$\bullet$] On note $f: \begin{array}{lll}
			%E & \rightarrow & F\vsec\\
			%x&\mapsto & f(x)
			%\end{array}$ ou $f: x\mapsto f(x)$ s'il n'y a pas d'ambigu\"it\'e sur les ensembles.
	\end{itemize}
\end{defi}




\textbf{Notations}: On note $f: \left| \begin{array}{lll}
		E & \rightarrow & F\vsec \\
		x & \mapsto     & f(x)
	\end{array} \right.$ ou $f: x\mapsto f(x)$ s'il n'y a pas d'ambigu\"it\'e sur les ensembles.




\begin{rems}
	\begin{itemize}
		\item[$\bullet$]  \warning  Ne pas confondre $f$ : la fonction, la "machine" qui fait passer de $x$ à $f(x)$
			\hspace*{2.8cm} et $f(x)$:  l'élement de l'espace d'arrivée.

			C'est comme confondre des carottes rapées et le mixeur qui a été utilisé pour les raper.

	\end{itemize}
\end{rems}


{\footnotesize
\begin{exercice}
	\begin{enumerate}
		\item Soit $f: \left| \begin{array}{lll} \R & \rightarrow & \R     \\
             x        & \mapsto     & x^2 +2\end{array} \right.$. Donner l'image de 4 par $f$. Donner s'ils existent les ant\'ec\'edents de $11, 0$ et $5$ par $f$.
		\item Soit $g: \left| \begin{array}{lll} \Z & \rightarrow & \N    \\
             n        & \mapsto     & n^2+2\end{array} \right.$. Donner l'image de 4 par $g$. Donner s'ils existent les ant\'ec\'edents de $11, 0$ et $5$ par $g$.

		\item Soit $F: \left| \begin{array}{lll} \R^2 & \rightarrow & \R^3            \\
             (x,y)      & \mapsto     & (2x+y,x-y,x+2y)\end{array} \right.$. Donner l'image de $(1,2)$ par $F$. Donner s'ils existent les ant\'ec\'edents de $(1,2,3)$ et $(2,1,1)$
	\end{enumerate}
\end{exercice}}


%

\paragraph{Exemples usuels}

\begin{enumerate}
	\item[\ding{182}] \textbf{L'identit\'e:}\\
		\noindent Soit $E$ un ensemble. On d\'efinit la fonction identit\'e de $E$ not\'ee $\hbox{Id}_E$ par:

	\item[\ding{183}] \textbf{La fonction caract\'eristique ou fonction indicatrice d'un ensemble:}\\
		\noindent Soit $E$ un ensemble et $A$ un sous ensemble de $E$. On d\'efinit la fonction caract\'eristique de $A$ ou fonction indicatrice de $A$ par
		$$\chi_A: \left| \begin{array}{lll} E & \rightarrow & \{0,1\}                         \\
             x       & \mapsto     & \left\{ \begin{array}{ll}
				                                             1 & \text{ si $x\in A$} \\
				                                             0 & \text{ sinon}
			                                             \end{array}		\right.\end{array} \right. $$
	\item[\ding{184}] \textbf{Les fonctions num\'eriques:}\\
		\noindent Exemples: (ne pas oublier les ensembles de départ et d'arrivée)
		$f(x) = \exp(x)$



	\item[\ding{185}] \textbf{Les suites}

		Ce sont  les fonctions de $\N$ dans $\R$


	\item[\ding{186}] \textbf{Exemples de fonctions d'une variable complexe et/ou \`{a} valeurs complexes}
		\begin{itemize}
			\item[$\bullet$] $\theta\mapsto e^{i\theta}$
			\item[$\bullet$] $z\mapsto \frac{z-i}{z+2}$
			\item[$\bullet$]  $z \mapsto |z|$
		\end{itemize}
		\vsec


	\item[\ding{187}] \textbf{Exemples de fonctions \`{a} plusieurs variables}
		\begin{itemize}
			\item[$\bullet$]  $(x,y) \mapsto (2x+y, x-y)$
			\item[$\bullet$]  $(x,y) \mapsto (exp(x)+\frac{y}{\cos(x)}, 1)$
		\end{itemize}

\end{enumerate}


%-----------------------------------------------------------
%----------------------------------------------------------
%\subsection{\'Egalit\'e entre applications}

\noindent  {\noindent

	\begin{defi} Deux applications $f$ et $g$ sont \'egales si
		\begin{itemize}
			\item[$\bullet$]  Elles ont même ensemble de départ (E) et d'arrivée (F).
			\item[$\bullet$]  Et pour tout $x\in E$, $f(x)=g(x)$
				%\item[$\bullet$] \dotfill \vsec
		\end{itemize}
	\end{defi}

}

\begin{rem}
	\warning  Les domaines de d\'epart et d'arriv\'ee sont  importants !
\end{rem}
\vsec
%
%\noindent  {\noindent  
%
%\begin{defi} Soient $E$ et $F$ deux ensembles, et $f: E\rightarrow F$. 
%\begin{itemize}
%\item[$\bullet$] Soit $E_1$ un sous-ensemble de $E$. On appelle restriction de $f$ \`a $E_1$ l'application :
%$$f_{E_1} : \left|\begin{array}{r}
%\hspace*{3cm}\vsec\\
%\vsec
%\end{array}\right.$$
%\item[$\bullet$] Soit $E_2$ un ensemble contenant $E$. On appelle prolongement de $f$ \`a $E_2$ toute application :
%$h : \left|\begin{array}{rcl}
%E_2 & \to & F\\
%x & \mapsto & h(x)
%\end{array}\right.$ telle que $\forall x \in E$, \dotfill
%\end{itemize}
%\end{defi}
% 
%}
%
%
%
%\begin{rem}
%\warning  La restriction de $f$ \`a un sous-ensemble est unique, mais on peut trouver beaucoup de prolongements \`a une application si on n'impose pas de contrainte !
%\end{rem}
%
%
%{\footnotesize 
%\begin{exercice}
%Soit $f : \left|\begin{array}{rcl}
%\R^+ & \to & \R\\
%x & \mapsto & x^2
%\end{array}\right.$. Donner la restriction de $f$ \`a $[0,1]$, ainsi que deux prolongements de $f$ sur $\R$.
%\end{exercice}
%}
%On verra plus tard le prolongement par continuité, lui est unique. 

%-----------------------------------------------------------
%----------------------------------------------------------


%-----------------------------------------------------------
%----------------------------------------------------------
\subsection{Image directe }


\begin{defi}
	Soit $f$ une application de $E$ dans $F$ et  $A$ un sous-ensemble de $E$.\\
	On appelle image directe de $A$ par $f$:
	$$f(A)=\{ f(x)| x\in A\}$$
	Ainsi on a toujours $f(A)\subset F$
\end{defi}

Avec des quantificateurs :
$y\in f(A)\Longleftrightarrow  \exists x\in A \, f(x)=y$





\begin{exemples}
	$$\exp{(\R)}=\hspace{2cm} \cos{\left( \left\lbrack 0,\ddp\frac{\pi}{2}\right\rbrack \right)}=\hspace{2cm} \tan{\left( \left\rbrack -\ddp\frac{\pi}{2} \ddp\frac{\pi}{2}\right\lbrack \right)}=\hspace{2cm} \ln{\left( \rbrack 0,1\rbrack \right)}=\hspace{2cm}$$
\end{exemples}


%
%
%\noindent\ {D\'efinition de l'image r\'eciproque}\\
%
%\noindent 
%\noindent  {\noindent  
%
%\begin{defi}
%Soit $f$ une application de $E$ dans $F$.\\
%\noindent Soit $B$ un sous-ensemble de $F$.\\
%\noindent On appelle image r\'eciproque de $B$ par $f$:
%$$f^{-1}(B)=\hspace{12cm}$$
%Ainsi on a toujours $f^{-1}(B)\subset \dots$
%\end{defi}
% 
%}
%$x\in f^{-1}(B)\Longleftrightarrow f(x)\in B $
% 
%
%
%
%\begin{exemples}
%$$\exp^{-1}{(\R^-)}=\hspace{2cm} \cos^{-1}{\left| \lbrack -1,1\rbrack \right)}=\hspace{2cm} \tan^{-1}{\left| \R^+ \right)}=\hspace{4cm} \ln^{-1}{\left| \rbrack 0,1\rbrack \right)}=\hspace{2cm}$$
%\end{exemples}
%
% \noindent \warning  L'\'ecriture $f^{-1}(B)$ est abusive!!! Elle n'implique pas du tout \dotfill
%
%{\footnotesize 
%\begin{exercice}
%\begin{enumerate}
%\item Soit $f: \N\rightarrow \N$ d\'efinie par $n\mapsto 2n$. Calculer $f(A)$ et $f^{-1}(B)$ avec $A=\lbrace 1,2,4,10\rbrace$ et $B=\lbrace 0,1,2,3\rbrace$.
%\item Soit $g: \R\rightarrow \R$ la fonction carr\'ee. Calculer $g^{-1}(\R^-)$, $g^{-1}(\R^{-\star})$, $g^{-1}(\lbrack 1,4\rbrack)$, $g^{-1}(\lbrack -1,4\rbrack)$.%, $g(\lbrack 1,2\rbrack)$ et $g(\lbrack 4,8\rbrack)$.
%\end{enumerate}
%\end{exercice}}
%{\footnotesize 
%\begin{exercice}
%Soit $f$ une application de $E$ dans $F$ et $B$ un sous-ensemble de $F$. Montrer que $f\left| f^{-1}(B)\right)\subset B$.
%\end{exercice}}

%-----------------------------------------------------------
%----------------------------------------------------------
%\subsection{Compatibilit\'e avec l'inclusion, l'union et l'intersection }



\begin{prop}
	Soit $f$ une application de $E$ dans $F$. Soient $A$, $B$ sous-ensembles de $E$.% et $C$, $D$ sous ensembles de $F$.\vsec\\

	\begin{minipage}{0.4\textwidth}
		\begin{itemize}
			\item[$\bullet$] Si $A\subset B$ alors  $f(A)\subset f(B)$
			\item[$\bullet$] $f(A\cup B)= f(A) \cup f(B)$
			\item[$\bullet$] $f(A\cap B)\cup f(A) \cap f(B)$
		\end{itemize}
	\end{minipage}
	%\begin{minipage}{0.3\textwidth}
	%\begin{itemize}
	%\item[$\bullet$] Si $C\subset D$ alors  \vsec
	%\item[$\bullet$] $f^{-1}(C\cup D)=$ \vsec
	%\item[$\bullet$] $f^{-1}(C\cap D)=$ \vsec
	%\end{itemize}
	%\end{minipage}

\end{prop}

\begin{exemples}
	Calculer
	$$ \ddp \cos{\left( \left\lbrack 0, \ddp\frac{\pi}{2}\right\rbrack \right)} \cap
		\cos{\left( \left\lbrack  \frac{-\pi}{2},0 \right\rbrack \right)}
	$$

	puis

	$$ \cos{\left( \left\lbrack 0,\ddp\frac{\pi}{2}\right\rbrack \cap
			\left\lbrack \frac{-\pi}{2},\ddp 0 \right\rbrack \right)}
	$$
\end{exemples}



\begin{exemple}
	\begin{itemize}
		\item Soit la fonction $f$ d\'efinie par $f(x)=\ddp\frac{3x}{1-x^2}$. Calculer  $f(\rbrack -1,1\lbrack)$ et $f(\rbrack -2,0\rbrack\setminus\lbrace -1\rbrace)$.%, puis $f^{-1}(\lbrack 0,1\rbrack)$ et $f^{-1}(\R^+)$. 

		      %\item Soit $f: z\in\bC\mapsto |z|\in\R$. Calculer $f^{-1}(\lbrace 1\rbrace)$.
		      %\item Soit $g: z\in\bC^*\mapsto \arg{(z)}\in\R$. Calculer $g(\R^{\star})$, $g(i\R^{\star})$, $g(\R^{-\star})$ et $g(i\R^{+\star})$. 
		      %\item Soit $h: z\in\bC\setminus\lbrace 1\rbrace\mapsto \ddp\frac{z+1}{z-1}\in\bC$. Montrer que $h(\mathbb{U}\setminus\lbrace 1\rbrace)\subset i\R$.

	\end{itemize}
\end{exemple}


\subsection{Composition des applications}

\noindent  {\noindent

	\begin{defi} Soient $E,\ F$ et $G$ trois ensembles, $f: E\rightarrow F$ et $g: F\rightarrow G$.
		\begin{itemize}
			\item[$\bullet$] On appelle application compos\'ee de $f$ par $g$ not\'ee $g\circ f$ l'application d\'efinie par
				$$g\circ f:  \left| \begin{array}{lll} E & \rightarrow & G       \\
             x       & \mapsto     & g(f(x))\end{array} \right.$$
			\item[$\bullet$] Cela correspond au diagramme suivant:
				\begin{center}
					\includegraphics[scale=0.5]{Compo.jpg}
				\end{center}

		\end{itemize}
	\end{defi}

}



\begin{exemples}
	Calculer pour chacun des cas suivants et lorsque cela a un sens $f\circ g$ et $g\circ f$.
	\begin{enumerate}
		\item On consid\`ere les deux applications $f:
			      \left|\begin{array}{lll}
				      \R & \rightarrow & \R\vsec \\
				      x  & \mapsto     & x^2-1
			      \end{array} \right. $ et $g:
			      \left|\begin{array}{lll}
				      \R & \rightarrow & \R\vsec   \\
				      x  & \mapsto     & \ddp x-2.
			      \end{array} \right.$

		\item On consid\`ere les deux applications $f:
			      \left|\begin{array}{lll}
				      \R & \rightarrow & \R^{+}\vsec \\
				      x  & \mapsto     & x^2
			      \end{array} \right. $ et $g:
			      \left|\begin{array}{lll}
				      \R^+ & \rightarrow & \R\vsec        \\
				      x    & \mapsto     & \ddp \sqrt{x}.
			      \end{array}\right.$

		\item On consid\`ere les deux applications $f:
			      \left|\begin{array}{lll}
				      ]0,+\infty[ & \rightarrow & \R\vsec \\
				      x           & \mapsto     & \ln x
			      \end{array} \right. $ et $g:
			      \left|\begin{array}{lll}
				      \R & \rightarrow & \R^{-}\vsec \\
				      x  & \mapsto     & \ddp -x^2.
			      \end{array} \right.$

	\end{enumerate}
\end{exemples}
%\begin{rems}
%\begin{itemize}
%\item[$\bullet$] \noindent \warning  \dotfill
%\item[$\bullet$] \noindent \warning  \dotfill
%\end{itemize}
%\end{rems}








% 
%------------------------------------------------
%-------------------------------------------------
%-------------------------------------------------
%--------------------------------------------------
%------------------------------------------------
\section{Applications injectives}

%-----------------------------------------------------------
%----------------------------------------------------------
%\subsection{Applications injectives: d\'efinition et propri\'et\'es}
\subsection{D\'efinition et exemples}

%\noindent\ {D\'efinition et exemples}\\

\noindent  {\noindent

	\begin{defi} Soient $E,\ F$ deux ensembles et $f: E\rightarrow F$ une application.\\
		\noindent On dit que $f$ est injective de $E$ dans $F$ ou que $f$ est une injection de $E$ dans $F$ si chaque élément image a au plus un antécédent.
	\end{defi}

}



\begin{exemple}
	Parmi les dessins suivants, lequel repr\'esente une fonction injective ? \\

\end{exemple}

Avec des quantificateurs :
%$$ \forall y \in F, Card(f^{-1}(\{ y\} ) \leq 1$$
$$\forall x, y \in E^2, f(x)= f(y) \implique x= y$$
$$\forall x, y \in E^2, x\neq y \implique f(x)\neq f(y)$$





\begin{exercice}
	Soit $f: A\rightarrow B$ et $g: B\rightarrow C$. Montrer que $g\circ f$ injective $\Rightarrow$ $f$ injective.
\end{exercice}

\begin{exercice}
	Montrer que l'application $z \in \bC \mapsto \overline{z} \in \bC$ est injective.
\end{exercice}



\begin{exercice}
	\'Etudier l'injectivit\'e des fonctions suivantes:
	\begin{enumerate}

		\item $x\in\R\mapsto |x|\in\R$
		\item $z\in\bC\mapsto |z|\in\R^+$.
		      %\item[$\bullet$] $z\in\bC\mapsto \overline{z}\in\bC$.


		      %\item[$\bullet$] $x\in\R\mapsto x^2\in\R$
		\item  $z\in\bC\mapsto \Re{(z)}\in\R$.
		\item $\theta \in\R\mapsto e^{i\theta} \in\bC$.

	\end{enumerate}
\end{exercice}





\begin{rem}
	Pour les fonctions num\'eriques, l'injectivit\'e s'observe graphiquement en balayant le plan par une droite horizontale qui doit rencontrer au plus une fois le graphe de la fonction.\\
	\noindent Exemples :

\end{rem}


\subsection{Cas particulier des fonctions num\'eriques}

\noindent Si $f:I\to \R$ avec $I \subset \R$, on peut utiliser la propri\'et\'e suivante :\\


\begin{prop} Soient $I\subset \R$ et $f: I\rightarrow \R$ une application.\vsec\\
	Si $f$ est strictement monotone sur $I$ alors  $f$ est injective sur $I$.
\end{prop}




\begin{rem} Lien avec la composition d'\'egalit\'es par une fonction:
	\begin{itemize}
		\item[$\bullet$] $a=b\Longrightarrow f(a)=f(b)$ est toujours vrai.
			%\item[$\bullet$] \dotfill
		\item[$\bullet$] $a=b \Longleftrightarrow f(a)=f(b)$ seulement quand $f$ est injective.
	\end{itemize}
\end{rem}





M\'ethode : toujours commencer par \'etudier la fonction et faire son tableau de variation. Cela permet:
\begin{itemize}
	\item[$\bullet$] De prouver directement l'injectivit\'e l\`{a} o\`{u} elle est strictement monotone.
	\item[$\bullet$] De donner une id\'ee du contre-exemple l\`{a} o\`{u} elle n'est pas injective.\vsec
\end{itemize}

\setlength\fboxrule{0.5pt}

{\footnotesize
	\begin{exercice}
		\'Etudier l'injectivit\'e des fonctions $f: x \mapsto \ddp\frac{e^x-1}{e^x+1}$ et $g:x\mapsto \ddp\frac{x^2}{1+x}$.
	\end{exercice}}




% 

\begin{rem}
	Le caract\`{e}re injectif d'une fonction est li\'e \`{a} son ensemble de d\'epart. Ainsi si $f$ n'est pas injective, il est possible de restreindre $f$ sur un ensemble de d\'epart plus petit pour obtenir une injection.\\
	\noindent Exemple : trouver un intervalle $I$ tel que la fonction $f : \left| \begin{array}{ccl}
			I & \rightarrow & \R^+ \\
			x & \mapsto     & x^2
		\end{array} \right.$ soit injective.

\end{rem}

%

%-----------------------------------------------------------
%----------------------------------------------------------
%\subsection{Caract\'erisations}
%%\noindent\ {Caract\'erisations}\\
%
%%\noindent \textbf{\large{\ding{182}}} \underline{\textbf{\large{ Avec la d\'efinition.}}}
%\noindent\ {Avec la d\'efinition}\\
%%
%
%\noindent  {\noindent  
%
%\begin{prop} Soient $E,\ F$ deux ensembles et $f: E\rightarrow F$ une application.\vsec\\
%$f$ est injective de $E$ dans $F$ $\Longleftrightarrow$\dotfill\vsec
%\end{prop}
% 
%}
%\vsec
%
%\noindent M\'ethode :\\
%\noindent 
%
%\begin{itemize}
%\item[$\bullet$] \dotfill\vsec
%\item[] \dotfill\vsec
%\item[] 
%\item[$\bullet$] \dotfill\vsec
%%\item[] \dotfill\vsec
%\end{itemize}
% 
%\quad
%%\hspace{0.5cm}
%
%\setlength\fboxrule{1pt}
%\noindent  {
%
%Soit $(x_1,x_2)\in E^2$ tel que $f(x_1)=f(x_2)$. \\
%\phantom{\hspace{0cm}}\dotfill \\
%\phantom{\hspace{0cm}}\dotfill \\
%\phantom{\hspace{0cm}}\dotfill \\
%Donc $x_1=x_2$.\\
%Conclusion: $f$ est injective de $E$ dans $F$.\\
% }
%\setlength\fboxrule{0.5pt}
% 
%
%{\footnotesize 
%-}
%
%
%
%
%%\noindent \textbf{\large{\ding{183}}} \underline{\textbf{\large{Par la contrapos\'ee de la d\'efinition.}}}
%%
%\noindent\ {Par la contrapos\'ee de la d\'efinition}\\
%
%\noindent  {\noindent  
%
%\begin{prop} Soient $E,\ F$ deux ensembles et $f: E\rightarrow F$ une application.\vsec\\
%$f$ est injective de $E$ dans $F$ $\Longleftrightarrow$\dotfill\vsec
%\end{prop}
% 
%}
%\vsec
%
%\noindent M\'ethode :\\
%\noindent 
%%
%\begin{itemize}
%\item[$\bullet$] \dotfill\vsec
%\item[] \dotfill\vsec
%\item[] 
%\item[$\bullet$] \dotfill\vsec
%\item[] \dotfill\vsec
%\end{itemize}
% 
%\quad
%%\hspace{0.5cm}
%
%\setlength\fboxrule{1pt}
%\noindent  {
%
%Soit $(x_1,x_2)\in E^2$ tel que $x_1\not= x_2$. \\
%\phantom{\hspace{0cm}}\dotfill \\
%\phantom{\hspace{0cm}}\dotfill \\
%\phantom{\hspace{0cm}}\dotfill \\
%Donc $f(x_1)\not=f(x_2)$.\\
%Conclusion: $f$ est injective de $E$ dans $F$.\\
% }
%\setlength\fboxrule{0.5pt}
% 
%
%{\footnotesize 
%\begin{exercice}
%Montrer que l'application $z \in \bC \mapsto \overline{z} \in \bC$ est injective.
%\end{exercice}}
%



%\noindent \textbf{\large{\ding{184}}} \underline{\textbf{\large{Pour montrer que $f$ n'est pas injective.}}}
%%
%\noindent\ {Pour montrer que $f$ n'est pas injective}\\
%
%\noindent En niant la d\'efinition, on obtient la m\'ethode pour montrer qu'une application $f$ n'est pas injective de $E$ dans $F$:\\
%
%\noindent  {\noindent  
%
%\begin{prop} Soient $E,\ F$ deux ensembles et $f: E\rightarrow F$ une application.\vsec\\
%$f$ n'est pas injective de $E$ dans $F$ $\Longleftrightarrow$ \dotfill\vsec
%\end{prop}
% 
%}







%\noindent \textbf{\large{\ding{184}}} \underline{\textbf{\large{Cas particulier des fonctions strictement monotones.}}}
%
\noindent
%-----------------------------------------------------------
%----------------------------------------------------------
%\subsection{M\'ethodes pour montrer qu'une application est injective}
%
%
%\noindent\ {M\'ethodes dans le cas g\'en\'eral}\\
%
%\begin{enumerate}
%\item[\textbf{\large{\ding{182}}}] \underline{\textbf{\large{ M\'ethode 1: par la d\'efinition:}}}\\
%
%\noindent C'est la m\'ethode la plus courante pour r\'esoudre des exercices (hors exercices avec des fonctions num\'eriques).\\
%
%\noindent 
%
%\begin{itemize}
%\item[$\bullet$] \dotfill\vsec
%\item[] \dotfill\vsec
%\item[] 
%\item[$\bullet$] \dotfill\vsec
%\item[] \dotfill\vsec
%\end{itemize}
% 
%\quad
%%\hspace{0.5cm}
%
%\setlength\fboxrule{1pt}
%\noindent  {
%
%Soit $(x_1,x_2)\in E^2$ tel que $f(x_1)=f(x_2)$. \\
%\phantom{\hspace{0cm}}\dotfill \\
%\phantom{\hspace{0cm}}\dotfill \\
%\phantom{\hspace{0cm}}\dotfill \\
%Donc $x_1=x_2$.\\
%Conclusion: $f$ est injective de $E$ dans $F$.\\
% }
%\setlength\fboxrule{0.5pt}
% 
%
%{\footnotesize 
%\begin{exercice}
%Soit $f: A\rightarrow B$ et $g: B\rightarrow C$. Montrer que $g\circ f$ injective $\Rightarrow$ $f$ injective.
%\end{exercice}}
%
%
%\item[\textbf{\large{\ding{183}}}] \underline{\textbf{\large{ M\'ethode 2: par la contrapos\'ee:}}}\\
%
%\noindent On l'utilise moins.
%
%\noindent 
%
%\begin{itemize}
%\item[$\bullet$] \dotfill\vsec
%\item[] \dotfill\vsec
%\item[] 
%\item[$\bullet$] \dotfill\vsec
%\item[] \dotfill\vsec
%\end{itemize}
% 
%\quad
%%\hspace{0.5cm}
%
%\setlength\fboxrule{1pt}
%\noindent  {
%
%Soit $(x_1,x_2)\in E^2$ tel que $x_1\not= x_2$. \\
%\phantom{\hspace{0cm}}\dotfill \\
%\phantom{\hspace{0cm}}\dotfill \\
%\phantom{\hspace{0cm}}\dotfill \\
%Donc $f(x_1)\not=f(x_2)$.\\
%Conclusion: $f$ est injective de $E$ dans $F$.\\
% }
%\setlength\fboxrule{0.5pt}
% 
%
%% 
%\item[\textbf{\large{\ding{184}}}] \underline{\textbf{\large{ M\'ethode 3: en trouvant un contre-exemple pour montrer la non injectivit\'e:}}}\\
%
%\setlength\fboxrule{1pt}
%\noindent  {
%
%Trouver deux \'el\'ements $(x_1,x_2)\in E^2$ tels que:
%\begin{itemize}
%\item[$\bullet$] \dotfill 
%\item[$\bullet$] ET \dotfill
%\end{itemize}
%Conclusion: $f$ n'est pas injective de $E$ dans $F$.\\
% }
%\setlength\fboxrule{0.5pt}
%
%{\footnotesize 
%\begin{exercice}
%Montrer que les deux applications suivantes ne sont pas injectives: $f : \theta\in\R\mapsto e^{i\theta}\in\bC$ et $g: z\in\bC\mapsto \im{(z)}\in\R$.
%\end{exercice}}
%
%\end{enumerate}
%
%
%\noindent\ {Cas particulier des fonctions num\'eriques: M\'ethodes}\\
%
%\setlength\fboxrule{1pt}
%\noindent  {
%
%Toujours commencer par \'etudier la fonction et faire son tableau de variation.\\
%Cela permet:
%\begin{itemize}
%\item[$\bullet$] De prouver directement l'injectivit\'e l\`{a} o\`{u} elle est strictement monotone.
%\item[$\bullet$] De donner une id\'ee du contre-exemple l\`{a} o\`{u} elle n'est pas injective.\vsec
%\end{itemize}
% }
%\setlength\fboxrule{0.5pt}
%
%{\footnotesize 
%\begin{exercice} 
%\'Etudier l'injectivit\'e des applications $f: x\in\R\mapsto \sin{(x)}+2x\in\R$, $g: x\in\R^{+\star}\mapsto \ddp\frac{\sqrt{x}}{x}\in\R$ et $h : x \in \R \backslash \{3\} \mapsto \ddp\frac{2x^2}{x-3} \in \R$.
%\end{exercice}}
%
%
% 
%------------------------------------------------
%-------------------------------------------------
%-------------------------------------------------
%--------------------------------------------------
%------------------------------------------------
\section{Applications surjectives}

%-----------------------------------------------------------
%----------------------------------------------------------
\subsection{D\'efinition et exemples}


\begin{defi} Soient $E,\ F$ deux ensembles et $f: E\rightarrow F$ une application.
	On dit que $f$ est surjective de $E$ dans $F$ ou que $f$ est une surjection de $E$ dans $F$ si
\end{defi}

Avec des quantificateurs :
%$$ \forall y \in F, Card(f^{-1}(\{ y\} ) \geq 1$$
$$\forall  y \in F, \, \exists x\in E, \,  y=f(x)$$




\begin{exemple}
	Parmi les dessins suivants, lequel repr\'esente une fonction surjective ?

\end{exemple}




\begin{rem}
	Le caract\`{e}re surjectif d'une fonction est ainsi li\'e \`{a} son ensemble d'arriv\'ee. Ainsi si $f$ n'est pas surjective, il est possible de restreindre l'ensemble d'arriv\'ee pour obtenir une application surjective.\\
	Exemple :
\end{rem}



\begin{exercice}
	Soit $f: A\rightarrow B$ et $g: B\rightarrow C$. Montrer que $g\circ f$ surjective $\Rightarrow$ $g$ surjective.
\end{exercice}

\begin{exercice}
	\'Etudier la surjectivit\'e des fonctions suivantes:
	\begin{enumerate}

		\item $x\in\R\mapsto \cos x \in\R$.
		\item $x\in\R\mapsto x^2+4x\in\R$.


		\item $\theta\in\R\mapsto e^{i\theta}\in\bC$.
		\item $z\in\bC\mapsto |z|\in\R$.

	\end{enumerate}
\end{exercice}



\subsection{Cas des fonctions num\'eriques}

\begin{rem}
	Pour les fonctions num\'eriques, la surjectivit\'e s'observe graphiquement en balayant le plan par une droite horizontale qui doit rencontrer \dotfill le graphe de la fonction.\\
	\noindent Exemples:%\dotfill



\end{rem}


\noindent On peut alors utiliser le th\'eor\`eme des valeurs interm\'ediaires.\\

\noindent  {\noindent

	\begin{prop} Soit $f$ une fonction \dotfill sur un intervalle $[a,b]$.\\
		Alors pour tout $y$ compris entre $f(a)$ et $f(b)$, il existe $c \in [a,b]$ tel que $y=f(c)$.
	\end{prop}

}\vsec

\noindent Il est possible de g\'en\'eraliser avec des intervalles ouverts ou semi-ouverts, en utilisant les limites.

	{\footnotesize
		\begin{exercice}
			Soit la fonction d\'efinie par $f(x)=\ddp\frac{x^3}{x^2+1}$. Montrer que $f$ est surjective de $\R$ dans $\R$.
		\end{exercice}}





\subsection{Lorsque l'on conna\^it l'expression de $f$}

\noindent Dans le cas d'exercices plus concrets dans lesquels l'expression de la fonction est connue, on peut raisonner par analyse-synth\`ese en r\'esolvant l'\'equation $f(x)=y$.\\

\begin{itemize}
	\item[$\bullet$] \textbf{Analyse :} Soit $y\in F$. On r\'esout l'\'equation $f(x)=y$ d'inconnue $x \in E$.
	\item[$\bullet$] \textbf{Synth\`{e}se:} deux cas sont possibles.
		\begin{itemize}
			\item[$\star$] Si pour tout $y \in F$, l'\'equation $f(x)=y$ admet au moins une solution $x$ dans $E$, alors $f$ est surjective de $E$ dans $F$.
			\item[$\star$] S'il existe au moins un $y \in F$ tel que l'\'equation $f(x)=y$ n'admet pas de solution $x$ dans $E$, alors $f$ n'est pas surjective de $E$ dans $F$.
		\end{itemize}
\end{itemize}




\begin{exercice}
	On consid\`ere la fonction $f$ d\'efinie sur $\bC$ par $f(z)=\exp(z)$. Montrer que $f$ est surjective de $\bC$ dans $\bC^\star$.
\end{exercice}

\begin{exercice}
	On consid\`ere la fonction $f$ d\'efinie sur $\bC$ par $f(z)=\frac{2z+i}{z-i}$. Montrer que $f$ est surjective de $D_f$ dans $\bC\setminus\{2\}$.
\end{exercice}


%{\footnotesize 
%\begin{exercice}
%Soit la fonction d\'efinie par $f(x)=\ddp\frac{2x+1}{x-1}$. Montrer que $f$ est surjective de $\R\setminus\lbrace 1\rbrace$ dans $\R\setminus\lbrace 2\rbrace$.
%\end{exercice}}
%
%\noindent \warning  Dans tous les cas, lorsque l'on doit \'etudier la surjectivit\'e d'une fonction num\'erique dont on conna\^{i}t l'expression, on commence toujours par \'etudier la fonction et tracer son tableau de variation. Cela donne une id\'ee du r\'esultat que l'on doit montrer.\\


%-----------------------------------------------------------
%----------------------------------------------------------
%-----------------------------------------------------------
%----------------------------------------------------------
\section{Applications bijectives}

%-----------------------------------------------------------
%----------------------------------------------------------
\subsection{D\'efinition et exemples}

%\noindent\ {D\'efinition et exemples}\\



\begin{defi} Soient $E,\ F$ deux ensembles et $f: E\rightarrow F$ une application.
	\begin{itemize}
		\item[$\bullet$] On dit que $f$ est bijective de $E$ dans $F$ ou que $f$ est une bijection de $E$ dans $F$ si $f$ est bijective et injective.
	\end{itemize}
\end{defi}


Avec des quantificateurs :
%$$ \forall y \in F, Card(f^{-1}(\{ y\} ) = 1$$
$$\forall  y \in F, \, \exists ! x\in E, \,  y=f(x)$$





\begin{rems}
	\begin{itemize}
		\item[$\bullet$] Le caract\`{e}re bijectif d'une application est ainsi li\'e \`{a} son ensemble de d\'epart et d'arriv\'ee. Ainsi si $f$ n'est pas bijective, il est possible de restreindre les ensemble de d\'epart et d'arriv\'ee pour obtenir une application bijective.
		\item[$\bullet$] \noindent \warning  Que ce soit pour l'injectivit\'e, la surjectivit\'e ou la bijectivit\'e d'une application, il est indispensable de toujours bien pr\'eciser \dotfill
	\end{itemize}
\end{rems}

%

%\noindent\ {Caract\'erisation des applications bijectives et application r\'eciproque}\\
%-----------------------------------------------------------
%----------------------------------------------------------
\subsection{Application r\'eciproque}


\noindent  {\noindent

	\begin{prop} Soient $E,\ F$ deux ensembles et $f: E\rightarrow F$ une application.\\
		L'application $f$ est bijective de $E$ dans $F $ s'il existe une application $g$ de $F$ dans $E$ telle que l'on ait
		$$\hspace{8cm}$$
		L'application $g$ est alors not\'ee \ldots \ldots\ et est appel\'ee \dotfill \vsec
	\end{prop}

}



\noindent  {\noindent

	\begin{prop}
		Si $f$ est une application bijective de $E$ dans $F$ alors:
		\begin{itemize}
			\item[$\bullet$] $f^{-1}: $\dotfill \vsec
			\item[$\bullet$] $\forall x\in E,\ \forall y\in F:$\dotfill \vsec
		\end{itemize}
	\end{prop}

}



\noindent  {\noindent

	\begin{prop}
		Soient $E,\ F$ et $G$ trois ensembles. Soient $f$ une application bijective de $E$ dans $F$ et $g$ une application bijective de $F$ dans $G$. Alors:\vsec
		\begin{itemize}
			\item[$\bullet$] \dotfill \vsec
			\item[$\bullet$] \dotfill \vsec
		\end{itemize}
	\end{prop}

}


\begin{proof}

\end{proof}

\begin{exemple}
	Montrer que l'application $h : \left| \begin{array}{ccc}
			\R^+ & \to     & [1,+\infty[\vsec \\
			x    & \mapsto & e^{\sqrt{x}}
		\end{array}\right.$ est bijective et calculer $h^{-1}$.

\end{exemple}



%% 
%
%\noindent\ {Cas particulier des fonctions num\'eriques: th\'eor\`{e}me de la bijection}\\
%
%\noindent  {\noindent  
%
%\begin{theorem} Soit $f: I\subset \R\rightarrow \R$ une fonction num\'erique, $I=\lbrack a,b\lbrack$ (par exemple).\\
%\noindent SI\vsec
%\begin{itemize} 
%\item[$\bullet$] \dotfill \vsec
%\item[$\bullet$] \dotfill \vsec
%\item[$\bullet$] \begin{itemize}
%\item[$\star$] $f(a)=\dots$
%\item[$\star$] $\lim\limits_{x\to b^-}f(x)=\dots$\vsec
%\end{itemize}
%\end{itemize}
%ALORS \vsec
%\begin{itemize} 
%\item[$\bullet$] \dotfill \vsec
%\item[$\bullet$] \dotfill \vsec
%\end{itemize}
%\end{theorem}
% 
%}
%
%{\footnotesize 
%\begin{exercice}
%\'Etudier la bijectivit\'e des applications suivantes: $f: x\mapsto x^2+2x-3$ et $g: x\mapsto \ddp\frac{x+1}{x-1}$.
%\end{exercice}}
%
% 
%-----------------------------------------------------------
%----------------------------------------------------------
\subsection{Caract\'erisations}

%\ {M\'ethodes pour les autres exercices}\\

%\item[\textbf{\large{\ding{182}}}] \underline{\textbf{\large{ M\'ethode 1: en montrant qu'elle est injective et surjective:}}}\vsec
\paragraph{Montrer que $f$ est injective et surjective}


Voir les m\'ethodes pour montrer qu'une fonction est injective et surjective.



%\item[\textbf{\large{\ding{183}}}] \underline{\textbf{\large{ M\'ethode 2: par la caract\'erisation (rare):}}}\vsec
\paragraph{Utiliser la caract\'erisation de l'application r\'eciproque (rare)}


Trouver une application $g$ v\'erifiant $f\circ g=Id$ et $g\circ f=Id$.

%\item[\textbf{\large{\ding{184}}}] \underline{\textbf{\large{ M\'ethode pour montrer une non bijectivit\'e:}}}\vsec

% 
%
%
% Voir les m\'ethodes pour montrer qu'une application est non injective ou non surjective.



\paragraph{Par analyse-synth\`ese}

Lorsque l'on conna\^it l'expression de $f$, on peut raisonner par analyse-synth\`ese en r\'esolvant $f(x)=y$. Il faut y penser en particulier lorsque l'on vous demande l'expression de la bijection r\'eciproque.\\

\setlength\fboxrule{1pt}
{

\begin{itemize}
	\item[$\bullet$] Soit $y\in F$. On r\'esout l'\'equation $f(x)=y$ d'inconnue $x \in E$.
	\item[$\bullet$] \textbf{Synth\`{e}se:} deux cas sont possibles.
		\begin{itemize}
			\item[$\star$] Si pour tout $y \in F$, l'\'equation $f(x)=y$ admet une UNIQUE solution $x$ dans $E$, alors $f$ est bijective de $E$ dans $F$.\\
				On obtient alors l'expression de l'application r\'eciproque :  $f^{-1}:  \left| \begin{array}{ccc}
						F & \rightarrow & E\vsec \\
						y & \mapsto     & g(y)
					\end{array}\right.$, o\`u $g(y)$ est l'unique solution dans $E$ de $f(x)=y$.
			\item[$\star$] S'il existe au moins un $y \in F$ tel que l'\'equation $f(x)=y$ admette $0$ ou au moins $2$ solutions dans $E$, alors $f$ n'est pas bijective de $E$ dans $F$.
		\end{itemize}
\end{itemize}

\begin{exercice}
	\'Etudier la bijectivit\'e de la fonction $f$ d\'efinie par $f(x)=1+\ddp\frac{1}{x}$ et calculer l'expression de $f^{-1}$ lorsqu'elle existe.
\end{exercice}



\paragraph{Montrer que $f$ n'est pas bijective}
%
%
%
$f$ non bijective de $E$ dans $F$ $\Leftrightarrow$ \dotfill
%-----------------------------------------------------------
%----------------------------------------------------------
\subsection{Cas particulier des fonctions num\'eriques}


\begin{rem}
	Pour les fonctions num\'eriques, la bijectivit\'e s'observe graphiquement en balayant le plan par une droite horizontale qui doit rencontrer \dotfill le graphe de la fonction.\\
	\noindent Exemples:



\end{rem}

\ {Th\'eor\`eme de la bijection}\\

{

\begin{theorem} Soit $f: [a,b] \rightarrow \R$ une fonction num\'erique v\'erifiant :
	\begin{itemize}
		\item[$\bullet$] \dotfill \vsec
		\item[$\bullet$] \dotfill \vsec
	\end{itemize}
	Alors \dotfill \vsec
\end{theorem}

}
\vsec

On peut g\'en\'eraliser \`a des intervalles ouverts ou semi-ouverts en utilisant les limites.\\
Y penser lorsque l'on demande uniquement de montrer que la fonction est bijective (utiliser directement la m\'ethode par analyse-synth\`ese si l'\'enonc\'e demande aussi l'expression de $f^{-1}$).

%% M\'ethode:\\
%
%\hspace{-1.2cm} 
%\setlength\fboxrule{1pt}
%  {
%
%\begin{itemize}
%\item[$\bullet$] \textbf{On montre que $f$ est continue sur $I$:}\\
%En g\'en\'eral comme somme, produit, compos\'ee... de fonctions.
%\item[$\bullet$] \textbf{On montre que $f$ est strictement monotone sur $I$:}\\
%En g\'en\'eral en \'etudiant ses variations \`a l'aide de la d\'eriv\'ee.
%\item[$\bullet$] \textbf{On calcule les limites ou valeurs de $f$ aux bornes de $I$}.
%\item[$\bullet$] \textbf{On cite le th\'eor\`eme et on conclut}:\\
%D'apr\`{e}s le th\'eor\`{e}me de la bijection, $f$ est bijective de $I$ sur $f(I)$.
%\end{itemize}
% }
%\setlength\fboxrule{0.5pt}
% 
%\quad
%
%\setlength\fboxrule{1pt}
%  {
%
%Y penser lorsque:\vsec
%\begin{itemize}
%\item[$\bullet$] \dotfill\vsec
%\item[$\bullet$] \dotfill\vsec
%\item[$\bullet$] \dotfill\vsec
%\end{itemize}
% }
%\setlength\fboxrule{0.5pt}
% 
%


\begin{exercice}
	\'Etudier la bijectivit\'e de $x\mapsto 1+\ddp\frac{1}{x}$.
\end{exercice}


\begin{exercice}
	\'Etudier la bijectivit\'e des applications suivantes: $f: x\mapsto \ddp\frac{x+1}{x-1}$ et $g: x\mapsto x^2+2x-3$.
\end{exercice}




\subsection{\'Etude de la fonction r\'eciproque}

{

	\begin{prop}[Graphe d'une fonction r\'eciproque] \quad\\
		Si $f$ est bijective de $I$ sur $J$, alors  $\mathcal{C}_f$ et $\mathcal{C}_{f^{-1}}$ sont \dotfill%est monotone de m\^eme sens sur $J$.
	\end{prop}
}

\begin{exemple}
	Tracer les graphes des fonctions cube et racine cubique.

\end{exemple}



{

\begin{prop}[Monotonie d'une fonction r\'eciproque] \quad\\
	Si $f$ est bijective et monotone de $I$ sur $J$, alors $f^{-1}$ \dotfill%est monotone de m\^eme sens sur $J$.
\end{prop}
}


{\footnotesize
\begin{exercice} Calculer les variations et \'etudier la bijectivit\'e de la fonction $f: x\mapsto x^2-x\ln{x}-1$ de $\R^{+\star}$ sur un intervalle \`{a} d\'eterminer. En d\'eduire les variations de $f^{-1}$.
\end{exercice}}



{

\begin{theorem}[Th\'eor\`{e}me de d\'erivabilit\'e d'une fonction r\'eciproque] \quad \\
	Soit $f: I\rightarrow J$ une fonction bijective, et $f^{-1}: J\rightarrow I$ sa fonction r\'eciproque. Soit $x_0 \in I$ tel que :
	\begin{itemize}
		\item[$\bullet$] \dotfill%$f$ est d\'erivable en $x$
		\item[$\bullet$] \dotfill%$f'(x)\not= 0$
	\end{itemize}
	Alors $f^{-1}$ est d\'erivable en $y_0=f(x_0)$ et $(f^{-1})'(y_0) =$ \vsec\vsec\vsec
	%\begin{enumerate}
	%\item[\ding{182}] \underline{Formulation 1:}\\
	% \begin{tabular}{ll} SI: & $\star$ $f$ est d\'erivable en $x_0$\\
	%& $\star$ $f'(x_0)\not= 0$\end{tabular}\\
	% ALORS: $f^{-1}$ est d\'erivable en $y_0=f(x_0)$ et: $ \left| f^{-1} \right)^{\prime}(y_0)=\ddp\frac{1}{f^{\prime}\left| f^{-1}(y_0)\right)}=\ddp\frac{1}{f^{\prime}\left| x_0\right)}$.
	%\item[\ding{183}] \underline{Formulation 2:}\\
	% \begin{tabular}{ll} SI: & $\star$ $f$ est d\'erivable sur $I$\\
	%& $\star$ $\forall x\in I,\ f'(x)\not= 0$: la d\'eriv\'ee de $f$ ne s'annule pas sur $I$\end{tabular}\\
	% ALORS: $f^{-1}$ est d\'erivable sur $J$ et: $\forall y\in J,\ \left| f^{-1} \right)^{\prime}(y)=\ddp\frac{1}{f^{\prime}\left| f^{-1}(y)\right)}$.
	%\end{enumerate}
\end{theorem}
}




\setlength\fboxrule{1pt}


\textbf{M\'ethode pour \'etudier la d\'erivabilit\'e d'une r\'eciproque:}
\begin{itemize}
	\item[$\bullet$]  Justifier que $f$ est d\'erivable sur $I$ et calculer sa d\'eriv\'ee.
		%\item[$\bullet$] on calcule la d\'eriv\'ee $f^{\prime}$ de $f$.
	\item[$\bullet$] D\'eterminer tous les points $x_0$ o\`u $f^{\prime}$ s'annule, et calculer les $y_0=f(x_0)$ correspondants.
		%\item[$\bullet$] on calcule tous les $y_0=f(x_0)$ correspondant.
	\item[$\bullet$] On sait alors d'apr\`{e}s le th\'eor\`{e}me de d\'erivabilit\'e d'une r\'eciproque que:
		\begin{itemize}
			\item[$\star$] $f^{-1}$ est d\'erivable sur l'intervalle $J$ priv\'e de tous les points $y_0$ calcul\'es ci-dessus.
			\item[$\star$] Et pour tout $y\not= y_0$, on a: $(f^{-1})^{\prime}(y)=\ddp\frac{1}{f^{\prime}(f^{-1}(y))}$.
		\end{itemize}
\end{itemize}

\setlength\fboxrule{0.5pt}

%

{\footnotesize
	\begin{exercice}
		\'Etudier la d\'erivabilit\'e et calculer la d\'eriv\'ee de la fonction racine cubique.
	\end{exercice}}


%\ {M\'ethodes pour les exercices concrets}\\
%
% Il s'agit de deux m\'ethodes tr\`{e}s classiques, \`{a} conna\^{i}tre absolument et qu'on utilise d\`{e}s que l'on conna\^{i}t l'expression de l'application $f$.\vsec
%
%\begin{enumerate}
%\item[\textbf{\large{\ding{182}}}] \underline{\textbf{\large{ M\'ethode 1: par le th\'eor\`{e}me de la bijection:}}}\\
%
%
%\hspace{-1.7cm} 
%\setlength\fboxrule{1pt}
%  {
%
%\begin{itemize}
%\item[$\bullet$] \textbf{On montre que $f$ est continue sur $I$:}\\
%En g\'en\'eral comme somme, produit, compos\'ee... de fonctions continues.
%\item[$\bullet$] \textbf{On montre que $f$ est strictement monotone sur $I$:}\\
%En g\'en\'eral en \'etudiant ses variations \`a l'aide de la d\'eriv\'ee.
%\item[$\bullet$] \textbf{On calcule les limites ou valeurs de $f$ aux bornes de $I$}.
%\item[$\bullet$] \textbf{On cite le th\'eor\`eme et on conclut}:\\
%Donc d'apr\`{e}s le th\'eor\`{e}me de la bijection, $f$ est bijective de $I$ sur $f(I)$.
%\end{itemize}
% }
%\setlength\fboxrule{0.5pt}
% 
%\quad
%
%\setlength\fboxrule{1pt}
%  {
%
%Y PENSER lorsque:\vsec
%\begin{itemize}
%\item[$\bullet$] \dotfill\vsec
%\item[$\bullet$] \dotfill\vsec
%\item[$\bullet$] \dotfill\vsec
%\end{itemize}
% }
%\setlength\fboxrule{0.5pt}
% 
%
%
%{\footnotesize 
%\begin{exercice}
%\'Etudier la bijectivit\'e de $x\mapsto 1+\ddp\frac{1}{x}$.
%\end{exercice}}
%
%% 
%
%\item[\textbf{\large{\ding{183}}}] \underline{\textbf{\large{ M\'ethode 2: par analyse-synth\`{e}se:}}}\\
%
%
%\hspace{-1.7cm} 
%\setlength\fboxrule{1pt}
%  {
%
%\begin{itemize}
%\item[$\bullet$] Soit $y\in F$.
%\item[$\bullet$] \textbf{Analyse:} On suppose qu'il existe $x\in E$ tel que $y=f(x)$.
%$$\begin{array}{lll}
%y=f(x) & \Longleftrightarrow & \dotfill\\
%& \Longleftrightarrow & \dotfill\\
%& \Longleftrightarrow & x=\phantom{\hspace{5cm}}
%\end{array}$$
%OBJECTIF: Trouver l'ensemble $B\subset F$, ensemble des r\'eels $y$ pour lesquels l'\'equation admet une UNIQUE solution $x$ DANS $E$.\\
%Cette solution $x\in E$ s'exprime alors en fonction de $y$ sous la forme $x=g(y)$.\\
%\item[$\bullet$] \textbf{Synth\`{e}se:} Soit $y\in B$. On pose $x=\dotfill$  \phantom{\hspace{2cm}}\\
% On a bien:
%\begin{itemize}
%\item[$\star$]  $x$ existe car...
%\item[$\star$] $x\in E$ car ...
%\item[$\star$] $y=f(x)$ avec $x$ unique d'apr\`{e}s l'analyse..
%\end{itemize}
%Ainsi il existe bien un unique $x\in E$ tel que $y=f(x)$.\\
%\item[$\bullet$] \textbf{Conclusion:} $f$ est bijective de $E$ dans $B$.\vsec\\
%On obtient alors l'expression de l'application r\'eciproque $f^{-1}:  B\rightarrow E$ qui v\'erifie: $ y  \mapsto  g(y)$.\vsec
%\end{itemize}
% }
%\setlength\fboxrule{0.5pt}
% 
%\quad
%
%
%\setlength\fboxrule{1pt}
%  {
%
%Y PENSER lorsque:\vsec
%\begin{itemize}
%\item[$\bullet$] \dotfill\vsec
%\item[$\bullet$] \dotfill\vsec
%\item[$\bullet$] \dotfill\vsec
%\end{itemize}
% }
%\setlength\fboxrule{0.5pt}
% 
%
%{\footnotesize 
%\begin{exercice}
%\'Etudier la bijectivit\'e de $x\mapsto 1+\ddp\frac{1}{x}$ et lorsqu'elle est bijective, donner l'expression de $f^{-1}$.
%\end{exercice}}
%
%
%\end{enumerate}
%
%% 
%\ {M\'ethodes pour les autres exercices}\\
%
%
%\begin{enumerate}
%\item[\textbf{\large{\ding{182}}}] \underline{\textbf{\large{ M\'ethode 1: en montrant qu'elle est injective et surjective:}}}\vsec
%
% Voir les m\'ethodes pour montrer qu'une fonction est injective et surjective.\vsec
%
%\item[\textbf{\large{\ding{183}}}] \underline{\textbf{\large{ M\'ethode 2: par la caract\'erisation (rare):}}}\vsec
%
%\setlength\fboxrule{1pt}
%  {
%
%Trouver une application $g$ v\'erifiant $f\circ g=Id$ et $g\circ f=Id$.
% }
%\setlength\fboxrule{0.5pt}
%\vsec
%
%\item[\textbf{\large{\ding{184}}}] \underline{\textbf{\large{ M\'ethode pour montrer une non bijectivit\'e:}}}\vsec
%
%\setlength\fboxrule{1pt}
%  {
%
%$f$ NON bijective de $E$ dans $F$ $\Leftrightarrow$ \dotfill
% }
%\setlength\fboxrule{0.5pt}\vsec
%
% Voir les m\'ethodes pour montrer qu'une application est non injective ou non surjective.
%
%
%\end{enumerate}
%------------------------------------------------
%-------------------------------------------------
%-------------------------------------------------
%--------------------------------------------------
%------------------------------------------------
%\section{Images directes et r\'eciproques}
%
%%-----------------------------------------------------------
%%----------------------------------------------------------
%\subsection{D\'efinitions et exemples}
%
%\ {D\'efinition de l'image directe}\\
%
%
%  {  
%
%\begin{defi}
%Soit $f$ une application de $E$ dans $F$.\\
% Soit $A$ un sous-ensemble de $E$.\\
% On appelle image directe de $A$ par $f$:
%$$f(A)=\hspace{12cm}$$
%Ainsi on a toujours $f(A)\subset \dots$
%\end{defi}
% 
%}
% 
%
%\setlength\fboxrule{1pt}
%
%
%  {
%
%$y\in f(A)\Longleftrightarrow $
% }
%\setlength\fboxrule{0.5pt}\vsec
% 
%
%
%
%\begin{exemples}
%$$\exp{(\R)}=\hspace{2cm}, \cos{\left| \left\lbrack 0,\ddp\frac{\pi}{2}\right\rbrack \right)}=\hspace{2cm}, \tan{\left| \left\rbrack -\ddp\frac{\pi}{2},\ddp\frac{\pi}{2}\right\lbrack \right)}=\hspace{2cm}, \ln{\left| \rbrack 0,1\rbrack \right)}=\hspace{2cm}$$
%\end{exemples}
%
%
%
%
%\ {D\'efinition de l'image r\'eciproque}\\
%
%
%  {  
%
%\begin{defi}
%Soit $f$ une application de $E$ dans $F$.\\
% Soit $B$ un sous-ensemble de $F$.\\
% On appelle image r\'eciproque de $B$ par $f$:
%$$f^{-1}(B)=\hspace{12cm}$$
%Ainsi on a toujours $f^{-1}(B)\subset \dots$
%\end{defi}
% 
%}
% 
%
%\setlength\fboxrule{1pt}
%
%
%  {
%
%$x\in f^{-1}(B)\Longleftrightarrow $
% }
%\setlength\fboxrule{0.5pt}\vsec
% 
%
%
%
%\begin{exemples}
%$$\exp^{-1}{(\R^-)}=\hspace{2cm}, \cos^{-1}{\left| \lbrack -1,1\rbrack \right)}=\hspace{2cm}, \tan^{-1}{\left| \R^+ \right)}=\hspace{4cm}, \ln^{-1}{\left| \rbrack 0,1\rbrack \right)}=\hspace{2cm}$$
%\end{exemples}
%
%  \warning  L'\'ecriture $f^{-1}(B)$ est abusive!!! Elle n'implique pas du tout \dotfill
%
%{\footnotesize 
%\begin{exercice}
%\begin{enumerate}
%\item Soit $f: \N\rightarrow \N$ d\'efinie par $n\mapsto 2n$. Calculer $f(A)$ et $f^{-1}(B)$ avec $A=\lbrace 1,2,4,10\rbrace$ et $B=\lbrace 0,1,2,3\rbrace$.
%\item Soit $g: \R\rightarrow \R$ la fonction carr\'ee. Calculer $g^{-1}(\R^-)$, $g^{-1}(\R^{-\star})$, $g^{-1}(\lbrack 1,4\rbrack)$, $g^{-1}(\lbrack -1,4\rbrack)$, $g(\lbrack 1,2\rbrack)$ et $g(\lbrack 4,8\rbrack)$.
%\end{enumerate}
%\end{exercice}}
%{\footnotesize 
%\begin{exercice}
%Soit $f$ une application de $E$ dans $F$ et $B$ un sous-ensemble de $F$. Montrer que $f\left| f^{-1}(B)\right)\subset B$.
%\end{exercice}}
%
%%-----------------------------------------------------------
%%----------------------------------------------------------
%\subsection{Compatibilit\'e avec l'inclusion, l'union et l'intersection }
%
%  {  
%
%\begin{prop}
%Soit $f$ une application de $E$ dans $F$.\\
% Soient $A$, $B$ sous-ensembles de $E$ et $C$, $D$ sous ensembles de $F$.
%
%
%\begin{itemize}
%\item[$\bullet$] Si $A\subset B$ alors  \vsec
%\item[$\bullet$] $f(A\cup B)=$ \vsec
%\item[$\bullet$] $f(A\cap B)$ \vsec
%\end{itemize}
% 
%\qquad \quad
%
%\begin{itemize}
%\item[$\bullet$] Si $C\subset D$ alors  \vsec
%\item[$\bullet$] $f^{-1}(C\cup D)=$ \vsec
%\item[$\bullet$] $f^{-1}(C\cap D)=$ \vsec
%\end{itemize}
% 
%\end{prop}
% 
%}
%
%\begin{proof}
%
%
%\end{proof}
%
%%-----------------------------------------------------------
%%----------------------------------------------------------
%\subsection{M\'ethodes pour d\'eterminer une image directe et une image r\'eciproque}
%
%
%\ {M\'ethodes avec les fonctions num\'eriques}\vsec\\
%
%\hspace{-1cm} 
%\setlength\fboxrule{1pt}
%  {
%
%\begin{itemize}
%\item[$\bullet$] M\'ethode 1 pour calculer une image directe:\\
%Th\'eor\`{e}me de la bijection.\\
%Parfois on a aussi besoin de $f(A\cup B)=f(A)\cup f(B)$.
%\item[$\bullet$] M\'ethode 2 pour calculer une image directe $f(A)$:\\
%Par la d\'efinition.
%\begin{itemize}
%\item[$\star$] \'Ecrire que $x\in A$.
%\item[$\star$] \'Ecrire des in\'egalit\'es successives afin obtenir $y=f(x)$.
%\end{itemize}
%\end{itemize}
% }
%\setlength\fboxrule{0.5pt}\vsec
% 
%\qquad \quad
%
%\setlength\fboxrule{1pt}
%
%
%
%  {
%
%Image r\'eciproque = r\'esolutions d'in\'equations.
% }
%\setlength\fboxrule{0.5pt}\vsec
% 
%
%{\footnotesize 
%\begin{exercice}
%Soit la fonction $f$ d\'efinie par $f(x)=\ddp\frac{3x}{1-x^2}$. Calculer $f^{-1}(\lbrack 0,1\rbrack)$, $f^{-1}(\R^+)$ puis $f(\rbrack -1,1\lbrack)$, $f(\rbrack 1,+\infty\lbrack)$ et $f(\rbrack -2,0\rbrack\setminus\lbrace -1\rbrace)$. 
%\end{exercice}}
%
%
%
%\ {M\'ethodes dans le cas g\'en\'eral: par la d\'efinition}\vsec\\
%
%
%\setlength\fboxrule{1pt}
%  {
%
%Pour l'image directe, utiliser:
%$$y\in f(A) \Leftrightarrow \hspace{3cm}$$
% }
%\setlength\fboxrule{0.5pt}\vsec
% 
%\qquad \quad
%
%\setlength\fboxrule{1pt}
%  {
%
%Pour l'image r\'eciproque, utiliser:
%$$x\in f^{-1}(B) \Leftrightarrow \hspace{3cm}$$
% }
%\setlength\fboxrule{0.5pt}\vsec
% 
%
%{\footnotesize 
%\begin{exercice}
%Finir de d\'emontrer les propri\'et\'es de compatibilit\'e avec l'inclusion, l'union et l'intersection.
%\end{exercice}}
%{\footnotesize 
%\begin{exercice}
%\begin{itemize}
%\item[$\bullet$] Soit $f: z\in\bC\mapsto |z|\in\R$. Calculer $f^{-1}(\lbrace 1\rbrace)$.
%\item[$\bullet$] Soit $g: z\in\bC^*\mapsto \arg{(z)}\in\R$. Calculer $g(\R^{\star})$, $g(i\R^{\star})$, $g(\R^{-\star})$ et $g(i\R^{+\star})$. 
%\item[$\bullet$] Soit $h: z\in\bC\setminus\lbrace 1\rbrace\mapsto \ddp\frac{z+1}{z-1}\in\bC$. Montrer que $h(\mathbb{U}\setminus\lbrace 1\rbrace)\subset i\R$.
%\end{itemize}
%\end{exercice}}
%
%

%%------------------------------------------------
%%-------------------------------------------------
%%-------------------------------------------------
%%--------------------------------------------------
%%------------------------------------------------
%\section{Restriction, prolongement, stabilit\'e et point fixe}

%%-----------------------------------------------------------
%%----------------------------------------------------------
%\subsection{}




\begin{exemple} Fonctions usuelles :\\

	\renewcommand{\arraystretch}{3.2}
	%\setlength{\tabcolsep}{1cm}
	\begin{tabular}{|c|c|c|c|}
		\hline
		Fonction                                                                        & Injective ? & Surjective ? & Bijective ? \\
		\hline
		sinus $: \R \to \R$                                                             &             &              &             \\
		\hline
		cosinus $: \R \to \R$                                                           &             &              &             \\
		\hline
		tangente $: \R\backslash \left\{ \frac{\pi}{2} + k\pi, k \in \Z\right\} \to \R$ &             &              &             \\
		\hline
		exponentielle $: \R \to \R$                                                     &             &              &             \\
		\hline
		logarithme $: \; ]0,+\infty[ \to \R$                                            &             &              &             \\
		\hline
		carr\'ee $: \R \to \R$                                                          &             &              &             \\
		\hline
		cube $: \R \to \R$                                                              &             &              &             \\
		\hline
		racine carr\'ee $: \R^+ \to \R$                                                 &             &              &             \\
		\hline
		inverse $: \R^\star \to \R$                                                     &             &              &             \\
		\hline
		valeur absolue $: \R \to \R$                                                    &             &              &             \\
		\hline
	\end{tabular}
	%\begin{enumerate}
	%\item Parmi les dessins suivants, lequel repr\'esente une fonction injective ? \\
	%
	%\item Exemples avec les fonctions usuelles:
	%\begin{itemize}
	%\item[$\bullet$] Fonction sinus: \dotfill\vsec
	%\item[$\bullet$] Fonction cosinus: \dotfill\vsec
	%\item[$\bullet$] Fonction tangente: \dotfill\vsec
	%\item[$\bullet$] Fonction exponentielle: \dotfill\vsec
	%\item[$\bullet$] Fonction logarithme: \dotfill\vsec
	%\item[$\bullet$] Fonction carr\'ee: \dotfill\vsec
	%\item[$\bullet$] Fonction cube: \dotfill\vsec
	%\item[$\bullet$] Fonction racine carr\'ee: \dotfill\vsec
	%\item[$\bullet$] Fonction inverse: \dotfill\vsec
	%\item[$\bullet$] Fonction valeur absolue: \dotfill\vsec
	%\end{itemize}
	%\end{enumerate}
	\label{ex:fct_usuelles}
\end{exemple}





\end{document}