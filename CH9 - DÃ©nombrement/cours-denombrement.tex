\documentclass[a4paper, 11pt]{article}
\usepackage[utf8]{inputenc}
\usepackage{amssymb,amsmath,amsthm}
\usepackage{geometry}
\usepackage[T1]{fontenc}
\usepackage[french]{babel}
\usepackage{fontawesome}
\usepackage{pifont}
\usepackage{tcolorbox}
\usepackage{fancybox}
\usepackage{bbold}
\usepackage{tkz-tab}
\usepackage{tikz}
\usepackage{fancyhdr}
\usepackage{sectsty}
\usepackage[framemethod=TikZ]{mdframed}
\usepackage{stackengine}
\usepackage{scalerel}
\usepackage{xcolor}
\usepackage{hyperref}
\usepackage{listings}
\usepackage{enumitem}
\usepackage{stmaryrd} 
\usepackage{comment}


\hypersetup{
    colorlinks=true,
    urlcolor=blue,
    linkcolor=blue,
    breaklinks=true
}





\theoremstyle{definition}
\newtheorem{probleme}{Problème}
\theoremstyle{definition}


%%%%% box environement 
\newenvironment{fminipage}%
     {\begin{Sbox}\begin{minipage}}%
     {\end{minipage}\end{Sbox}\fbox{\TheSbox}}

\newenvironment{dboxminipage}%
     {\begin{Sbox}\begin{minipage}}%
     {\end{minipage}\end{Sbox}\doublebox{\TheSbox}}


%\fancyhead[R]{Chapitre 1 : Nombres}


\newenvironment{remarques}{ 
\paragraph{Remarques :}
	\begin{list}{$\bullet$}{}
}{
	\end{list}
}




\newtcolorbox{tcbdoublebox}[1][]{%
  sharp corners,
  colback=white,
  fontupper={\setlength{\parindent}{20pt}},
  #1
}







%Section
% \pretocmd{\section}{%
%   \ifnum\value{section}=0 \else\clearpage\fi
% }{}{}



\sectionfont{\normalfont\Large \bfseries \underline }
\subsectionfont{\normalfont\Large\itshape\underline}
\subsubsectionfont{\normalfont\large\itshape\underline}



%% Format théoreme, defintion, proposition.. 
\newmdtheoremenv[roundcorner = 5px,
leftmargin=15px,
rightmargin=30px,
innertopmargin=0px,
nobreak=true
]{theorem}{Théorème}

\newmdtheoremenv[roundcorner = 5px,
leftmargin=15px,
rightmargin=30px,
innertopmargin=0px,
]{theorem_break}[theorem]{Théorème}

\newmdtheoremenv[roundcorner = 5px,
leftmargin=15px,
rightmargin=30px,
innertopmargin=0px,
nobreak=true
]{corollaire}[theorem]{Corollaire}
\newcounter{defiCounter}
\usepackage{mdframed}
\newmdtheoremenv[%
roundcorner=5px,
innertopmargin=0px,
leftmargin=15px,
rightmargin=30px,
nobreak=true
]{defi}[defiCounter]{Définition}

\newmdtheoremenv[roundcorner = 5px,
leftmargin=15px,
rightmargin=30px,
innertopmargin=0px,
nobreak=true
]{prop}[theorem]{Proposition}

\newmdtheoremenv[roundcorner = 5px,
leftmargin=15px,
rightmargin=30px,
innertopmargin=0px,
]{prop_break}[theorem]{Proposition}

\newmdtheoremenv[roundcorner = 5px,
leftmargin=15px,
rightmargin=30px,
innertopmargin=0px,
nobreak=true
]{regles}[theorem]{Règles de calculs}


\newtheorem*{exemples}{Exemples}
\newtheorem{exemple}{Exemple}
\newtheorem*{rem}{Remarque}
\newtheorem*{rems}{Remarques}
% Warning sign

\newcommand\warning[1][4ex]{%
  \renewcommand\stacktype{L}%
  \scaleto{\stackon[1.3pt]{\color{red}$\triangle$}{\tiny\bfseries !}}{#1}%
}


\newtheorem{exo}{Exercice}
\newcounter{ExoCounter}
\newtheorem{exercice}[ExoCounter]{Exercice}

\newcounter{counterCorrection}
\newtheorem{correction}[counterCorrection]{\color{red}{Correction}}


\theoremstyle{definition}

%\newtheorem{prop}[theorem]{Proposition}
%\newtheorem{\defi}[1]{
%\begin{tcolorbox}[width=14cm]
%#1
%\end{tcolorbox}
%}


%--------------------------------------- 
% Document
%--------------------------------------- 






\lstset{numbers=left, numberstyle=\tiny, stepnumber=1, numbersep=5pt}




% Header et footer

\pagestyle{fancy}
\fancyhead{}
\fancyfoot{}
\renewcommand{\headwidth}{\textwidth}
\renewcommand{\footrulewidth}{0.4pt}
\renewcommand{\headrulewidth}{0pt}
\renewcommand{\footruleskip}{5px}

\fancyfoot[R]{Olivier Glorieux}
%\fancyfoot[R]{Jules Glorieux}

\fancyfoot[C]{ Page \thepage }
\fancyfoot[L]{1BIOA - Lycée Chaptal}
%\fancyfoot[L]{MP*-Lycée Chaptal}
%\fancyfoot[L]{Famille Lapin}



\newcommand{\Hyp}{\mathbb{H}}
\newcommand{\C}{\mathcal{C}}
\newcommand{\U}{\mathcal{U}}
\newcommand{\R}{\mathbb{R}}
\newcommand{\T}{\mathbb{T}}
\newcommand{\D}{\mathbb{D}}
\newcommand{\N}{\mathbb{N}}
\newcommand{\Z}{\mathbb{Z}}
\newcommand{\F}{\mathcal{F}}




\newcommand{\bA}{\mathbb{A}}
\newcommand{\bB}{\mathbb{B}}
\newcommand{\bC}{\mathbb{C}}
\newcommand{\bD}{\mathbb{D}}
\newcommand{\bE}{\mathbb{E}}
\newcommand{\bF}{\mathbb{F}}
\newcommand{\bG}{\mathbb{G}}
\newcommand{\bH}{\mathbb{H}}
\newcommand{\bI}{\mathbb{I}}
\newcommand{\bJ}{\mathbb{J}}
\newcommand{\bK}{\mathbb{K}}
\newcommand{\bL}{\mathbb{L}}
\newcommand{\bM}{\mathbb{M}}
\newcommand{\bN}{\mathbb{N}}
\newcommand{\bO}{\mathbb{O}}
\newcommand{\bP}{\mathbb{P}}
\newcommand{\bQ}{\mathbb{Q}}
\newcommand{\bR}{\mathbb{R}}
\newcommand{\bS}{\mathbb{S}}
\newcommand{\bT}{\mathbb{T}}
\newcommand{\bU}{\mathbb{U}}
\newcommand{\bV}{\mathbb{V}}
\newcommand{\bW}{\mathbb{W}}
\newcommand{\bX}{\mathbb{X}}
\newcommand{\bY}{\mathbb{Y}}
\newcommand{\bZ}{\mathbb{Z}}



\newcommand{\cA}{\mathcal{A}}
\newcommand{\cB}{\mathcal{B}}
\newcommand{\cC}{\mathcal{C}}
\newcommand{\cD}{\mathcal{D}}
\newcommand{\cE}{\mathcal{E}}
\newcommand{\cF}{\mathcal{F}}
\newcommand{\cG}{\mathcal{G}}
\newcommand{\cH}{\mathcal{H}}
\newcommand{\cI}{\mathcal{I}}
\newcommand{\cJ}{\mathcal{J}}
\newcommand{\cK}{\mathcal{K}}
\newcommand{\cL}{\mathcal{L}}
\newcommand{\cM}{\mathcal{M}}
\newcommand{\cN}{\mathcal{N}}
\newcommand{\cO}{\mathcal{O}}
\newcommand{\cP}{\mathcal{P}}
\newcommand{\cQ}{\mathcal{Q}}
\newcommand{\cR}{\mathcal{R}}
\newcommand{\cS}{\mathcal{S}}
\newcommand{\cT}{\mathcal{T}}
\newcommand{\cU}{\mathcal{U}}
\newcommand{\cV}{\mathcal{V}}
\newcommand{\cW}{\mathcal{W}}
\newcommand{\cX}{\mathcal{X}}
\newcommand{\cY}{\mathcal{Y}}
\newcommand{\cZ}{\mathcal{Z}}







\renewcommand{\phi}{\varphi}
\newcommand{\ddp}{\displaystyle}


\newcommand{\G}{\Gamma}
\newcommand{\g}{\gamma}

\newcommand{\tv}{\rightarrow}
\newcommand{\wt}{\widetilde}
\newcommand{\ssi}{\Leftrightarrow}

\newcommand{\floor}[1]{\left \lfloor #1\right \rfloor}
\newcommand{\rg}{ \mathrm{rg}}
\newcommand{\quadou}{ \quad \text{ ou } \quad}
\newcommand{\quadet}{ \quad \text{ et } \quad}
\newcommand\fillin[1][3cm]{\makebox[#1]{\dotfill}}
\newcommand\cadre[1]{[#1]}
\newcommand{\vsec}{\vspace{0.3cm}}

\DeclareMathOperator{\im}{Im}
\DeclareMathOperator{\cov}{Cov}
\DeclareMathOperator{\vect}{Vect}
\DeclareMathOperator{\Vect}{Vect}
\DeclareMathOperator{\card}{Card}
\DeclareMathOperator{\Card}{Card}
\DeclareMathOperator{\Id}{Id}
\DeclareMathOperator{\PSL}{PSL}
\DeclareMathOperator{\PGL}{PGL}
\DeclareMathOperator{\SL}{SL}
\DeclareMathOperator{\GL}{GL}
\DeclareMathOperator{\SO}{SO}
\DeclareMathOperator{\SU}{SU}
\DeclareMathOperator{\Sp}{Sp}


\DeclareMathOperator{\sh}{sh}
\DeclareMathOperator{\ch}{ch}
\DeclareMathOperator{\argch}{argch}
\DeclareMathOperator{\argsh}{argsh}
\DeclareMathOperator{\imag}{Im}
\DeclareMathOperator{\reel}{Re}



\renewcommand{\Re}{ \mathfrak{Re}}
\renewcommand{\Im}{ \mathfrak{Im}}
\renewcommand{\bar}[1]{ \overline{#1}}
\newcommand{\implique}{\Longrightarrow}
\newcommand{\equivaut}{\Longleftrightarrow}

\renewcommand{\fg}{\fg \,}
\newcommand{\intent}[1]{\llbracket #1\rrbracket }
\newcommand{\cor}[1]{{\color{red} Correction }#1}

\newcommand{\conclusion}[1]{\begin{center} \fbox{#1}\end{center}}


\renewcommand{\title}[1]{\begin{center}
    \begin{tcolorbox}[width=14cm]
    \begin{center}\huge{\textbf{#1 }}
    \end{center}
    \end{tcolorbox}
    \end{center}
    }

    % \renewcommand{\subtitle}[1]{\begin{center}
    % \begin{tcolorbox}[width=10cm]
    % \begin{center}\Large{\textbf{#1 }}
    % \end{center}
    % \end{tcolorbox}
    % \end{center}
    % }

\renewcommand{\thesection}{\Roman{section}} 
\renewcommand{\thesubsection}{\thesection.  \arabic{subsection}}
\renewcommand{\thesubsubsection}{\thesubsection. \alph{subsubsection}} 

\newcommand{\suiteu}{(u_n)_{n\in \N}}
\newcommand{\suitev}{(v_n)_{n\in \N}}
\newcommand{\suite}[1]{(#1_n)_{n\in \N}}
%\newcommand{\suite1}[1]{(#1_n)_{n\in \N}}
\newcommand{\suiteun}[1]{(#1_n)_{n\geq 1}}
\newcommand{\equivalent}[1]{\underset{#1}{\sim}}

\newcommand{\demi}{\frac{1}{2}}
\geometry{hmargin=2.0cm, vmargin=3.5cm}

%Section
\pretocmd{\section}{%
  \ifnum\value{section}=0 \else\clearpage\fi
}{}{}



\begin{document}
   
 \title{Chapitre 12 : Dénombrement} 
 % debut
 %------------------------------------------------
\vspace{0.5cm}


%-----------------------------------------------------------
%----------------------------------------------------------
%-----------------------------------------------------------
%----------------------------------------------------------
%-----------------------------------------------------------
%----------------------------------------------------------
%-----------------------------------------------------------


%-----------------------------------------------------------
%----------------------------------------------------------
%-----------------------------------------------------------
%----------------------------------------------------------
\section{Cardinal d'un ensemble fini}
%----------------------------------------------------------
%-----------------------------------------------------------
%----------------------------------------------------------
\subsection{D\'efinition}

 

\begin{defi} 
\begin{itemize}
\item[$\bullet$] Soit $E$ un ensemble fini comportant $n$ éléments. 
On dit alors que $E$ est de cardinal $n$ et on note $\Card(E)=n$.
\item[$\bullet$] L'ensemble vide est un ensemble fini et son cardinal est $\Card(\emptyset)=0$.
\item[$\bullet$] Un singleton est un ensemble $E$ vérifiant  $\Card(E)=1$.
\end{itemize}
\end{defi}
 
 \warning Ne pas confondre le nombre d'éléments et la "dimension" des objets à l'intérieur de l'ensemble. 
 
 ex $E= \{ (1,2,3), (3,4,0)\}$ est un ensemble à \underline{2} éléments : 
 \begin{enumerate}
 \item $(1,2,3)$
 \item $(3,4,0)$ 
 \end{enumerate}
 Chaque élément est une liste contenant 3 nombres. \warning

\begin{exemple}
OG n'a r\'eussi \`a corriger que le quart de la moiti\'e de son paquet de $48$ copies. Quel est le cardinal de l'ensemble des copies restantes ?
\end{exemple}



%----------------------------------------------------------
%-----------------------------------------------------------
%----------------------------------------------------------
\subsection{Cardinal d'une union}




\begin{prop} 
Deux ensembles $A$ et $B$ sont disjoints  lorsque $A\cap B=\emptyset$
On a alors : $$\Card(A\cup B) =\Card(A)+\Card(B)$$
\end{prop}



\noindent {\footnotesize \begin{exercice} 
On tire une carte d'un jeu de 32 cartes. Quel est le nombre $N$ de possibilit\'es d'obtenir un 7 ou une figure ?
\end{exercice}
}



\begin{prop} 
Des ensembles finis $A_1,\ A_2,\dots,A_n$ sont deux \`{a} deux disjoints lorsque 
$$\forall i,j \in \intent{1,n}\, i\neq j \implique A_i \cap A_j =\emptyset$$
On a alors : 

$$\Card(A_1 \cup \cdots \cup A_n)  =\sum_{k=1}^n \Card(A_i)$$
\end{prop}
 



\noindent \warning  Ne pas confondre des ensembles deux \`{a} deux disjoints et l'intersection de tous les ensembles est vide.\\


 



\begin{prop} 
Soient $A$ et $B$ deux ensembles finis alors
$$\Card(A\cup B) =\Card(A) +\Card(B)- \Card(A\cap B)$$

\end{prop}
 



%----------------------------------------------------------
%-----------------------------------------------------------
%----------------------------------------------------------
\subsection{Cardinal d'un compl\'ementaire}

\begin{prop} 
Soit $A$ un sous ensemble d'un ensemble fini $E$.\\
\noindent On note $\overline{A}$ son compl\'ementaire dans $E$. Alors:

$$\Card(\bar{A}) =\Card(E) -\Card(A)$$
\end{prop}

\begin{rem}
Penser au compl\'ementaire d\`es que il y a: "au moins" dans l'énoncé. 
\end{rem}


 {\footnotesize \begin{exercice} 
Dans un centre de vacances, il y a 50 personnes plus ou moins sportives et de nombreuses activit\'es leur sont propos\'ees : 15 personnes font du tennis, 20 de la piscine, 30 du volley-ball, 10 du tennis de table, 5 du cheval et 4 restent allong\'ees au bord de la piscine toute la journ\'ee. Combien de personnes pratiquent au moins un sport ?
\end{exercice}
}


 
%----------------------------------------------------------
%-----------------------------------------------------------
%----------------------------------------------------------
\subsection{Cardinal d'un produit cart\'esien}


\begin{defi} Rappels sur le produit cart\'esien:
\begin{itemize}
\item[$\bullet$] Soient $A$ et $B$ deux ensembles. On note:
$$A\times B=\{ (a,b)\,|\,  a\in A, b\in B\}$$
et on a 
$$Card(A\times B) = \Card(A)\times \Card(B)$$
\item[$\bullet$] Soit $E$ un ensemble, on note:
 $$E^p=\{ (e_1,e_2,\dots, e_p)\,|\,  e_i\in E \}$$
 et on a 
 $$\Card(E^p) = \Card(E)^p$$
\end{itemize} 
\end{defi}






\noindent \begin{exemple} 
Calcul du cardinal de $A\times B$ avec $A=\lbrace  2,6,8\rbrace$ et $B=\lbrace1,3,5,6,8\rbrace$:\\

\end{exemple}

\noindent {\footnotesize \begin{exercice} 
On tire successivement 2 cartes d'un jeu de 32 cartes.
\begin{enumerate}
\item Quel est le nombre $N$ de possibilit\'es d'obtenir un roi suivi d'une dame ? 
\item On tire maintenant successivement 4 cartes du m\^eme jeu. Quel est le nombre $M$ de possibilit\'es d'obtenir dans l'ordre un as, un roi, une dame, et un valet ? 
\end{enumerate}
\end{exercice}
}

\vspace*{0.2cm}
%----------------------------------------------------------
%-----------------------------------------------------------
%----------------------------------------------------------
\subsection{Cardinal de l'ensemble des parties d'un ensemble}

\begin{defi} Soit $E$ un ensemble.
\begin{itemize}
\item[$\bullet$] On note $\mathcal{P}(E)$ l'ensemble des parties de $E$ 
On  a $$\Card(P(E)) = 2^{\Card(E)}$$
\end{itemize}
\end{defi}
 

%\begin{exemples}
%\begin{itemize}
%\item[$\bullet$] Si $E=\lbrace a,b,c\rbrace$ alors $\mathcal{P}(E)=$\dotfill \vsec
%\item[$\bullet$] Si $E=\lbrace a\rbrace$ alors $\mathcal{P}(E)=$\dotfill \vsec
%\end{itemize}
%\end{exemples}


%
%\begin{prop} Soit $E$ un ensemble fini, alors:  
%$$\Card (\mathcal{P}(E)) =2^{\Card(E)}$$
%\end{prop}



\begin{exemple}
Calculer le nombre de parties des ensembles pr\'ec\'edents.
\end{exemple}


\subsection{Cardinal et applications}

\begin{prop}
Soient $E$ et $F$ deux ensembles finis. 
\begin{itemize}
\item Il existe une injection entre $E$ et $F$ ssi $\Card(E) \leq \Card(F)$
\item Il existe une surjection entre $E$ et $F$ ssi $\Card(E) \geq \Card(F)$
\item Il existe une bijection entre $E$ et $F$ ssi $\Card(E) = \Card(F)$

\end{itemize}

\end{prop}
 


%-----------------------------------------------------------
%----------------------------------------------------------
%-----------------------------------------------------------
%----------------------------------------------------------
\section{Choix de $p$ objets parmi $n$}

\noindent La plupart des exercices de d\'enombrement peuvent se ramener au cas de tirages de $p$ \'el\'ements parmi les $n$ \'el\'ements d'un ensemble $E$. Il y a alors essentiellement quatre fa\c{c}ons diff\'erentes de tirer $p$ \'el\'ements parmi $n$:
\begin{itemize}
\item[$\bullet$] Avec ordre et r\'ep\'etition ($n^p$) (Nombre de  codes secrets de carte bleue)
\item[$\bullet$] Avec ordre et sans r\'ep\'etition, ($\frac{n!}{(n-p)!}$) (Nombre de possiblités au tiercé) 
\item[$\bullet$] Sans ordre et sans r\'ep\'etition, ($\binom{n}{p}$) (nombre de possibilités au loto)
\item[$\bullet$] Sans ordre et avec r\'ep\'etition. (plus rare et compliqué) 
\end{itemize}

%----------------------------------------------------------
%-----------------------------------------------------------
%----------------------------------------------------------
%----------------------------------------------------------
%-----------------------------------------------------------
%----------------------------------------------------------
\subsection{Choix successifs}

%----------------------------------------------------------
%-----------------------------------------------------------
%----------------------------------------------------------
\subsubsection{Listes avec r\'ep\'etitions \'eventuelles (avec ordre et r\'ep\'etition)}

 {\noindent 

\begin{defi} 
Soit $E$ un ensemble de cardinal fini $n$. \vsec\\
Une $p$-liste de $E$ est un élément de $E^p$
\end{defi}
 }\vsec

\begin{exemple}
Soit $E=\lbrace 0,2,4,6,8,10,12\rbrace$. \\
Donner une $2$-liste de $E$ : \dotfill et une $5$-liste de $E$ :\dotfill
\end{exemple}

\begin{rem}
\noindent \warning  Ne pas confondre $p$-liste avec ensemble \`a $p$ \'el\'ements. Dans une $p$-liste, l'ordre est important et si on change l'ordre, on change la $p$-liste. Dans un ensemble \`a $p$ \'el\'ements, l'ordre n'intervient pas et l'ensemble est toujours le m\^eme lorsque l'on intervertit des \'el\'ements.\\
Exemple avec les points de coordonn\'ees $(2,3)$ et $(3,2)$ :

\end{rem}




\begin{prop} Avec ordre et r\'ep\'etition:\\
Le nombre de fa\c{c}ons de choisir $p$ objets pris parmi $n$ objets distincts avec r\'ep\'etition possible et avec ordre est $n^p$ c'est-\`a-dire $\Card(E^p)$
\end{prop}




\noindent \begin{exemples} 
\begin{itemize}
\item[$\bullet$] \textbf{Exemple fondamental: Tirage successif avec remise}\\
\noindent Soient une urne contenant $n$ boules diff\'erentes et un entier $p$. On tire successivement $p$ boules dans l'urne, on note le num\'ero de la boule tir\'ee \`a chaque fois et l'on remet la boule dans l'urne. Le nombre de r\'esultats possibles d'un tel tirage est \dotfill\vsec
\item[$\bullet$] Nombre de fa\c{c}ons de ranger 2 chemises de couleurs diff\'erentes dans 3 tiroirs discernables : \dotfill\vsec
%Plus g\'en\'eralement quel est le nombre de r\'epartitions possibles de $p$ chemises de couleurs diff\'erentes dans $n$ tiroirs discernables: \dotfill \vsec\\
\item[$\bullet$] Nombre de mots de 5 lettres \'ecrits avec les lettres A,B,C,D,E et F : \dotfill \vsec
\item[$\bullet$] Nombre de r\'epartitions possibles de $5$ billes diff\'erentes dans $10$ bo\^{i}tes distinctes avec 0, 1 ou plusieurs billes par bo\^{i}te : \dotfill \vsec
% \dotfill
\end{itemize}
\end{exemples}

 



%----------------------------------------------------------
%-----------------------------------------------------------
%----------------------------------------------------------
\subsubsection{Listes sans r\'ep\'etition (avec ordre et sans r\'ep\'etition)}


\begin{defi} 
Soit $E$ un ensemble de cardinal fini $n$.\\
Une $p$-liste de $E$ sans r\'ep\'etition (ou arrangement de $p$ \'el\'ements de $E$) est 
\end{defi}

\begin{exemple} 
Soit $E=\lbrace 0,2,4,6,8,10,12\rbrace$. \\
Donner une $5$-liste sans r\'ep\'etition de $E$ : \dotfill 
\end{exemple}

\begin{rem}
Pour qu'il n'y ait pas de r\'ep\'etition, il faut n\'ecessairement que l'on ait \dotfill
\end{rem}



\begin{prop} Avec ordre et sans r\'ep\'etition:\\
Le nombre de fa\c{c}ons de choisir $p$ objets pris parmi $n$ objets distincts sans r\'ep\'etition possible et avec ordre est \dotfill \\
c'est-\`a-dire :\vsec\vsec%\ldots \ldots \ldots \ldots \vsec
\end{prop}

\begin{rem}
On utilise parfois la notation $\mathcal{A}_n^p$ (pour ``arrangement'') pour d\'esigner le nombre de $p$-listes sans r\'ep\'etition d'un ensemble \`a $n$ \'el\'ements. Ainsi, on a $\mathcal{A}_n^p =$ \dotfill\\
Par convention, on pose $\mathcal{A}_n^0 = 1$ et $\mathcal{A}_n^p=0$ si $p>n$.
\end{rem}

\noindent  \begin{exemples} 
\begin{itemize}
\item[$\bullet$] \textbf{Exemple fondamental: Tirage successif sans remise}\\
\noindent Soient une urne contexnant $n$ boules diff\'erentes et un entier $p$. On tire successivement et sans remise $p$ boules dans l'urne. Le nombre de r\'esultats possibles d'un tel tirage est \dotfill\vsec
\item[$\bullet$] Nombre de r\'epartitions possibles de $5$ billes diff\'erentes dans $10$ bo\^{i}tes distinctes avec au plus une bille par bo\^{i}te : \dotfill\vsec
\item[$\bullet$] Nombre de paris possibles au tierc\'e dans une course o\`u 15 chevaux sont en comp\'etition : \dotfill\vsec
\item[$\bullet$] Nombre de mots de 3 lettres distinctes avec les lettres A,B,C et D : \dotfill
\end{itemize}
\end{exemples}\vsec



\begin{defi}  
Soit $E$ un ensemble fini \`a $n$ \'el\'ements. \\
\noindent On appelle permutation de $E$ \dotfill
\end{defi}
 

\begin{exemple}
Soit $E=\lbrace 0,2,4,6,8,10,12\rbrace$. \\
Donner un exemple de permutation de $E$ : \dotfill\\
Doner un exemple de $7$-liste de $E$ qui n'est pas une permutation : \dotfill
\end{exemple}

 


\begin{prop}
Soit $E$ un ensemble fini de cardinal $n$. 
Le nombre de permutations de $E$ est \dotfill \vsec
\end{prop}
 


\noindent {\footnotesize \begin{exercice} 
Nombre de permutations de $\intent{ 1,3}$ ? de $\intent{ 1,6}$ ? \dotfill.
\end{exercice}
}
%
%
%\vspace*{0.5cm}
%
%\noindent\ {Exemple : Anagrammes}\\
%
% {\noindent 
%
%\begin{defi} 
%On appelle anagramme d'un mot \dotfill\vsec
%\end{defi}
% }\vsec
%
%
%\begin{exemple}
%Exemples d'anagramme du mot BCPST: \dotfill
%\end{exemple}\vsec
%
%
%\noindent \textbf{Calcul du nombre d'anagramme d'un mot:}
%
%\noindent \begin{itemize}
% \item[$\bullet$] Le cas de $k$ lettres distinctes: quel est le nombre d'anagrammes du mot cheval ?\\
%\vspace*{1cm}
%\item[$\bullet$] Le cas de lettres r\'ep\'et\'ees: quel est le nombre d'anagrammes du mot mouton ?\\
%\vspace*{3cm}
%\end{itemize}
%
%{\footnotesize\begin{exercice}
%Quel est le nombre d'anagrammes du mot mississippi ?
%\end{exercice}}

%----------------------------------------------------------
%-----------------------------------------------------------
%----------------------------------------------------------
%----------------------------------------------------------
%-----------------------------------------------------------
%----------------------------------------------------------
\subsection{Choix simultan\'es}


%----------------------------------------------------------
%-----------------------------------------------------------
%----------------------------------------------------------
\subsubsection{Combinaisons (sans ordre et sans r\'ep\'etition)}



\begin{defi}  
Soient $E$ un ensemble fini de cardinal $n$.\\
\noindent  On appelle combinaison de $p$ \'el\'ements pris parmi $n$ \'el\'ements de $E$ \dotfill \\
\noindent \phantom{ \hspace{-0.1cm}}\dotfill \vsec
\end{defi}
 



\begin{exemple} 
Soit $E=\lbrace 0,2,4,6,8,10,12\rbrace$. \\
Donner une combinaison \`a $5$ \'el\'ements de $E$ : \dotfill 
\end{exemple}


\begin{rem}
On doit n\'ecessairement avoir \dotfill
\end{rem}




\begin{prop} Sans ordre et sans r\'ep\'etition :\\
Le nombre de fa\c{c}ons de choisir $p$ objets pris parmi $n$ objets distincts sans r\'ep\'etition possible et sans ordre est \dotfill\\
c'est-\`a-dire :\vsec\vsec
\end{prop}





\noindent  \begin{exemples} 
\begin{itemize}
\item[$\bullet$] \textbf{Exemple fondamental: Tirage simultan\'e non ordonn\'e}\\
\noindent Soient une urne contenant $n$ boules diff\'erentes et un entier $p$. On tire simultan\'ement $p$ boules. Le nombre de r\'esultats possibles d'un tel tirage est \dotfill\vsec
\item[$\bullet$] Nombre de r\'epartitions possibles de $5$ billes identiques dans $10$ bo\^{i}tes distinctes avec au plus une bille par bo\^{i}te : \dotfill\vsec
\end{itemize}
\end{exemples}


{\footnotesize\begin{exercice}
Jeu de cartes: on distribue 5 cartes d'un jeu de 32 cartes \`a un joueur, celui-ci dispose donc d'une main de 5 cartes.
\begin{enumerate}
\item D\'eterminer le nombre de mains possibles.
\item D\'eterminer le nombre de mains contenant exactement 2 coeurs.
\item D\'eterminer le nombre de mains contenant exactement 2 cartes de pique, 2 cartes de coeur et 1 carte de carreau.
\end{enumerate}
\end{exercice}}

{\footnotesize\begin{exercice}
Formule des "chefs" : un s\'electionneur de foot doit choisir $k$ joueurs parmi $n$ candidats, et d\'esigner un capitaine parmi les joueurs. En comptant de deux fa\c cons diff\'erentes le nombre de possibilit\'es, red\'emontrer la formule des "chefs".
\end{exercice}}


\vspace*{0.2cm}



%----------------------------------------------------------
%-----------------------------------------------------------
%----------------------------------------------------------
\subsubsection{Choix sans ordre et avec r\'ep\'etition}

\noindent Ce cas l\`{a} est plus rare mais il appara\^{i}t parfois. On verra quelques exemples en TD.

\begin{exemple}  On consid\`ere $5$ boules indiscernables que l'on veut placer dans $3$ tiroirs distincts, chaque tiroir pouvant contenir de 0 \`a $5$ boules. Donner le nombre de r\'epartitions possibles.\\
\vspace*{5cm}
\end{exemple}



 
%-----------------------------------------------------------
%----------------------------------------------------------
%-----------------------------------------------------------
%----------------------------------------------------------
%\section{Cardinal et application}
%
%\noindent Dans toute cette partie, on consid\`{e}re deux ensembles finis $E$ et $F$ de cardinal respectivement $p$ et $n$.
%
%
%%----------------------------------------------------------
%%-----------------------------------------------------------
%%----------------------------------------------------------
%%----------------------------------------------------------
%%-----------------------------------------------------------
%%----------------------------------------------------------
%\subsection{Injectivit\'e, surjectivit\'e, bijectivit\'e et d\'enombrement}
%
%
% {\noindent  
%
%\begin{prop}  \hspace*{0.5cm}
%\begin{itemize}
%\item[$\bullet$] Il existe une injection de $E$ dans $F$ si et seulement si \dotfill\vsec
%\item[$\bullet$] Il existe une surjection de $E$ dans $F$ si et seulement si \dotfill\vsec
%\item[$\bullet$] Il existe une bijection de $E$ dans $F$ si et seulement si \dotfill\vsec
%\end{itemize}
%\end{prop}
% 
%}\vsec
%
%\begin{proof} 
%\vspace*{10cm}
%\end{proof}
%
%
%%----------------------------------------------------------
%%-----------------------------------------------------------
%%----------------------------------------------------------
%%----------------------------------------------------------
%%-----------------------------------------------------------
%%----------------------------------------------------------
%\subsection{Nombre d'applications de $E$ dans $F$}
%
% {\noindent  
%
%\begin{prop}  Le nombre d'applications de $E$ dans $F$ est:\vsec
%\end{prop}
% 
%}\vsec
%
%\begin{proof} 
%\vspace*{7cm}
%\end{proof}
%
%
%
%%----------------------------------------------------------
%%-----------------------------------------------------------
%%----------------------------------------------------------
%\subsection{Nombre d'injections de $E$ dans $F$}
%
%
% {\noindent  
%
%\begin{prop} On suppose $p\leq n$.\\
%Le nombre d'applications injectives de $E$ dans $F$ est: \vsec
%\end{prop}
% 
%}\vsec
%
%\begin{proof} 
%\vspace*{5cm}
%\end{proof}
%
%
%%
%%----------------------------------------------------------
%%-----------------------------------------------------------
%%----------------------------------------------------------
%\subsection{Nombre de bijections de $E$ dans $F$}
%
%
% {\noindent  
%
%\begin{prop}  
%Soient $E$ et $F$ deux ensembles finis non vides ayant le m\^{e}me cardinal $n$.\\
%\noindent Alors le nombre d'applications bijectives de $E$ dans $F$ est:\vsec
%\end{prop}
% 
%}\vsec
%
%\begin{proof} 
%\vspace*{4cm}
%\end{proof}
%



\end{document}