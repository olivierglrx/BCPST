\documentclass[a4paper, 11pt]{article}
\input{macro/package.tex}
\input{macro/environement}
% Header et footer

\pagestyle{fancy}
\fancyhead{}
\fancyfoot{}
\renewcommand{\headwidth}{\textwidth}
\renewcommand{\footrulewidth}{0.4pt}
\renewcommand{\headrulewidth}{0pt}
\renewcommand{\footruleskip}{5px}

\fancyfoot[R]{Olivier Glorieux}
%\fancyfoot[R]{Jules Glorieux}

\fancyfoot[C]{ Page \thepage }
\fancyfoot[L]{1BIOA - Lycée Chaptal}
%\fancyfoot[L]{MP*-Lycée Chaptal}
%\fancyfoot[L]{Famille Lapin}

\input{macro/newcommand.tex}
\geometry{hmargin=2.0cm, vmargin=2.5cm}




\begin{document}
\tableofcontents
\title{Chapitre : derivation}
% debut
%------------------------------------------------
\vspace{0.5cm}



%----------------------------------------------------
%-----------------------------------------------------
%-------------------------------------------------------


\noindent Dans tout ce chapitre, $I$ d\'esigne un intervalle de $ \R$ non r\'eduit \`a un point.


%------------------------------------------------
%-----------------------------------------------------------
%----------------------------------------------------------
%-----------------------------------------------------------
%----------------------------------------------------------
\section{D\'erivabilit\'e en un point}


%-----------------------------------------------------------
%----------------------------------------------------------
%-----------------------------------------------------------
%----------------------------------------------------------
\subsection{D\'erivabilit\'e en un point: d\'efinition}

{\noindent

	\begin{defi} Taux d'accroissement:\\
		\noindent Soient $f: I\rightarrow \R$ et $x_0\in I$. Pour tout $x\in I$ avec $x\not= x_0$, on appelle taux d'accroissement de $f$ entre $x$ et $x_0$ le quotient:\vsec\\
		\noindent $\tau_{f,x_0}(x)=$\dotfill \phantom{   \hspace{7cm}  }\vsec
	\end{defi}

}

\vspace{0.3cm}

{\noindent

	\begin{defi} D\'erivabilit\'e d'une fonction en un point:\\
		\noindent Soient $f: I\rightarrow \R$ et $x_0\in I$. \vsec
		\begin{itemize}
			\item[$\bullet$] On dit que $f$ est d\'erivable en $x_0$ si \dotfill \vsec\\
			      \phantom{\hspace{0cm}} \dotfill \vsec
			\item[$\bullet$] Si cette limite existe, elle est not\'ee \dotfill et est appel\'ee le nombre d\'eriv\'ee de $f$ en $x_0$:\\
			      \vspace{1cm}
		\end{itemize}
	\end{defi}

	Remarque  changtement de variable $(f(x_0+h) -f(x_0))/h$
}

Exemple : exp. ln. sin. racine carrée.


{\footnotesize \begin{exercice}
	\begin{enumerate}
		\item \'Etudier la d\'erivabilit\'e de la fonction carr\'ee en $1$.
		\item \'Etudier la d\'erivabilit\'e de la fonction cube en $2$.
		      %\item[$\bullet$] \'Etudier la d\'erivabilit\'e en 0 de la fonction $f$ d\'efinie par: 
		      %$f(x)=\left\lbrace\begin{array}{cl}  x^2\ln{(x)} & \mathrm{si}\ x>0\vsec\\ 0 & \mathrm{si}\ x=0.   \end{array}\right.$
	\end{enumerate}
\end{exercice}}


%-----------------------------------------------------------
%----------------------------------------------------------
%-----------------------------------------------------------
%----------------------------------------------------------
%\subsection{D\'erivabilit\'e \`{a} gauche et \`{a} droite en un point}

% {\noindent  
%
%\begin{defi} D\'erivabilit\'e d'une fonction \`{a} droite et \`{a} gauche en un point:\\
%\noindent Soient $f: I\rightarrow \R$ et $x_0\in I$ non une borne de $I$. \vsec
%\begin{itemize}
% \item[$\bullet$] On dit que $f$ est d\'erivable \`a gauche en $x_0$ si \dotfill \vsec\\
%\phantom{\hspace{0cm}} \dotfill \vsec
%\item[$\bullet$] On dit que $f$ est d\'erivable \`a droite en $x_0$ si \dotfill \vsec\\
%\phantom{\hspace{0cm}} \dotfill \vsec
%\end{itemize}
%\end{defi}
% 


\begin{exercice}
	\begin{enumerate}
		\item \'Etudier la d\'erivabilit\'e de la fonction valeur absolue en 0.
		\item \'Etudier la d\'erivabilit\'e en 0 de la fonction $f$ d\'efinie par $f(x)=\left\lbrace\begin{array}{cl}  e^{\frac {1}{x}} & \mathrm{si}\ x\not=0\vsec\\ 0 & \mathrm{si}\ x=0.   \end{array}\right.$\vsec
		\item \'Etudier la d\'erivabilit\'e en 0 de la fonction $f$ d\'efinie par $f(x)=\left\lbrace\begin{array}{cl}  e^{x} & \mathrm{si}\ x\geq 0\vsec\\ \ddp\frac{\sin{x}}{x} & \mathrm{si}\ x<0.   \end{array}\right.$
	\end{enumerate}
\end{exercice}



%
% {\noindent  
%
%\begin{prop} Lien entre d\'erivabilit\'e en un point et d\'erivabilit\'e \`{a} droite et \`{a} gauche en un point:\\
%\noindent Soient $f: I\rightarrow \R$ et $x_0\in I$ non une borne de $I$. \vsec\\
%\noindent 
%$f$ est d\'erivable en $x_0$ $\Longleftrightarrow$ \dotfill \vsec
%\end{prop}
% }
%
%\begin{rem}
%Dans le cas o\`u les deux limites valent soit toutes les deux $+\infty$, soit toutes les deux $-\infty$, la fonction n'est pas d\'erivable en $x_0$ mais la limite existe et $\lim\limits_{x\to x_0} \ddp\frac{f(x)-f(x_0)}{x-x_0}=\pm\infty$.
%\end{rem}

\begin{exercice}
	\begin{enumerate}
		\item \'Etude de la d\'erivabilit\'e en 2 de la fonction d\'efinie par $f(x)=-x+\sqrt{(x-2)^2(x-1)}$.
		\item \'Etude de la d\'erivabilit\'e en 0 de la fonction d\'efinie sur $\R$ par $f(x)= \texttt{signe}(x)\sqrt[3]{|x|}$, o\`u $\texttt{signe}(x) = \left\lbrace\begin{array}{cl}
				      1  & \hbox{si}\ x>0  \\
				      0  & \hbox{si}\ x=0  \\
				      -1 & \hbox{si}\ x< 0\end{array}\right.$.
		      \vspace*{-0.2cm}
		\item \'Etude de la d\'erivabilit\'e en 0 de la fonction d\'efinie sur $\R$ par $f(x)=\left\lbrace\begin{array}{cl} e^{-\frac{1}{x}} & \hbox{si}\ x>0\vsec \\
             x^2                   & \hbox{si}\ x\leq 0\end{array}\right.$.
	\end{enumerate}
\end{exercice}

%-----------------------------------------------------------
%----------------------------------------------------------
%-----------------------------------------------------------
%----------------------------------------------------------
\subsection{Interpr\'etation graphique}



\begin{prop} Tangente:\\
	\noindent Si la fonction $f$ est d\'erivable en $x_0$ alors la courbe $\mathcal{C}_f$ admet au point d'abscisse $x_0$ une tangente qui a pour \'equation: \dotfill  \vsec
\end{prop}



\begin{rem}
	La connaissance de la tangente $T$ \`a $\mathcal{C}_f$ permet de tracer la courbe au voisinage du point $M$ d'abscisse $x_0$. Pour un trac\'e encore plus pr\'ecis, on \'etudie souvent la position de la courbe par rapport \`a la tangente, \`a savoir le signe de $f(x)-y=f(x)-f(x_0)-f^{\prime}(x_0)(x-x_0)$.
\end{rem}


\begin{exemples} Repr\'esentation graphique des fonctions exponentielle et logarithme n\'ep\'erien. On pr\'ecisera leur tangente \`{a} leur courbe respectivement aux points d'abscisse 0 et 1.\vsec
\end{exemples}



{\footnotesize \begin{exercice} Soit la fonction $f$ d\'efinie par $f(x)=\ddp\frac{\ln{(1+x)}}{(x+1)e^x}$. \'Equation de la tangente \`{a} la courbe $\mathcal{C}_f$ au point d'abscisse 0 .
\end{exercice}}

\vsec\vsec




{\noindent

	\begin{defi} Tangente verticale:\\
		\noindent Soient $f: I\rightarrow \R$ et $x_0\in I$. On suppose que la fonction $f$ est continue en $x_0$ et que $\lim\limits_{x\to x_0} \ddp\frac{f(x)-f(x_0)}{x-x_0}=\pm\infty$, \vsec alors $\mathcal{C}_f$ \dotfill\vsec
	\end{defi}
}

%
%
%\noindent\ {Demi-tangente}\\
%
%\noindent Dans le cas o\`u $f$ est d\'erivable \`a droite ou \`a gauche en $x_0$, on parle de demi-tangente \`a droite ou \`a gauche. Ces demi-tangentes sont distinctes si les nombres d\'eriv\'es \`a droite et \`a gauche sont distincts.
%On parle aussi de demi-tangente verticale \`a droite ou \`a gauche si l'un des taux d'accroissement tend vers l'infini quand $x$ tend vers $x_0$ \`a droite ou \`a gauche.\\
%
% {\noindent  
%
%\begin{defi} Point anguleux:\\
%\noindent Soient $f: I\rightarrow \R$ et $x_0\in I$. On suppose que $f$ est d\'erivable \`a gauche et \`a droite en $x_0$ avec $f^{\prime}_g(x_0)\not= f^{\prime}_d(x_0)$.\vsec\\ On dit alors que $\mathcal{C}_f$ \dotfill\vsec
%\end{defi}
% }
%
%\vspace{0.3cm}
%
%\begin{exemple} Repr\'esentation graphique de la fonction valeur absolue.
%\end{exemple}
%\vspace{4cm}


{\footnotesize \begin{exercice} Soit la fonction $f$ d\'efinie par $f(x)=x^2+x+|2x(x+2)|$. \'Etude de la d\'erivabilit\'e en 0.
	\end{exercice}}

\vspace{0.3cm}

% {\noindent  
%
%\begin{defi} Point de rebroussement:\\
%\noindent Soient $f: I\rightarrow \R$ et $x_0\in I$. Si la limite \`a droite et \`{a} gauche du taux d'accroissement en $x_0$ sont toutes les deux infinies mais de signe oppos\'e, \vsec on dit alors que $\mathcal{C}_f$ admet \dotfill \vsec
%\end{defi}
% }
%
%{\footnotesize \begin{exercice} Repr\'esentation graphique de la fonction d\'efinie par $f(x)=\left\lbrace\begin{array}{ll}  \sqrt{x} & \mathrm{si}\ x\geq 0\vsec\\  -\sqrt[3]{x} & \mathrm{si}\ x< 0.\end{array}\right.$
%\end{exercice}}
%
%\begin{rem} On peut aussi \^etre dans le cas o\`u la limite \`a droite ou \`a gauche quand $x$ tend vers $x_0$ du taux d'accroissement est fini et l'autre limite est infinie. 
%\end{rem}
%{\footnotesize \begin{exercice}
%Donner un exemple d'une telle fonction et en faire la repr\'esentation graphique.
%\end{exercice}}
%
%
%\vspace{0.5cm}
%
%\noindent\ {Pas de tangente}\\
%
%\noindent Il reste le cas o\`u le taux d'accroissement n'a pas de limite ni finie ni infinie, ni \`a droite, ni \`a gauche.
%{\footnotesize \begin{exercice} \'Etude d'un \'eventuel prolongement par continuit\'e en 0 de la fonction $f$ d\'efinie par $f(x)=x\sin{\left( \ddp\frac{1}{x}\right)}$. \'Etude de la d\'erivabilit\'e en 0 et repr\'esentation graphique.
%\end{exercice}}


%----------------------------------------------------------
%-----------------------------------------------------------
%----------------------------------------------------------
\subsection{Lien entre la d\'erivabilit\'e et la continuit\'e}

{\noindent

	\begin{prop} Soient $f: I\rightarrow \R$ et $x_0\in I$.\vsec\\
		\noindent  Si $f$ est d\'erivable en $x_0$ alors \dotfill  \vsec
	\end{prop}
}

\noindent \warning  La r\'eciproque est fausse : \dotfill



\begin{rem} Par contrapos\'ee, on obtient \dotfill
\end{rem}



%-----------------------------------------------------------
%----------------------------------------------------------
\section{D\'erivabilit\'e sur un intervalle}

%-----------------------------------------------------------
%----------------------------------------------------------
%-----------------------------------------------------------
%----------------------------------------------------------
\subsection{D\'erivabilit\'e sur un intervalle: d\'efinition}


{\noindent

	\begin{defi} D\'erivabilit\'e d'une fonction sur un intervalle:
		\begin{itemize}
			\item[$\bullet$] La fonction $f$ est d\'erivable sur l'intervalle $I$ si elle \dotfill
			\item[$\bullet$] On appelle alors fonction d\'eriv\'ee de $f$ et on note $f^{\prime}$ la fonction qui \`a tout $x$ de $I$ associe $f^{\prime}(x)$.
		\end{itemize}
	\end{defi}

}


%-----------------------------------------------------------
%----------------------------------------------------------
%-----------------------------------------------------------
%----------------------------------------------------------
%\subsection{D\'erivabilit\'e des fonctions usuelles}
%
%\begin{itemize}
%\item[$\bullet$] Les fonctions trigonom\'etriques: \vsec
%\begin{itemize}
%\item[$\star$] Les fonctions sinus et cosinus sont d\'erivables sur \dotfill \vsec
%\item[$\star$] La fonction tangente est d\'erivable sur \dotfill \vsec
%%\item[$\star$] La fonction cotangente est \dotfill \vsec
%\end{itemize}
%\item[$\bullet$] Les fonctions exponentielle et logarithme n\'ep\'erien: \vsec
%\begin{itemize}
%\item[$\star$] La fonction exponentielle est d\'erivable sur \dotfill \vsec
%\item[$\star$] La fonction logarithme n\'ep\'erien est d\'erivable sur \dotfill \vsec
%\end{itemize}
%\item[$\bullet$] Les fonctions puissances: $x\mapsto x^{\alpha}$: \vsec
%\begin{itemize}
%\item[$\star$] Si $\alpha=n\in\N$, d\'erivable sur \dotfill \vsec
%\item[$\star$] Si $\alpha=n\in\Z\setminus\N$, d\'erivable sur  \dotfill \vsec
%\item[$\star$] Si $\alpha\in\R\setminus\Z$, $x^\alpha = \ldots \ldots \ldots \ldots \ldots$, la fonction est d\'erivable sur  \dotfill \vsec
%\end{itemize}
%\item[$\bullet$] La fonction valeur absolue est d\'erivable sur \dotfill \vsec
%\item[$\bullet$] La fonction partie enti\`{e}re est d\'erivable sur \dotfill  \vsec
%\end{itemize}
%
%\begin{proof} 
%\begin{itemize}
%\item[$\bullet$] \'Etude de la d\'erivabilit\'e de la fonction carr\'ee et expression de sa d\'eriv\'ee:
%\vspace*{3cm}
%\item[$\bullet$] \'Etude de la d\'erivabilit\'e de la fonction inverse et expression de sa d\'eriv\'ee:
%\vspace*{3cm}
%\end{itemize}
%\end{proof}


%-----------------------------------------------------------
%----------------------------------------------------------
%-----------------------------------------------------------
%----------------------------------------------------------
\subsection{Op\'erations alg\'ebriques sur les d\'eriv\'ees}

{\noindent

	\begin{prop}
		Soient $I$ un intervalle de $\R$, $f$ et $g$ deux fonctions d\'erivables sur $I$ et $\lambda\in\R$. On a alors:\vsec
		\begin{itemize}
			\item[$\bullet$] $f+g$ est d\'erivable sur $I$ et $(f+g)^{\prime}=$\dotfill \phantom{\hspace{5cm}}\vsec
			\item[$\bullet$] $\lambda f$ est d\'erivable sur $I$ et $(\lambda f)^{\prime}=$\dotfill \phantom{\hspace{5cm}} \vsec
			\item[$\bullet$] $fg$ est d\'erivable sur $I$ et $(fg)^{\prime}=$\dotfill \phantom{\hspace{5cm}} \vsec
			\item[$\bullet$] Si la fonction $g$ ne s'annule pas sur $I$ alors $\ddp\frac{1}{g}$ est d\'erivable sur $I$ et $\left(\ddp\frac{1}{g}\right)^{\prime}=$\dotfill \vsec
			\item[$\bullet$] Si la fonction $g$ ne s'annule pas sur $I$ alors $\ddp\frac{f}{g}$ est d\'erivable sur $I$ et $\left(\ddp\frac{f}{g}\right)^{\prime}=$\dotfill \vsec
		\end{itemize}
	\end{prop}
}

{\footnotesize \begin{exercice}
		\'Etudier la d\'erivabilit\'e des fonctions suivantes et calculer leur d\'eriv\'ee: $ f(x)=\ddp\frac{xe^x}{\ln{x}}$ et $g(x)=\ddp\frac{x^{\alpha}e^x+x^2}{\cos{x}}$.
	\end{exercice}}


%-----------------------------------------------------------
%----------------------------------------------------------
%-----------------------------------------------------------
%----------------------------------------------------------
\subsection{D\'erivabilit\'e d'une compos\'ee}

{\noindent

	\begin{prop}
		Soient $I$ et $J$ deux intervalles de $\R$. \\
		\noindent Si $f$ est d\'erivable sur $I$, $g$ est d\'erivable sur $J$ et \dotfill alors:\vsec
		\begin{itemize}
			\item[$\bullet$] $g \circ f$ est d\'erivable sur \ldots\ldots\ldots \vsec
			\item[$\bullet$] \dotfill \phantom{  \hspace{1cm}  }\vsec
		\end{itemize}
		\vsec
	\end{prop}
}

\begin{exemples} Soient $I$ un intervalle, $n\in\N^{\star}$ et $\alpha\in\R\setminus\Z$. Soit $u$ une fonction d\'erivable sur $I$. Alors : \vsec
	\begin{itemize}
		\item[$\bullet$] La fonction $\sin{(u)}$ est d\'erivable sur $I$ et $(\cos (u))'=$ \dotfill \vsec
		\item[$\bullet$] La fonction $\cos{(u)}$ est d\'erivable sur $I$ et $(\sin(u))'=$ \dotfill \vsec
		\item[$\bullet$] La fonction $e^{u}$ est d\'erivable sur $I$ et $(e^u)'=$ \dotfill \vsec
		\item[$\bullet$] Si $\forall x \in I$ \dotfill alors la fonction $\ln{{|u|}}$ est d\'erivable sur $I$ et $(\ln |u|)'= $ \dotfill  \vsec
		\item[$\bullet$] La fonction $u^n$ est d\'erivable sur $I$ et $(u^n)'=$ \dotfill \vsec
		\item[$\bullet$] Si $\forall x \in I$ \dotfill  alors la fonction $\ddp\frac{1}{u^n}$ est d\'erivable sur $I$ et $\left(\frac{1}{u^n}\right)'=$ \dotfill \vsec
		\item[$\bullet$] Si  $\forall x \in I$ \dotfill  alors $u^{\alpha}$ est d\'erivable sur $I$ et $(u^\alpha)'=$ \dotfill \vsec
		\item[$\bullet$] Si $\forall x \in I$ \dotfill  alors $\sqrt{u}$ est d\'erivable sur $I$ et $(\sqrt{u})'=$ \dotfill \vsec
		\item[$\bullet$] Si $v$ est d\'erivable sur $I$ et $\forall x \in I$ \dotfill  alors $u^v$ est d\'erivable sur $I$ et $(u^v)'=$ \dotfill \vsec
	\end{itemize}
\end{exemples}

{\footnotesize \begin{exercice}
	\'Etudier la d\'erivabilit\'e des fonctions suivantes et calculer leur d\'eriv\'ee:
	\begin{enumerate}
		\item $f(x)=\sqrt{1-2\cos{(x)}}$ \vsec
		\item $f(x)=\ln{\left( \sqrt{1-x^2} +1 \right)}$\vsec
		      %\item[$\bullet$] $f_2(x)=\ddp\frac{(x^3-4x)^ne^{-6x}}{\ln{( \sqrt{ x^2-4 } )}}$
		\item $f_3(x)=\left( 1+e^{-x^2} \right)^{x^2-3x}$


		      %\item[$\bullet$] $f_4(x)=\ddp\frac{1}{\sqrt{3}\cos^2{(x)}-2\sin{(x)}\cos{(x)}-\sqrt{3}\sin^2{(x)}-\sqrt{2}    }$
		\item $f(x)=\ln{\left( \ddp\frac{ x^x-1 }{x^x+1} \right)}$
		\item $f(x)=\ln{\left( \ddp\frac{x+3}{-x+1} \right)}$
		      %\item[$\bullet$] $f(x)=$.

	\end{enumerate}
\end{exercice}}

%-----------------------------------------------------------
%----------------------------------------------------------
%-----------------------------------------------------------
%----------------------------------------------------------
\subsection{D\'erivabilit\'e d'une fonction r\'eciproque}


{\noindent

	\begin{theorem} Th\'eor\`{e}me de d\'erivabilit\'e d'une fonction r\'eciproque: \\
		\noindent Soit $f: I\rightarrow J$ une fonction bijective de $I$ dans $J$.  Elle admet ainsi  une fonction r\'eciproque $f^{-1}: J\rightarrow I$.\\
		Si:
		\begin{itemize}
			\item[$\bullet$] \dotfill\vsec
			\item[$\bullet$] \dotfill\vsec
		\end{itemize}
		\noindent Alors \dotfill et $\forall y \in J, \; (f^{-1})'(y) = $ \dotfill \vsec\vsec
	\end{theorem}
}

%ARctan et racine n eme au programme. 
\vspace{0.3cm}

\setlength\fboxrule{1pt}
\noindent  {

	\textbf{M\'ethode pour \'etudier la d\'erivabilit\'e d'une r\'eciproque:}
	\begin{itemize}
		\item[$\bullet$] On justifie que $f$ est d\'erivable sur un intervalle et on calcule la d\'eriv\'ee $f^{\prime}$ de $f$.
		\item[$\bullet$] On d\'etermine tous les points $x_0$ o\`u $f^{\prime}$ s'annule: cela nous donne l'intervalle $I$ qui permet d'appliquer le th\'eor\`{e}me.
		\item[$\bullet$] On calcule tous les $y_0=f(x_0)$ correspondants : cela nous donne tous les points \`{a} enlever et on conna\^{i}t ainsi l'intervalle $J$ sur lequel $f^{-1}$ va \^{e}tre d\'erivable.
		\item[$\bullet$] On applique le th\'eor\`{e}me en commen\c{c}ant par \'enoncer les deux hypoth\`{e}ses.\\
		      On sait alors que $f^{-1}$ est d\'erivable sur l'intervalle $J$, et que $\forall y \in J, \; (f^{-1})^{\prime}(y)=\ddp\frac{1}{f^{\prime}(f^{-1}(y))}$.
	\end{itemize}
}
\setlength\fboxrule{0.5pt}

\begin{exemples}
	\begin{itemize}
		\item[$\bullet$] \'Etudier la d\'erivabilit\'e et calculer la d\'eriv\'ee de la fonction arctangente.
		      %\vspace*{4cm}
		\item[$\bullet$] \'Etudier la d\'erivabilit\'e et calculer la d\'eriv\'ee de la fonction racine $n$-i\`eme.
		      \begin{itemize}
			      \item[$\star$] Cas o\`u $n$ est pair :
			            %\vspace*{5cm}
			      \item[$\star$] Cas o\`u $n$ est impair :
			            %\vspace*{4cm}
		      \end{itemize}
	\end{itemize}
\end{exemples}

%{\footnotesize
%\begin{exercice}
% \'Etudier la d\'erivabilit\'e de $f^{-1}$ o\`u la fonction $f$ est d\'efinie par
%$f(x)=\left\lbrace\begin{array}{cl} \ddp\frac{x}{e^x-1} & \hbox{si}\ x\not= 0\vsec\\
%1  & \hbox{si}\ x = 0.    \end{array}\right.$.
%%$f(x)=\ddp\frac{x}{x-\ln{(x)}}$. 
%\end{exercice}}


\vspace*{0.5cm}


%----------------------------------------------------------
%-----------------------------------------------------------
%----------------------------------------------------------
\section{Th\'eor\`{e}mes utilisant la d\'erivabilit\'e sur un intervalle}


%\noindent On fait ici un r\'ecapitulatif de tous les th\'eor\`{e}mes utilisant la d\'erivabilit\'e d'une fonction sur un intervalle.
%-----------------------------------------------------------
%----------------------------------------------------------
%-------------------------------------------------------
\subsection{Lien entre le signe de la d\'eriv\'ee et le sens de variation d'une fonction}

{\noindent

	\begin{theorem}
		Soit $f$ une fonction d\'erivable sur un intervalle $I$. On a les \'equivalences suivantes:\vsec
		\begin{itemize}
			\item[$\bullet$] La fonction $f$ est croissante sur $I$ \; $\Longleftrightarrow $ \; \dotfill\vsec
			\item[$\bullet$] La fonction $f$ est d\'ecroissance sur $I$  \; $\Longleftrightarrow $ \;  \dotfill\vsec
			\item[$\bullet$] La fonction $f$ est constante sur $I$  \; $\Longleftrightarrow $ \;  \dotfill \vsec
			\item[$\bullet$] Si $f^{\prime}$ est strictement positive sur $I$ sauf \'eventuellement en un nombre fini de points alors \vsec\\
			      \hspace*{0cm}\dotfill \vsec
			\item[$\bullet$] Si $f^{\prime}$ est strictement n\'egative sur $I$ sauf \'eventuellement en un nombre fini de points alors \vsec\\
			      \hspace*{0cm}\dotfill \vsec
		\end{itemize}
	\end{theorem}
}\vsec

\noindent \textbf{Application:} \dotfill

\begin{rem}
	\noindent \warning  La r\'eciproque des derni\`eres propri\'et\'es est fausse. Contre-exemple : \dotfill
\end{rem}

{\footnotesize
\begin{exercice}
	\'Etudier les variations des fonctions d\'efinies par : $f(x) = \ln{(e-e^{-\frac{1}{x}})}$, $g(x) = \ddp\frac{\ln{x}}{1+x}$, et $h(x)= x^n\ln{(x)}$.
\end{exercice}}


%-------------------------------------------------------
\subsection{Recherche d'extremum}

\noindent\ {Condition n\'ecessaire d'existence d'extremum local:}\\

{\noindent

\begin{prop}
	Soient $f:\ I\rightarrow \R$ une fonction d\'efinie sur un intervalle $I$ et $x_0\in I$. Si : \vsec
	\begin{itemize}
		\item[$\bullet$] $f$ est dérivable sur I
		\item[$\bullet$] et $f$ admet un extremum en $x_0 \in I$  $x_0$ n'est pas une borne de l'intervalle
	\end{itemize}
	Alors $f'(x_0) =0$
\end{prop}
}

\begin{rems}
	\begin{enumerate}
		\item \noindent \warning  Ce n'est qu'une condition n\'ecessaire d'existence d'extremum:\dotfill \\
		      \dotfill \\
		\item Toutes les hypoth\`eses sont importantes, en particulier ce th\'eor\`eme est faux si $x_0$ est une borne de $I$.\\
		      Exemple: \dotfill
		\item Le th\'eor\`eme nous dit que si une fonction est d\'erivable sur un intervalle ouvert alors \\
		      \dotfill
	\end{enumerate}
\end{rems}

\vspace{0.4cm}
\noindent\ {Condition n\'ecessaire et suffisante d'extremum local:}\\

{\noindent

\begin{prop}
	Soient $f:\ I\rightarrow \R$ une fonction d\'efinie sur un intervalle $I$ et $x_0\in I$. Si : \vsec
	\begin{itemize}
		\item[$\bullet$] \dotfill \vsec
		\item[$\bullet$] \dotfill \vsec
		\item[$\bullet$] \dotfill \vsec
	\end{itemize}
	Alors \dotfill \vsec
\end{prop}
}

{\footnotesize
\begin{exercice}
	Montrer que: $\forall x>0,\ \ln{x}<\sqrt{x}$ et $\forall x\geq 0,\ (1+x)^{\alpha} \geq 1+\alpha x$ avec $\alpha>1$.
\end{exercice}}

%-------------------------------------------------------
\subsection{Th\'eor\`eme de Rolle}

{\noindent

	\begin{theorem}
		Soient $a,b$ deux r\'eels tels que $a<b$ et $f: \lbrack a,b\rbrack\rightarrow \R$. Si : \vsec
		\begin{itemize}
			\item[$\bullet$] \dotfill\vsec
			\item[$\bullet$] \dotfill\vsec
			\item[$\bullet$] \dotfill\vsec
		\end{itemize}
		Alors \dotfill \vsec
	\end{theorem}
}

\begin{rems}
	\noindent \warning  Importance de toutes les hypoth\`eses: si on retire une des hypoth\`eses du th\'eor\`eme de Rolle, celui-ci ne s'applique plus.
	\begin{itemize}
		\item[$\bullet$] Si $f(a)\not= f(b)$:
		      \vspace{2cm}

		\item[$\bullet$] Si on supprime l'hypoth\`ese de d\'erivabilit\'e sur $\rbrack a,b\lbrack$:
		      \vspace{2cm}

		\item[$\bullet$] Si on supprime la continuit\'e sur $\rbrack a,b\lbrack$,
		      \vspace{2cm}

		\item[$\bullet$] Si on supprime la continuit\'e aux extremit\'es, par exemple en $b$:
		      \vspace{2cm}

	\end{itemize}
\end{rems}

\vspace{0.3cm}

\setlength\fboxrule{1pt}
\noindent  {

\textbf{Quand penser au th\'eor\`{e}me de Rolle:}
\begin{itemize}
	%\item[$\bullet$] La d\'eriv\'ee s'annule en un point.
	\item[$\bullet$] Type d'exercices: Exercices plut\^{o}t th\'eoriques lorsque l'on ne conna\^{i}t pas l'expression de la fonction.
	\item[$\bullet$] Y penser d\`es que l'on veut d\'eterminer l'existence de racines pour la d\'eriv\'ee ou les d\'eriv\'ees successives d'un polyn\^ome.
	\item[$\bullet$] Y penser d\`es que l'on parle de valeurs d'annulation pour la d\'eriv\'ee ou les d\'eriv\'ees successives d'une fonction.
\end{itemize}
}
\setlength\fboxrule{0.5pt}

{\footnotesize
	\begin{exercice} Soit $P$ un polyn\^ome ayant deux racines r\'eelles distinctes. Montrer que $P^{\prime}$ admet au moins une racine.
	\end{exercice}}

% 
%-------------------------------------------------------
%-------------------------------------------------------
%-------------------------------------------------------
%-------------------------------------------------------
\subsection{Th\'eor\`{e}me des accroissements finis}

%-------------------------------------------------------
\noindent\ {Th\'eor\`{e}me des accroissements finis}\\

{\noindent

\begin{theorem}
	Soient $a,\ b$ deux r\'eels tels que $a<b$ et $f: \lbrack a,b\rbrack\rightarrow \R$ une fonction. Si :\vsec
	\begin{itemize}
		\item[$\bullet$] \dotfill \vsec
		\item[$\bullet$] \dotfill \vsec
	\end{itemize}
	Alors \dotfill \vsec
\end{theorem}
}

\begin{rem}
	Lorsque $a \not= b$, la conclusion est qu'il existe $c \in \; ]a,b[$ tel que : \dotfill\vsec\\
	Autrement dit, il existe un point de $]a,b[$ o\`u la d\'eriv\'ee est \'egale au taux d'accroissement entre $a$ et $b$.
\end{rem}

\begin{proof}
	$g(x) = f(x) - f(a) - \left(\frac{f(b)-f(a)}{b-a} \right) (x-a)$
\end{proof}

%\begin{exemples}
%Soit $f$ une fonction continue sur $[0,1]$ et d\'erivable sur $]0,1[$, et soit $x \in \; ]0,1[$ fix\'e. Appliquer l'\'egalit\'e des accroissements finis sur les intervalles suivants :
%\begin{itemize}
%\item[$\bullet$] Sur $\lbrack 0,1\rbrack$ :
%\vspace{0.5cm}
%
%\item[$\bullet$] Sur $\lbrack 0,x\rbrack$ :
%\vspace{0.5cm}
%
%\item[$\bullet$] Sur $\lbrack x,1 \rbrack$ :
%\vspace{0.5cm}
%
%\end{itemize}
%\end{exemples}

{\footnotesize
\begin{exercice} Lorsque la d\'eriv\'ee de la fonction est born\'ee, le th\'eor\`eme des accroissements finis permet de montrer des in\'egalit\'es appel\'ees ``in\'egalit\'es des accroissements finis''. On consid\`ere une fonction $f$ continue sur un intervalle $[a,b]$ et d\'erivable sur $]a,b[$.
	\begin{enumerate}
		\item Montrer que s'il existe $(m,M)\in\R^2$ tels que : $\forall x\in \; \rbrack a,b\lbrack,\ m\leq f^{\prime}(x)\leq M$, alors : \; $m(b-a)\leq f(b)-f(a)\leq M(b-a).$
		\item Montrer que s'il existe $K\in\R$ tel que: $\forall x\in\rbrack a,b\lbrack,\  \left| f^{\prime}(x)\right| \leq K$,
		      alors : \; $\left| f(b)-f(a) \right| \leq K |b-a|.$
	\end{enumerate}
\end{exercice}}
\vsec\vsec

%-------------------------------------------------------
\noindent\ {Applications}


\begin{itemize}
	\item[{\large\ding{182}}] \underline{{\large\textbf{Obtenir des in\'egalit\'es:}}}\\


	      \setlength\fboxrule{1pt}
	      \noindent  {

		      \begin{itemize}
			      \item[$\bullet$] On fixe un r\'eel $x$.
			      \item[$\bullet$] On applique le TAF
			            \`a une fonction sur un intervalle de type $\lbrack 0,x\rbrack$, $\lbrack x,x+1\rbrack$, $\lbrack 1,x\rbrack$...
			            %\item[$\bullet$] Parfois, si besoin, on passe \`{a} la valeur absolue.
			      \item[$\bullet$] On encadre alors la d\'eriv\'ee de la fonction afin d'obtenir l'encadrement cherch\'e.
			            %\begin{itemize}
			            %\item[$\star$] Directement
			            %\item[$\star$] En utilisant le fait que $c\in\lbrack 0,x\rbrack$, $\lbrack x,x+1\rbrack$, $\lbrack 1,x\rbrack$...
			            %\end{itemize}
			            %\item[$\bullet$] On conclut.
		      \end{itemize}
	      }
	      \setlength\fboxrule{0.5pt}\vsec

	      {\footnotesize
		      \begin{exercice} Montrer les in\'egalit\'es suivantes:
			      \begin{enumerate}
				      \item Montrer que pour tout $x>0$: $x<e^x-1<xe^x$.
				      \item Montrer que: $\forall x\in\R, \quad |\sin{x}|\leq |x| \quad \hbox{et} \quad |\cos{x}-1|\leq |x|.$
				      \item Montrer que pour tout $x>0$, on a $\ddp\frac{x}{1+x}<\ln{(1+x)}<x$.
			      \end{enumerate}
		      \end{exercice}}
	      \vsec

	\item[{\large{\ding{183}}}]  \underline{{\large\textbf{Obtenir la convergence de suite d\'efinie par r\'ecurrence}}}\\


	      \setlength\fboxrule{1pt}
	      \noindent  {

		      \begin{itemize}
			      \item[$\bullet$] Penser au TAF pour montrer que: $\left| u_{n+1}-\alpha\right|\leq C\left| u_n-\alpha\right|$
			            avec $\alpha$ point fixe de la fonction $f$ associ\'ee \`a la suite, lorsque l'on sait que la d\'eriv\'ee $f^{\prime}$ est born\'ee.
			      \item[$\bullet$] Appliquer alors le TAF \`{a} la fonction $f$ entre $u_n$ et $\alpha$.
		      \end{itemize}
	      }
	      \setlength\fboxrule{0.5pt}
	      \vsec

	      {\footnotesize
		      \begin{exercice}
			      Soit $f:\R\rightarrow \R$ une fonction d\'efinie par $f(x)=\ddp\demi\sin{x}+1$.
			      \begin{enumerate}
				      \item Montrer que $f$ a un unique point fixe que l'on notera $\alpha$.
				      \item Soit la suite $(u_n)_{n\in\N}$ d\'efinie par $u_0=0$ et: $\forall n\in\N,\ u_{n+1}=f(u_n)$. Montrer que: $\forall n\in\N,\ |u_{n+1}-\alpha|\leq \ddp\demi|u_n-\alpha|$.\\
				      \item Conclure quand \`a la convergence de la suite.
			      \end{enumerate}
		      \end{exercice}}
	      \vsec

	      {\footnotesize
		      \begin{exercice}
			      Soit $f: \, \rbrack -2,+\infty\lbrack \rightarrow \R$ une fonction d\'efinie par $f(x)=\ln{(2+x)}$.
			      \begin{enumerate}
				      \item \'Etudier la fonction $f$ et montrer que $f$ a un unique point fixe dans $\lbrack 1,2\rbrack$ que l'on notera $\alpha$.
				      \item Soit la suite $(u_n)_{n\in\N}$ d\'efinie par $u_0=1$ et: $\forall n\in\N,\ u_{n+1}=f(u_n)$.
				      \item Montrer que la suite est bien d\'efinie et que pour tout $n\in\N$: $1\leq u_n\leq 2$.
				      \item Montrer que: $\forall n\in\N,\ |u_{n+1}-\alpha|\leq \ddp\ddp\frac{1}{3} |u_n-\alpha|$ et conclure quant \`a la convergence de la suite.
			      \end{enumerate}
		      \end{exercice}}

\end{itemize}

%-------------------------------------------------------
%-------------------------------------------------------
%-------------------------------------------------------
%\subsection{Th\'eor\`eme de la limite de la d\'eriv\'ee}
%
%\noindent Lorsque la fonction $f$ est d\'efine par des raccords ou un prolongement par continuit\'e ou encore avec des fonctions ayant des racine carr\'ee, valeur absolue, $\arccos{}$, $\arcsin{}$ (dont le domaine de d\'efinition est plus gros que le domaine de d\'erivabilit\'e), il existe alors des points \`{a} probl\`{e}me pour lesquels l'\'etude de la d\'erivabilit\'e ne marche pas avec les th\'eor\`{e}mes g\'en\'eraux (comme somme, produit, compos\'ee, quotient...).\\
%\noindent Il existe alors deux m\'ethodes permettant d'\'etudier la d\'erivabilit\'e en ces points \`{a} probl\`{e}me:
%\begin{itemize}
%\item[$\bullet$] Par la d\'efinition avec le taux d'accroissement.
%\item[$\bullet$] Par le th\'eor\`{e}me de la limite de la d\'eriv\'ee (ci-dessous) qui a l'avantage de donner en plus la continuit\'e de la d\'eriv\'ee.\vsec\vsec
%\end{itemize}
%
%
%\hspace{-0.5cm}  {\noindent  
%
%\begin{theorem}
%Soit $I$ un intervalle de $\R$ et $a\in I$. 
%\begin{enumerate}
%\item SI \vsec
%\begin{itemize}
% \item[$\bullet$] \dotfill \vsec
%\item[$\bullet$]  \dotfill \vsec
%\item[$\bullet$]  \dotfill \vsec
%\end{itemize}
%ALORS
%\begin{itemize}
% \item[$\bullet$] \dotfill \vsec
%\item[$\bullet$]  \dotfill \vsec
%\item[$\bullet$]  \dotfill \vsec
%\end{itemize}
%\item SI \vsec
%\begin{itemize}
% \item[$\bullet$] \dotfill \vsec
%\item[$\bullet$]  \dotfill \vsec
%\item[$\bullet$]  \dotfill \vsec
%\end{itemize}
%ALORS
%\begin{itemize}
% \item[$\bullet$] \dotfill \vsec
%\item[$\bullet$]  \dotfill \vsec
%\item[$\bullet$]  \dotfill \vsec
%\end{itemize}
%\end{enumerate}
%\end{theorem}
% }
%\vsec\vsec
%
%\noindent \warning  Le th\'eor\`{e}me NE DIT RIEN si la limite de $f^{\prime}$ n'existe pas lorsque $x$ tend vers $a$. La fonction peut \^{e}tre ou ne pas \^{e}tre d\'erivable et il faut revenir \`{a} l'\'etude du taux d'accroissement.
%\begin{exemple}
%\'Etudier la d\'erivabilit\'e en 0 de la fonction $f$ d\'efinie sur $\R$ par
%$f(x)=\left\lbrace\begin{array}{ll} x^2\sin{\left( \ddp\frac{1}{x} \right)}& \hbox{si}\ x\not= 0\vsec\\ 0 & \hbox{si}\ x=0.\end{array}\right.$\\
%%\noindent\'Etudier la d\'erivabilit\'e de cette fonction en 0.
%\end{exemple}
%
%{\footnotesize
%\begin{exercice} 
%\begin{enumerate}
%\item \'Etude de la d\'erivabilit\'e en 0 de la fonction $f$ d\'efinie sur $\lbrack 0,1\rbrack$ par: $x\mapsto \cos{(\sqrt{x})}$.
%\item On d\'efinit la fonction $f$ par $f(x)=x^2\ln{(x)}$. \'Etudier un \'eventuel prolongement par continuit\'e de la fonction $f$ en 0. \'Etudier alors la d\'erivabilit\'e de la fonction $f$ prolong\'ee en 0.
%\item On d\'efinit la fonction $g$ par $g(x)=|x|^3$. \'Etudier la d\'erivabilit\'e de la fonction $g$ sur son domaine de d\'efinition.
%\end{enumerate}
%\end{exercice}}


\vspace*{0.5cm}

%-----------------------------------------------------
%-------------------------------------------------------
%-----------------------------------------------------
%-------------------------------------------------------
%-----------------------------------------------------
%-------------------------------------------------------
%-----------------------------------------------------
%-------------------------------------------------------
\section{D\'eriv\'ees d'ordre sup\'erieur}


%-----------------------------------------------------
%-------------------------------------------------------
\subsection{D\'efinitions}

%Soit une fonction $f: I\rightarrow \R$. 
%\begin{itemize}
%\item[$\bullet$] Si $f$ est d\'erivable sur $I$, sa d\'eriv\'ee $f^{\prime}$ peut elle-m\^{e}me \^{e}tre d\'erivable sur $I$. \\
%\noindent La d\'eriv\'ee de $f^{\prime}$ s'appelle alors \dotfill not\'ee \dotfill.
%\item[$\bullet$] Cette fonction peut elle m\^{e}me \^{e}tre d\'erivable sur $I$. \\
%\noindent La d\'eriv\'ee de $f^{(2)}$ s'appelle alors \dotfill not\'ee \dotfill.
%\item[$\bullet$] ainsi de suite
%\end{itemize}\vsec\vsec

{\noindent

	\begin{defi}
		Soit $I$ un intervalle de $\R$ et $f: I\rightarrow \R$.
		\begin{itemize}
			\item[$\bullet$] On note $f^{(0)}=f$ (d\'eriv\'ee d'ordre 0 de $f$).
			      %\item[$\bullet$] Si $f$ est d\'erivable sur $I$, on note $f^{(1)}=f^{\prime}$.
			\item[$\bullet$] Les d\'eriv\'ees successives de $f$ sont alors d\'efinies par r\'ecurrence.\\
			      \noindent La fonction $f$ est $n+1$ fois d\'erivable sur $I$ si \vsec
			      \begin{itemize}
				      \item[$\star$]  \dotfill \vsec
				      \item[$\star$] \dotfill \vsec
			      \end{itemize}
			      Dans ce cas, on note \dotfill \vsec
		\end{itemize}
	\end{defi}
}
\vsec\vsec

\noindent \warning  Ne pas confondre \dotfill \vsec
{\footnotesize
	\begin{exercice}
		Calculer la d\'eriv\'ee premi\`ere, seconde, troisi\`eme et quatri\`eme du cosinus et de l'exponentielle.
	\end{exercice}}\vsec\vsec

{\noindent

	\begin{defi} Fonctions de classe $C^n$ et $C^{\infty}$:
		\begin{itemize}
			\item[$\bullet$]  On dit que $f$ est de classe $\mathcal{C}^n$ sur $I$ si  \vsec
			      \begin{itemize}
				      \item[$\star$] \dotfill\vsec
				      \item[$\star$] \dotfill\vsec
			      \end{itemize}
			      On note alors \dotfill l'ensemble des fonctions de classe $\mathcal{C}^n$ sur $I$.\vsec
			\item[$\bullet$] $\mathcal{C}^0(I)$ est l'ensemble des fonctions \dotfill \vsec
			\item[$\bullet$] $\mathcal{C}^1(I)$ est l'ensemble des fonctions \dotfill \vsec
			\item[$\bullet$] On dit que $f$ est de classe $\mathcal{C}^{\infty}$ sur $I$ \dotfill\vsec\\
			      \noindent  On note \dotfill l'ensemble des fonctions de classe $\mathcal{C}^{\infty}$ sur $I$. \vsec
		\end{itemize}
	\end{defi}
}


%-----------------------------------------------------
%-------------------------------------------------------
\subsection{D\'eriv\'ees successives des fonctions usuelles}

%-------------------------------------------------------
\noindent\ {R\'egularit\'e des fonctions usuelles:}\\

\begin{itemize}
	\item[$\bullet$] Continuit\'e des fonctions usuelles:\vsec\\
	      \phantom{\hspace{0cm}} \dotfill\vsec
	\item[$\bullet$] Caract\`{e}re $C^{\infty}$ des fonctions usuelles:\vsec\\
	      La plupart des fonctions que  nous rencontrerons cette ann\'ee seront de classe $C^{\infty}$. Par exemple:\vsec
	      \begin{itemize}
		      \item[$\star$] \dotfill\vsec
		      \item[$\star$] \dotfill\vsec
		      \item[$\star$] \dotfill\vsec
	      \end{itemize}
\end{itemize}

\vspace{0.5cm}
%-------------------------------------------------------





\begin{dboxminipage}{12cm}
	M\'ethode pour trouver les dérivées successives :
	\begin{itemize}
		\item[$\bullet$] Calcul des premi\`{e}res d\'eriv\'ees.
		\item[$\bullet$] Conjecture de l'expression de $f^{(n)}$ pour tout $n\in\N$.
		\item[$\bullet$] D\'emonstration de la conjecture par r\'ecurrence.
	\end{itemize}
\end{dboxminipage}

\begin{enumerate}
	\item \textbf{D\'eriv\'ees successives de la fonction inverse:}\\

	      \vspace{2cm}

	\item \textbf{D\'eriv\'ees successives de la fonction logarithme n\'ep\'erien:}\\

	      \vspace{2cm}

	\item \textbf{D\'eriv\'ees successives des fonctions sinus et cosinus:}\\

	      \vspace{2cm}

\end{enumerate}



\begin{dboxminipage}{15cm}
	Expressions des d\'eriv\'ees successives des fonctions usuelles:
	\begin{itemize}
		\item[$\bullet$] Fonction exponentielle: $f\in C^{\infty}(\R)$ et $\forall n\in\N,\ \forall x\in\R,$
		      $$f^{(n)}(x)=e^x.$$
		\item[$\bullet$] Fonction $\ln{}$: $f\in C^{\infty}(\R^{+\star})$ et $\forall n\in\N^{\star},\ \forall x>0,$ $$ f^{(n)}(x)=\ddp\frac{(-1)^{n-1} (n-1)!}{x^n}.$$
		\item[$\bullet$] Fonction polynomiale: cf cours polyn\^{o}me.
		\item[$\bullet$] Fonction inverse: $f\in C^{\infty}(\R^{\star})$ et $\forall n\in\N,\ \forall x\in\R^{\star},$ $$ f^{(n)}(x)=\ddp\frac{(-1)^{n} n!}{x^{n+1}}.$$
		\item[$\bullet$] Fonction $x\mapsto x^{\alpha},\ \alpha\in\R$: $f\in C^{\infty}(\R^{+\star})$ et $\forall n\in\N,\ \forall x>0,$  $$f^{(n)}(x)=\alpha(\alpha-1)(\alpha-2)\dots (\alpha-n+1)x^{\alpha-n}.$$
		\item[$\bullet$] Fonction cosinus: $f\in C^{\infty}(\R)$ et $\forall n\in\N,\ \forall x\in\R,$ $$f^{(n)}(x)=\cos{\left( x+n\ddp\frac{\pi}{2}\right)}.$$
		\item[$\bullet$] Fonction sinus: $f\in C^{\infty}(\R)$ et $\forall n\in\N,\ \forall x\in\R,$ $$f^{(n)}(x)=\sin{\left( x+n\ddp\frac{\pi}{2}\right)}.$$
	\end{itemize}
\end{dboxminipage}


{\footnotesize
\begin{exercice}
	Calculer les d\'eriv\'ees successives des fonctions suivantes: $x\mapsto e^x$ puis $x\mapsto e^{ax}$, $a\in\R^{\star}$, $x\mapsto |x|$, $x\mapsto x^n$, $n\in\N^{\star}$, $x\mapsto x^{\alpha}$, $\alpha\in\R\setminus\Z$ et $x\mapsto \ddp\frac{1}{a-x}$, $a\in\R$.
\end{exercice}}\vsec\vsec
%-----------------------------------------------------
%-------------------------------------------------------
\subsection{Op\'erations et d\'eriv\'ees successives}

%-------------------------------------------------------
\subsubsection{D\'eriv\'ees successives d'une somme}

{\noindent

	\begin{prop} Somme et multiplication par un r\'eel :
		\begin{itemize}
			\item[$\bullet$] Si $f$ et $g$ sont deux fonctions de classe $C^n$ alors \dotfill \vsec\\
			      Et de plus $(f+g)^{(n)}=$\dotfill \vsec
			\item[$\bullet$] Si $f$ est une fonctions de classe $C^n$ et $\lambda\in\R$ alors \dotfill \vsec\\
			      Et de plus $(\lambda f)^{(n)}=$\dotfill \vsec
		\end{itemize}
	\end{prop}
}


{\footnotesize
	\begin{exercice}
		Calculer les d\'eriv\'ees successives de la fonction suivante: $f: x\mapsto x^5-\ln{(x-2)}+5e^{-x}$.
	\end{exercice}}\vsec\vsec

%-------------------------------------------------------
\subsubsection{D\'eriv\'ees successives d'un produit:}

{\noindent

	\begin{prop} Produit de deux fonctions:\\
		Si $f$ et $g$ sont deux fonctions de classe $C^n$ alors \dotfill \vsec
		%\begin{itemize}
		%\item[$\bullet$] Si $f$ et $g$ sont deux fonctions de classe $C^n$ alors \dotfill \vsec
		%\item[$\bullet$] $(fg)^{(n)}=$\dotfill \vsec
		%\end{itemize}
	\end{prop}
}

{\footnotesize
	\begin{exercice}
		La formule de la d\'eriv\'ee d'un produit, appel\'ee ``formule de Leibniz'' n'est pas au programme. Il faut savoir la d\'emontrer pour pouvoir l'utiliser.
		\begin{enumerate}
			\item Montrer que si $f$ et $g$ sont de classe $C^n$, alors $(fg)^{(n)} = \ddp \sum_{k=0}^n \binom{n}{k} f^{(k)} g^{(n-k)}.$
			\item En d\'eduire les d\'eriv\'ees successives des fonctions suivantes: $f: x\mapsto x^3e^{-4x}$, $g: x\mapsto x^4\sin{(x)}$ et $h: x\mapsto \ddp\frac{x^2}{4-x}$.
		\end{enumerate}
	\end{exercice}}\vsec\vsec


%-------------------------------------------------------
\subsubsection{D\'eriv\'ees successives d'un quotient:}

{\noindent

	\begin{prop} Quotient de deux fonctions:\\
		\noindent Si $f$ et $g$ sont deux fonctions de classe $C^n$ sur $I$ et $g$ ne s'annule pas sur $I$ alors \dotfill \vsec
	\end{prop}
}


\begin{rem}
	Si l'on conna\^it les d\'eriv\'ees successives de $\ddp \frac{1}{g}$, on peut utiliser la formule de Leibniz pour calculer les d\'eriv\'ees successives de $\ddp \frac{f}{g}$.
\end{rem}

{\footnotesize
\begin{exercice}
	\'Etudier la r\'egularit\'e de la fonction tangente.
\end{exercice}}\vsec\vsec\vsec


%-------------------------------------------------------
\subsubsection{D\'eriv\'ees successives d'une compos\'ee:}

{\noindent

	\begin{prop} Compos\'ee de deux fonctions:\\
		\noindent Si $f$ et $g$ est de classe $C^n$ sur I, $g$ est de classe $C^n$ sur J avec \dotfill  alors \dotfill \vsec
	\end{prop}
}

\begin{rem}
	Le calcul des d\'eriv\'ees successives d'une compos\'ee utilise g\'en\'eralement une r\'ecurrence.
\end{rem}

{\footnotesize
\begin{exercice}
	On consid\`ere la fonction $f$ d\'efinie sur $\R$ par $f(x) = \left\{ \begin{array}{cl}
			e^{-\frac{1}{x^2}} & \textmd{ si } x \not =0\vsec \\
			0                  & \textmd{ si } x=0
		\end{array}\right.$
	\begin{enumerate}
		\item Montrer que $f$ est continue sur $\R$.
		\item Montrer que $f$ est de classe $\mathcal{C}^{\infty}$ sur $\R^{\star}$.
		\item Montrer que pour tout $n\in\N$ il existe un polyn\^ome $P_n$ tel que : $ \forall x>0, \, \ddp f^{(n)}(x)=P_n\left(\frac{1}{x}\right) f(x).$
	\end{enumerate}
\end{exercice}}\vsec\vsec


%-------------------------------------------------------
\subsubsection{D\'eriv\'ees successives d'une fonction r\'eciproque:}

\noindent Soit $f: I\rightarrow \R$ une fonction bijective de $I$ sur $J=f(I)$. Ainsi $f^{-1}: J\rightarrow I$ existe et on cherche \`{a} conna\^{i}tre sa r\'egularit\'e sur $J$.\vsec


{\noindent

	\begin{prop} Fonction r\'eciproque:\\
		\noindent Si :
		\begin{itemize}
			\item[$\bullet$] \dotfill \vsec
			\item[$\bullet$] \dotfill \vsec
		\end{itemize}
		Alors \dotfill \vsec
	\end{prop}
}
\vsec\vsec

%{\footnotesize
%\begin{exercice} 
%\'Etudier la r\'egularit\'e de la fonction  racine cubique puis la r\'egularit\'e de la r\'eciproque de la fonction $f: x\mapsto x^2-x\ln{x}-1$ (apr\`{e}s avoir justifi\'e son existence).
%\end{exercice}}\vsec\vsec








\end{document}