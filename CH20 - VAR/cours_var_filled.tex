\documentclass[a4paper, 11pt]{article}
\input{macro/package.tex}
\input{macro/environement}
% Header et footer

\pagestyle{fancy}
\fancyhead{}
\fancyfoot{}
\renewcommand{\headwidth}{\textwidth}
\renewcommand{\footrulewidth}{0.4pt}
\renewcommand{\headrulewidth}{0pt}
\renewcommand{\footruleskip}{5px}

\fancyfoot[R]{Olivier Glorieux}
%\fancyfoot[R]{Jules Glorieux}

\fancyfoot[C]{ Page \thepage }
\fancyfoot[L]{1BIOA - Lycée Chaptal}
%\fancyfoot[L]{MP*-Lycée Chaptal}
%\fancyfoot[L]{Famille Lapin}

\input{macro/newcommand.tex}
\geometry{hmargin=2.0cm, vmargin=2.5cm}




\begin{document}
\tableofcontents
 \title{Chapitre 22  : Variables Aléatoires Réelles} 



 Dans tout ce chapitre, on se place dans un espace probabilis\'e fini $(\Omega, \mathcal{P}(\Omega),P)$.

%\vspace{0.4cm}
%--------------------------------------------------
%------------------------------------------------
%-----------------------------------------------------------
%----------------------------------------------------------
%-----------------------------------------------------------
%----------------------------------------------------------
\section{G\'en\'eralit\'es sur les variables al\'eatoires r\'eelles finies}

%--------------------------------------------------
%------------------------------------------------
%-----------------------------------------------------------
\subsection*{D\'efinition et notations}

 {  

\begin{defi} On appelle variable al\'eatoire r\'eelle finie sur $\Omega$, not\'ee varf, toute application de $\Omega$ dans $\R$. 


\end{defi}
 
}

\begin{exemple}
Pour chacune des exp\'eriences al\'eatoires suivantes, donner l'univers $\Omega$ et un exemple de varf sur $\Omega$.
\begin{enumerate}
\item On lance une pi\`ece de monnaie deux fois de suite, et \`a chaque lancer on gagne un euro si on obtient pile, on perd $2$ euros si on obtient face.

$$G \left|\begin{array}{ccc}
 \intent{P,F}& \tv &\R \\
 P&\mapsto & 1\\
 F&\mapsto & -2\\
\end{array}\right.$$

\item On lance deux d\'es non truqu\'es de couleurs diff\'erentes.
$$S \left|\begin{array}{ccc}
 \intent{1,6}^2& \tv &\R \\
 (i,j)&\mapsto &i+j
\end{array}\right.$$

\end{enumerate}
\end{exemple}



%--------------------------------------------------
%------------------------------------------------
%-----------------------------------------------------------


 {  

\begin{defi} On appelle univers image d'une varf, not\'e  $X(\Omega)$ l'ensemble des images de l'application $\Omega$. 

$$X(\Omega) = \{X(\omega) \, |\, \omega \in \Omega\}.$$ 
\end{defi}
 
}

\begin{exemple}
Donner les univers images pour les $2$ exemples pr\'ec\'edents.

\end{exemple}

\begin{rem}
En BCPST1, on ne travaille que sur des univers $\Omega$ finis. 

\end{rem}




%--------------------------------------------------
%------------------------------------------------
%-----------------------------------------------------------


%\paragraph{Notations des \'ev\`{e}nements associ\'es \`{a} une varf}

 \textbf{Notations}: Pour tout $a\in\R$, on note:\vsec
\begin{itemize}
\item[$\bullet$] $\lbrack X=a\rbrack=X^{-1}(\{a\})=\{ \omega \in \Omega \, |\, X(\omega) = a\}$
\item[$\bullet$]  $\lbrack X\leq a\rbrack=\{ \omega \in \Omega \, |\, X(\omega) \leq a\}$
\item[$\bullet$]  $\lbrack X\geq a\rbrack=\{ \omega \in \Omega \, |\, X(\omega) \geq  a\}$
\item[$\bullet$]  $\lbrack X< a\rbrack=\{ \omega \in \Omega \, |\, X(\omega) < a\}$
\item[$\bullet$]  $\lbrack X> a\rbrack=\{ \omega \in \Omega \, |\, X(\omega) > a\}$
\end{itemize}
Soit $A\subset \Omega$ on note :
\begin{itemize}
\item[$\bullet$] $\lbrack X\in A \rbrack=X^{-1}(A)=\{ \omega \in \Omega \, |\, X(\omega) \in A\}$
\end{itemize}

Comme ce sont des \'ev\'enements, on peut calculer leur probabilit\'e.

\begin{exemple} 
On reprend le deuxi\`eme exemple. Calculer $P(\lbrack Y=5\rbrack)$, $P(\lbrack Y \geq 10 \rbrack)$, $P(\lbrack 1\leq Y< 5\rbrack)$, $P(\lbrack Y>12\rbrack)$.
\end{exemple}

{\footnotesize \begin{exercice} Soit $X$ une varf et $a\in\R$. Exprimer $P(\lbrack X \leq a\rbrack)$ en fonction de  $P(\lbrack X\geq a\rbrack)$.
\end{exercice}}




%\subsection{Sce associ\'e \`{a} une var}

 {  

\begin{prop} Syst\`{e}me complet d'\'ev\`{e}nements associ\'e \`{a} une varf
\begin{itemize}
\item[$\bullet$] Si $X$ est une varf avec $X(\Omega)=\lbrace x_1,\dots, x_n\rbrace$.
 Alors $\ddp \bigcup_{i=1}^n [X=x_i]$ est un sce.
\item[$\bullet$] En particulier, on a
$$P(X\in A)=\sum_{x\in A} P([X=a])$$


\end{itemize}
\end{prop}
 
}

\begin{exemple} 
Donner le sce associ\'e dans les exemples du d\'ebut.

\end{exemple}

%--------------------------------------------------
%------------------------------------------------
%-----------------------------------------------------------
\subsection{Loi d'une variable al\'eatoire r\'eelle finie}

%\subsection{D\'efinition de la loi d'une var}

 {  

\begin{defi} Soit $X$ une variable al\'eatoire r\'eelle.
 On appelle loi de probabilit\'e de $X$ not\'ee $f_X$ l'application
$$f_X \left|\begin{array}{ccc}
 X(\Omega) & \tv &[0,1] \\
x &\mapsto &P([X=x])
\end{array}\right.$$

\end{defi}
 
}
\vsec\vsec

\setlength\fboxrule{1pt}
  {

D\'eterminer la loi de probabilit\'e de $X$:
\begin{itemize}
\item[$\bullet$] Donner l'univers image $X(\Omega)=\lbrace x_1,x_2,x_3,\dots,x_n\rbrace$.
\item[$\bullet$] Pour chaque \'el\'ement $x$ de $X(\Omega)$, calculer sa probabilit\'e $P(X=x)$:

$$\hbox{Calculs de }\ P(X=x_1),\ P(X=x_2),\ P(X=x_3),\ \dots,\ P(X=x_n).$$
\end{itemize}
 }
\setlength\fboxrule{0.5pt}

\begin{exemple}  
Donner les lois des varf des exemples du d\'ebut.
\end{exemple} 


 

%\subsection{Repr\'esentation graphique d'une loi}


 Deux repr\'esentations sont utilis\'ees pour d\'efinir la loi d'une var:
\begin{itemize}
\item[$\bullet$] \textbf{Sous la forme d'un tableau}


\item[$\bullet$] \textbf{Sous la forme d'un diagramme en b\^{a}tons}


\end{itemize}

\begin{exemple}  
Donner les deux repr\'esentations graphiques possibles pour les exemples du d\'ebut.
\end{exemple} 

%--------------------------------------------------
%------------------------------------------------
%-----------------------------------------------------------
\subsection{Fonction de r\'epartition d'une variable al\'eatoire r\'eelle finie}

%\subsection{D\'efinition}

 

\begin{defi} Soit $X$ une variable al\'eatoire r\'eelle.
 On appelle fonction de r\'epartition de $X$ not\'ee $F_X$ l'application
 
$$F_X \left|\begin{array}{ccc}
 X(\Omega) & \tv &[0,1] \\
x &\mapsto &P([X\leq x])
\end{array}\right.$$
 
\end{defi}
 


\begin{exemple} 
Calculer la fonction de r\'epartition $F_X$ pour les $2$ exemples du d\'ebut. Faire de plus \`{a} chaque fois la repr\'esentation graphique de $F_X$.
\end{exemple} 
\begin{center}
\includegraphics[scale=0.4]{fonction_repartition.png}

\end{center}





%\subsection{Propri\'et\'es}

 
\begin{prop}
Soit $X$ une variable aléatoire et $F_X$ sa fonction de répartition alors :
\begin{enumerate}
\item $F$ est croissante. 
\item $F$ est continue à droite en tout point de $\R$. 
\item $\ddp \lim_{x\tv -\infty} F(x) = 0$  et $\ddp \lim_{x\tv +\infty} F(x) = 1$
\end{enumerate}
De plus on a 
$$\forall x\in \R, \, F(x) =\sum_{z \leq x, z\in X(\Omega)} P(X=z)$$
\end{prop}



\paragraph{Lien entre la fonction de r\'epartition et la loi}

 D'apr\`{e}s la propri\'et\'e ci-dessus, on remarque que si l'on conna\^it la loi de $X$, on peut calculer la fonction de r\'epartition de $X$. Mais la r\'eciproque est vraie aussi. Si on conna\^it la fonction de r\'epartition de $X$, on peut en d\'eduire la loi de $X$. Cela peut \^etre utile lorsqu'il est plus facile de d\'eterminer la fonction de r\'epartition que la loi. \vsec\vsec

   

\begin{prop} 
Soit $X$ une varf avec $X(\Omega)=\lbrace x_1,\dots, x_n\rbrace$ tels que $x_1<x_2<\dots<x_n$. Alors on a:\vsec
 
 $$P(X=x_i) = F(x_i) -F(x_{i-1})$$

\end{prop}
 


\begin{proof}

\end{proof}

{\footnotesize \begin{exercice} Un joueur pr\'el\`eve successivement et avec remise $n$ boules dans une urne contenant $N$ boules num\'erot\'ees de 1 \`a $N$. On consid\`ere les varf $X$ et $Y$ \'egales respectivement au plus grand et au plus petit num\'ero des $n$ boules tir\'ees. Donner les univers images de $X$ et de $Y$. Calculer la fonction de r\'epartition de $X$. En d\'eduire la loi de $X$. Calculer ensuite pour tout $k\in Y(\Omega)$: $P(\lbrack Y>k\rbrack)$. En d\'eduire la loi de $Y$.
\end{exercice}}\vsec\vsec



Penser \`{a} passer par la fonction de r\'epartition pour obtenir la loi de var d\'efinie avec des min ou des max.


%--------------------------------------------------
%------------------------------------------------
%-----------------------------------------------------------
\subsection{Compos\'ee d'une variable al\'eatoire r\'eelle finie: $Y=g(X)$}


 Si $X$ est une variable al\'eatoire finie et $g$ une fonction num\'erique dont l'ensemble de d\'efinition contient $X(\Omega)$, on peut \'etudier la variable al\'eatoire $Y=g(X)$.\vsec\vsec 

\begin{dboxminipage}{0.9 \textwidth}
M\'ethode pour \'etudier les var de type $Y=g(X)$:
\begin{itemize}
\item[$\bullet$] Calcul de l'univers image $Y(\Omega)$:
 Si $X(\Omega)=\lbrace x_1,x_2,x_3,\dots,x_n\rbrace$ alors $Y(\Omega)$ s'obtient en calculant $g(x_1), g(x_2), g(x_3),\dots, g(x_n)$.
\item[$\bullet$] Calcul de la loi de $Y=g(X)$:
 Pour tout $y\in Y(\Omega)$, on \'ecrit que: $P(\lbrack Y=y\rbrack)=P(\lbrack g(X)=y\rbrack)$, et on se ram\`{e}ne alors \`{a} $X$.
\end{itemize}
\end{dboxminipage}

%
%\begin{exercice} On suppose que $X$ est une varf qui suit la loi d\'efinie par
%$$\begin{array}{|c|c|c|c|c|c|}
%\hline
%\rule[-2mm]{0pt}{7mm}  x& -2&-1&0&1&2
%\hline
%\rule[-3mm]{0pt}{8mm}  P(\lbrack X=x\rbrack) & \ddp\frac{1}{10} & \ddp\frac{3}{10} & \ddp\frac{1}{5}&\ddp\frac{1}{3}&\ddp\frac{1}{15}
%\hline
%\end{array}$$
%On note $Y=|X|$, $Z=X^2$ et $T=X(X+1)(X+2)$. Calculer les lois de $Y,\ Z$ et $T$. 
%
%\end{exercice}

\begin{exercice} 
Reprendre l'exemple 1 du d\'ebut et \'etudier $U=Z^2-Z-2$.
\end{exercice}


%-----------------------------------------------------------
%----------------------------------------------------------
%-----------------------------------------------------------
%----------------------------------------------------------



\section{Moments d'une variable al\'eatoire r\'eelle finie}

%--------------------------------------------------
%------------------------------------------------
%-----------------------------------------------------------
\subsection{Esp\'erance d'une variable al\'eatoire r\'eelle finie}

 On s'int\'eresse ici \`a la moyenne d'un ph\'enom\`ene al\'eatoire. Mais la moyenne au sens arithm\'etique semble sans int\'er\^et car certaines valeurs ont une probabilit\'e plus forte que d'autres d'arriver. On va ainsi s'int\'eresser \`a une moyenne pond\'er\'ee par la probabilit\'e qu'a la valeur d'arriver.


%\subsection{D\'efinition de l'esp\'erance}

 {  

\begin{defi} 
Soit $X$ une varf avec $X(\Omega)=\lbrace x_1,\dots, x_n\rbrace$. \vsec
\begin{itemize}
\item[$\bullet$] On appelle esp\'erance de $X$ not\'ee $E(X)$  le nombre r\'eel:
$$E(X) =\sum_{i=1}^n x_i P(X=x_i)$$
\item[$\bullet$] Une variable al\'eatoire d'esp\'erance nulle est dite centrée.
\end{itemize}
\end{defi}
 
}


\begin{exemples}
\begin{itemize}
\item[$\bullet$] Si $X(\Omega)=\intent{0,n}$, $n\in\N$ fix\'e alors $E(X)=\sum_{i=1}^n i P(X=i)$
\item[$\bullet$] Si $X(\Omega)=\lbrace -3,-1,1,2\rbrace$ alors $E(X)=-3 P(X=-3) -P(X=-1) +P(X=1)+2P(X=2)$
\item[$\bullet$] Si $X$ est une varf constante \'egale \`{a} $a$: $X=a$, alors $E(X)=a$
\end{itemize}


\begin{itemize}
\item[$\bullet$] Si $X(\Omega)=\lbrace 0,1\rbrace$ alors $$E(X)= $$
\item[$\bullet$] Si $X(\Omega)=\intent{0,n}$, $n\in\N$ fix\'e alors $$E(X)= $$
\item[$\bullet$] Si $X$ est une varf constante \'egale \`{a} $a$: $X=a$, alors $$E(X)= $$
\end{itemize}
\end{exemples}




\begin{rem}
 L'esp\'erance est ainsi la moyenne de chacune des valeurs prises par la variable al\'eatoire $X$ pond\'er\'ee par la probabilit\'e que $X$ prenne cette valeur.
\end{rem}

\begin{exemple}
Calculer l'esp\'erance dans chacun des exemples du d\'ebut.

\end{exemple}




\paragraph{Th\'eor\`{e}me de transfert}

 

\begin{theorem} 
Soit $X$ une var avec $X(\Omega)=\lbrace x_1,\dots, x_n\rbrace$, soit $g: \R\rightarrow \R$ une fonction num\'erique et $Y=g(X)$. Alors, on a
$$E(Y)= \sum_{i=1}^n g(x_i) P(X=x_i)$$

\end{theorem}
 


\begin{exemples}
\begin{itemize}
\item[$\bullet$] Si $X(\Omega)=\intent{ 0,n}$, $n\in\N$ fix\'e alors $E(X^2)= \sum_{i=0}^n i^2 P(X=i)$
%\item[$\bullet$] Si $X(\Omega)=\lbrace -3,-1,1,2\rbrace$ alors $E(e^X)=$\dotfill  \vsec
\item[$\bullet$] Si $X(\Omega)=\intent{ 1,6}$ alors $E(\ln{(X)})= \sum_{i=1}^6 \ln(i) P(X=i) $
\item[$\bullet$] Si $X(\Omega)=\intent{ 1,n}$, $n\in\N$ fix\'e alors $E(X^3)= \sum_{i=1}^n i^3 P(X=i)$
\end{itemize}
\end{exemples}

{\footnotesize \begin{exercice} 
Reprendre l'exercice 3 et calculer l'esp\'erance des varf $X,Y,Z$ et $T$.
\end{exercice}}\vsec\vsec



Penser au th\'eor\`{e}me de transfert pour calculer l'esp\'erance de var de type $Y=g(X)$ avec $X$ connu.


\begin{defi} 
Soit $X$ une varf avec $X(\Omega)=\lbrace x_1,\dots, x_n\rbrace$. \vsec
 Pour tout $k\in\N^{\star}$, on appelle moment d'ordre $k$ de $X$, not\'e parfois $m_k(X)$, le nombre r\'eel: 

 $$m_k(X)=E(X^k)= \sum_{x_i\in X(\Omega)} x_i^k P(X=x_i)$$
 

\end{defi}
 


{\footnotesize \begin{exercice} 
Calculer les moments d'ordre 0, 1, 2 et 3 de la var $X$ dont la loi est d\'efinie dans l'exercice 3.
\end{exercice}}\vsec\vsec






\paragraph{Lin\'earit\'e de l'esp\'erance}


 {  

\begin{prop} Lin\'earit\'e de l'esp\'erance:
\begin{itemize}
\item[$\bullet$] Soit $X$ une var et $(a,b)\in\R^2$. On a:

$$E(aX+b) =aE(X)+b$$
\item[$\bullet$] Soient $X_1,X_2,\dots,X_n$ des var et $(a_1,a_2,\dots,a_n)\in\R^n$. On a:
$$E(\sum_{i=1}^n a_i X_i) = \sum_{i=1}^n a_iE(X_i) $$

\end{itemize}
\end{prop}
 
}

\begin{proof}
D\'emontrons que $E(aX+b)=aE(X)+b$.


\end{proof}


{\footnotesize \begin{exercice} On consid\`ere $r$ boules num\'erot\'ees de 1 \`a $r$ et $n$ tiroirs num\'erot\'es de 1 \`a $n$. On place au hasard chacune des $r$ boules dans l'un des $n$ tiroirs, chaque tiroir pouvant contenir 0, 1 ou plusieurs boules. On note $V$ la var \'egale au nombre de tiroirs rest\'es vides. D\'eterminer l'esp\'erance de $V$.
\end{exercice}}



%--------------------------------------------------
%------------------------------------------------
%-----------------------------------------------------------
\subsection{Variance, \'ecart-type d'une variable al\'eatoire r\'eelle finie}


%\subsection{D\'efinition de la variance}

 {  

\begin{defi} 
Soit $X$ une variable al\'eatoire finie. 
 On appelle variance de $X$, not\'ee $V(X)$, le nombre r\'eel d\'efini par
$$V(X) = E( (X-E(X))^2)$$


\end{defi}
 
}

\begin{rem}
La variance est la moyenne du carr\'e de la distance entre les valeurs de $X$ et de sa moyenne $E(X)$. Ainsi, la variance mesure l'\'ecart entre $X$ et sa moyenne c'est-\`a-dire la dispersion de la var $X$.
\end{rem}


%
 
\paragraph{Formule de Koenig-Huygens}

 En pratique, on n'utilise quasiment jamais la d\'efinition de la variance pour calculer la variance car cela entraine des calculs compliqu\'es. Ainsi, dans la quasi-totalit\'e des cas, on utilise la formule de Koenig-Huygens donn\'ee ci-dessous pour calculer la variance.\vsec\vsec


 {  

\begin{prop} 
Soit $X$ une variable al\'eatoire finie. 

$$V(X) = E(X^2) - E(X)^2$$

\end{prop}
 
}



\begin{proof}


\end{proof}

\begin{exemple}
Calculer la variance dans chacun des $2$ exemples du d\'ebut.
\end{exemple}\vsec\vsec

\setlength\fboxrule{1pt}
  {

M\'ethode pour calculer la variance: utiliser la formule de Koenig-Huygens:
\begin{itemize}
\item[$\bullet$] \'Etape 1: calculer l'esp\'erance $E(X)$.
\item[$\bullet$] \'Etape 2: calculer en utilisant le th\'eor\`{e}me de transfert $E(X^2)$.
\item[$\bullet$] \'Etape 3: conclure en utilisant la formule de Koenig-Huygens: $V(X)=E(X^2)-E(X)^2$.
\end{itemize}
 }
\setlength\fboxrule{0.5pt}



%\paragraph{Propri\'et\'es de la variance}

 {  

\begin{prop} 
Soit $X$ une variable al\'eatoire finie et $(a,b)\in\R^2$. 
\begin{itemize}
\item[$\bullet$] $V(X) \geq 0$
\item[$\bullet$] $V(aX+b) = a^2 V(X)$
\item[$\bullet$] En particulier on a: $V(X+b)=V(X)$
\end{itemize}
\end{prop}
 
}




\

\begin{defi} 
Soit $X$ une variable al\'eatoire finie.
 On appelle \'ecart-type de $X$ le nombre r\'eel not\'e $\sigma$ le nombre 
 $$\sigma=\sigma(X) =\sqrt{V(X)}.$$


\end{defi}
 

\begin{exemple}
\begin{itemize}
\item[$\bullet$] Une var d'esp\'erance nulle et d'\'ecart-type (ou de variance) \'egal \`{a} 1 est dite réduite
\item[$\bullet$] Si $X$ est une var d'esp\'erance $m$ et d'\'ecart-type $\sigma$ alors $Y=\frac{X-m}{\sigma}$ est une var centr\'ee r\'eduite.
En effet :


\end{itemize}
\end{exemple}



\subsection{In\'egalit\'e de Bienaym\'e-Tchebychev}

 Cette in\'egalit\'e permet de donner une estimation de l'\'ecart de la variable al\'eatoire avec son esp\'erance.\vsec

 {  

\begin{prop} 
In\'egalit\'e de Bienaym\'e-Tchebychev. 
Soit $X$ une variable al\'eatoire finie, et $\varepsilon >0$. On a alors :
$$P(|X-E(X)| \geq \varepsilon) \; \leq \;  \frac{V(X)}{\varepsilon^2} $$
\end{prop}
 
}

\begin{rem}
Intuivement, on obtient que la probabilit\'e que l'erreur entre $X$ et son esp\'erance soit sup\'erieur \`a $\varepsilon$ est major\'ee par $\ddp \frac{V(X)}{\varepsilon^2}$. On remarque que :
\begin{itemize}
\item[$\bullet$] pour $\varepsilon$ grand (un grand \'ecart avec l'esp\'erance), la probabilit\'e est  très petite. 
\item[$\bullet$] pour $\varepsilon$ petit (un petit \'ecart avec l'esp\'erance), l'in\'egalit\'e peut ne pas \^etre utile, en particulier si $\ddp \frac{V(X)}{\varepsilon^2}>1$.
\end{itemize}
\end{rem}

{\footnotesize \begin{exercice} 
On lance $n$ fois de suite un d\'e \'equilibr\'e.
\begin{enumerate}
\item Soit $X$ le nombre d'apparition du nombre $6$. Quelle loi suit $X$ ? Donner son esp\'erance et sa variance.
\item Soit $Y$ la fr\'equence d'apparition du nombre $6$. Exprimer $Y$ en fonction de $X$, et donner son esp\'erance et sa variance.
\item Soit $p_n$ la probabilit\'e que $Y$ soit proche de $\ddp \frac{1}{6}$ \`a $0.1$ pr\`es. Combien de lancers doit-on effectuer pour que $p_n$ soit sup\'erieur \`a $0.9$ ?
\end{enumerate}
\end{exercice}}








%------------------------------------------------
%-----------------------------------------------------------
%----------------------------------------------------------
%-----------------------------------------------------------
%----------------------------------------------------------
\section{Lois usuelles}
%--------------------------------------------------
%------------------------------------------------


%--------------------------------------------------
%------------------------------------------------
%-----------------------------------------------------------
\subsection{Loi uniforme}



\paragraph{Mod\'elisation type}

\begin{itemize}
\item[$\bullet$] Lancer d'un dé équilibré à $6$ faces. On note $X$ la variable aléatoire correspondant au numéro obtenu. On obtient 
$$\forall k\in \intent{1,6},\,  P(X=k ) =\frac{1}{6}$$
\item[$\bullet$] Autres exemples:
\begin{itemize}
\item[$\star$]  Tirer une boule dans une urne contenant $N$  boules numérotées de $1$ à $N$. On note $X$ la variable aléatoire correspondant au numéro obtenu. On obtient 
$$\forall k\in \intent{1,N},\,  P(X=k ) =\frac{1}{N}$$
\end{itemize}
\end{itemize}

\begin{dboxminipage}{0.7 \textwidth}
\textbf{\large{Sch\'ema uniforme:}}\\
 Exp\'eriences dont toutes les issues sont \'equiprobables.
\end{dboxminipage}




\hspace{-0.7cm} {  

\begin{defi} Soit $X$ une var.
\begin{itemize}
\item[$\bullet$] Soit $X(\Omega)=\lbrace x_1,x_2,\dots,x_n\rbrace$:
 On dit que $X$ suit la loi uniforme sur $X(\Omega)$ si
 $$\forall  k\in \intent{1,n},\,  P(X=x_k ) =\frac{1}{n}$$


 On note alors $X\sim U(n)$
\item[$\bullet$] Cas particulier o\`{u} $X(\Omega)=\intent{ 1,n}$:
 $$\forall  k\in \intent{1,n},\,  P(X=k ) =\frac{1}{n}$$
 On note alors $X\sim U(\intent{1,n})$
\end{itemize}
\end{defi}
 
}
 

\textbf{Repr\'esentation graphique }
\begin{center}
\includegraphics[scale=0.4]{loiunif.png}
\end{center}

 










\begin{prop} Soit $X$ une var.
\begin{itemize}
\item[$\bullet$] Si $X\hookrightarrow \mathcal{U}(\intent{1,n})$, $n\in\N^{\star}$, alors:
$$E(X) = \frac{n+1}{2} \quad V(X) = $$

\item[$\bullet$] Si $X\hookrightarrow \mathcal{U}(X(\Omega))$ avec $X(\Omega)=\intent{ a,b}$ avec $a<b$, alors:
$$E(X) =\frac{b+a}{2} $$ 

\end{itemize}
\end{prop}
 

\begin{proof}


\end{proof}

 

%--------------------------------------------------
%------------------------------------------------
%-----------------------------------------------------------
\subsection{Loi de Bernoulli}



\paragraph{Mod\'elisation type}

 Mod\'elisation type: 
 
Une pièce truquée possède $p$ chances de tomber sur pile ( succés) et $q=1-p$ chances de tomber sur face (échec). On définit la VARF X qui teste si pile est sortie : $X$ vaut $1$ si on obtient pile et $0$ sinon. 
$$P(X=1) =p \quad P(X=0) = 1-p$$


\begin{dboxminipage}{0.7 \textwidth}
\textbf{\large{Sch\'ema de Bernoulli:}}\\
 Exp\'eriences n'ayant que 2 r\'esultats possibles: succ\`{e}s ou \'echec.
\end{dboxminipage}



\paragraph{Loi}


\hspace{-0.7cm}  {  

\begin{defi} Soit $X$ une var et $p\in\lbrack 0,1\rbrack$. On dit que $X$ suit une loi de Bernoulli de param\`{e}tre $p$ si
$$P(X=1) = p \quad P(X=0) =1-p$$
 On note alors $X \sim  \cB(p)$
\end{defi}
 
}


\begin{exemples}
\begin{itemize}
\item[$\bullet$] Soit $A\subset \Omega$ alors la loi $1_A$ suit une loi de Bernouilli de paramètre $p=P(A)$. 
\end{itemize}
\end{exemples}
 

\textbf{Repr\'esentation graphique}

 \begin{center}
 \includegraphics[scale=0.4]{bernouilli}
 \end{center}


%
%\begin{exemples}
%\begin{itemize}
%\item[$\bullet$] \dotfill \vsec
%\item[$\bullet$] \dotfill \vsec
%\item[$\bullet$] \dotfill \vsec
%\end{itemize}
%\end{exemples}





%\paragraph{Fonction de r\'epartition}
%
% Calculer la fonction de r\'epartition de $X$ avec $X\hookrightarrow \mathcal{B}(p)$, $p\in\lbrack 0,1\rbrack$. Faire la repr\'esentation graphique pour $p=\ddp\frac{1}{4}$.
%




\begin{prop} Soit $X$ une var avec $X\hookrightarrow \mathcal{B}(p)$, $p\in\lbrack 0,1\rbrack$. Alors
$$E(X) = p\quad V(X) = p(1-p)= pq$$

\end{prop}
 





%--------------------------------------------------
%------------------------------------------------
%-----------------------------------------------------------
\subsection{Loi binomiale}



\paragraph{Mod\'elisation type}
\begin{enumerate}
\item [$\bullet$] Exemple 1 On dispose d'une urne avec $N$ boules, $R$ rouges et $N-R$ jaunes. On effectue $n$ tirages successifs avec remise, la probabilité d'obtenir exactement $k$ boules rouges parmi les $n$ tirages est 

\begin{itemize}
\item[$\star$] $X(\Omega)=\intent{0,n}$
\item[$\star$] Loi de $X$:
$$P(X=k ) = \binom{n}{k} \frac{R^k(N-R)^{n-k}}{N^n}$$

\end{itemize}

\item [$\bullet$] Exemple 2 (Sch\'ema binomial) On effectue $n$ lancers d'une pièce non équilibrée satisfaisant $P( pile) =p $ et $P(face)= q$ on effectue $n$ lancer et $X$ correspond au nombre de pile on obtient : 

\begin{itemize}
\item[$\star$] $X(\Omega)=\intent{0,n}$
\item[$\star$] Loi de $X$:
$$P(X=k ) = \binom{n}{k} p^k(1-p)^{n-k}$$

\end{itemize}
\end{enumerate}





\paragraph{Loi}


\hspace{-0.7cm}  {  

\begin{defi} Soient $X$ une var, $p\in\lbrack 0,1\rbrack$ et $n\in\N^{\star}$. On dit que $X$ suit une loi binomiale de param\`{e}tre $n$ et $p$ si
$$\forall k\in \intent{1,n}, \quad P(X=k ) = \binom{n}{k} p^k(1-p)^{n-k}$$

 On note alors $X\sim B(n,p) $ ou $X\hookrightarrow B(n,p)$
\end{defi}
 
}
 
\textbf{Repr\'esentation graphique}

 \begin{center}
 \includegraphics[scale=0.4]{binomiale}
 \end{center}



\begin{prop} Soit $X$ une var avec $X\hookrightarrow \mathcal{B}(n,p)$, $p\in\lbrack 0,1\rbrack$ et $n\in\N^{\star}$. Alors

$$E(X)= np \quad V(X) = np(1-p) =npq$$
\end{prop}
 




%--------------------------------------------------
%------------------------------------------------
%-----------------------------------------------------------
%\subsection{Loi hyperg\'eom\'etrique}
%
%
%
%
%\paragraph{Mod\'elisation type}
%
%On dispose d'une urne avec $N$ boules, $R$ rouges et $N-R$ jaunes. On effectue $n$ tirages successifs avec remise, la probabilité d'obtenir exactement $k$ boules rouges parmi les $n$ tirages est 
%
%\begin{itemize}
%\item[$\star$] $X(\Omega)=\intent{0,N}$
%\item[$\star$] Loi de $X$:
%$$P(X=k ) = \frac{\binom{R}{k} \binom{N-R}{n-k} }{\binom{N}{n} }$$
%
%\end{itemize}
%
%C'est le même résultat que pour le tirage simultané de $n$ boules. 
%
%
%
%\paragraph{Loi}
%
% {  
%
%\begin{defi} Loi hyperg\'eom\'etrique:
% Soient $X$ une var, $(n,N)\in\N^2$ avec $n\in\intent{ 1,N}$, $p\in\lbrack 0,1\lbrack$ tel que $Np\in\N$. On dit que $X$ suit une loi hyperg\'eom\'etrique de param\`{e}tre $N,n,p$ si 
%$$ \forall k\in \intent{0,n} \quad P(X=k ) = \frac{\binom{pN}{k} \binom{(1-p)N}{n-k} }{\binom{N}{n} }$$
%
% On note alors  $X \sim \cH (N,n,p)$ ou $X \hookrightarrow \cH (N,n,p)$
%\end{defi}
% 
%}\vsec\vsec
%
%\setlength\fboxrule{1pt}
%  {
%
%\textbf{\large{Sch\'ema hyperg\'eom\'etrique:}}
%Le cadre de la loi $\mathcal{H}(N,n,p)$ est celui
%\begin{itemize}
%\item[$\bullet$] d'un ensemble de $N$ objets s\'epar\'es en 2 cat\'egories:
%(Np objets du premier type et $Nq$ objets du second type).
%\item[$\bullet$] dans lequel on extrait $n$ \'echantillons:
%\begin{itemize}
%\item[$\star$] soit simultan\'ement
%\item[$\star$] soit successivement sans remise.
%\end{itemize}
%\item[$\bullet$] La loi $\mathcal{H}(N,n,p)$ permet de conna\^{i}tre le nombre de repr\'esentants de l'une des cat\'egories.
%\end{itemize}
% }
%\setlength\fboxrule{0.5pt}
%\vsec\vsec
%
%
%\begin{exemples}
%\begin{itemize}
%\item[$\bullet$] \dotfill \vsec
%\item[$\bullet$] \dotfill \vsec
%\end{itemize}
%\end{exemples}
%
%\begin{rem}
%\dotfill \vsec
%\dotfill \vsec
%\dotfill
%\end{rem}
%
%
%   
%
%\begin{prop} Soit $X$ une var avec $X\hookrightarrow \mathcal{H}(N,n,p)$, $p\in\lbrack 0,1\rbrack$. Alors
%
%$$E(X)=np$$
%\end{prop}
% 
%\begin{prop}[Formule de Vandermonde]
%Soient $n,m,k \in \N^*$  avec $k\leq n+m$ 
%
%$$\binom{n+m}{k}  =\sum_{i=0}^k \binom{n}{i}\binom{m}{k-i}$$ 
%\end{prop}
%\begin{proof}
%\begin{align*}
%(1+X)^{n+m} &=(1+X)^n (1+X)^m\\
%\equivaut \sum_{k=0}^{m+n} \binom{n}{k} X^k &=\sum_{i=0}^{m} \binom{n}{i} X^i \sum_{j=0}^{m} \binom{m}{j} X^j\\
%\equivaut \sum_{k=0}^{m+n} \binom{n}{k} X^k &= \sum_{i=0}^{m} \sum_{j=0}^{m}  \binom{n}{i} \binom{m}{j} X^{i+j}
%\end{align*}
%On identifie le coefficient de $X^k$ dans les deux expressions. Dans la deuxième expression, pour $i\geq 0$, fixé $j=k-i\geq 0$. Donc i varie dans $\intent{0,k}$
%$$\binom{n}{k} = \sum_{i=0}^{k} \binom{n}{i} \binom{m}{k-i}$$ 
%\end{proof}
%\begin{proof}[Preuve sur l'espèrance de la loi hypergéomtrique]
%\begin{align*}
%E(X) &= \sum_{k=0}^n k P(X=k) \\
%		&= \sum_{k=0}^n k \frac{\binom{R}{k} \binom{N-R}{n-k} }{\binom{N}{n} }\\
%		&=\frac{1}{\binom{N}{n} } \sum_{k=1}^n k \binom{R}{k} \binom{N-R}{n-k}\\
%&=\frac{1}{\binom{N}{n} } \sum_{k=1}^n  R \binom{R-1}{k-1} \binom{N-R}{n-k}\\
%	&=\frac{1}{\binom{N}{n} } \sum_{k=1}^n k  \binom{R-1}{k-1} \binom{N-R}{n-1-(k-1)}\\
%		&=\frac{1}{\binom{N}{n} } \sum_{j=0}^{n-1} R \binom{R-1}{j} \binom{N-R}{n-1-j }\\
%				&= \frac{1}{\binom{n}{k} } R \binom{R-1+N-R}{n-1}\\
%		&=\frac{R}{\binom{n}{k} } \binom{N-1}{n-1}\\
%		&= \frac{Rn}{N}	
%\end{align*}
%
%\end{proof}





\section{Ind\'ependance de var}

%----------------------------------------------------------
%-----------------------------------------------------------
\subsection{Ind\'ependance de deux var}

\noindent\ {D\'efinition}\\

 {\noindent  

\begin{defi} 
Soient $X$ et $Y$ deux var. On dit que $X$ et $Y$ sont ind\'ependantes si:\\
\vspace{1.5cm}
\end{defi}
 
}\vsec

{\footnotesize \begin{exercice} 
\begin{enumerate}
\item Une urne contient $n$ jetons num\'erot\'es de 1 \`a $n$. On en tire 2 successivement avec remise. 
Soient $X$ le num\'ero du premier jeton, et $Y$ le num\'ero du second. \'Etudier l'ind\'ependance de $X$ et $Y$.
\item Reprendre l'exercice pr\'ec\'edent avec deux tirages sans remise.
\end{enumerate}
\end{exercice}
}



\vspace{0.3cm}

\noindent\ {Cons\'equence sur les trois types de lois}\\

 {\noindent  

\begin{prop} 
Soit $(X,Y)$ un couple de var. Il y a \'equivalence entre:
\begin{itemize}
 \item[$\bullet$] Les var $X$ et $Y$ sont ind\'ependantes.\vsec
\item[$\bullet$] $\forall (x,y)\in X(\Omega)\times Y(\Omega)$ \dotfill \vsec
\item[$\bullet$] Pour tout $y\in Y(\Omega)$ tel que $P(\lbrack Y=y\rbrack)\not= 0$, \dotfill \vsec
\item[$\bullet$] Pour tout $x\in X(\Omega)$ tel que $P(\lbrack X=x\rbrack)\not= 0$, \dotfill \vsec
\end{itemize}
\end{prop}
 
}\vsec

\begin{prop}
$(X,Y)$ couple, esperance et varaince de $X+Y$ et $XY$. 
\end{prop}
\vspace{0.3cm}

\noindent\ {Lemme des coalitions}\\


\begin{prop} 
Soient $X$ et $Y$ deux var ind\'ependantes.\vsec\\ 
Alors $u(X)$ et $v(Y)$ sont indépendants. 
\end{prop}
 


\begin{exemples}
Si $X$ et $Y$ sont ind\'ependantes, alors, par exemple
\begin{itemize}
 \item[$\bullet$] \dotfill\vsec
\item[$\bullet$] \dotfill\vsec
\end{itemize}
\end{exemples}

%-----------------------------------------------------------
%----------------------------------------------------------
%-----------------------------------------------------------
%----------------------------------------------------------
%-----------------------------------------------------------
\subsection{G\'en\'eralisation: ind\'ependance de $n$ var}


\begin{defi} 
Soient $X_1,\dots X_n$ $n$ var. Elles sont mutuellement ind\'ependantes si:\\
\vspace{1.2cm}
\end{defi}
 

\begin{rems}
\begin{itemize}
\item[$\bullet$] On vous demandera rarement de montrer que des var sont mutuellement ind\'ependantes. C'est plut\^ot une donn\'ee de l'exercice qui permet de faire le calcul de la probabilit\'e d'intersections d'\'ev\'enements.

\item[$\bullet$] Exemples types de mod\`{e}les donnant des var mutuellement ind\'ependantes:\vsec
\begin{itemize}
\item[$\star$] \dotfill\vsec
\item[$\star$] \dotfill\vsec
\item[$\star$] De fa\c{c}on g\'en\'erale: \dotfill\vsec
\end{itemize}
\end{itemize}
\end{rems}


\begin{prop} 
Soient $X_1,\dots, X_n,X_{n+1},\dots, X_p$ des var mutuellement ind\'ependantes. Alors:\vsec
\begin{itemize}
\item[$\bullet$] \dotfill\vsec
\item[$\bullet$] \dotfill\vsec
\end{itemize}
\end{prop}


\begin{exemples}
Si $X,Y,Z,T$ sont mutuellement ind\'ependantes, alors, par exemple\vsec
\begin{itemize}
 \item[$\bullet$] \dotfill\vsec
\item[$\bullet$] \dotfill\vsec
\end{itemize}
\end{exemples}






%----------------------------------------------------------
%----------------------------------------------------
%-----------------------------------------------------
%-------------------------------------------------------

\end{document}