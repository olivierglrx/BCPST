\documentclass[a4paper, 11pt]{article}
\usepackage[utf8]{inputenc}
\usepackage{amssymb,amsmath,amsthm}
\usepackage{geometry}
\usepackage[T1]{fontenc}
\usepackage[french]{babel}
\usepackage{fontawesome}
\usepackage{pifont}
\usepackage{tcolorbox}
\usepackage{fancybox}
\usepackage{bbold}
\usepackage{tkz-tab}
\usepackage{tikz}
\usepackage{fancyhdr}
\usepackage{sectsty}
\usepackage[framemethod=TikZ]{mdframed}
\usepackage{stackengine}
\usepackage{scalerel}
\usepackage{xcolor}
\usepackage{hyperref}
\usepackage{listings}
\usepackage{enumitem}
\usepackage{stmaryrd} 
\usepackage{comment}


\hypersetup{
    colorlinks=true,
    urlcolor=blue,
    linkcolor=blue,
    breaklinks=true
}





\theoremstyle{definition}
\newtheorem{probleme}{Problème}
\theoremstyle{definition}


%%%%% box environement 
\newenvironment{fminipage}%
     {\begin{Sbox}\begin{minipage}}%
     {\end{minipage}\end{Sbox}\fbox{\TheSbox}}

\newenvironment{dboxminipage}%
     {\begin{Sbox}\begin{minipage}}%
     {\end{minipage}\end{Sbox}\doublebox{\TheSbox}}


%\fancyhead[R]{Chapitre 1 : Nombres}


\newenvironment{remarques}{ 
\paragraph{Remarques :}
	\begin{list}{$\bullet$}{}
}{
	\end{list}
}




\newtcolorbox{tcbdoublebox}[1][]{%
  sharp corners,
  colback=white,
  fontupper={\setlength{\parindent}{20pt}},
  #1
}







%Section
% \pretocmd{\section}{%
%   \ifnum\value{section}=0 \else\clearpage\fi
% }{}{}



\sectionfont{\normalfont\Large \bfseries \underline }
\subsectionfont{\normalfont\Large\itshape\underline}
\subsubsectionfont{\normalfont\large\itshape\underline}



%% Format théoreme, defintion, proposition.. 
\newmdtheoremenv[roundcorner = 5px,
leftmargin=15px,
rightmargin=30px,
innertopmargin=0px,
nobreak=true
]{theorem}{Théorème}

\newmdtheoremenv[roundcorner = 5px,
leftmargin=15px,
rightmargin=30px,
innertopmargin=0px,
]{theorem_break}[theorem]{Théorème}

\newmdtheoremenv[roundcorner = 5px,
leftmargin=15px,
rightmargin=30px,
innertopmargin=0px,
nobreak=true
]{corollaire}[theorem]{Corollaire}
\newcounter{defiCounter}
\usepackage{mdframed}
\newmdtheoremenv[%
roundcorner=5px,
innertopmargin=0px,
leftmargin=15px,
rightmargin=30px,
nobreak=true
]{defi}[defiCounter]{Définition}

\newmdtheoremenv[roundcorner = 5px,
leftmargin=15px,
rightmargin=30px,
innertopmargin=0px,
nobreak=true
]{prop}[theorem]{Proposition}

\newmdtheoremenv[roundcorner = 5px,
leftmargin=15px,
rightmargin=30px,
innertopmargin=0px,
]{prop_break}[theorem]{Proposition}

\newmdtheoremenv[roundcorner = 5px,
leftmargin=15px,
rightmargin=30px,
innertopmargin=0px,
nobreak=true
]{regles}[theorem]{Règles de calculs}


\newtheorem*{exemples}{Exemples}
\newtheorem{exemple}{Exemple}
\newtheorem*{rem}{Remarque}
\newtheorem*{rems}{Remarques}
% Warning sign

\newcommand\warning[1][4ex]{%
  \renewcommand\stacktype{L}%
  \scaleto{\stackon[1.3pt]{\color{red}$\triangle$}{\tiny\bfseries !}}{#1}%
}


\newtheorem{exo}{Exercice}
\newcounter{ExoCounter}
\newtheorem{exercice}[ExoCounter]{Exercice}

\newcounter{counterCorrection}
\newtheorem{correction}[counterCorrection]{\color{red}{Correction}}


\theoremstyle{definition}

%\newtheorem{prop}[theorem]{Proposition}
%\newtheorem{\defi}[1]{
%\begin{tcolorbox}[width=14cm]
%#1
%\end{tcolorbox}
%}


%--------------------------------------- 
% Document
%--------------------------------------- 






\lstset{numbers=left, numberstyle=\tiny, stepnumber=1, numbersep=5pt}




% Header et footer

\pagestyle{fancy}
\fancyhead{}
\fancyfoot{}
\renewcommand{\headwidth}{\textwidth}
\renewcommand{\footrulewidth}{0.4pt}
\renewcommand{\headrulewidth}{0pt}
\renewcommand{\footruleskip}{5px}

\fancyfoot[R]{Olivier Glorieux}
%\fancyfoot[R]{Jules Glorieux}

\fancyfoot[C]{ Page \thepage }
\fancyfoot[L]{1BIOA - Lycée Chaptal}
%\fancyfoot[L]{MP*-Lycée Chaptal}
%\fancyfoot[L]{Famille Lapin}



\newcommand{\Hyp}{\mathbb{H}}
\newcommand{\C}{\mathcal{C}}
\newcommand{\U}{\mathcal{U}}
\newcommand{\R}{\mathbb{R}}
\newcommand{\T}{\mathbb{T}}
\newcommand{\D}{\mathbb{D}}
\newcommand{\N}{\mathbb{N}}
\newcommand{\Z}{\mathbb{Z}}
\newcommand{\F}{\mathcal{F}}




\newcommand{\bA}{\mathbb{A}}
\newcommand{\bB}{\mathbb{B}}
\newcommand{\bC}{\mathbb{C}}
\newcommand{\bD}{\mathbb{D}}
\newcommand{\bE}{\mathbb{E}}
\newcommand{\bF}{\mathbb{F}}
\newcommand{\bG}{\mathbb{G}}
\newcommand{\bH}{\mathbb{H}}
\newcommand{\bI}{\mathbb{I}}
\newcommand{\bJ}{\mathbb{J}}
\newcommand{\bK}{\mathbb{K}}
\newcommand{\bL}{\mathbb{L}}
\newcommand{\bM}{\mathbb{M}}
\newcommand{\bN}{\mathbb{N}}
\newcommand{\bO}{\mathbb{O}}
\newcommand{\bP}{\mathbb{P}}
\newcommand{\bQ}{\mathbb{Q}}
\newcommand{\bR}{\mathbb{R}}
\newcommand{\bS}{\mathbb{S}}
\newcommand{\bT}{\mathbb{T}}
\newcommand{\bU}{\mathbb{U}}
\newcommand{\bV}{\mathbb{V}}
\newcommand{\bW}{\mathbb{W}}
\newcommand{\bX}{\mathbb{X}}
\newcommand{\bY}{\mathbb{Y}}
\newcommand{\bZ}{\mathbb{Z}}



\newcommand{\cA}{\mathcal{A}}
\newcommand{\cB}{\mathcal{B}}
\newcommand{\cC}{\mathcal{C}}
\newcommand{\cD}{\mathcal{D}}
\newcommand{\cE}{\mathcal{E}}
\newcommand{\cF}{\mathcal{F}}
\newcommand{\cG}{\mathcal{G}}
\newcommand{\cH}{\mathcal{H}}
\newcommand{\cI}{\mathcal{I}}
\newcommand{\cJ}{\mathcal{J}}
\newcommand{\cK}{\mathcal{K}}
\newcommand{\cL}{\mathcal{L}}
\newcommand{\cM}{\mathcal{M}}
\newcommand{\cN}{\mathcal{N}}
\newcommand{\cO}{\mathcal{O}}
\newcommand{\cP}{\mathcal{P}}
\newcommand{\cQ}{\mathcal{Q}}
\newcommand{\cR}{\mathcal{R}}
\newcommand{\cS}{\mathcal{S}}
\newcommand{\cT}{\mathcal{T}}
\newcommand{\cU}{\mathcal{U}}
\newcommand{\cV}{\mathcal{V}}
\newcommand{\cW}{\mathcal{W}}
\newcommand{\cX}{\mathcal{X}}
\newcommand{\cY}{\mathcal{Y}}
\newcommand{\cZ}{\mathcal{Z}}







\renewcommand{\phi}{\varphi}
\newcommand{\ddp}{\displaystyle}


\newcommand{\G}{\Gamma}
\newcommand{\g}{\gamma}

\newcommand{\tv}{\rightarrow}
\newcommand{\wt}{\widetilde}
\newcommand{\ssi}{\Leftrightarrow}

\newcommand{\floor}[1]{\left \lfloor #1\right \rfloor}
\newcommand{\rg}{ \mathrm{rg}}
\newcommand{\quadou}{ \quad \text{ ou } \quad}
\newcommand{\quadet}{ \quad \text{ et } \quad}
\newcommand\fillin[1][3cm]{\makebox[#1]{\dotfill}}
\newcommand\cadre[1]{[#1]}
\newcommand{\vsec}{\vspace{0.3cm}}

\DeclareMathOperator{\im}{Im}
\DeclareMathOperator{\cov}{Cov}
\DeclareMathOperator{\vect}{Vect}
\DeclareMathOperator{\Vect}{Vect}
\DeclareMathOperator{\card}{Card}
\DeclareMathOperator{\Card}{Card}
\DeclareMathOperator{\Id}{Id}
\DeclareMathOperator{\PSL}{PSL}
\DeclareMathOperator{\PGL}{PGL}
\DeclareMathOperator{\SL}{SL}
\DeclareMathOperator{\GL}{GL}
\DeclareMathOperator{\SO}{SO}
\DeclareMathOperator{\SU}{SU}
\DeclareMathOperator{\Sp}{Sp}


\DeclareMathOperator{\sh}{sh}
\DeclareMathOperator{\ch}{ch}
\DeclareMathOperator{\argch}{argch}
\DeclareMathOperator{\argsh}{argsh}
\DeclareMathOperator{\imag}{Im}
\DeclareMathOperator{\reel}{Re}



\renewcommand{\Re}{ \mathfrak{Re}}
\renewcommand{\Im}{ \mathfrak{Im}}
\renewcommand{\bar}[1]{ \overline{#1}}
\newcommand{\implique}{\Longrightarrow}
\newcommand{\equivaut}{\Longleftrightarrow}

\renewcommand{\fg}{\fg \,}
\newcommand{\intent}[1]{\llbracket #1\rrbracket }
\newcommand{\cor}[1]{{\color{red} Correction }#1}

\newcommand{\conclusion}[1]{\begin{center} \fbox{#1}\end{center}}


\renewcommand{\title}[1]{\begin{center}
    \begin{tcolorbox}[width=14cm]
    \begin{center}\huge{\textbf{#1 }}
    \end{center}
    \end{tcolorbox}
    \end{center}
    }

    % \renewcommand{\subtitle}[1]{\begin{center}
    % \begin{tcolorbox}[width=10cm]
    % \begin{center}\Large{\textbf{#1 }}
    % \end{center}
    % \end{tcolorbox}
    % \end{center}
    % }

\renewcommand{\thesection}{\Roman{section}} 
\renewcommand{\thesubsection}{\thesection.  \arabic{subsection}}
\renewcommand{\thesubsubsection}{\thesubsection. \alph{subsubsection}} 

\newcommand{\suiteu}{(u_n)_{n\in \N}}
\newcommand{\suitev}{(v_n)_{n\in \N}}
\newcommand{\suite}[1]{(#1_n)_{n\in \N}}
%\newcommand{\suite1}[1]{(#1_n)_{n\in \N}}
\newcommand{\suiteun}[1]{(#1_n)_{n\geq 1}}
\newcommand{\equivalent}[1]{\underset{#1}{\sim}}

\newcommand{\demi}{\frac{1}{2}}
\geometry{hmargin=2.0cm, vmargin=2cm}




\begin{document}

 \title{Chapitre 22  : Variables Aléatoires Réelles - exemples de cours} 

 Dans tout ce chapitre, on se place dans un espace probabilis\'e fini $(\Omega, \mathcal{P}(\Omega),P)$.

%\vspace{0.4cm}
%--------------------------------------------------
%------------------------------------------------
%-----------------------------------------------------------
%----------------------------------------------------------
%-----------------------------------------------------------
%----------------------------------------------------------
\section{G\'en\'eralit\'es sur les variables al\'eatoires r\'eelles finies}

\begin{exemple}
Pour chacune des exp\'eriences al\'eatoires suivantes, donner un exemple de variable aléatoire adaptée à l'expérience. (On les appellera $X_1$ à $X_4$)
\begin{enumerate}
\item On lance une pi\`ece de monnaie  et on gagne un euro si on obtient pile, on perd $2$ euros si on obtient face. 
\item On lance deux d\'es à 6 faces de couleurs différentes.
\item Une urne contient $n$ boules numérotées de $1$ à $n$. On tire une boule au hasard.
\item Une urne contient $R$ boules rouges et $B$ boules blanches. On effectue $10$ tirages successifs avec remise.
\end{enumerate}
\end{exemple}

\vspace{12cm}
%--------------------------------------------------
%------------------------------------------------
%-----------------------------------------------------------




\begin{exemple}
Donner les univers images pour les $4$ exemples pr\'ec\'edents.

\end{exemple}

\vspace{4cm}



%--------------------------------------------------
%------------------------------------------------
%-----------------------------------------------------------


%\paragraph{Notations des \'ev\`{e}nements associ\'es \`{a} une varf}
\begin{tcolorbox}
 \textbf{Notations}: Pour tout $a\in\R$, on note:\vsec
\begin{itemize}
\item[$\bullet$] $\lbrack X=a\rbrack=X^{-1}(\{a\})=\{ \omega \in \Omega \, |\, X(\omega) = a\}$
\item[$\bullet$]  $\lbrack X\leq a\rbrack=\{ \omega \in \Omega \, |\, X(\omega) \leq a\}$
\item[$\bullet$]  $\lbrack X\geq a\rbrack=\{ \omega \in \Omega \, |\, X(\omega) \geq  a\}$
\item[$\bullet$]  $\lbrack X< a\rbrack=\{ \omega \in \Omega \, |\, X(\omega) < a\}$
\item[$\bullet$]  $\lbrack X> a\rbrack=\{ \omega \in \Omega \, |\, X(\omega) > a\}$
\end{itemize}
Soit $A\subset \Omega$ on note :
\begin{itemize}
\item[$\bullet$] $\lbrack X\in A \rbrack=X^{-1}(A)=\{ \omega \in \Omega \, |\, X(\omega) \in A\}$
\end{itemize}   
\end{tcolorbox}
 

\begin{exemple} 
On reprend le deuxi\`eme exemple (somme des valeurs des dés). Exprimer sous forme d'ensemble 
\begin{itemize}
    \item $\lbrack X_2=5\rbrack$
    \item $\lbrack X_2 \geq 10 \rbrack$
    \item $\lbrack 1\leq X_2< 5\rbrack$
    \item $\lbrack X_2>13\rbrack$
\end{itemize}
En déduire la valeur de 
\begin{itemize}
    \item $P(\lbrack X_2=5\rbrack)$
    \item $P(\lbrack X_2 \geq 10 \rbrack)$
    \item $P(\lbrack 1\leq X_2< 5\rbrack)$
    \item $P(\lbrack X_2>13\rbrack)$
\end{itemize}
\end{exemple}


\begin{exercice} Soit $X$ une varf et $a\in\R$. Exprimer $P(\lbrack X \leq a\rbrack)$ en fonction de  $P(\lbrack X> a\rbrack)$.
\end{exercice}
\vspace{2cm}







%--------------------------------------------------
%------------------------------------------------
%-----------------------------------------------------------
\subsection{Loi d'une variable al\'eatoire r\'eelle finie}



\begin{exemple}  
Donner les lois de $X_1$ et $X_3$.
\end{exemple} 

\vspace{8cm}
 



\begin{exemple}  
Donner les diagrammes en bâtons de $X_1$ et $X_3$.
\end{exemple} 
\vspace{4cm}

%--------------------------------------------------
%------------------------------------------------
%-----------------------------------------------------------
\subsection{Fonction de r\'epartition d'une variable al\'eatoire r\'eelle finie}

%\subsection{D\'efinition}

 




\begin{exemple} 
Calculer les fonctions de r\'epartition $F_{X_1}$ et $F_{X_3}$ puis en faire  la repr\'esentation graphique.
\end{exemple} 


\vspace{10cm}






\begin{exercice} Un joueur pr\'el\`eve successivement et avec remise $n$ boules dans une urne contenant $N$ boules num\'erot\'ees de 1 \`a $N$. On consid\`ere les varf $X$ et $Y$ \'egales respectivement au plus grand et au plus petit num\'ero des $n$ boules tir\'ees. Donner les univers images de $X$ et de $Y$. Calculer la fonction de r\'epartition de $X$. En d\'eduire la loi de $X$. Calculer ensuite pour tout $k\in Y(\Omega)$: $P(\lbrack Y>k\rbrack)$. En d\'eduire la loi de $Y$.
\end{exercice}

\vspace{14cm}



Penser \`{a} passer par la fonction de r\'epartition pour obtenir la loi de var d\'efinie avec des min ou des max.


%--------------------------------------------------
%------------------------------------------------
%-----------------------------------------------------------

\begin{exercice} 
Donner loi de  $U=G^2-G-2$ où $G$ est la variable aléatoire de l'exemple 1.
\end{exercice}

\vspace{3cm}
%-----------------------------------------------------------
%----------------------------------------------------------
%-----------------------------------------------------------
%----------------------------------------------------------



\section{Moments d'une variable al\'eatoire r\'eelle finie}

%--------------------------------------------------
%------------------------------------------------
%-----------------------------------------------------------
\subsection{Esp\'erance d'une variable al\'eatoire r\'eelle finie}



\begin{exemple}
Calculer l'espérance de $X_1$ et $X_3$. 
\end{exemple}



 \vspace{7cm}


 



%\begin{exercice} 
%Calculer les moments d'ordre 0, 1, 2 et 3 de la var $X$ dont la loi est d\'efinie dans l'exercice 3.
%\end{exercice}








\begin{exercice} On consid\`ere $r$ boules num\'erot\'ees de 1 \`a $r$ et $n$ tiroirs num\'erot\'es de 1 \`a $n$. On place au hasard chacune des $r$ boules dans l'un des $n$ tiroirs, chaque tiroir pouvant contenir 0, 1 ou plusieurs boules. On note $Y$ la var \'egale au nombre de tiroirs rest\'es vides. 
\begin{enumerate}
\item On note $X_i$ la var qui vaut 1 si le tiroir $i$ est vide et $0$ sinon. Donner la loi  et l'espérance de $X_i$.
\item Exprimer $Y$ en fonction des $X_i$.
\item En déduire l'espérance de $Y$
\end{enumerate}
\end{exercice}

\vspace{7cm}

%--------------------------------------------------
%------------------------------------------------
%-----------------------------------------------------------
\subsection{Variance, \'ecart-type d'une variable al\'eatoire r\'eelle finie}


Penser au th\'eor\`{e}me de transfert pour calculer l'esp\'erance de var de type $Y=g(X)$ avec $X$ connue.

\begin{exemple}
Calculer les moments d'ordre 2 de $X_1$ et $X_3$
\end{exemple}





\vspace{3cm}


\begin{exemple}
Calculer la variance de $X_1$ et $X_3$
\end{exemple}


\vspace{3cm}




%
\subsection{In\'egalit\'e de Bienaym\'e-Tchebychev}

%
 \begin{exercice} 
On lance $n$ fois de suite un d\'e \'equilibr\'e.
\begin{enumerate}
\item Soit $X$ le nombre d'apparition du nombre $6$. Quelle loi suit $X$ ? Donner son esp\'erance et sa variance.
\item Soit $Y$ la fr\'equence d'apparition du nombre $6$. Exprimer $Y$ en fonction de $X$, et donner son esp\'erance et sa variance.
\item Soit $p_n$ la probabilit\'e que $Y$ soit proche de $\ddp \frac{1}{6}$ \`a $0.1$ pr\`es. Combien de lancers doit-on effectuer pour que $p_n$ soit sup\'erieur \`a $0.9$ ?
\end{enumerate}
\end{exercice}
\newpage







%------------------------------------------------
%-----------------------------------------------------------
%----------------------------------------------------------
%-----------------------------------------------------------
%----------------------------------------------------------
\section{Lois usuelles}
%--------------------------------------------------
%------------------------------------------------


%--------------------------------------------------
%------------------------------------------------
%-----------------------------------------------------------
\subsection{Loi uniforme}



\paragraph{Mod\'elisation type}

\begin{itemize}
\item[$\bullet$] Lancer d'un dé équilibré à $6$ faces. On note $X$ la variable aléatoire correspondant au numéro obtenu. On obtient 
$$\forall k\in \intent{1,6},\,  P(X=k ) =....$$
\item[$\bullet$] Autres exemples:
\begin{itemize}
\item[$\star$]  Tirer une boule dans une urne contenant $N$  boules numérotées de $1$ à $N$. On note $X$ la variable aléatoire correspondant au numéro obtenu. On obtient 
$$\forall k\in \intent{1,N},\,  P(X=k ) =....$$
\end{itemize}
\end{itemize}

 

%--------------------------------------------------
%------------------------------------------------
%-----------------------------------------------------------
\subsection{Loi de Bernoulli}



\paragraph{Mod\'elisation type}

 Mod\'elisation type: 
 
Une pièce truquée possède $p$ chances de tomber sur pile ( succés) et $q=1-p$ chances de tomber sur face (échec). On définit la VARF X qui teste si pile est sortie : $X$ vaut $1$ si on obtient pile et $0$ sinon. 
$$P(X=1) =.... \quad P(X=0) = ....$$






%
%\begin{exemples}
%\begin{itemize}
%\item[$\bullet$] \dotfill \vsec
%\item[$\bullet$] \dotfill \vsec
%\item[$\bullet$] \dotfill \vsec
%\end{itemize}
%\end{exemples}





%\paragraph{Fonction de r\'epartition}
%
% Calculer la fonction de r\'epartition de $X$ avec $X\hookrightarrow \mathcal{B}(p)$, $p\in\lbrack 0,1\rbrack$. Faire la repr\'esentation graphique pour $p=\ddp\frac{1}{4}$.
%










%--------------------------------------------------
%------------------------------------------------
%-----------------------------------------------------------
\subsection{Loi binomiale}



\paragraph{Mod\'elisation type}
\begin{enumerate}
\item [$\bullet$] Exemple 1 On dispose d'une urne avec $N$ boules, $R$ rouges et $N-R$ jaunes. On effectue $n$ tirages successifs avec remise, la probabilité d'obtenir exactement $k$ boules rouges parmi les $n$ tirages est 

\begin{itemize}
\item[$\star$] $X(\Omega)=\intent{0,n}$
\item[$\star$] Loi de $X$:
$$P(X=k ) =....$$

\end{itemize}

\item [$\bullet$] Exemple 2 (Sch\'ema binomial) On effectue $n$ lancers d'une pièce non équilibrée satisfaisant $P( pile) =p $ et $P(face)= q$ on effectue $n$ lancer et $X$ correspond au nombre de pile on obtient : 

\begin{itemize}
\item[$\star$] $X(\Omega)=\intent{0,n}$
\item[$\star$] Loi de $X$:
$$P(X=k ) =....$$

\end{itemize}
\end{enumerate}

\begin{exemple}
    Dire si les $X_i$ de l'exemple 1 suivent des lois usuelles. Si oui les expliciter.
\end{exemple}

\vspace{4cm}


\newpage
%------------------------------------------------
\begin{exemple}  \;
	Pour chacune des variables al\'eatoires d\'ecrites ci-dessous, donner la loi exacte, l'esp\'erance et la variance:
	\begin{enumerate}
		\item Nombre de piles au cours du lancer de 20 pi\`eces truqu\'ees dont la probabilit\'e d'obtenir face est 0.7.
		%\item On tire 8 cartes d'un jeu de 52 et on s'int\'eresse au nombre de carreaux.
		\item On lance 5 d\'es.
		      \begin{enumerate}
			      \item On s'int\'eresse au nombre de 6.
			      \item On s'int\'eresse au num\'ero obtenu avec le premier d\'e.
		      \end{enumerate}
		\item Nombre de filles dans les familles de 6 enfants sachant que la probabilit\'e d'obtenir une fille est 0.51.
		      %\item Nombre d'accidents par an \`a un carrefour donn\'e, sachant qu'il y a chaque jour une chance sur 125 d'accident.
		\item Nombre de voix d'un des candidats \`a une \'election pr\'esidentielle lors du d\'epouillement des 100 premiers bulletins dans un bureau de vote.
		\item On range au hasard 20 objets dans 3 tiroirs. Nombre d'objets dans le premier tiroir.
%		\item Un sac contient 26 jetons sur lesquels figurent les 26 lettres de l'alphabet. On en aligne 5 au hasard. Nombre de voyelles dans ce mot.
\item 	On consid\`ere une urne contenant 5 boules num\'erot\'ees: 2 rouges et 3 bleues.
	\begin{enumerate}
		%\item On tire simultan\'ement 3 boules de l'urne et on note $X$ le nombre de boules bleues obtenu. Donner la loi de $X$ ainsi que son esp\'erance et sa variance.
		\item On r\'ealise 3 tirages successifs avec remise et on note $Y$ le nombre de boules bleues obtenu au cours de ces tirages. Donner la loi de $Y$ ainsi que son esp\'erance et sa variance.
	%	\item On r\'ealise 3 tirages successifs sans remise et on note $Z$ le nombre de boules bleues obtenu au cours de ces tirages. Donner la loi de $Z$ ainsi que son esp\'erance et sa variance.
		\item On tire une boule de l'urne et on note $T$ le num\'ero de la boule obtenue. Donner la loi de $T$ ainsi que son esp\'erance et sa variance.
	\end{enumerate}
		\item Un enclos contient 15 lamas, 15 dromadaires et 15 chameaux. On sort un animal au hasard de cet enclos. Nombre de bosses.
		\item On suppose que $1\%$ des tr\`efles poss\`edent 4 feuilles. On cueille 1000 tr\`efles. Nombre de tr\`efles \`a 4 feuilles cueillis.
%		\item Dans une population de 20 personnes, dont 8 hommes, nombre de femmes pr\'esentes dans une d\'el\'egation de 6 personnes tir\'ees au sort.
		\item Il y a 128 boules num\'erot\'ees de 1 \`a 128. On en tire 10 parmi les 128, puis on en tire une parmi les 10. On s'int\'eresse au num\'ero de la boule obtenue.
  
	\end{enumerate}
\end{exemple}
%----------------------------------------------
%------------------------------------------------
\vspace{8cm}


%--------------------------------------------------
%------------------------------------------------
%-----------------------------------------------------------


%----------------------------------------------------------
%-----------------------------------------------------------
% \section{Ind\'ependance de deux var}
% \begin{exercice} 
% \begin{enumerate}
% \item Une urne contient $n$ jetons num\'erot\'es de 1 \`a $n$. On en tire 2 successivement avec remise. 
% Soient $X$ le num\'ero du premier jeton, et $Y$ le num\'ero du second. \'Etudier l'ind\'ependance de $X$ et $Y$.
% \item Reprendre l'exercice pr\'ec\'edent avec deux tirages sans remise.
% \end{enumerate}
% \end{exercice}




%\begin{exemple}
%Si $X$ et $Y$ sont ind\'ependantes, alors, par exemple
%\begin{itemize}
% \item[$\bullet$] \dotfill\vsec
%\item[$\bullet$] \dotfill\vsec
%\end{itemize}
%\end{exemple}
%
%%-----------------------------------------------------------
%%----------------------------------------------------------
%%-----------------------------------------------------------
%%----------------------------------------------------------
%
%\begin{exemple}
%Si $X,Y,Z,T$ sont mutuellement ind\'ependantes, alors, par exemple\vsec
%\begin{itemize}
% \item[$\bullet$] \dotfill\vsec
%\item[$\bullet$] \dotfill\vsec
%\end{itemize}
%\end{exemple}






%----------------------------------------------------------
%----------------------------------------------------
%-----------------------------------------------------
%-------------------------------------------------------

\end{document}