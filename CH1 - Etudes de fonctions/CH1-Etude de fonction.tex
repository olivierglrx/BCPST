\documentclass[a4paper, 11pt]{article}
\usepackage[utf8]{inputenc}
\usepackage{amssymb,amsmath,amsthm}
\usepackage{geometry}
\usepackage[T1]{fontenc}
\usepackage[french]{babel}
\usepackage{fontawesome}
\usepackage{pifont}
\usepackage{tcolorbox}
\usepackage{fancybox}
\usepackage{bbold}
\usepackage{tkz-tab}
\usepackage{tikz}
\usepackage{fancyhdr}
\usepackage{sectsty}
\usepackage[framemethod=TikZ]{mdframed}
\usepackage{stackengine}
\usepackage{scalerel}
\usepackage{xcolor}
\usepackage{hyperref}
\usepackage{listings}
\usepackage{enumitem}
\usepackage{stmaryrd} 
\usepackage{comment}


\hypersetup{
    colorlinks=true,
    urlcolor=blue,
    linkcolor=blue,
    breaklinks=true
}





\theoremstyle{definition}
\newtheorem{probleme}{Problème}
\theoremstyle{definition}


%%%%% box environement 
\newenvironment{fminipage}%
     {\begin{Sbox}\begin{minipage}}%
     {\end{minipage}\end{Sbox}\fbox{\TheSbox}}

\newenvironment{dboxminipage}%
     {\begin{Sbox}\begin{minipage}}%
     {\end{minipage}\end{Sbox}\doublebox{\TheSbox}}


%\fancyhead[R]{Chapitre 1 : Nombres}


\newenvironment{remarques}{ 
\paragraph{Remarques :}
	\begin{list}{$\bullet$}{}
}{
	\end{list}
}




\newtcolorbox{tcbdoublebox}[1][]{%
  sharp corners,
  colback=white,
  fontupper={\setlength{\parindent}{20pt}},
  #1
}







%Section
% \pretocmd{\section}{%
%   \ifnum\value{section}=0 \else\clearpage\fi
% }{}{}



\sectionfont{\normalfont\Large \bfseries \underline }
\subsectionfont{\normalfont\Large\itshape\underline}
\subsubsectionfont{\normalfont\large\itshape\underline}



%% Format théoreme, defintion, proposition.. 
\newmdtheoremenv[roundcorner = 5px,
leftmargin=15px,
rightmargin=30px,
innertopmargin=0px,
nobreak=true
]{theorem}{Théorème}

\newmdtheoremenv[roundcorner = 5px,
leftmargin=15px,
rightmargin=30px,
innertopmargin=0px,
]{theorem_break}[theorem]{Théorème}

\newmdtheoremenv[roundcorner = 5px,
leftmargin=15px,
rightmargin=30px,
innertopmargin=0px,
nobreak=true
]{corollaire}[theorem]{Corollaire}
\newcounter{defiCounter}
\usepackage{mdframed}
\newmdtheoremenv[%
roundcorner=5px,
innertopmargin=0px,
leftmargin=15px,
rightmargin=30px,
nobreak=true
]{defi}[defiCounter]{Définition}

\newmdtheoremenv[roundcorner = 5px,
leftmargin=15px,
rightmargin=30px,
innertopmargin=0px,
nobreak=true
]{prop}[theorem]{Proposition}

\newmdtheoremenv[roundcorner = 5px,
leftmargin=15px,
rightmargin=30px,
innertopmargin=0px,
]{prop_break}[theorem]{Proposition}

\newmdtheoremenv[roundcorner = 5px,
leftmargin=15px,
rightmargin=30px,
innertopmargin=0px,
nobreak=true
]{regles}[theorem]{Règles de calculs}


\newtheorem*{exemples}{Exemples}
\newtheorem{exemple}{Exemple}
\newtheorem*{rem}{Remarque}
\newtheorem*{rems}{Remarques}
% Warning sign

\newcommand\warning[1][4ex]{%
  \renewcommand\stacktype{L}%
  \scaleto{\stackon[1.3pt]{\color{red}$\triangle$}{\tiny\bfseries !}}{#1}%
}


\newtheorem{exo}{Exercice}
\newcounter{ExoCounter}
\newtheorem{exercice}[ExoCounter]{Exercice}

\newcounter{counterCorrection}
\newtheorem{correction}[counterCorrection]{\color{red}{Correction}}


\theoremstyle{definition}

%\newtheorem{prop}[theorem]{Proposition}
%\newtheorem{\defi}[1]{
%\begin{tcolorbox}[width=14cm]
%#1
%\end{tcolorbox}
%}


%--------------------------------------- 
% Document
%--------------------------------------- 






\lstset{numbers=left, numberstyle=\tiny, stepnumber=1, numbersep=5pt}




% Header et footer

\pagestyle{fancy}
\fancyhead{}
\fancyfoot{}
\renewcommand{\headwidth}{\textwidth}
\renewcommand{\footrulewidth}{0.4pt}
\renewcommand{\headrulewidth}{0pt}
\renewcommand{\footruleskip}{5px}

\fancyfoot[R]{Olivier Glorieux}
%\fancyfoot[R]{Jules Glorieux}

\fancyfoot[C]{ Page \thepage }
\fancyfoot[L]{1BIOA - Lycée Chaptal}
%\fancyfoot[L]{MP*-Lycée Chaptal}
%\fancyfoot[L]{Famille Lapin}



\newcommand{\Hyp}{\mathbb{H}}
\newcommand{\C}{\mathcal{C}}
\newcommand{\U}{\mathcal{U}}
\newcommand{\R}{\mathbb{R}}
\newcommand{\T}{\mathbb{T}}
\newcommand{\D}{\mathbb{D}}
\newcommand{\N}{\mathbb{N}}
\newcommand{\Z}{\mathbb{Z}}
\newcommand{\F}{\mathcal{F}}




\newcommand{\bA}{\mathbb{A}}
\newcommand{\bB}{\mathbb{B}}
\newcommand{\bC}{\mathbb{C}}
\newcommand{\bD}{\mathbb{D}}
\newcommand{\bE}{\mathbb{E}}
\newcommand{\bF}{\mathbb{F}}
\newcommand{\bG}{\mathbb{G}}
\newcommand{\bH}{\mathbb{H}}
\newcommand{\bI}{\mathbb{I}}
\newcommand{\bJ}{\mathbb{J}}
\newcommand{\bK}{\mathbb{K}}
\newcommand{\bL}{\mathbb{L}}
\newcommand{\bM}{\mathbb{M}}
\newcommand{\bN}{\mathbb{N}}
\newcommand{\bO}{\mathbb{O}}
\newcommand{\bP}{\mathbb{P}}
\newcommand{\bQ}{\mathbb{Q}}
\newcommand{\bR}{\mathbb{R}}
\newcommand{\bS}{\mathbb{S}}
\newcommand{\bT}{\mathbb{T}}
\newcommand{\bU}{\mathbb{U}}
\newcommand{\bV}{\mathbb{V}}
\newcommand{\bW}{\mathbb{W}}
\newcommand{\bX}{\mathbb{X}}
\newcommand{\bY}{\mathbb{Y}}
\newcommand{\bZ}{\mathbb{Z}}



\newcommand{\cA}{\mathcal{A}}
\newcommand{\cB}{\mathcal{B}}
\newcommand{\cC}{\mathcal{C}}
\newcommand{\cD}{\mathcal{D}}
\newcommand{\cE}{\mathcal{E}}
\newcommand{\cF}{\mathcal{F}}
\newcommand{\cG}{\mathcal{G}}
\newcommand{\cH}{\mathcal{H}}
\newcommand{\cI}{\mathcal{I}}
\newcommand{\cJ}{\mathcal{J}}
\newcommand{\cK}{\mathcal{K}}
\newcommand{\cL}{\mathcal{L}}
\newcommand{\cM}{\mathcal{M}}
\newcommand{\cN}{\mathcal{N}}
\newcommand{\cO}{\mathcal{O}}
\newcommand{\cP}{\mathcal{P}}
\newcommand{\cQ}{\mathcal{Q}}
\newcommand{\cR}{\mathcal{R}}
\newcommand{\cS}{\mathcal{S}}
\newcommand{\cT}{\mathcal{T}}
\newcommand{\cU}{\mathcal{U}}
\newcommand{\cV}{\mathcal{V}}
\newcommand{\cW}{\mathcal{W}}
\newcommand{\cX}{\mathcal{X}}
\newcommand{\cY}{\mathcal{Y}}
\newcommand{\cZ}{\mathcal{Z}}







\renewcommand{\phi}{\varphi}
\newcommand{\ddp}{\displaystyle}


\newcommand{\G}{\Gamma}
\newcommand{\g}{\gamma}

\newcommand{\tv}{\rightarrow}
\newcommand{\wt}{\widetilde}
\newcommand{\ssi}{\Leftrightarrow}

\newcommand{\floor}[1]{\left \lfloor #1\right \rfloor}
\newcommand{\rg}{ \mathrm{rg}}
\newcommand{\quadou}{ \quad \text{ ou } \quad}
\newcommand{\quadet}{ \quad \text{ et } \quad}
\newcommand\fillin[1][3cm]{\makebox[#1]{\dotfill}}
\newcommand\cadre[1]{[#1]}
\newcommand{\vsec}{\vspace{0.3cm}}

\DeclareMathOperator{\im}{Im}
\DeclareMathOperator{\cov}{Cov}
\DeclareMathOperator{\vect}{Vect}
\DeclareMathOperator{\Vect}{Vect}
\DeclareMathOperator{\card}{Card}
\DeclareMathOperator{\Card}{Card}
\DeclareMathOperator{\Id}{Id}
\DeclareMathOperator{\PSL}{PSL}
\DeclareMathOperator{\PGL}{PGL}
\DeclareMathOperator{\SL}{SL}
\DeclareMathOperator{\GL}{GL}
\DeclareMathOperator{\SO}{SO}
\DeclareMathOperator{\SU}{SU}
\DeclareMathOperator{\Sp}{Sp}


\DeclareMathOperator{\sh}{sh}
\DeclareMathOperator{\ch}{ch}
\DeclareMathOperator{\argch}{argch}
\DeclareMathOperator{\argsh}{argsh}
\DeclareMathOperator{\imag}{Im}
\DeclareMathOperator{\reel}{Re}



\renewcommand{\Re}{ \mathfrak{Re}}
\renewcommand{\Im}{ \mathfrak{Im}}
\renewcommand{\bar}[1]{ \overline{#1}}
\newcommand{\implique}{\Longrightarrow}
\newcommand{\equivaut}{\Longleftrightarrow}

\renewcommand{\fg}{\fg \,}
\newcommand{\intent}[1]{\llbracket #1\rrbracket }
\newcommand{\cor}[1]{{\color{red} Correction }#1}

\newcommand{\conclusion}[1]{\begin{center} \fbox{#1}\end{center}}


\renewcommand{\title}[1]{\begin{center}
    \begin{tcolorbox}[width=14cm]
    \begin{center}\huge{\textbf{#1 }}
    \end{center}
    \end{tcolorbox}
    \end{center}
    }

    % \renewcommand{\subtitle}[1]{\begin{center}
    % \begin{tcolorbox}[width=10cm]
    % \begin{center}\Large{\textbf{#1 }}
    % \end{center}
    % \end{tcolorbox}
    % \end{center}
    % }

\renewcommand{\thesection}{\Roman{section}} 
\renewcommand{\thesubsection}{\thesection.  \arabic{subsection}}
\renewcommand{\thesubsubsection}{\thesubsection. \alph{subsubsection}} 

\newcommand{\suiteu}{(u_n)_{n\in \N}}
\newcommand{\suitev}{(v_n)_{n\in \N}}
\newcommand{\suite}[1]{(#1_n)_{n\in \N}}
%\newcommand{\suite1}[1]{(#1_n)_{n\in \N}}
\newcommand{\suiteun}[1]{(#1_n)_{n\geq 1}}
\newcommand{\equivalent}[1]{\underset{#1}{\sim}}

\newcommand{\demi}{\frac{1}{2}}
\geometry{hmargin=1.5cm, vmargin=1.5cm}




\begin{document}




\title{CH 1 - Etude de fonctions}
\vspace{1cm}
\begin{defi}
Soit $f$ une fonction. On dit qu'elle est définie sur un ensemble $D$ si pour tout $x$ de $D$ on peut associer une valeur à $f(x)$. 
\end{defi}


\begin{defi}
On dit qu'une fonction $f$ est croissante (respectivement décroissante) sur un intervalle $I$ si pour tout $x,y\in I$, 
$$x\leq y \implique f(x)\leq f(y) \quad \text{ resp. $f(x) \geq f(y)$} $$
\end{defi}

%\subsection{Représentations graphiques et domaine d'étude}
\begin{defi}
Soit $f$ une fonction définie sur $D\subset \R$. On appelle courbe représentative de $f$ (ou graphe), la courbe du plan notée $\cC_f$ et définie par :
$$ \cC_f = \{ (x,f(x)) \in \R^2 \, |\, x\in D\}.$$
\end{defi}

\vspace{1cm}
\section{Dérivation}


	\begin{defi} D\'erivabilit\'e d'une fonction en un point:\\
		\noindent Soient $f: I\rightarrow \R$ et $x_0\in I$. \vsec
		\begin{itemize}
			\item[$\bullet$] On dit que $f$ est d\'erivable en $x_0$ si 
   $x\mapsto \frac{f(x)-f(x_0)}{x-x_0}$ admet une limite quand $x$ tends vers $x_0$
			\item[$\bullet$] Si cette limite existe, elle est not\'ee $f'(x_0)$ et est appel\'ee le nombre d\'eriv\'ee de $f$ en $x_0$:
$$ \lim_{x\tv x_0} \frac{f(x)-f(x_0)}{x-x_0} = f'(x_0).$$
\end{itemize}
	\end{defi}
 

\begin{prop} Soient $u$ et $v$ deux fonctions d\'erivables sur $I$.
\begin{itemize}
\item[$\star$] La somme $u+v$ est d\'erivable sur $I$ et $$(u+v)'=u' +v' $$

\item[$\star$] Le produit $uv$ est d\'erivable sur $I$ et $$(uv)'=u'v +uv'$$

\item[$\star$] Si $v$ ne s'annule pas sur $I$, alors le quotient de $u$ par $v$ est d\'erivable sur $I$ et $$\left(\dfrac{u}{v}\right)'=\frac{u'v-v'u}{v^2}$$

\item[$\star$] En particulier si $v$ ne s'annule pas sur $I$, alors l'inverse de $v$ est d\'erivable sur $I$ et $$\left(\dfrac{1}{v}\right)^{\prime}=\frac{-v'}{v^2}$$

\end{itemize}
\end{prop}


\begin{prop} \underline{Si  $f$ est d\'erivable} alors
\begin{itemize}
\item[$\bullet$] $f$ est croissante sur $I$ $\Longleftrightarrow$ $\forall x\in I,\ f'(x)\geqslant 0$.
\item[$\bullet$] $f$ est d\'ecroissante sur $I$ $\Longleftrightarrow$ $\forall x\in I,\ f'(x)\leqslant 0$.
\item[$\bullet$] $f$ est constante sur $I$ $\Longleftrightarrow$ $\forall x\in I,\ f'(x)= 0$.
\end{itemize}
\end{prop}



\begin{prop} Tangente:\\
\noindent Si la fonction $f$ est d\'erivable en $x_0$ alors la courbe $\mathcal{C}_f$ admet au point d'abscisse $x_0$ une tangente qui a pour \'equation: 
$$ y- f(x_0) = f'(x_0)(x-x_0)$$
\end{prop}



\begin{rem}
La connaissance de la tangente $T$ \`a $\mathcal{C}_f$ permet de tracer la courbe au voisinage du point $M$ d'abscisse $x_0$. Pour un trac\'e encore plus pr\'ecis, on \'etudie souvent la position de la courbe par rapport \`a la tangente, \`a savoir le signe de $f(x)-y=f(x)-f(x_0)-f^{\prime}(x_0)(x-x_0)$.
\end{rem}


\section{Composition}

\begin{defi}
Soit $E$,  $F$ et $G$ trois ensembles. Soit $f : E\tv F$ et $g : F \tv G$ deux fonctions, alors on définit 
$g\circ f$, dite composée de  $f$ et $g$, et prononcée  "g rond f". C'est la fonction 
de $E$ vers $G$ qui vérifie pour tout $x\in E$ :
$$g\circ f (x) = g(f(x))$$
\end{defi}

$\warning$ L'ordre est important !  Avec les notations de la définition, on ne pourrait pas considérer $f\circ g$ en effet $g$ 'mange' un élément de $F$ et renvoie un élément de $G$ mais alors $f(g(x))$ n'a pas de sens, car $f$ 'mange' un élément de $E$ et non de $G$. Quand bien même les deux ont du sens, les fonctions $f\circ g$ et $g\circ f$ ne sont généralement pas égale comme le montre l'exemple suivant. 
 $\warning$
 
 \begin{exemples}
     \item $f(x)=x+1$ et $g(x)=x^2$ On a $$f\circ g(x)= x^2+1 \quadet g\circ f (x) = (x+1)^2=x^2+2x+1$$
 \end{exemples}
 
 
\begin{prop}
Soit $I$,  $J$ et $K$ trois intervalles de $\R$ . Soit $f : I\tv J$ et $g : J \tv K$ deux fonctions dérivables. Alors $ g\circ f$ est dérivable et pour tout $x\in I$:
$$(g\circ f)'(x) = f'(x) \times g'\circ f (x) = f'(x)  g'(f (x) )$$
\end{prop} 
 

\begin{exemples} D\'eriv\'ees des compos\'ees de r\'ef\'erence : soit $u$ une fonction d\'erivable sur $I$ et $n\in\N^{\star}$. 
\begin{itemize}
\item[$\bullet$] La fonction $u^n$ est d\'erivable sur $I$ et $$(u^n)'=n u' u^{n-1}$$
%\item[$\bullet$] Si $\forall x \in I, u(x)\neq 0$ alors la fonction $\ddp\frac{1}{u^n}$ est d\'erivable sur $I$ et $$\ddp \left(\frac{1}{u^n}\right)'=\frac{-n u' u^{n-1}}{u^{2n}}=\frac{n u' }{u^{n+1}} $$ \\
\item[$\bullet$] Si $\forall x \in I, u(x)>0 $ alors $\sqrt{u}$ est d\'erivable sur $I$ et $$(\sqrt{u})'=\frac{u'}{2\sqrt{u}} $$ 
\item[$\bullet$] Si $\forall x \in I,  u(x)>0 $ alors la fonction $\ln{{u}}$ est d\'erivable sur $I$ et $$(\ln u)'= \frac{u'}{u}$$ 
\item[$\bullet$] La fonction $e^{u}$ est d\'erivable sur $I$ et $$(e^u)'=u' \times e^u$$ 
% \item[$\bullet$] La fonction $\cos{(u)}$ est d\'erivable sur $I$ et $$(\cos (u))'=-u' \sin(u)$$
% \item[$\bullet$] La fonction $\sin{(u)}$ est d\'erivable sur $I$ et $$(\sin(u))'=u' \cos(u)$$ 
% \item[$\bullet$] Si $\forall x \in I, u(x)\not\equiv \frac{\pi}{2} [\pi]$ alors la fonction $\tan{{u}}$ est d\'erivable sur $I$ et $$(\tan u)'= u' \frac{1}{\cos^2(u)}$$
\end{itemize}
\end{exemples}
\vspace{-1cm}
\section{Limites}

Pour l'instant vous devez connaître les limites suivantes : 
\begin{itemize}
    \item[$\bullet$]$ \ddp \lim_{x\tv +\infty} \exp(x) = +\infty\quadet \lim_{x\tv -\infty} \exp(x) = 0$ 
    \item[$\bullet$]$ \ddp \lim_{x\tv +\infty} \ln(x) = +\infty\quadet \lim_{x\tv0} \ln(x) = -\infty$
     \item[$\bullet$]$ \ddp\lim_{x\tv +\infty} x^n= +\infty\quadet \lim_{x\tv -\infty} x^n = \pm \infty \quad ( \text{en fonction de la parité de $n$})$
\item[$\bullet$]$ \ddp\lim_{x\tv +\infty} \sqrt{x}= +\infty \quadet  \ddp\lim_{x\tv +\infty} \frac{1}{x}=0$
      \item[$\bullet$]$ \ddp\lim_{x\tv 0^+} \frac{1}{x}=+\infty \quadet  \ddp\lim_{x\tv 0^-} \frac{1}{x}=-\infty$
            
\end{itemize}


\begin{prop}
Soient $I$ et $J$ deux intervalles de $\R$, $f: I\rightarrow \R$ et $g: J\rightarrow \R$. Soient $x_0$ un \'el\'ement de $I$ ou une borne (finie ou infinie) de $I$, $y_0$ un \'el\'ement de $J$ ou une borne (finie ou infinie) de $J$ et $\ell\in\R\cup\lbrace -\infty,+\infty\rbrace$. Alors 
$$\left\lbrace \begin{array}{lll}
\lim\limits_{x\to x_0} f(x)&=&y_0 \vsec\\
\lim\limits_{y\to y_0} g(y)&=&\ell
\end{array}\right. \quad 
\Longrightarrow \quad \lim\limits_{x\to x_0} g\circ f(x)=\ell.$$
\end{prop}


\begin{theorem} Th\'eor\`eme des croissances compar\'ees. Pour tous r\'eels $a>0$ et $b>0$, on a: 
\begin{itemize}
\item[$\bullet$] $\lim\limits_{x\to +\infty}\ddp\frac{(\ln{x})^b}{x^a}= 0$ \hspace{3cm}et \quad $\lim\limits_{x\to 0^+}x^a(\ln{x})^b= 0$\vsec
\item[$\bullet$]  $\lim\limits_{x\to +\infty} \ddp \frac{x^{a}}{e^{bx}} = 0$ \hspace{3.2cm}et \quad $\lim\limits_{x\to -\infty} \ddp x^a e^{bx} = 0$\vsec
\end{itemize}
\end{theorem}

\begin{proof}
Remarquez tout d'abord que l'on peut se ramener à $a=b=1$.
    \begin{itemize}
        \item[$\bullet$] Considérer $f(x) = \frac{\ln(x)}{\sqrt{x}}$ pour montrer que $\lim\limits_{x\to +\infty}\ddp\frac{\ln{x}}{x}= 0$ et changement de variable $y=\frac{1}{x}$

         \item[$\bullet$] Changement de variable $y=e^x$ puis $y=-x$
    \end{itemize}
\end{proof}

\begin{prop} Taux d'accroissements en 0
\begin{itemize}
\begin{minipage}{0.4\textwidth}
   \item[$\bullet$] $\ddp\lim_{x\tv0} \frac{\sin(x)}{x}= 1$ 
\item[$\bullet$] $\ddp\lim_{x\tv0} \frac{\ln(x+1)}{x}= 1$ 
\end{minipage}
\begin{minipage}{0.4\textwidth}
\item[$\bullet$] $\ddp\lim_{x\tv0} \frac{\exp(x)-1}{x}= 1$
\item[$\bullet$] $\ddp\lim_{x\tv0} \frac{(1+x)^{\alpha}-1 }{x}= 1$
\end{minipage}

\end{itemize}
\end{prop}
\begin{proof}
    Considérer la bonne fonction 
\end{proof}

\begin{prop}
$\lim_{x\tv0} \ddp\frac{\cos(x)-1}{x^2}= \frac{-1}{2}$    
\end{prop}
    \begin{proof}
        Changement de variable $x=2y$ puis manipulation algébrique sur $\cos$ 
    \end{proof}
\section{Théorème des valeurs intermédiaires}

\begin{theorem} Soit $f$ une fonction continue sur un intervalle $[a,b]$.\\
Alors pour tout $y$ compris entre $f(a)$ et $f(b)$, il existe $c \in [a,b]$ tel que $y=f(c)$.\vsec

 Si de plus la fonction est strictement monotone alors le réel $c$ est unique
\end{theorem}


\end{document}
