\documentclass[a4paper, 11pt,reqno]{article}
\input{macro/package.tex}
\input{macro/environement}
% Header et footer

\pagestyle{fancy}
\fancyhead{}
\fancyfoot{}
\renewcommand{\headwidth}{\textwidth}
\renewcommand{\footrulewidth}{0.4pt}
\renewcommand{\headrulewidth}{0pt}
\renewcommand{\footruleskip}{5px}

\fancyfoot[R]{Olivier Glorieux}
%\fancyfoot[R]{Jules Glorieux}

\fancyfoot[C]{ Page \thepage }
\fancyfoot[L]{1BIOA - Lycée Chaptal}
%\fancyfoot[L]{MP*-Lycée Chaptal}
%\fancyfoot[L]{Famille Lapin}

\input{macro/newcommand.tex}
\geometry{hmargin=1.0cm, vmargin=2.5cm}



\begin{document}
\tableofcontents
\title{Chapitre 15  : Polynômes}
% debut
%------------------------------------------------


%----------------------------------------------------
%-----------------------------------------------------
%-------------------------------------------------------
%----------------------------------------------------
%-----------------------------------------------------
%-------------------------------------------------------


\noindent Dans tout ce chapitre, $\bK$ d\'esigne $\R$ ou $\bC$.

%----------------------------------------------------
%-----------------------------------------------------
%----------------------------------------------------
%-----------------------------------------------------
%-------------------------------------------------------
\section{G\'en\'eralit\'es sur les polyn\^omes. Degr\'e et coefficient dominant.}

%----------------------------------------------------
%-----------------------------------------------------
\subsection{D\'efinitions et notations}

{\noindent

	\begin{defi} \textbf{Notion de polyn\^omes:}
		\begin{itemize}
			\item[$\bullet$] On appelle polyn\^ome \`a coefficients dans $\bK$ toute fonction $P:\ \bK\rightarrow\bK$ qui est de type:\\


			\item[$\bullet$] $a_0,a_1,\dots,a_n$ sont appel\'es \dotfill \vsec
			\item[$\bullet$] $\bK[X]$ est l'ensemble des polyn\^omes \`a coefficients dans $\bK$.
		\end{itemize}
	\end{defi}

}


\begin{exemples}
	\begin{itemize}
		\item[$\bullet$] Donner un exemple de polyn\^{o}me de $\R[X]$: \dotfill\vsec
		\item[$\bullet$] Donner un exemple de polyn\^{o}me de $\bC[X]$: \dotfill\vsec
		\item[$\bullet$] Donner des exemples de non polyn\^{o}me: \dotfill\vsec
	\end{itemize}
\end{exemples}
%\vsec

%----------------------------------------------------
%-----------------------------------------------------
{\noindent

\begin{defi} \textbf{Polyn\^omes particuliers: notations usuelles:}\\
	\noindent On d\'efinit les polyn\^omes particuliers suivants:
	\begin{itemize}
		\item[$\bullet$] le polyn\^ome $\mathbf{1}:\ x\mapsto 1$\dotfill \vsec
		\item[$\bullet$] le polyn\^ome $X:\ x\mapsto x$ \dotfill \vsec
		\item[$\bullet$] pour tout $n\in\N^{\star}$, le polyn\^ome $X^n:\ x\mapsto x^n$ appel\'e \dotfill\vsec
	\end{itemize}

\end{defi}

}\vsec

\noindent Par exemple, le polyn\^ome r\'eel $P=2X^4-5X^3+10X-6$ est une fonction qui est d\'efinie par\\\vspace{2cm}


\begin{rem} \noindent \warning  Il faut bien comprendre ici que $\mathbf{1},X,X^2,\dots X^n$ d\'esignent des fonctions et pas des nombres. En particulier on ne confondra pas les notations suivantes:\\
	\begin{itemize}
		\item[$\star$] $ax^2+bx+c$  qui est \dotfill\vsec
		\item[$\star$] $aX^2+bX+c$  qui est \dotfill\vsec
		\item[$\star$] $ax^2+bx+c=0$  qui est \dotfill\vsec
		\item[$\star$] $aX^2+bX+c=0$  qui est \dotfill\vsec
	\end{itemize}
	De m\^{e}me, $P$ est un polyn\^{o}me donc une \dotfill alors que $P(x)$ est un \dotfill
\end{rem}


%----------------------------------------------------
%-----------------------------------------------------
\subsection{Unicit\'e des coefficients}

{\noindent

	\begin{theorem} \textbf{Identification des coefficients d'un polyn\^ome}
		\begin{itemize}
			\item[$\bullet$] $P=0\Longleftrightarrow \forall x\in\bK,\ P(x)=0\Longleftrightarrow$ \dotfill\vsec

			\item[$\bullet$] $P=Q\Longleftrightarrow \left\|\begin{array}{ll} \forall x\in\bK & P(x)=Q(x) \\ \forall x\in\bK & \sum\limits_{k=0}^n a_kx^k =\sum\limits_{k=0}^n b_k x^k\end{array}\right.
				      \vsec \\ \phantom{sdf} \quad \hspace{0.2cm}\Longleftrightarrow \left\|\begin{array}{l} \hbox{les coefficients de}\ P \hbox{ et de}\ Q\ \hbox{sont \'egaux.} \\
					      \phantom{'} \dotfill
				      \end{array}\right.$
		\end{itemize}
	\end{theorem}
}



{\footnotesize \begin{exercice}
		\begin{enumerate}
			\item Factoriser le polyn\^ome $P=2X^3-6X^2+5X-1$.
			\item Montrer qu'il existe trois r\'eels $(a,b,c)\in\R^3$ tels que l'on ait: $\forall x\in\R\setminus\left\lbrace  -1,-\ddp\frac{2}{3}\right\rbrace,\quad \ddp\frac{6x^2+10x+3}{(3x+2)(x+1)}=a+\ddp\frac{b}{3x+2}+\ddp\frac{c}{x+1}.$
		\end{enumerate}
	\end{exercice}
}

%----------------------------------------------------
%-----------------------------------------------------
\subsection{Op\'erations sur les polyn\^omes}

\noindent Comme les polyn\^{o}mes sont des cas particuliers de fonctions, on peut d\'efinir la somme de deux polyn\^{o}mes, le produit et la compos\'ee.\vsec

{\noindent

	\begin{theorem} \textbf{Somme, multiplication, composition de polyn\^omes}\\
		\noindent Soit $P=\sum\limits_{k=0}^n a_k X^k\in\bK[X]$, $Q=\sum\limits_{k=0}^n b_k X^k\in\bK[X]$ et $\lambda\in\bK$.\vsec
		\begin{itemize}
			\item[$\bullet$] Somme: $P+Q=$\dotfill est un polyn\^ome \vsec
			\item[$\bullet$] Multiplication par un scalaire: $\lambda P=$\dotfill est un polyn\^ome  \vsec
			\item[$\bullet$] Multiplication de deux polyn\^omes: $PQ: x\in\bK\mapsto (PQ)(x)=P(x)Q(x)$ est un polyn\^ome\vsec
			      %\item[$\bullet$] Composition de deux polyn\^omes: $P(Q): x\in\bK\mapsto (P\circ Q)(x)=P(Q(x))$ est un polyn\^ome  \vsec
		\end{itemize}
	\end{theorem}
}

{\footnotesize \begin{exercice}
		\begin{enumerate}
			\item $P=X^3+2X^2+1$ et $Q=X^2-2X+1$. Calcul de $PQ$.
			      %\item[$\star$] $P=X^3+1$ et $Q=X^2+1$. Calcul de $P(Q)$
			\item  On d\'efinit souvent: $P(X+a)$ qui est la compos\'ee du polyn\^ome $P$ et du polyn\^ome $X+a$. Calculer $P(X+a)$ avec $P=X^3-3X^2+1$.
		\end{enumerate}
	\end{exercice}
}\vsec\vsec

{\noindent

	\begin{prop} Simplification avec les polyn\^{o}mes:\vsec
		\begin{itemize}
			\item[$\bullet$] $\forall (P,Q)\in (\bK[X])^2,\quad P\times Q=0 \Longleftrightarrow$ \dotfill\vsec
			\item[$\bullet$] $\forall (P,Q)\in (\bK[X])^2,\quad \left(P\times Q=P\times R\ \hbox{et}\ P\not= 0\right) \Longrightarrow $ \dotfill \vsec
		\end{itemize}
	\end{prop}
}


%----------------------------------------------------
%-----------------------------------------------------
\subsection{Degr\'e et coefficient dominant d'un polyn\^ome}



\begin{defi} \textbf{Degr\'e d'un polyn\^ome}\\
	\noindent Soit $P\in\bK\lbrack X\rbrack$ s'\'ecrivant: $P=\sum\limits_{k=0}^n a_kX^k$.
	\begin{itemize}
		\item[$\bullet$] Le degr\'e de $P$ est not\'e $\deg (P)$ et est d\'efini par:\vsec
		      \begin{itemize}
			      \item[$\star$] si $P$ est le polyn\^{o}me nul alors \dotfill \vsec
			      \item[$\star$] Sinon \dotfill\vsec
		      \end{itemize}
		\item[$\bullet$] Ainsi, si $P$ n'est pas le polyn\^ome nul:
		      $$\deg(P)=n \Longleftrightarrow  \hspace{11cm}   $$
		\item[$\bullet$] $a_n$ est alors appel\'e \dotfill \vsec
		\item[$\bullet$] $a_nX^n$ est alors appel\'e \dotfill \vsec
		\item[$\bullet$] On dit qu'un polyn\^ome est unitaire si \dotfill \vsec
		\item[$\bullet$] On note $\bK_n\lbrack X\rbrack$ l'ensemble des polyn\^omes de $\bK\lbrack X\rbrack$ \dotfill \vsec
	\end{itemize}
\end{defi}


\begin{prop}
	Si $P=Q$ alors $$ $$
	Réciproquement si $ deg(P) \neq deg(Q)$ alors
	$$ $$
\end{prop}
\begin{remarques}
	\item Bien s\^{u}r deux polyn\^{o}mes peuvent avoir le m\^{e}me degr\'e et ne pas \^etre \'egaux !

	\item[$\bullet$] Si $P$ est un polyn\^ome constant non nul alors \dotfill \vsec

	\item[$\bullet$]  Si $P=\sum\limits_{k=0}^n a_kX^k$ alors le degr\'e de $P$ n'est pas forc\'ement $n$.
	\begin{itemize}
		\item[$\star$] Si $a_n\not= 0$ alors \dotfill \vsec
		\item[$\star$] Si $a_n=0$, on a \dotfill \vsec
	\end{itemize}
	Ainsi la notation $P=\sum\limits_{k=0}^n a_k X^k$ signifie juste que \dotfill

\end{remarques}\vsec



\begin{prop} \textbf{Degr\'e d'une somme, d'un produit, d'une compos\'ee, d'un polyn\^{o}me d\'eriv\'ee}\\
	\noindent Soient $(P,Q)\in \bK[X]^2$ et $\lambda\in\bK$.
	\begin{itemize}
		\item[$\bullet$] Somme: $\deg(P+Q)$ \dotfill\vsec
		      \begin{itemize}
			      \item[$\star$] Si $\deg(P)\not= \deg(Q)$ alors: $ \deg(P+Q)=$\dotfill\vsec
			      \item[$\star$] Si $\deg(P)= \deg(Q)$ alors: $ \deg(P+Q)$\dotfill\vsec
		      \end{itemize}
		\item[$\bullet$] Produit par un scalaire: $\deg(\lambda P)=$\dotfill \vsec
		\item[$\bullet$] Produit: $\deg(PQ)=$\dotfill \vsec
		\item[$\bullet$] Composition: $\deg(P(Q))=$\dotfill \vsec
		\item[$\bullet$] D\'erivation: $\deg (P^{\prime})=$\dotfill \vsec
	\end{itemize}
\end{prop}


{\footnotesize \begin{exercice}
	\begin{enumerate}
		\item Soit $P$ un polyn\^{o}me de degr\'e $n$ et $Q=P(X+1)-P$. Montrer que $\deg (Q)\leq n$, puis que $\deg (Q)\leq n-1$.
		\item Soit $P$ un polyn\^{o}me de degr\'e $n$. D\'eterminer le degr\'e de $Q=\sum\limits_{k=0}^n P^{(k)}$.
	\end{enumerate}
\end{exercice}
}\vsec





\colorbox{gray!40}{M\'ethodes pour d\'eterminer le coefficient dominant et le degr\'e d'un polyn\^ome $P$}\\

\hspace{-1cm}
\begin{dboxminipage}{1\textwidth}
	\begin{itemize}
		\item[$\bullet$] Calculer explicitement $P$ (lorsque cela est possible, plut\^{o}t pour les polyn\^{o}mes de petit degr\'e).
		\item[$\bullet$] \'Etudier uniquement les mon\^omes de plus haut degr\'e du polyn\^ome et regarder s'ils sont nuls ou pas:
		      \begin{itemize}
			      \item[$\star$] On \'ecrit $P$ sous la forme $P=a_nX^n+Q$ avec $Q\in\bK_{n-1}\lbrack X\rbrack$.
			      \item[$\star$]  On calcule $a_n$:
			            \begin{itemize}
				            \item[$\circ$] Si $a_n\not= 0$ alors $\deg{P}=\ldots \ldots$ et le coefficient dominant est \ldots \ldots.
				            \item[$\circ$] Si $a_n= 0$ alors $\deg{P}\leq \ldots \ldots$ et on \'ecrit $P$ sous la forme: $P=a_{n-1}X^{n-1}+Q$ avec $Q\in\bK_{n-2}\lbrack X\rbrack$.\\
				                  \hspace{1.3cm} On calcule $a_{n-1}$ pour savoir s'il est nul ou pas.
			            \end{itemize}
			      \item[$\star$] On continue ainsi jusqu'\`{a} tomber sur un terme non nul.
		      \end{itemize}
		\item[$\bullet$] Utiliser les r\'esultats sur le degr\'e d'une somme, d'un produit, d'une compos\'ee et d'une d\'eriv\'ee.
		\item[$\bullet$] Lorsqu'il s'agit d'une suite de polyn\^omes d\'efinie par r\'ecurrence, on raisonne par r\'ecurrence.
		      \begin{itemize}
			      \item[$\star$] Pour deviner le r\'esultat, il est conseill\'e de regarder les premiers termes de la suite: $P_0,P_1,P_2\dots$.
			      \item[$\star$] On conjecture alors le r\'esultat.
			      \item[$\star$] On le d\'emontre par r\'ecurrence en utilisant la relation de r\'ecurrence.
		      \end{itemize}
	\end{itemize}

\end{dboxminipage}

\setlength\fboxrule{0.5pt}

{\footnotesize \begin{exercice}
		\begin{enumerate}
			\item \'Etudier le degr\'e et le coefficient dominant de $P=(X+1)^6-X^6+2$.
			\item On d\'efinit $(P_n)_{n\in\N}$ par $\left\lbrace\begin{array}{l}
					      P_0=1\vsec   \\
					      P_1=-2X\vsec \\
					      \forall n\in\N,\ P_{n+2}=-2XP_{n+1}-2(n+1)P_n.
				      \end{array}\right.$ \\
			      Calculer $P_2$ et $P_3$. Trouver le degr\'e et le coefficient de plus haut degr\'e de $P_n$.
			      %\item[$\bullet$] Soit $(P_n)_{n\in\N}$ la suite de polyn\^omes d\'efinie par r\'ecurrence par $\left\lbrace\begin{array}{l}
			      %P_0=1\qquad P_1=X\vsec\\ 
			      %\forall n\in\N,\quad P_{n+2}=2XP_{n+1}-P_n.
			      %\end{array}\right.$
			      %Ces polyn\^omes sont appel\'es les polyn\^omes de Tchebychev. Calculer $P_2$, $P_3$ et $P_4$. Trouver le degr\'e et le coefficient de plus haut degr\'e de $P_n$.
			      %\item[$\bullet$]  Soit la fonction $f$ d\'efinie sur $\left\rbrack -\ddp\frac{\pi}{2},\ddp\frac{\pi}{2}\right\lbrack$ par: $f(x)=\ddp\frac{1}{\cos{x}}$. Calculer $f^{\prime}$ et $f^{\prime\prime}$. Montrer l'existence, pour tout $n\in\N$, d'un polyn\^ome $P_n$ tel que, pour tout $x\in\left\rbrack -\ddp\frac{\pi}{2},\ddp\frac{\pi}{2}\right\lbrack$: $f^{(n)}(x)=\ddp\frac{P_n(\sin{x})}{(\cos{x})^{n+1}}.$
			      %Trouver une relation entre $P_{n+1}$, $P_n$ et $P^{\prime}_n$. D\'eterminer le degr\'e de $P_n$ et le mon\^ome de plus haut degr\'e de $P_n$.
			      %\item[$\bullet$] D\'eterminer le degr\'e et le coefficient dominant de $P=(X+1)^n-(X-1)^n$. 
			      %\item[$\bullet$] Soit $a$ un nombre r\'eel non nul et $P$ un polyn\^ome de degr\'e $n$. D\'eterminer le degr\'e du polyn\^ome $P(X+a)-P.$ 
			      %\item[$\bullet$] On d\'efinit $(P_n)_{n\in\N}$ par $\left\lbrace\begin{array}{l} 
			      %P_0=1\vsec\\
			      %P_1=X\vsec\\
			      %\forall n\in\N,\ P_{n+2}=XP_{n+1}+\left( 1-\ddp\frac{X^2}{4} \right)P_n.
			      %              \end{array}\right.$ 
			      %\begin{itemize}
			      %\item[$\star$] Calculer $P_2$ et $P_3$.
			      %\item[$\star$] Montrer par r\'ecurrence que pour tout $n\in\N$: $\deg (P_n)\leq n$.
			      %\item[$\star$] En d\'eduire que pour tout $n\in\N$: $a_{n+2}=a_{n+1}-\ddp\frac{a_n}{4}$ avec $a_n$ coefficient d'indice $n$ de $P_n$.
			      %\item[$\star$] Calculer $(a_n)_{n\in\N}$ puis montrer que $\deg (P_n)=n$.
			      %\end{itemize}              
		\end{enumerate}
	\end{exercice}
}

% 

%----------------------------------------------------
%-----------------------------------------------------
%----------------------------------------------------
%-----------------------------------------------------
%-------------------------------------------------------
\section{Polyn\^omes et d\'erivation}

\noindent Les polyn\^omes sont des fonctions tr\`es r\'eguli\`eres qui sont continues, et d\'erivables une infinit\'e de fois. On peut donc calculer leurs  d\'eriv\'ees successives.


%----------------------------------------------------
%-----------------------------------------------------
\subsection{Polyn\^ome d\'eriv\'e, d\'eriv\'ees successives d'un polyn\^ome}




\begin{defi} D\'erivation d'un polyn\^ome\\

	$\bullet$ D\'eriv\'ee des fonctions $X^n$, $n\in\N$:\vsec\vsec
	\begin{itemize}
		\item[$\star$] $n=0$: $(X^0)^{\prime}=$\dotfill \vsec
		\item[$\star$] $n>0$: $(X^n)^{\prime}=$\dotfill \vsec
	\end{itemize}

	\quad

	$\bullet$ $P=\sum\limits_{k=0}^n a_kX^k$ un polyn\^ome de $\bK$ de degr\'e $n$ ($a_n\not= 0$).\vsec
	\begin{itemize}
		\item[$\star$] $n=0$: la d\'eriv\'ee d'un polyn\^ome constant est \dotfill  \vsec
		\item[$\star$] $n>0$: $P^{\prime}=$\dotfill \vsec
	\end{itemize}

\end{defi}
\vsec\vsec

%\noindent\ {D\'eriv\'ees d'ordre sup\'erieur: exemples usuels \`a conna\^itre}\vsec

\begin{defi}
	La dérivée $n$-éme d'un polynôme est définie par réccurrence. On a
	$$\left\{
		\begin{array}{ccl}
			P^{(0)}   & = & P                                \\
			P^{(n+1)} & = & (P^{(n)})' \quad \forall n\geq 0
		\end{array}
		\right.$$
\end{defi}


{\footnotesize \begin{exercice}
	\begin{enumerate}
		\item Calculer les d\'eriv\'ees suivantes: $(X^2)^{\prime}$, $(X^2)^{(2)}$, $(X^2)^{(3)}$.
		\item Calculer les d\'eriv\'ees suivantes: $(X^n)^{(0)}$, $(X^n)^{\prime}$, $(X^n)^{(2)}$, $(X^n)^{(3)}$, $(X^n)^{(n)}$, $(X^n)^{(n+1)}$.
	\end{enumerate}
\end{exercice}
}\vsec

{\noindent

	\begin{prop} Exemples usuels de d\'eriv\'ees successives d'un polyn\^{o}me:
		\begin{itemize}
			\item[$\bullet$] Pour tous entiers naturels $n$ et $p$, on a
			      $$(X^n)^{(p)}= \hspace{8cm}  $$
			\item[$\bullet$] Pour tous entiers naturels $n$ et $p$, on a
			      $$((X-a)^n)^{(p)}= \hspace{8cm}  $$
			\item[$\bullet$] Soit $P$ un polyn\^ome de $\bK$ de degr\'e $n$. Si $P=\ddp\sum\limits_{k=0}^n a_kX^k$ avec $a_n\not= 0$ alors, on a
			      $$P^{(p)}= \hspace{8cm}  $$
		\end{itemize}
	\end{prop}
}

{\footnotesize \begin{exercice}
		\begin{enumerate}
			\item Calculer les d\'eriv\'ees premi\`{e}re, seconde et troisi\`{e}me de $P=5X^6+4X^3+X^2$.
			\item Faire de m\^{e}me avec $Q=(2+3i)X^4-(5-i)X^3+iX-8$.
		\end{enumerate}
	\end{exercice}
}\vsec

% 


\begin{prop}
	Soit $P\in\bK$, on a:\vsec
	\begin{itemize}
		\item[$\bullet$] $P^{\prime}=0\Longleftrightarrow$ \dotfill \vsec
		\item[$\bullet$]  Si $P$ n'est pas constant alors: $\deg(P^{\prime})=$\dotfill\vsec
		\item[$\bullet$]  Si $\deg(P)=n$ avec $n\in\N$ alors $P^{(n+1)} = $\dotfill \vsec
	\end{itemize}
\end{prop}




%----------------------------------------------------
%-----------------------------------------------------
\subsection{Op\'erations sur la d\'erivation}%, formule de Leibniz}

{\noindent

	\begin{prop}
		Soient $P$ et $Q$ deux polyn\^omes de $\bK[X]$ et soit $\lambda \in\bK$.
		\begin{itemize}
			\item[$\bullet$] $(P+Q)^{\prime}=$\dotfill\vsec
			\item[$\bullet$]  $(\lambda P)^{\prime}=$\dotfill\vsec
			\item[$\bullet$]  $(PQ)^{\prime}=$\dotfill\vsec
			      %\item[$\bullet$] $\left(P(Q)\right)^{\prime}=$\dotfill\vsec 
			      %\item[$\bullet$] Formule de Leibniz: $(PQ)^{(n)}=$\dotfill\vsec 
		\end{itemize}
	\end{prop}
}

{\footnotesize \begin{exercice}
		Soit $(P_n)_{n\in\N}$ la suite de polyn\^omes d\'efinie par r\'ecurrence par: $\left\lbrace\begin{array}{l}
				P_0=1\vsec \\
				\forall n\in\N,\quad P_{n+1}=(2n+1)XP_n-(X^2+1)P_n^{\prime}.
			\end{array}\right.$ Calculer $P_1$, $P_2$ et $P_3$. Trouver le degr\'e et le coefficient dominant de $P_n$.
	\end{exercice}}

%{\footnotesize \begin{exercice} 
%\begin{itemize}
%\item[$\bullet$] Calculer la d\'eriv\'ee $n$-i\`eme de $P=Q(aX+b)$ avec $a$ et $b$ deux scalaires.
%\item[$\bullet$] Calculer la d\'eriv\'ee $n$-i\`eme de $P=(X-a)^n(X-b)^n$. En d\'eduire $((X-a)^{2n})^{(n)}$. En d\'eduire la relation $\ddp \sum\limits_{k=0}^n \binom{n}{k}^2=\binom{2n}{n}$.
%\end{itemize}
%\end{exercice}}


%----------------------------------------------------
%-----------------------------------------------------
%\subsection{Formule de Taylor-Lagrange pour les polyn\^omes}
%
%\hspace{-0.5cm}  {\noindent  
%
%\begin{theorem}
%Soit $P$ un polyn\^ome de $\bK$ de degr\'e $n$ et soit $a$ un \'el\'ement quelconque de $\bK$.\vsec
%\begin{itemize}
%\item[$\bullet$] $P=$\dotfill \vsec\vsec
%\item[$\bullet$] Pour $a=0$: $P=$\dotfill \vsec
%\end{itemize}
%\end{theorem}
% }
%
%
%\begin{rem}
%Cette formule dit en particulier que pour tout polyn\^ome $P\in\bK$ de degr\'e $n$ avec $P=\sum\limits_{k=0}^n a_kX^k$, alors\\
%
%
%\end{rem}
%
%
%{\footnotesize 
%\begin{exercice}
%{\footnotesize Montrer qu'il existe un unique polyn\^ome de degr\'e inf\'erieur ou \'egal \`a 3 et le calculer tel que $P(2)=5$, $P^{\prime}(2)=10$, $P^{(2)}(2)=9$, $P^{(3)}(2)=6$. }
%\end{exercice}}


%

%----------------------------------------------------
%-----------------------------------------------------
%----------------------------------------------------
%-----------------------------------------------------
%-------------------------------------------------------
\section{Racines d'un polyn\^ome}

%----------------------------------------------------
%-----------------------------------------------------
\subsection{D\'efinition et caract\'erisation}

{\noindent

	\begin{defi}
		Soit $(P,Q)\in\bK[X]^2$. On dit que $Q$ divise $P$ si:\\


	\end{defi}
}\vsec\vsec

\begin{defi}
	Soit $P\in\bK[X]$ et $a\in\bK$. \\
	On dit que $a$ est \emph{racine de $P$} (ou un zéro de $P$) si  $$ $$
	Soit $k\in \N^*$. On dit que $a$ est\emph{ racine de multiplicité $k$  de  $P$} si  $$ $$\vspace{1cm}
\end{defi}

\begin{theorem}\textbf{Caract\'erisation d'une racine d'un polyn\^ome}\\
	Soit $P\in\bK[X]$ et $a\in\bK$\\
	\noindent \begin{tabular}{llll}
		$\bullet$ & $a$ racine de $P$                                               & $\Longleftrightarrow$ & \hspace{5cm} \vsec\vsec \\

		$\bullet$ & \begin{tabular}{l} $a$ racine\\ d'ordre $k$ de $P$\end{tabular} & $\Longleftrightarrow$ &

		\vsec\vsec\vsec
	\end{tabular}
\end{theorem}



\begin{rems}
	\begin{itemize}
		\noindent Ainsi, d\`es que l'on conna\^it une racine (ou z\'ero) $a$ de $P$, on peut factorier $P$ par $X-a$.
		\item[$\bullet$] Pour $k=1$, on dit que la racine est \dotfill, pour $k=2$, on dit que la racine est \dotfill\\
		      \noindent D\`es que $k>1$, on dit que la racine est \dotfill. \\
		      Et pour $k=0$, $a$ n'est pas racine de $P$.
		\item[$\bullet$]  \noindent \warning  On distinguera bien les racines r\'eelles et les racines complexes selon que $P$ est vu comme un polyn\^ome r\'eel ou complexe. En effet, un polyn\^ome dont tous les coefficients sont r\'eels peut \^etre consid\'er\'e comme un polyn\^ome r\'eel mais aussi comme un polyn\^ome complexe. Cela change tout au niveau des racines.\\
		      \noindent  Exemple avec $P=X^2+1$:

	\end{itemize}
\end{rems}

%{\footnotesize 
%\begin{exercice}
%Soit $n$ un entier non nul. Montrer que $1$ est racine de $P=X^{2n}-nX^{n+1}+nX^{n-1}-1$ et d\'eterminer l'ordre de multiplicit\'e de cette racine. 
%\end{exercice}
%}


%----------------------------------------------------
%-----------------------------------------------------
\subsection{M\'ethodes pour obtenir les racines d'un polyn\^ome}



\colorbox{gray!40}{M\'ethodes pour d\'eterminer les racines d'un polyn\^ome $P$:}\\

\begin{dboxminipage}{0.9\textwidth}
	\begin{itemize}
		\item[$\bullet$] Si $P$ est un polyn\^ome de degr\'e 0, 1 ou 2: on conna\^it ses racines.
		\item[$\bullet$] Si $P$ est un polyn\^ome de degr\'e sup\'erieur:
		      \begin{itemize}
			      \item[$\star$] Recherche de racines \'evidentes: on teste avec -2,-1,0,1,2,-i,i....
			      \item[$\star$] D\'etermination de leur multiplicit\'e en calculant les d\'eriv\'ees successives.
			      \item[$\star$] On utilise parfois: Si $P\in\R$ et si $P$ a une racine complexe, son conjugu\'e sera aussi racine de $P$ avec la m\^eme multiplicit\'e.
		      \end{itemize}
		\item[$\bullet$] Si $P\in\R$, utiliser le th\'eor\`eme des valeurs interm\'ediaires ou le th\'eor\`eme de la bijection sur la fonction $x\mapsto P(x)$ pour d\'eterminer l'existence et/ou l'unicit\'e de racines.
		      %\begin{itemize}
		      %\item[$\star$] \'Etude de la fonction $x\mapsto P(x)$ sur $\R$ pour d\'eterminer l'existence et/ou l'unicit\'e de racines:\\
		      %On utilise pour cela le TVI ou le  th\'eor\`eme de la bijection.
		      %\item[$\star$] Utilisation du th\'eor\`eme de Rolle pour faire le lien entre les racines de $P$ et celles de $P^{\prime}$.
		      %\end{itemize}
		      %\item[$\bullet$] Utilisation des racines $n$-i\`eme de l'unit\'e pour les polyn\^omes de type $X^n-a$.
		\item[$\bullet$] Utilisation des identit\'es remarquables.
	\end{itemize}
\end{dboxminipage}


{\footnotesize
\begin{exercice}
	\begin{enumerate}
		\item D\'eterminer les racines complexes de $P=X^5-X^4+2X^3-2X^2+X-1$
		\item D\'eterminer les racines complexes de $P=X^5-1$
		\item Soit $n$ un entier non nul. Montrer que $1$ est racine de $P=X^{2n}-nX^{n+1}+nX^{n-1}-1$ et d\'eterminer l'ordre de multiplicit\'e de cette racine.
		      %\item[$\star$] Montrer que si $P\in\R$ admet au moins deux racines r\'eelles alors $P^{\prime}$ a au moins une racine r\'eelle.
		      %\item[$\star$] Montrer que le polyn\^ome $P=X^4-9X^3+30X^2-44X+24$ a une racine d'ordre 3 que l'on pr\'ecisera. En d\'eduire toutes les racines de $P$.
		      %\item[$\star$] D\'eterminer les racines de $A=(X+1)^n-(X-1)^n$.
	\end{enumerate}
\end{exercice}}

{\footnotesize
\begin{exercice}
	Montrer que tout polyn\^ome de $\R$ de degr\'e impair admet au moins une racine r\'eelle.
\end{exercice}}


%----------------------------------------------------
%-----------------------------------------------------
\subsection{M\'ethodes pour montrer l'\'egalit\'e entre deux polyn\^omes ou la nullit\'e d'un polyn\^ome}

\noindent
Soient $P\in\bK$ et $\alpha_1,\alpha_2,\dots,\alpha_p$ des \'el\'ements de $\bK$ deux \`a deux distincts.
$$\alpha_1,\alpha_2,\dots,\alpha_p\ \hbox{sont des racines de}\ P\Longleftrightarrow \hspace{9.5cm}  $$
On en d\'eduit le r\'esultat suivant:\\

{\noindent

\begin{theorem} \textbf{Lien entre le nombre de racines et la nullit\'e d'un polyn\^ome}\vsec
	\begin{itemize}
		\item[$\bullet$] Tout polyn\^ome de degr\'e inf\'erieur ou \'egal \`a $n$ et qui admet au moins $n+1$ racines est \dotfill \vsec
		\item[$\bullet$] Deux polyn\^omes de degr\'e  $\leq n$ ayant les m\^emes valeurs pour au moins $n+1$ valeurs sont \dotfill \vsec
	\end{itemize}
\end{theorem}
}\vsec\vsec



\colorbox{gray!30}{M\'ethodes pour montrer qu'un polyn\^ome $P$ est nul}\\

\begin{dboxminipage}{0.9\textwidth}
	\begin{itemize}
		\item[$\bullet$] M\'ethode 1: On montre que tous ses coefficients sont nuls.
		\item[$\bullet$] M\'ethode 2: On montre qu'il a strictement plus de racines que son degr\'e ou qu'il a une infinit\'e de racines.
	\end{itemize}
\end{dboxminipage}



\vspace{1cm}

\colorbox{gray!30}{M\'ethodes pour montrer que deux polyn\^omes sont \'egaux}\\


\begin{dboxminipage}{0.9\textwidth}
	\begin{itemize}
		\item[$\bullet$] M\'ethode 1: On montre qu'ils ont les m\^emes coefficients.
		\item[$\bullet$] M\'ethode 2: On montre qu'ils sont \'egaux sur strictement plus de valeurs que leur degr\'e.
	\end{itemize}

\end{dboxminipage}


{\footnotesize
\begin{exercice}
	D\'eterminer tous les polyn\^omes de $\R$ tels que $\forall x\in\R,\ P(x+1)=P(x).$
\end{exercice}
}

%----------------------------------------------------
%-----------------------------------------------------
%----------------------------------------------------
%-----------------------------------------------------
%-------------------------------------------------------
\section{Factorisation d'un polyn\^ome dans $\R$ ou $\bC$}

%----------------------------------------------------
%-----------------------------------------------------
\subsection{Factorisation d'un polyn\^ome dans $\bC$}

{\noindent

	\begin{theorem} \textbf{Th\'eor\`eme de d'Alembert - Hors Programme ! mais tellement fondamental...}
		\begin{itemize}
			\item[$\bullet$] Dans $\bC$, un polyn\^ome de degr\'e $n$ a exactement $n$ racines, en comptant leur multiplicit\'e.
			\item[$\bullet$] Dans $\bC$, un polyn\^ome $P$ non nul de degr\'e $n$ se factorise selon
			      $$P = a_n \prod_{k=1}^n (X-z_k)^{p_k}$$


			      avec
			      \begin{itemize}
				      \item[$\star$] $a_n$ coefficient dominant de $P$,
				      \item[$\star$] $z_1,z_2,\dots,z_k$ sont toutes les racines de $P$ complexes distinctes et
				      \item[$\star$] $p_1,p_2,\dots,p_k$ sont les multiplicit\'es respectives de ces racines: $p_1+\dots+p_k=n$.
			      \end{itemize}
		\end{itemize}
	\end{theorem}

}\vsec\vsec

\setlength\fboxrule{1pt}
\noindent  {

	Factoriser un polyn\^ome dans $\bC$ revient \`a d\'eterminer toutes ses racines complexes avec leur ordre.
}
\setlength\fboxrule{0.5pt}

{\footnotesize
	\begin{exercice}
		\begin{enumerate}
			\item Factorisation dans $\bC$ de $P=X^3+X^2+X+1$.
			\item Factorisation dans $\bC$ de $P=2X^4+6X^2+4$.
			      %\item[$\bullet$] Factorisation dans $\bC$ de $P=X^6-1-i$. 
			\item Factorisation dans $\bC$ de $P=X^n+1$ avec $n\in\N$.
		\end{enumerate}
	\end{exercice}
}



\begin{rem}
	Factorisation dans $\R[X]$ :

\end{rem}

%----------------------------------------------------
%-----------------------------------------------------
%\subsection{Factorisation d'un polyn\^ome dans $\R$}
%
%\noindent La factorisation est plus d\'elicate dans $\R$ que dans $\bC$ car il existe des polyn\^omes \`a coefficients r\'eels qui n'ont pas de racines r\'eelles (exemple: $P=X^2+1$) et qui ne sont donc pas factorisables dans $\R$. \vsec
%
%\hspace{-0.5cm}  {\noindent  
%
%\begin{prop}
%Soit $P$ un polyn\^ome r\'eel et $z_0$ une racine complexe d'ordre $n$. Alors $\overline{z_0}$ est aussi une racine de $P$, de m\^eme ordre $n$.
%\end{prop}
% }
%
%\hspace{-0.5cm}  {\noindent  
%
%\begin{theorem} \textbf{Factorisation dans $\R$}
%Tout polyn\^ome $P$ non constant \`a coefficients r\'eels se factorise dans $\R$ par\\
%
%
%
%avec 
%\begin{itemize}
%\item[$\bullet$] $a_n$ coefficient dominant de $P$, 
%\item[$\bullet$] $x_1,x_2,\dots,x_r$ sont les racines r\'eelles de $P$ distinctes, 
%\item[$\bullet$] $p_1,p_2,\dots,p_r$ sont les multiplicit\'es de ces racines r\'eelles, 
%\item[$\bullet$] les polyn\^omes $X^2+\alpha_j X+\beta_j$ sont \`a coefficients r\'eels mais sans racines r\'eelles \`a savoir de discriminant strictement n\'egatif, 
%\item[$\bullet$] $q_1,q_2,\dots, q_s$ sont des entiers naturels non nuls. 
%\end{itemize}
%De plus: $p_1+\dots+p_r+2q_1+\dots+2q_s=\deg P$.
%\end{theorem}
% 
%}\vsec\vsec
%
%
%\setlength\fboxrule{1pt}
%\noindent  {
%
%\textbf{M\'ethodes pour factoriser un polyn\^ome r\'eel $P$ dans $\R$}
%\begin{itemize}
%\item[$\bullet$]  On factorise $P$ dans $\bC$.
%\item[$\bullet$]  On conserve les termes $X-x_i$ pour toutes les racines $x_i$ r\'eelles.
%\item[$\bullet$] Pour les racines complexes NON r\'eelles:
%\begin{itemize}
%\item[$\star$] On regroupe les termes contenant les racines conjugu\'ees: $\alpha$ et $\overline{\alpha}$.
%\item[$\star$] On obtient ainsi des polyn\^omes de degr\'e 2 \`a discriminant strictement n\'egatif car\\
%\noindent $(X-\alpha)(X-\overline{\alpha})=\underbrace{X^2-2\reel{(\alpha)} X+|\alpha|^2}_{\in\R}$.
%\end{itemize}
%\end{itemize}
% }
%\setlength\fboxrule{0.5pt}


{\footnotesize
\begin{exercice}
	\begin{enumerate}
		\item Factorisation dans $\R$ de $P=X^4-1$
		\item Factorisation dans $\R$ de $P=2X^4+6X^2+4$
		      %\item[$\star$] Factorisation dans $\R$ de $P=X^5-X^4+X-1$ 
		      %\item[$\star$] Factorisation dans $\R$ de $P=X^3-3X^2+3X-2$ 
	\end{enumerate}
\end{exercice}}

%----------------------------------------------------
%-----------------------------------------------------
\subsection{Relation coefficients-racines}

\noindent On rappelle le r\'esultat suivant donnant une relation entre les racines d'un polyn\^ome de degr\'e 2 et les coefficients de ce polyn\^ome.\vsec

{\noindent

	\begin{prop}
		Soit $P=aX^2+bX+c \in\bC$ avec $a\not=0$.\\
		Les complexes $(z_1,z_2)$ sont racines de $P$ si et seulement si : \vsec

		%\begin{itemize}
		%\item[$\bullet$]  \dotfill\vsec
		%\item[$\bullet$] \dotfill\vsec
		%\end{itemize} 
	\end{prop}
}\vsec

{\footnotesize
	\begin{exercice}
		Soit $P$ un polyn\^ome de degr\'e 3 ayant pour racines $x_1, x_2, x_3$. Exprimer $x_1 + x_2 + x_3$, $x_1 x_2 + x_1 x_3 + x_2 x_3$ et $x_1 x_2 x_3$ en fonction des coefficients de $P$.
	\end{exercice}}

%
%\begin{rem}
%\noindent On peut g\'en\'eraliser ce r\'esultat \`a des polyn\^omes de degr\'e $n$ : soit $P$ le polyn\^ome de degr\'e $n$ d\'efini par $P=a_nX^n+a_{n-1}X^{n-1}+\dots+a_1X+a_0$ avec $a_n\not= 0$. On note $x_1, \ldots , x_n$ ses racines. On a alors :
%$$\left\{ \begin{array}{rcl}
%\ddp \sum_{i=1}^n x_i  & = &\hspace*{10cm} \vsec\\
%\ddp \sum_{1\leq i < j \leq n} x_i  x_j & = & \vsec\\
%\vdots\vsec\\
%\ddp \sum_{1\leq i_1 < i_2 < \ldots < i_k \leq n} x_{i_1} x_{i_2} \ldots x_{i_k} & = & \vsec\\
%\vdots\vsec\\
%\ddp \prod_{i=1}^n x_i & = & \vsec
%\end{array}\right.$$
%\end{rem}
%
%%\hspace{-0.5cm}  {\noindent  
%%
%%\begin{prop} Relation entre coefficients et racines\\
%%\noindent Soient $n\in\N$ et $P\in\bC$ un polyn\^ome de degr\'e $n$, $P=a_nX^n+a_{n-1}X^{n-1}+\dots+a_1X+a_0$ avec $a_n\not= 0$, 
%%et $x_1,x_2,\dots,x_n$ ses racines. Alors, on a:\\
%%
%%\end{prop}
%% }\vsec
%


\end{document}