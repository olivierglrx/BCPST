\documentclass[a4paper, 11pt]{article}
\usepackage[utf8]{inputenc}
\usepackage{amssymb,amsmath,amsthm}
\usepackage{geometry}
\usepackage[T1]{fontenc}
\usepackage[french]{babel}
\usepackage{fontawesome}
\usepackage{pifont}
\usepackage{tcolorbox}
\usepackage{fancybox}
\usepackage{bbold}
\usepackage{tkz-tab}
\usepackage{tikz}
\usepackage{fancyhdr}
\usepackage{sectsty}
\usepackage[framemethod=TikZ]{mdframed}
\usepackage{stackengine}
\usepackage{scalerel}
\usepackage{xcolor}
\usepackage{hyperref}
\usepackage{listings}
\usepackage{enumitem}
\usepackage{stmaryrd} 
\usepackage{comment}


\hypersetup{
    colorlinks=true,
    urlcolor=blue,
    linkcolor=blue,
    breaklinks=true
}





\theoremstyle{definition}
\newtheorem{probleme}{Problème}
\theoremstyle{definition}


%%%%% box environement 
\newenvironment{fminipage}%
     {\begin{Sbox}\begin{minipage}}%
     {\end{minipage}\end{Sbox}\fbox{\TheSbox}}

\newenvironment{dboxminipage}%
     {\begin{Sbox}\begin{minipage}}%
     {\end{minipage}\end{Sbox}\doublebox{\TheSbox}}


%\fancyhead[R]{Chapitre 1 : Nombres}


\newenvironment{remarques}{ 
\paragraph{Remarques :}
	\begin{list}{$\bullet$}{}
}{
	\end{list}
}




\newtcolorbox{tcbdoublebox}[1][]{%
  sharp corners,
  colback=white,
  fontupper={\setlength{\parindent}{20pt}},
  #1
}







%Section
% \pretocmd{\section}{%
%   \ifnum\value{section}=0 \else\clearpage\fi
% }{}{}



\sectionfont{\normalfont\Large \bfseries \underline }
\subsectionfont{\normalfont\Large\itshape\underline}
\subsubsectionfont{\normalfont\large\itshape\underline}



%% Format théoreme, defintion, proposition.. 
\newmdtheoremenv[roundcorner = 5px,
leftmargin=15px,
rightmargin=30px,
innertopmargin=0px,
nobreak=true
]{theorem}{Théorème}

\newmdtheoremenv[roundcorner = 5px,
leftmargin=15px,
rightmargin=30px,
innertopmargin=0px,
]{theorem_break}[theorem]{Théorème}

\newmdtheoremenv[roundcorner = 5px,
leftmargin=15px,
rightmargin=30px,
innertopmargin=0px,
nobreak=true
]{corollaire}[theorem]{Corollaire}
\newcounter{defiCounter}
\usepackage{mdframed}
\newmdtheoremenv[%
roundcorner=5px,
innertopmargin=0px,
leftmargin=15px,
rightmargin=30px,
nobreak=true
]{defi}[defiCounter]{Définition}

\newmdtheoremenv[roundcorner = 5px,
leftmargin=15px,
rightmargin=30px,
innertopmargin=0px,
nobreak=true
]{prop}[theorem]{Proposition}

\newmdtheoremenv[roundcorner = 5px,
leftmargin=15px,
rightmargin=30px,
innertopmargin=0px,
]{prop_break}[theorem]{Proposition}

\newmdtheoremenv[roundcorner = 5px,
leftmargin=15px,
rightmargin=30px,
innertopmargin=0px,
nobreak=true
]{regles}[theorem]{Règles de calculs}


\newtheorem*{exemples}{Exemples}
\newtheorem{exemple}{Exemple}
\newtheorem*{rem}{Remarque}
\newtheorem*{rems}{Remarques}
% Warning sign

\newcommand\warning[1][4ex]{%
  \renewcommand\stacktype{L}%
  \scaleto{\stackon[1.3pt]{\color{red}$\triangle$}{\tiny\bfseries !}}{#1}%
}


\newtheorem{exo}{Exercice}
\newcounter{ExoCounter}
\newtheorem{exercice}[ExoCounter]{Exercice}

\newcounter{counterCorrection}
\newtheorem{correction}[counterCorrection]{\color{red}{Correction}}


\theoremstyle{definition}

%\newtheorem{prop}[theorem]{Proposition}
%\newtheorem{\defi}[1]{
%\begin{tcolorbox}[width=14cm]
%#1
%\end{tcolorbox}
%}


%--------------------------------------- 
% Document
%--------------------------------------- 






\lstset{numbers=left, numberstyle=\tiny, stepnumber=1, numbersep=5pt}




% Header et footer

\pagestyle{fancy}
\fancyhead{}
\fancyfoot{}
\renewcommand{\headwidth}{\textwidth}
\renewcommand{\footrulewidth}{0.4pt}
\renewcommand{\headrulewidth}{0pt}
\renewcommand{\footruleskip}{5px}

\fancyfoot[R]{Olivier Glorieux}
%\fancyfoot[R]{Jules Glorieux}

\fancyfoot[C]{ Page \thepage }
\fancyfoot[L]{1BIOA - Lycée Chaptal}
%\fancyfoot[L]{MP*-Lycée Chaptal}
%\fancyfoot[L]{Famille Lapin}



\newcommand{\Hyp}{\mathbb{H}}
\newcommand{\C}{\mathcal{C}}
\newcommand{\U}{\mathcal{U}}
\newcommand{\R}{\mathbb{R}}
\newcommand{\T}{\mathbb{T}}
\newcommand{\D}{\mathbb{D}}
\newcommand{\N}{\mathbb{N}}
\newcommand{\Z}{\mathbb{Z}}
\newcommand{\F}{\mathcal{F}}




\newcommand{\bA}{\mathbb{A}}
\newcommand{\bB}{\mathbb{B}}
\newcommand{\bC}{\mathbb{C}}
\newcommand{\bD}{\mathbb{D}}
\newcommand{\bE}{\mathbb{E}}
\newcommand{\bF}{\mathbb{F}}
\newcommand{\bG}{\mathbb{G}}
\newcommand{\bH}{\mathbb{H}}
\newcommand{\bI}{\mathbb{I}}
\newcommand{\bJ}{\mathbb{J}}
\newcommand{\bK}{\mathbb{K}}
\newcommand{\bL}{\mathbb{L}}
\newcommand{\bM}{\mathbb{M}}
\newcommand{\bN}{\mathbb{N}}
\newcommand{\bO}{\mathbb{O}}
\newcommand{\bP}{\mathbb{P}}
\newcommand{\bQ}{\mathbb{Q}}
\newcommand{\bR}{\mathbb{R}}
\newcommand{\bS}{\mathbb{S}}
\newcommand{\bT}{\mathbb{T}}
\newcommand{\bU}{\mathbb{U}}
\newcommand{\bV}{\mathbb{V}}
\newcommand{\bW}{\mathbb{W}}
\newcommand{\bX}{\mathbb{X}}
\newcommand{\bY}{\mathbb{Y}}
\newcommand{\bZ}{\mathbb{Z}}



\newcommand{\cA}{\mathcal{A}}
\newcommand{\cB}{\mathcal{B}}
\newcommand{\cC}{\mathcal{C}}
\newcommand{\cD}{\mathcal{D}}
\newcommand{\cE}{\mathcal{E}}
\newcommand{\cF}{\mathcal{F}}
\newcommand{\cG}{\mathcal{G}}
\newcommand{\cH}{\mathcal{H}}
\newcommand{\cI}{\mathcal{I}}
\newcommand{\cJ}{\mathcal{J}}
\newcommand{\cK}{\mathcal{K}}
\newcommand{\cL}{\mathcal{L}}
\newcommand{\cM}{\mathcal{M}}
\newcommand{\cN}{\mathcal{N}}
\newcommand{\cO}{\mathcal{O}}
\newcommand{\cP}{\mathcal{P}}
\newcommand{\cQ}{\mathcal{Q}}
\newcommand{\cR}{\mathcal{R}}
\newcommand{\cS}{\mathcal{S}}
\newcommand{\cT}{\mathcal{T}}
\newcommand{\cU}{\mathcal{U}}
\newcommand{\cV}{\mathcal{V}}
\newcommand{\cW}{\mathcal{W}}
\newcommand{\cX}{\mathcal{X}}
\newcommand{\cY}{\mathcal{Y}}
\newcommand{\cZ}{\mathcal{Z}}







\renewcommand{\phi}{\varphi}
\newcommand{\ddp}{\displaystyle}


\newcommand{\G}{\Gamma}
\newcommand{\g}{\gamma}

\newcommand{\tv}{\rightarrow}
\newcommand{\wt}{\widetilde}
\newcommand{\ssi}{\Leftrightarrow}

\newcommand{\floor}[1]{\left \lfloor #1\right \rfloor}
\newcommand{\rg}{ \mathrm{rg}}
\newcommand{\quadou}{ \quad \text{ ou } \quad}
\newcommand{\quadet}{ \quad \text{ et } \quad}
\newcommand\fillin[1][3cm]{\makebox[#1]{\dotfill}}
\newcommand\cadre[1]{[#1]}
\newcommand{\vsec}{\vspace{0.3cm}}

\DeclareMathOperator{\im}{Im}
\DeclareMathOperator{\cov}{Cov}
\DeclareMathOperator{\vect}{Vect}
\DeclareMathOperator{\Vect}{Vect}
\DeclareMathOperator{\card}{Card}
\DeclareMathOperator{\Card}{Card}
\DeclareMathOperator{\Id}{Id}
\DeclareMathOperator{\PSL}{PSL}
\DeclareMathOperator{\PGL}{PGL}
\DeclareMathOperator{\SL}{SL}
\DeclareMathOperator{\GL}{GL}
\DeclareMathOperator{\SO}{SO}
\DeclareMathOperator{\SU}{SU}
\DeclareMathOperator{\Sp}{Sp}


\DeclareMathOperator{\sh}{sh}
\DeclareMathOperator{\ch}{ch}
\DeclareMathOperator{\argch}{argch}
\DeclareMathOperator{\argsh}{argsh}
\DeclareMathOperator{\imag}{Im}
\DeclareMathOperator{\reel}{Re}



\renewcommand{\Re}{ \mathfrak{Re}}
\renewcommand{\Im}{ \mathfrak{Im}}
\renewcommand{\bar}[1]{ \overline{#1}}
\newcommand{\implique}{\Longrightarrow}
\newcommand{\equivaut}{\Longleftrightarrow}

\renewcommand{\fg}{\fg \,}
\newcommand{\intent}[1]{\llbracket #1\rrbracket }
\newcommand{\cor}[1]{{\color{red} Correction }#1}

\newcommand{\conclusion}[1]{\begin{center} \fbox{#1}\end{center}}


\renewcommand{\title}[1]{\begin{center}
    \begin{tcolorbox}[width=14cm]
    \begin{center}\huge{\textbf{#1 }}
    \end{center}
    \end{tcolorbox}
    \end{center}
    }

    % \renewcommand{\subtitle}[1]{\begin{center}
    % \begin{tcolorbox}[width=10cm]
    % \begin{center}\Large{\textbf{#1 }}
    % \end{center}
    % \end{tcolorbox}
    % \end{center}
    % }

\renewcommand{\thesection}{\Roman{section}} 
\renewcommand{\thesubsection}{\thesection.  \arabic{subsection}}
\renewcommand{\thesubsubsection}{\thesubsection. \alph{subsubsection}} 

\newcommand{\suiteu}{(u_n)_{n\in \N}}
\newcommand{\suitev}{(v_n)_{n\in \N}}
\newcommand{\suite}[1]{(#1_n)_{n\in \N}}
%\newcommand{\suite1}[1]{(#1_n)_{n\in \N}}
\newcommand{\suiteun}[1]{(#1_n)_{n\geq 1}}
\newcommand{\equivalent}[1]{\underset{#1}{\sim}}

\newcommand{\demi}{\frac{1}{2}}
\geometry{hmargin=2.0cm, vmargin=3.5cm}




\begin{document}
\tableofcontents
\title{CH6 : Systèmes linéaires }



%-----------------------------------------------------------
%----------------------------------------------------------
%-----------------------------------------------------------
%----------------------------------------------------------
%-----------------------------------------------------------
%----------------------------------------------------------
%-----------------------------------------------------------
%----------------------------------------------------------
%-----------------------------------------------------------
%----------------------------------------------------------
%-----------------------------------------------------------
%----------------------------------------------------------
\section{G\'en\'eralit\'es: notations, d\'efinitions}
%-----------------------------------------------------------
%----------------------------------------------------------
%-----------------------------------------------------------
%----------------------------------------------------------
\subsection{D\'efinitions: syst\`{e}mes lin\'eaires}

%------------------------------------------------------------------------------------
%\noindent\subsubsection{Syst\`eme lin\'eaire de $n$ \'equations \`a $p$ inconnues}

{\noindent  

\begin{defi} 
On appelle syst\`eme lin\'eaire de $n$ \'equations \`a $p$ inconnues tout syst\`eme de la forme 
$$
(\mathcal{S})
\left\lbrace\begin{array}{lllllllll}
a_{11}x_1 & + & a_{12}x_2 & + & \dots & + & a_{1p}x_p & = & b_1\vsec\\
a_{21}x_1 & + & a_{22}x_2 & + & \dots & + & a_{2p}x_p & = & b_2\vsec\\
          &   &            &   &      &    &          & \vdots &   \vsec\\
a_{n1}x_1 & + & a_{n2}x_2 & + & \dots & + & a_{np}x_p & = & b_n.\vsec\\
\end{array}\right.$$
\begin{itemize}
\item[$\bullet$] $(x_1,\dots,x_p)\in\R^p$ : les inconnues du syst\`{e}me
\item[$\bullet$] $(a_{ij})_{1\leq i\leq n,\ 1\leq j\leq p}$ : les coefficients du syst\`eme
\item[$\bullet$] $(b_i)_{1\leq i\leq n}$ : les seconds membres du syst\`eme
\item[$\bullet$] $\mathcal{L}_i$: $a_{i1}x_1+a_{i2}x_2+\dots+a_{ip}x_p=b_i$ : = la i-\`eme \'equation du syst\`eme  = la i-\`eme ligne du syst\`eme.
\end{itemize}
\end{defi}

}

\begin{exemples} 
\begin{itemize}
\item[$\bullet$] $(\mathcal{S}_1)\left\lbrace\begin{array}{rcrcr}
2x & + & 3y & = & 4\\
-x & + &  y  & = &1
\end{array}\right.$ est un syst\`{e}me lin\'eaire \dotfill \vsec
\item[$\bullet$] $(\mathcal{S}_2)\left\lbrace\begin{array}{rcrcrcr}
x&+&3y&-&z&=&6\\
 & &y&+&6z&=&1
\end{array}\right.$ est un syst\`{e}me lin\'eaire \dotfill
\item[$\bullet$] $(\mathcal{S}_3)\left\lbrace\begin{array}{rcrcrcr}
x&+&3y&-&z&=&6
\end{array}\right.$ est un syst\`{e}me lin\'eaire \dotfill
\item[$\bullet$] $(\mathcal{S}_4)\left\lbrace\begin{array}{rcrcrcr}
x&+&3y&-&z&=&6\\
  & &&&0&=&0\\
\end{array}\right.$ est un syst\`{e}me lin\'eaire \dotfill
\item[$\bullet$] $(\mathcal{S}_5)\left\lbrace\begin{array}{rcrcr}
x&+&3y&=&1\\
x&+&3y&=&-1\\
\end{array}\right.$ est un syst\`{e}me lin\'eaire \dotfill
\end{itemize}
\end{exemples}

\vspace{0.4cm}

%------------------------------------------------------------------------------------
%\noindent\subsubsection{Syst\`eme lin\'eaire homog\`{e}ne}

{\noindent  

\begin{defi}
\begin{itemize}
\item[$\bullet$] Si le second membre d'un syst\`eme lin\'eaire de $n$ \'equations \`a $p$ inconnues est nul, c'est-\`a-dire si:\\ 
\noindent $\forall i\in\intent{ 1,n},\ b_i=0,$
le syst\`eme est dit  \underline{homog\`ene.}
\item[$\bullet$] On appelle syst\`eme homog\`ene associ\'e \`a $(\mathcal{S})$ le syst\`eme $(\mathcal{S}_0)$ obtenu en rempla\c{c}ant le second membre du syst\`eme lin\'eaire $(\mathcal{S})$ par un second membre nul:
$$(\mathcal{S})
\left\lbrace\begin{array}{lllllllll}
a_{11}x_1 & + & a_{12}x_2 & + & \dots & + & a_{1p}x_p & = & b_1\vsec\\
a_{21}x_1 & + & a_{22}x_2 & + & \dots & + & a_{2p}x_p & = & b_2\vsec\\
          &   &            &   &      &    &          & \vdots &   \vsec\\
a_{n1}x_1 & + & a_{n2}x_2 & + & \dots & + & a_{np}x_p & = & b_n.\vsec\\
\end{array}\right.
$$
$$
\mapsto
(\mathcal{S}_0)
\left\lbrace\begin{array}{lllllllll}
a_{11}x_1 & + & a_{12}x_2 & + & \dots & + & a_{1p}x_p & = & 0\vsec\\
a_{21}x_1 & + & a_{22}x_2 & + & \dots & + & a_{2p}x_p & = & 0\vsec\\
          &   &            &   &      &    &          & \vdots &   \vsec\\
a_{n1}x_1 & + & a_{n2}x_2 & + & \dots & + & a_{np}x_p & = & 0.\vsec\\
\end{array}\right.
$$

\end{itemize}
\end{defi}

}


\begin{exemples} Donner les syst\`{e}mes lin\'eaires homog\`{e}nes associ\'es \`{a} $(\mathcal{S}_1)$ et $(\mathcal{S}_2)$: 
\begin{itemize}

\item[$\bullet$] $(\mathcal{S}_{1,0})\left\lbrace\begin{array}{rcrcr}
2x & + & 3y & = & 0\\
-x & + &  y  & = &0
\end{array}\right.$ 
\item[$\bullet$] $(\mathcal{S}_{2,0})\left\lbrace\begin{array}{rcrcrcr}
x&+&3y&-&z&=&0\\
 & &y&+&6z&=&0
\end{array}\right.$ 

\end{itemize}
\end{exemples}



%-----------------------------------------------------------
%----------------------------------------------------------
%-----------------------------------------------------------
%----------------------------------------------------------
\subsection{Ensemble solution d'un syst\`eme lin\'eaire}

%------------------------------------------------------------------------------------
%\noindent\subsubsection{Solution d'un syst\`{e}me lin\'eaire}

\vspace{0.4cm}

{\noindent  

\begin{defi} Ensemble solution: 
\begin{itemize}
\item[$\bullet$] Une solution d'un syst\`eme lin\'eaire de $n$ \'equations \`a $p$ inconnues $(\mathcal{S})$ est un $p$-uplet (un élément de $\R^p$, bref $p$ réels) qui vérifient toutes les équations.
\item[$\bullet$] 
R\'esoudre $(\mathcal{S})$, c'est d\'eterminer l'ensemble des solutions.
\end{itemize}
\end{defi}

}


\begin{exemples} 
\begin{itemize}
\item[$\bullet$] Un syst\`eme lin\'eaire \underline{homog\`ene} de $n$ \'equation \`a $p$ inconnues a toujours au moins une solution: $(0,0,\cdots,0)$
\item[$\bullet$] R\'esoudre les syst\`emes lin\'eaires $(\mathcal{S}_1)$ et $(\mathcal{S}_2)$.
\end{itemize}
\end{exemples}

\vspace{0.4cm}

%------------------------------------------------------------------------------------
%\noindent\subsubsection{Syst\`{e}mes lin\'eaires compatibles}

\vspace{0.4cm}

{\noindent  

\begin{defi} 
Un syst\`eme lin\'eaire est dit compatible si il admet au moins une solution.
\noindent Sinon, il est dit incompatible.
\end{defi}

}
\begin{defi} 
Un syst\`eme de Cramer est un syst\`eme lin\'eaire admettant une unique solution. 
\end{defi}


\begin{exemples} 
\begin{itemize}
\item[$\bullet$] Les systèmes homogènes sont compatibles. 
\item[$\bullet$] $(\mathcal{S}_1)$ et $(\mathcal{S}_2)$ sont \dotfill
\item[$\bullet$] Le syst\`{e}me $(\mathcal{S}_3)\left\lbrace\begin{array}{rcrcrcr}
x & + & y & + & z &= & 1\\
x & + &  2y  &+ & z & = &2\\
  &   &     y &   &  & = &5
\end{array}\right.$ est-il compatible ?
\dotfill
\end{itemize}
\end{exemples}


\vspace{0.4cm}

%------------------------------------------------------------------------------------
%\noindent\subsubsection{Syst\`{e}mes lin\'eaires \'equivalents}

\vspace{0.4cm}

{\noindent  

\begin{defi} 
Deux syst\`emes lin\'eaires sont dits \'equivalents  si ils ont même ensemble de solutions. 
\end{defi}

}

\begin{rem}
%\vspace*{3cm}
Si on r\'esout un syst\`eme d'\'equations par implications successives, il faut alors v\'erifier que les solutions candidates trouv\'ees sont bien solutions du syst\`eme. En effet, dans un raisonnement par implication, des solutions parasites peuvent appara\^itre. Les solutions obtenues ne sont que des solutions \'eventuelles. Il faut donc alors toujours v\'erifier si les solutions trouv\'ees sont bien solutions du syst\`eme de d\'epart.\\
En cons\'equence, on cherchera plut\^ot \`a raisonner par \'equivalence et \`a transformer un syst\`eme en un syst\`eme \'equivalent. Ainsi, aucune v\'erification ne sera \`a faire, les deux syst\`emes ayant le m\^eme ensemble de solutions. On verra tout \`a l'heure quelles sont les op\'erations qui transforment un syst\`eme en un syst\`eme \'equivalent. 
\end{rem}

%-----------------------------------------------------------
%----------------------------------------------------------
%-----------------------------------------------------------
%----------------------------------------------------------
%\subsection{Syst\`eme de Cramer}


%-----------------------------------------------------------
%----------------------------------------------------------
%-----------------------------------------------------------
%----------------------------------------------------------
%-----------------------------------------------------------
%----------------------------------------------------------
%-----------------------------------------------------------
%----------------------------------------------------------
\section{Cas particuliers importants: les syst\`{e}mes \'echelonn\'es}

\noindent Les syst\`{e}mes \'echelonn\'es sont des syst\`{e}mes lin\'eaires que l'on sait r\'esoudre facilement.\\
\noindent La partie 3 pr\'esentera alors une m\'ethode qui permet de ramener tout syst\`{e}me \`{a} un syst\`{e}me \'echelonn\'e.


%-----------------------------------------------------------
%----------------------------------------------------------
%-----------------------------------------------------------
%----------------------------------------------------------
\subsection{Syst\`emes lin\'eaires triangulaires - \'echelonn\'es}

Ce sont des syst\`{e}mes \'echelonn\'es particuliers.\\

\vspace{0.3cm}


{\noindent  

\begin{defi} 
Un syst\`eme lin\'eaire est dit triangulaire s'il est de la forme :
%\begin{itemize}
% \item[$\bullet$]
%Il a le m\^eme nombre d'inconnues que d'\'equations: $n=p$.
%\item[$\bullet$] 
%Pour tout $i\in\intent{ 2,n}$, \`a la ligne $\mathcal{L}_i$, les inconnues $x_1,\dots,x_{i-1}$ ont un coefficient nul.
%\end{itemize}
%Ce syst\`eme est donc de la forme
$$\left\lbrace\begin{array}{lllllllllll}
%\begin{array}{|c|}\hline a_{11}x_1\\ \hline\end{array} & + & a_{12}x_2 
\fbox{$a_{11}$} \, x_1 & + & a_{12}x_2 
& + & \dots &+&\dots &+&a_{1n}x_n &=&b_1\vsec\\
 &  & \fbox{$a_{22}$} \,  x_2 & + 
& \dots &+&\dots &+&a_{2n}x_n &=&b_2\vsec\\
 &  &  & \ddots &  && && &=&\vdots\vsec\\
 &  & &  & \fbox{$a_{ii}$} \, x_i &+&\dots &+&a_{in}x_n &=&b_i\vsec\\
 &  &  &  &  &\ddots&&& &&\vdots\vsec\\
 &  &  &  &  && &&\fbox{$a_{nn}$} \, x_n &=&b_n\vsec\\
\end{array}\right.
$$
Les coefficients encadr\'es doivent \^etre NON NULS et sont appel\'es les $\dotfill$.
\end{defi}

}

\begin{exemple} R\'esoudre le syst\`{e}me suivant: $(\mathcal{S}_4)\left\lbrace\begin{array}{lllllllll}
x &+ & 2y &+&3z&+&4t&=&1\\
 & & y &+&2z&+&3t&=&2\\
 & &  &&z&+&2t&=&3\\
 & & &&&&t&=&4\\
\end{array}\right.$
\end{exemple}

\vspace{0.5cm}

\setlength\fboxrule{1pt}
\noindent {

\textbf{M\'ethode:}
Calculer la solution de proche en proche en remontant de la derni\`ere ligne \`a la premi\`ere.
}
\setlength\fboxrule{0.5pt}


%%------------------------------------------------------------------------------------
%\noindent\subsubsection{R\'esolution: Ensemble solution}
%
%
%\flushleft {\noindent  
%
%\begin{prop}
%Un syst\`eme lin\'eaire triangulaire admet une unique solution. C'est le n-uplet $(x_1,\dots,x_n)\in\R^n$ d\'efini de proche en proche \`a partir de la derni\`ere ligne
%par
%$$\left\lbrace\begin{array}{lll}
%x_n & = & \dotfill \vsec\\
%x_{n-1} & = & \dotfill \vsec\\
% & \vdots  & \phantom{ \hspace{8cm}} \vsec\\
%x_2 & = &  \dotfill \vsec\\
%x_1 & = & \dotfill \vsec\\
%\end{array}\right.
%$$
%Ce syst\`eme est donc en particulier un syst\`eme de Cramer.
%\end{prop}
%
%}
%
%
%
%\noindent \warning Ne pas retenir la formule des solutions, elle se retrouve toute seule.

%
%-----------------------------------------------------------
%----------------------------------------------------------
%-----------------------------------------------------------
%----------------------------------------------------------
%\subsection{Syst\`emes lin\'eaires \'echelonn\'es}

%------------------------------------------------------------------------------------
%\noindent\subsubsection{D\'efinition}

%\vspace{0.4cm}

{\noindent  

\begin{defi} 
On dit qu'un syst\`eme lin\'eaire de $n$ \'equations \`a $p$ inconnues est un syst\`eme \'echelonn\'e s'il est de la forme :
v\'erifie les deux conditions suivantes:
\begin{itemize}
 \item[$\bullet$]
Lorsque les coefficients de $x_1,\dots, x_k$ sont nuls sur la ligne $i$, alors les coefficients de 
$x_1,\dots,x_k,x_{k+1}$ sont nuls sur la ligne $i+1$. Chaque ligne contient au moins une inconnue de moins que la pr\'ec\'edente de fa\c{c}on cons\'ecutive \`a compter du premier.
\item[$\bullet$] 
Lorsque le membre de gauche de la $i$-\`eme ligne est nul, alors c'est le cas de toutes les lignes suivantes.
\end{itemize}
On est ainsi dans une situation de type:
$$
\left\lbrace\begin{array}{lllllllllllllllll}
\fbox{$a_{11}$} \, x_1 & + & \ldots & + & a_{1j} x_j &+ & \dots & + & a_{1j_r}x_{j_r} &  + & \dots & + & a_{1p}x_p & = & b_1\vsec\\
 & & & & \fbox{$a_{2j}$} \, x_j & + & \dots & + & a_{2j_r}x_{j_r} & + & \dots & + & a_{2p}x_p & = & b_2\vsec\\
&&&&&&&& &&&&  \vdots &  &\vdots\vsec\\
&&&&&&&& \fbox{$a_{rj_r}$} \, x_{j_r} & + & \dots & + & a_{rp}x_p & = & b_r\vsec\\
&&&&&&&&&&&& 0 &= &b_{r+1}\vsec\\
&&&&&&&&&&&& \vdots &= &\vdots\vsec\\
&&&&&&&&&&&& 0 &= &b_{n}\vsec\\
\end{array}\right.
$$
Les coefficients encadr\'es doivent \^etre NON NULS et sont appel\'es les pivots.\\
\noindent D'une ligne \`a l'autre, il y a au moins une inconnue de moins de fa\c{c}on cons\'ecutive puis \'eventuellement des lignes triviales sans inconnues.
\end{defi}

}

Quitte à réordonner les inconnues ($x_i \leftrightarrow x_j$) un système échelonné est de la forme : 
$$
\left\lbrace\begin{array}{lllllllllllllll}
\fbox{$a_{11}$} \, x_1 &  + & a_{12} x_2 &+ & \dots & + & a_{1r}x_{r} &  + & \dots & + & a_{1p}x_p & = & b_1\vsec\\
  & & \fbox{$a_{22}$} \, x_2 & + & \dots & + & a_{2r}x_{r} & + & \dots & + & a_{2p}x_p & = & b_2\vsec\\
&&&&&& &&&&  \vdots &  &\vdots\vsec\\
&&&&&& \fbox{$a_{rr}$} \, x_{r} & + & \dots & + & a_{rp}x_p & = & b_r\vsec\\
&&&&&&&&&& 0 &= &b_{r+1}\vsec\\
&&&&&&&&&& \vdots &= &\vdots\vsec\\
&&&&&&&&&& 0 &= &b_{n}\vsec\\
\end{array}\right.
$$


\begin{exemples} 
%\begin{itemize}
%\item[$\bullet$] Le syst\`{e}me $(\mathcal{S}_5)
%\left\lbrace\begin{array}{lllll}
%x &+ & y &= &1\\
% & & y &= &0\\
%x &+ & z &= &2\vsec\\
%\end{array}\right.$ est-il \'echelonn\'e ?
%\item[$\bullet$] Donner trois exemples de syst\`{e}mes \'echelonn\'es.
%\vspace*{2cm}
%\end{itemize}
\end{exemples}

\vspace{0.6cm}


%------------------------------------------------------------------------------------
\noindent\subsection{Rang d'un syst\`eme lin\'eaire \'echelonn\'e}


\noindent C'est une notion tr\`es importante qui intervient dans de nombreux chapitres.\vsec\\



{\noindent  

\begin{defi} 
On appelle rang d'un syst\`eme lin\'eaire \'echelonn\'e, le nombre d'équations non triviales. C'est-à-dire l'entier $r$ dans la définition précédente. 
\end{defi}

}

\begin{exemples} 
Donner les rangs des syst\`{e}mes $(\mathcal{S}_4)$ et des trois exemples donn\'es ci-dessus.
\end{exemples}

\begin{rem}
Le rang est inf\'erieur ou \'egal  au  nombre d'inconnues
\end{rem}


\begin{prop}
Deux systèmes linéaires équivalents ont même rang. 
\end{prop}

\vspace{0.5cm}

%------------------------------------------------------------------------------------
%\noindent\subsection{R\'esolution: ensemble solution}

\vspace{0.2cm}

{\noindent  

\begin{prop} 
Soit $(\mathcal{S})$ un syst\`eme \textbf{\underline{\'echelonn\'e}} de $n$ \'equations \`a $p$ inconnues et soit $r$ son rang.
\begin{enumerate}
 \item 
Si $(\mathcal{S})$ comporte au moins une \'equation de type $0=b_k$ avec $b_k\not= 0$ : \dotfill\\
 alors le syst\`eme $(\mathcal{S})$ n'a pas de solution, il est incompatible.
\vsec\dotfill
\item 
Si $(\mathcal{S})$ ne comporte pas d'\'equation de type $0=b_k$ avec $b_k\not= 0$ (mais il peut comporter des \'equations de type 0=0), alors le syst\`eme $(\mathcal{S})$ a des solutions, il est compatible. Plus pr\'ecisemment,
\begin{enumerate}
\item 
Si {$r=p$}, alors le syst\`eme $(\mathcal{S})$ a une unique solution qui s'obtient en \'eliminant les \'equations $0=0$ et en d\'eterminant la valeur de chaque inconnue par une lecture de bas en haut. En particulier tout syst\`eme \'echelonn\'e avec $n=p=r$ est de Cramer.
\item 
Si {$r<p$}, alors le syst\`eme $(\mathcal{S})$ a une  infinité de solutions.
\begin{itemize}
 \item[$\bullet$]
Les inconnues $x_1,\dots,x_r$ correspondant aux pivots sont appel\'ees inconnues principales. 
\item[$\bullet$]  
Les $p-r$ autres inconnues $x_{r+1},\dots, x_p$ sont appel\'ees inconnues secondaires et vont jouer le r\^ole de param\`etres.
\item[$\bullet$]  
On passe en second membre les $p-r$ inconnues secondaires qui deviennent des param\`etres. On obtient un syst\`eme triangulaire que l'on sait r\'esoudre. L'ensemble des solutions est ainsi param\'etr\'e par les $p-r$ inconnues secondaires. On dit qu'il y a $p-r$ degrés de liberté (on dira de dimension $(p-r)$ un peu plus tard) 
\end{itemize}
\end{enumerate}
\end{enumerate}
\end{prop}

}

\vspace{0.3cm}

\setlength\fboxrule{1pt}
\noindent {

\textbf{M\'ethode:}
\begin{itemize}
\item[$\bullet$] Reconna\^itre un syst\`{e}me lin\'eaire \'echelonn\'e \`{a} $n$ \'equations et $n$ inconnues.
\item[$\bullet$] Calculer le rang.
\item[$\bullet$] Identifier les inconnues principales et des inconnues secondaires.
\item[$\bullet$] Les inconnues secondaires passent au second membre et jouent alors le r\^ole de param\`etre.
\item[$\bullet$] R\'esoudre le syst\`{e}me en remontant les calculs.
\end{itemize}
}
\setlength\fboxrule{0.5pt}

\begin{exemples} 
\begin{itemize}
\item[$\bullet$] R\'esoudre le syst\`{e}me $(\mathcal{S}_4)$.
\item[$\bullet$]R\'esoudre les syst\`{e}mes lin\'eaires suivants:\\
$(\mathcal{S}_5)
\left\lbrace\begin{array}{rcrcrcrcrcr}
x&+&y&+&z&+&t&+&w&=&1\\
&&y&&&+&t&&&=&2\\
&&&&2z&&&+&3w&=&6\\
\end{array}\right.$ \quad et \quad  
$(\mathcal{S}_6)
\left\lbrace\begin{array}{rcrcrcr}
x&+&y&-&2z&=&\alpha\\
&&3y&-&4z&=&\beta\\
&&&&5z&=&\gamma\\
\end{array}\right.$ \\
\vspace*{0.5cm}
$(\mathcal{S}_7)
\left\lbrace\begin{array}{rcrcrcrcrcr}
x&+&y&+&z&+&t&+&w&=&1\\
&&&&z&-&t&&&=&6\\
&&&&&&2t&+&w&=&8\\
&&&&&&&&0&=&0\\
&&&&&&&&0&=&9\\
\end{array}\right.$ \quad et \quad  
$(\mathcal{S}_8)
\left\lbrace\begin{array}{rcrcrcrcrcr}
a &- &b&+&2c&-&3d&+&e&=&0\\
 & &2b&+&4c&+&d&-&5e&=&3\\
 & &&&&&2d&+&3e&=&-1\\
\end{array}\right.$
\end{itemize}
\end{exemples}




%\vspace*{0.5cm}

%-----------------------------------------------------------
%----------------------------------------------------------
%-----------------------------------------------------------
%----------------------------------------------------------
%-----------------------------------------------------------
%----------------------------------------------------------
%-----------------------------------------------------------
%----------------------------------------------------------
\section{M\'ethode du pivot de Gauss}


\noindent Nous savons donc r\'esoudre les syst\`emes lin\'eaires \'echelonn\'es. Nous allons maintenant \'etablir que tout syst\`eme lin\'eaire peut \^etre mis sous la forme d'un syst\`eme \'echelonn\'e qui lui est \'EQUIVALENT par une succession de transformations \'el\'ementaires sur les lignes et les colonnes.
%-----------------------------------------------------------
%----------------------------------------------------------
%-----------------------------------------------------------
%----------------------------------------------------------
%-------------------------------------------------------------------------------
\subsection{Op\'erations \'el\'ementaires sur les lignes et les colonnes}


{\noindent  

\begin{prop} 
Un syst\`eme $(\mathcal{S}_1)$ est transform\'e en un syst\`eme $(\mathcal{S}_2)$ qui lui est \'EQUIVALENT si:
\begin{itemize}
 \item[$\bullet$]
on \'echange la colonne $i$ avec la colonne $j$: $C_i\leftrightarrow C_j$.
\item[$\bullet$]  
on \'echange la ligne $i$ avec la ligne $j$: $L_i\leftrightarrow L_j.$\\
\vsec

\vsec
\item[$\bullet$] 
on multiplie la ligne $i$ par un scalaire $\alpha$ NON NUL:
$$L_i\leftarrow \alpha L_i,\ \alpha\not= 0.$$
\item[$\bullet$]  
on remplace la ligne d'indice $i$ par la somme de la ligne $i$ et de $\beta$ fois la ligne $j$:
$$ L_i\leftarrow L_i+\beta L_j.$$

\item[$\bullet$]   On supprime une ligne triviale $0=0$. 
\quad

\vspace{0.01cm}

%\rule{0.5mm}{3.5cm} \rule{0.5mm}{3.5cm}
%
%\vspace{1cm}
%$\Longrightarrow$ $\left\lbrace\begin{array}{l} L_i\leftarrow \alpha L_i+\beta L_j\vsec\\
% \begin{array}{|c|}\hline \alpha\not= 0\\ \hline\end{array} \end{array}\right.$

\end{itemize}
\end{prop}

}


\begin{rem}
\begin{enumerate}
 \item Importance de ne proc\'eder que par \'EQUIVALENCES. Ainsi, les seules op\'erations \`a effectuer sur un syst\`eme sont les op\'erations ci-dessus.
\item 
On peut appliquer une combinaison \`a plusieurs lignes en m\^eme temps si on choisit une ligne pivot (par exemple $L_1$) qui n'est pas modifi\'ee et que toutes les combinaisons se font \`a partir de cette ligne:
$$\left\lbrace\begin{array}{lll}
L_1 &  &\vsec\\
L_2 &\leftarrow & \alpha_2 L_2+\beta_2 L_1 \quad (\alpha_2 \not=0)\vsec\\
%L_3 &\leftarrow & L_3+\lambda_3 L_1\vsec\\
\vdots  &&\vsec\\
L_n &\leftarrow &\alpha_n L_n+\beta_n L_1 \quad (\alpha_n \not=0).\vsec\\ 
\end{array}\right.
$$
\end{enumerate}
\end{rem}

\begin{exemple} 
R\'esoudre le syst\`eme suivant: $(\mathcal{S}_{9}) :
\left\lbrace\begin{array}{rcrcrcr}
x&+&y&+&z&=&-1\\
x&-&2y&-&z&=&4\\
-x&+&5y&+&2z&=&1\\
\end{array}\right.$
\end{exemple}


\paragraph{Syst\`emes lin\'eaires \`a param\`etres}

\vspace{0.3cm}

\setlength\fboxrule{1pt}
\noindent {

\textbf{M\'ethode:} Faire des cas selon les opérations élémentaires effectuées
\begin{itemize}
\item[$\bullet$] Ne pas confondre $L_i\leftarrow 0 L_j$ où l'on transforme une ligne non nulle, en ligne $0=0$... Evidemment ca va changer l'ensemble des solutions et 
 $L_i\leftarrow L_i+0L_j$ où l'on change $L_i$ en .... $L_i$. Ici rien ne change, c'est le même système. 
\item[$\bullet$] R\'esoudre le syst\`eme \`a part pour les valeurs particuli\`eres des param\`etres pour lesquelles les pivots s'annulaient. 
\end{itemize}
}

%\begin{exemple} 
%R\'esoudre le syst\`eme suivant en fonction des valeurs du param\`etre $m\in\R$: 
%
%%\item[$\bullet$] R\'esoudre le syst\`{e}me lin\'eaire suivant en discutant selon la valeur de $a\in\R$:
%%$(\mathcal{S}_{13}):
%%\left\lbrace\begin{array}{rcrcrcr}
%%(4-a)x & + & 4y & - & 4z & = & 0\\
%%-x & + & (5-a)y & - & 3z & = & 0\\
%%  x & + & 7y & -     & (5+a)z & = & 0\\
%%\end{array}\right.$
%%\end{itemize}
%\end{exemple}


\begin{exercice}
Résoudre 
 $$(\mathcal{S}_{10}):\left\{ 
\begin{array}{cc}
x+2y&=1\\
\lambda x+y&=0
\end{array}
\right.\quadet 
(\mathcal{S}_{11}):\left\lbrace\begin{array}{rcrcr}
(m+1)x & + & my & = & 2m\\
mx & + & (m-1)y & = & 1\\
\end{array}\right.
$$

$$(\mathcal{S}_{12}):
\left\lbrace\begin{array}{rcrcrcr}
-(2+m)x & - & 2y & + & z & = & 0\\
-2x & + & (1-m)y & - & 2z & = & 0\\
  x & - & 2y & -     & (2+m)z & = & 0\\
\end{array}\right.$$
\end{exercice}



%-----------------------------------------------------------
%----------------------------------------------------------
%-----------------------------------------------------------
%----------------------------------------------------------
%-------------------------------------------------------------------------------
\subsection{Algorithme du pivot de Gauss}

%La m\'ethode que l'on vient d'employer pour r\'esoudre le syst\`eme lin\'eaire pr\'ec\'edent correspond en fait \`a un algorithme g\'en\'eral de r\'esolution de syst\`eme lin\'eaire qui permet de r\'esoudre tous les syst\`emes lin\'eaires de fa\c{c}on syst\'ematique: c'est l'algorithme du pivot de Gauss.\\
%Cette m\'ethode permet de transformer un syst\`eme donn\'e en un syst\`eme \'echelonn\'e \'equivalent et ce gr\^ace aux op\'erations \'el\'ementaires.\\



%

%------------------------------------------------------------------------------------
%\noindent\subsubsection{Description de l'algorithme du pivot de Gauss}

\vspace{0.3cm}

\setlength\fboxrule{1pt}
\noindent {

\textbf{M\'ethode:}
\begin{itemize}
\item[$\bullet$] \`{A} chaque \'etape, on choisit une ligne pivot et un pivot. On fait tous les calculs par rapport \`a cette ligne pivot.
\item[$\bullet$] On utilise alors les op\'erations \'el\'ementaires sur les lignes:
\begin{itemize}
\item[$\star$] La ligne pivot n'est pas modifi\'ee.
\item[$\star$] Toutes les op\'erations \'el\'ementaires se font \`{a} partir de cette ligne.
\item[$\star$] Choix des op\'erations:  \'elimination d'une m\^{e}me inconnue dans toutes les lignes sauf la ligne pivot.
\end{itemize}
\item[$\bullet$] On transforme ainsi notre syst\`eme lin\'eaire de d\'epart en un syst\`eme lin\'eaire \'echelonn\'e \'equivalent.
\end{itemize}
}
\setlength\fboxrule{0.5pt}
\vsec

%
%\begin{exemple} 
%R\'esoudre le syst\`eme suivant: 
%$(\mathcal{S}_{11}):
%\left\lbrace\begin{array}{rcrcrcrcr}
%3x&+&6y&+&8z&+&12t&=&7\\
% x&+&2y&+&2z&+&3t&=&2\\
%2x&+&4y&+&2z&+&  t&=&4\\
%4x&+&8y&+&10z&+&15t&=&9\\
%\end{array}\right.$
%\end{exemple}


\begin{exemples}
R\'esoudre les syst\`{e}mes lin\'eaires suivants:\\

\begin{itemize}
%\item[$\bullet$] $(\mathcal{S}_{10}):\left\lbrace\begin{array}{rcrcrcr}
%2x&+&3y&-&5z&=&-4\\
%-x&+&2y&+&4z&=&5\\
%x&+&5y&-&z&=&1\\
%\end{array}\right.$
%\vsec
\item[$\bullet$] $(\mathcal{S}_{10}):\left\lbrace\begin{array}{rcrcrcr}
2x&-&y&-&z&=&0\\
3x&+&y&+&3z&=&1\\
\end{array}\right.$
\end{itemize}


\begin{itemize}
\item[$\bullet$] $(\mathcal{S}_{11}):
\left\lbrace\begin{array}{rcrcrcr}
2x &+ &y &- & z & = &1\\
3x &+ &3y &- & z & = &2\\
2x &+ &4y& &&= &2
\end{array}\right.$
%\vsec
%\item[$\bullet$] $(\mathcal{S}_{13}):\left\lbrace\begin{array}{rcrcr}
%x&+&y&=&0\\
%2x&-&3y&=&-1\\
%-x&+&4y&=&2\\
%\end{array}\right.$
\end{itemize}

\end{exemples}

\begin{rems}
\begin{enumerate}
 \item 
\noindent \warning \`{A} chaque \'etape, le pivot doit \^etre \dotfill \phantom{\hspace{5cm}}\\
\noindent Attention aux syst\`emes dont les coefficients d\'ependent d'un ou plusieurs param\`etres (voir plus loin).
\item 
On a int\'er\^et \`a choisir un pivot le plus simple possible, le mieux \'etant 1 ou -1. Ainsi, il est parfois int\'eressant d'\'echanger des lignes ou des inconnues pour faire appara\^itre un pivot plus simple.
\end{enumerate}
\end{rems}

\vspace{0.4cm}

%------------------------------------------------------------------------------------
%\noindent\subsubsection{Cons\'equence: solution d'un syst\`{e}me lin\'eaire}


{\noindent  

\begin{theorem}% Ensemble solution d'un syst\`{e}me lin\'eaire:
%\begin{itemize}
 Tout syst\`eme lin\'eaire est \'equivalent \`a un syst\`eme \'echelonn\'e de m\^eme taille.
%\item[$\bullet$]
%Ainsi, un syst\`eme lin\'eaire a \dotfill
%\end{itemize}
\end{theorem}
}


\vspace{0.5cm}

%------------------------------------------------------------------------------------
%\noindent\subsubsection{Cons\'equence: rang d'un syst\`{e}me lin\'eaire}

{\noindent  

\begin{defi}% Rang:\\
\noindent On appelle rang d'un syst\`eme lin\'eaire, \'echelonn\'e ou pas, le rang d'un système linéaire échelonné qui lui est équivalent. 
\end{defi}
}

\begin{rem}
Pour un m\^eme syst\`eme lin\'eaire, selon les choix faits lors de l'algorithme du pivot de Gauss, on peut obtenir des syst\`emes \'echelonn\'es diff\'erents. Ces syst\`emes \'echelonn\'es sont tous \'equivalents puisqu'ils sont \'equivalents \`a un m\^eme syst\`eme. Pour autant, il est parfois difficile de s'en rendre compte au premier coup d'oeil. Il existe n\'eanmoins une caract\'eristique commune \`a tous ces syst\`emes qui est justement leur rang. %Tous les syst\`emes \'echelonn\'es qui sont \'equivalents ont le m\^eme rang.
\end{rem}

%Le rang d'un syst\`eme est donc le nombre d'\'equations pertinentes pour r\'esoudre le syst\`eme. On calcule le rang en effectuant l'algorithme du pivot de Gauss sur le syst\`eme jusqu'au bout, jusqu'\`a obtenir un syst\`eme \'echelonn\'e \'equivalent.\\


\begin{exemples} 
Donner le rang des syst\`{e}mes lin\'eaires $\mathcal{S}_{9}$ \`a $\mathcal{S}_{11}$.
\end{exemples}

%-----------------------------------------------------------
%----------------------------------------------------------
%-----------------------------------------------------------
%----------------------------------------------------------
%---------------------------------------------------

\end{document}