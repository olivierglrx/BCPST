\documentclass[a4paper, 11pt]{article}
\input{macro/package.tex}
\input{macro/environement}
% Header et footer

\pagestyle{fancy}
\fancyhead{}
\fancyfoot{}
\renewcommand{\headwidth}{\textwidth}
\renewcommand{\footrulewidth}{0.4pt}
\renewcommand{\headrulewidth}{0pt}
\renewcommand{\footruleskip}{5px}

\fancyfoot[R]{Olivier Glorieux}
%\fancyfoot[R]{Jules Glorieux}

\fancyfoot[C]{ Page \thepage }
\fancyfoot[L]{1BIOA - Lycée Chaptal}
%\fancyfoot[L]{MP*-Lycée Chaptal}
%\fancyfoot[L]{Famille Lapin}

\input{macro/newcommand.tex}
\geometry{hmargin=2.0cm, vmargin=2.5cm}




\begin{document}
\tableofcontents
\title{Chapitre : Continuite}
% debut
%------------------------------------------------
\vspace{0.5cm}

%----------------------------------------------------------
%----------------------------------------------------
%-----------------------------------------------------
%-------------------------------------------------------
\vspace{0.4cm}


\noindent Dans tout ce qui suit, les fonctions consid\'er\'ees sont des fonctions num\'eriques.\\
\noindent $f$ d\'esigne une fonction de $\R$ dans $\R$. Son domaine de d\'efinition not\'e $\mathcal{D}_f$ est un intervalle ou une r\'eunion d'intervalles.

%-------------------------------------------------------------
%----------------------------------------------------
%-------------------------------------------------------------
%----------------------------------------------------
%-----------------------------------------------------
%-------------------------------------------------------
\section{Continuit\'e en un point}
%-------------------------------------------------------------
%----------------------------------------------------
\subsection{D\'efinition de la continuit\'e en un point:}
%-------------------------------------------------------------
%----------------------------------------------------

{\noindent

	\begin{defi}
		Soit $x_0\in\mathcal{D}_f$. On dit que la fonction $f$ est continue en $x_0$ si
		\vspace{1cm}
	\end{defi}

}

{\footnotesize \begin{exercice}
		\begin{enumerate}
			\item \'Etudier la continuit\'e en 0 de la fonction $f$ d\'efinie par $f(x)=\left\lbrace\begin{array}{ll} x\ln{(x)} &\hbox{si}\ x>0\\ 0 & \hbox{si}\ x=0.    \end{array}\right.$\vsec
			\item \'Etudier la continuit\'e en 1 de la fonction $f$ d\'efinie par $f(x)=\left\lbrace\begin{array}{ll} \ddp\frac{\ln{(x)}}{x-1} &\hbox{si}\ x>0,\ x\not=1\\ 2 & \hbox{si}\ x=1.    \end{array}\right.$
		\end{enumerate}
	\end{exercice}}

%--------------------------------
%--------------------------------
\subsection{Continuit\'e \`a droite et \`a gauche en un point}
%--------------------------------
%--------------------------------
\noindent Lorsque la fonction se comporte de fa\c{c}on diff\'erente \`{a} droite et \`{a} gauche de $x_0\in\mathcal{D}_f$, on doit regarder la continuit\'e \`{a} droite et \`{a} gauche en $x_0$.\\



Exemple type: les fonctions d\'efinies par des raccords.



\begin{defi}
	Soit $f$ une fonction d\'efinie en $x_0$. On suppose que $x_0$ n'est pas une borne de $\mathcal{D}_f$.\vsec
	\begin{itemize}
		\item[$\bullet$] On dit que la fonction $f$ est continue \`a droite en $x_0$ si \dotfill \vsec\vsec
		\item[$\bullet$]  On dit que la fonction $f$ est continue \`a gauche en $x_0$ si \dotfill \vsec\vsec
	\end{itemize}
\end{defi}




{\footnotesize \begin{exercice}
	\'Etudier la continuit\'e en $0$ \`{a} droite et \`{a} gauche de la fonction partie enti\`{e}re.
\end{exercice}}


%--------------------------------
%--------------------------------
\subsection{Lien entre continuit\'e \`a droite, \`a gauche et continuit\'e en un point}
%--------------------------------
%--------------------------------


\begin{prop}
	Soit $f$ une fonction d\'efinie en $x_0$, o\`u $x_0$ n'est pas une extr\'emit\'e de $\mathcal{D}_f$. On a alors:\vsec\vsec\\
	\noindent f continue en $x_0$ $\Longleftrightarrow $ \dotfill $\Longleftrightarrow $\dotfill\vsec\vsec
\end{prop}



\begin{exemple}
	\'Etudier la continuit\'e en $0$ de la fonction partie enti\`{e}re.
\end{exemple}

{\footnotesize \begin{exercice}
	\'Etudier la continuit\'e au point de raccord des fonctions suivantes:
	\begin{enumerate}

		\item $f(x)=\left\lbrace\begin{array}{ll} \ddp\frac{\sin{(x)}}{x} &\hbox{si}\ x>0\\ 1 & \hbox{si}\ x=0\\ \ddp\frac{e^x-1}{x} &\hbox{si}\ x<0  \end{array}\right.$\vsec
		\item $f(x)=\left\lbrace\begin{array}{ll} e^{-\frac{1}{x}} &\hbox{si}\ x\not=0\\ 0 & \hbox{si}\ x=0   \end{array}\right.$


		\item $f(x)=\left\lbrace\begin{array}{ll} \ddp\frac{\sin{(3x)}}{e^x-1} &\hbox{si}\ x>0 \vsec\\ -\ddp\demi & \hbox{si}\ x=0\vsec\\ \ddp\frac{\sqrt{1-x}-1}{x} &\hbox{si}\ x<0  \end{array}\right.$\vsec
		\item $f(x)=\left\lbrace\begin{array}{ll} \ddp\frac{\ln{(1+x^2)}}{x} &\hbox{si}\ x\not=0\\ 1 & \hbox{si}\ x=0 \end{array}\right.$

	\end{enumerate}
\end{exercice}}


%\textbf{M\'ethode pour montrer la continuit\'e de $\mathbf{f}$ en un point $\mathbf{x_0}$ de $\mathbf{\mathcal{D}_f}$ d\'efini par un raccord:}
%\begin{itemize}
%\item[$\bullet$] Calculer les limites \`a gauche et \`a droite.
%\item[$\bullet$] Montrer que $f(x_0) = \lim\limits_{x \to x_0^+} f(x) = \lim\limits_{x \to x_0^-} f(x) $.
%\end{itemize}


%-------------------------------------------------------------
%----------------------------------------------------
%-----------------------------------------------------
%-------------------------------------------------------
\section{Continuit\'e sur un intervalle}
%-------------------------------------------------------------
%----------------------------------------------------
\subsection{D\'efinition de la continuit\'e sur un intervalle}
%-------------------------------------------------------------
%----------------------------------------------------


\begin{defi}
	Soit $I$ un intervalle de $\R$ non r\'eduit \`a un point. \vsec
	\begin{itemize}
		\item[$\bullet$] Une application $f: I\rightarrow \R$ est dite continue sur $I$ si \dotfill\vsec
		\item[$\bullet$] L'ensemble des fonctions continues sur $I$ est not\'e \dotfill ou \dotfill\vsec
		\item[$\bullet$] On dit que $f$ \dotfill\vsec
	\end{itemize}
\end{defi}



%----------------------------------------------------------------
%----------------------------------------------------------------
\subsection{Continuit\'e des fonctions usuelles}
%----------------------------------------------------------------
%----------------------------------------------------------------

\noindent En repassant par la d\'efinition, on peut ainsi d\'emontrer la continuit\'e des fonctions usuelles suivantes :\vsec\\
\phantom{\hspace*{0cm}}\dotfill\vsec\\
\phantom{\hspace*{0cm}}\dotfill


%------------------------------------------------------
\subsection{Continuit\'e et op\'erations alg\'ebriques}



\begin{prop}
	Soient $I$ un intervalle quelconque de $\R$ et $\lambda\in\R$.\\
	\noindent Si $f$ et $g$ sont deux \'el\'ements de $\mathcal{C}(I)$ alors\vsec
	\begin{itemize}
		\item[$\bullet$] \dotfill \vsec
		\item[$\bullet$]  \dotfill\vsec
		\item[$\bullet$]  \dotfill \vsec
		\item[$\bullet$]  \dotfill \vsec
	\end{itemize}
\end{prop}



%---------------------------------------------------------
%---------------------------------------------------------
\subsection{Continuit\'e et composition entre deux fonctions}
%---------------------------------------------------------
%---------------------------------------------------------

{\noindent

\begin{prop}
	Soient $I$ et $J$ deux intervalles de $\R$. \vsec\\
	%\begin{itemize}
	%\item[$\bullet$]
	Si $f\in\mathcal{C}(I)$ et $g\in\mathcal{C}(J)$ avec $f(I)\subset J$ alors \dotfill \vsec
	%\item[$\bullet$] \dotfill\vsec
	%\end{itemize}
\end{prop}



\textbf{M\'ethode pour montrer la continuit\'e sur un intervalle:}\vsec\\
Par somme, produit, compos\'ee, quotient dont le d\'enominateur ne s'annule pas de fonctions usuelles.\vsec

{\footnotesize \begin{exercice} \'Etudier la continuit\'e sur $\mathcal{D}_f$ des fonctions suivantes:
		\begin{enumerate}

			\item $f(x)=\ddp\sqrt{\frac{1+x}{2}}$
			\item $f(x)=\ddp\frac{e^{-x^2}}{x}$
			\item $f(x)=\ln{(e^{-x^4} +3)}$


			\item $f(x)=\left\lbrace\begin{array}{ll} x\sin{\left(\ddp\frac{1}{x} \right)} &\hbox{si}\ x\not=0\\ 1 & \hbox{si}\ x=0  \end{array}\right.$
			\item $f(x)=\left\lbrace\begin{array}{ll} \ddp\frac{\ln{(x)}}{\tan{(\pi x)}} &\hbox{si}\ x\not=1\\ 1 & \hbox{si}\ x=1   \end{array}\right. $

		\end{enumerate}
	\end{exercice}}



%-------------------------------------------------------------
%----------------------------------------------------
%-----------------------------------------------------
%-------------------------------------------------------
%-------------------------------------------------------------
%----------------------------------------------------
%-----------------------------------------------------
%-------------------------------------------------------
\section{Prolongement par continuit\'e}

\noindent La fonction $f$ est d\'efinie au voisinage de $x_0$ MAIS PAS en $x_0$: \fbox{$x_0\notin\mathcal{D}_f$}. On calcule donc $\lim\limits_{x\to x_0} f(x)$ pour savoir si on peut prolonger par continuit\'e la fonction en $x_0$.


\subsection{Prolongement par continuit\'e}


\begin{defi}
	Soient $x_0\in\R$ et une fonction $f$ telle que :
	$\left\{\begin{array}{l}
			\textmd{\fbox{$f$ non d\'efinie en $x_0$}}\vsec \\
			\textmd{$f$ admet une limite finie en $x_0$ not\'ee $l$.}
		\end{array}\right.$\\
	Alors la fonction $\tilde f$ d\'efinie par :


	s'appelle le prolongement par continuit\'e de $f$ en $x_0$. Par abus de notation, on la note encore \ldots\ldots.
\end{defi}

%
%{\footnotesize \begin{exercice} \'Etudier les \'eventuels prolongements par continuit\'e des fonctions suivantes:
%\begin{enumerate}
%
%\item $f(x) = \ddp\frac{\sin{x}}{x}$
%\item $f(x) = x^{\alpha}$, $\alpha\in\R$
%\item $f(x) =  x^x$
%\item  $f(x)=\left\lbrace\begin{array}{ll}
%\sin{x} & \hbox{si}\ x>0\vsec\\
%\cos{x} & \hbox{si}\ x<0.
%\end{array}\right.
%$ \vsec
%%\item $f(x)=\sin{(x+1)}\ln{|x+1|}$
% 
%
%\item $f(x)=\left\lbrace\begin{array}{ll}
%e^x & \hbox{si}\ x<0\vsec\\
%1+x & \hbox{si} \ x>0.
%\end{array}\right.
%$\vsec
%\item $f(x)=\ddp\frac{\cos{(\pi x)}+1}{e^{ax^2}-e^{a}}$
%\item $f(x)=\ddp\frac{x^2-2x-3}{\sqrt{1+x}}$
% 
%\end{enumerate}
%\end{exercice}}

% 

%\begin{rem}
%\'Etant donn\'ee une fonction $f:I\rightarrow \R$, on ne confondra pas les situations suivantes:\vsec
%\begin{itemize}
%\item[$\bullet$]
%La fonction $f$ est continue en $x_0$: \dotfill \vsec\\
%Exemple : $f(x) = \left(\begin{array}{rcl}
%\R & \to & \R\vsec\\
%x & \to & \left\lbrace\begin{array}{cl} 
%\ddp \frac{\sin x }{x} & \hbox{si}\ x\not=0\vsec\\
%1 & \hbox{si}\ x=0.
%\end{array}\right.
%\end{array}\right.$ est \dotfill
%\item[$\bullet$]  
%La fonction $f$ est prolongeable par continuit\'e en $x_0$: \dotfill \vsec\\
%Exemple : $g(x) =  \left(\begin{array}{rcl}
%\R^\star & \to & \R\vsec\\
%x &\to & \ddp \frac{\sin x }{x} 
%\end{array}\right.$ est \dotfill
%\end{itemize}
%\end{rem}
%
%
% 

%---------------------------------------------
\subsection{Prolongement par continuit\'e \`a droite et \`a gauche}

\noindent Lorsque la limite en $x_0$ n'existe pas, cela vient le plus souvent du fait que les limites en $x_0$ \`{a} droite et \`{a} gauche ne sont pas les m\^{e}mes. On peut alors regarder si la fonction est prolongeable par continuit\'e \`{a} droite ou \`{a} gauche.\vsec\vsec

{\noindent

	\begin{prop} Existence du prolongement par continuit\'e \`{a} droite ou \`{a} gauche:\\
		Soit $f$ une fonction non d\'efinie en $x_0$.
		\begin{itemize}
			\item[$\bullet$] Prolongement par continuit\'e \`{a} droite:
			      \begin{itemize}
				      \item[$\star$] $f$ admet un prolongement par continuit\'e \`a droite en $x_0$ si \dotfill
				      \item[$\star$]
				            Dans ce cas, le prolongement $\tilde f$ est d\'efini par :
				            \vspace{1.5cm}

			      \end{itemize}
			\item[$\bullet$] Prolongement par continuit\'e \`{a} gauche:
			      \begin{itemize}
				      \item[$\star$] $f$ admet un prolongement par continuit\'e \`a gauche en $x_0$ si \dotfill
				      \item[$\star$] Dans ce cas, le prolongement $\tilde f$ est d\'efini par :
				            \vspace{1.5cm}

			      \end{itemize}
		\end{itemize}
	\end{prop}

}

{\footnotesize \begin{exercice} \'Etudier les \'eventuels prolongements par continuit\'e des fonctions suivantes:
		\begin{enumerate}
			\item $f(x)=e^{\frac{1}{x}}$
			\item $g(x)=\left\lbrace\begin{array}{ll}
					      \ddp\frac{\ln{x}}{x-1} & \hbox{si}\ x>1\vsec \\
					      \ddp\frac{x}{-x^2-x+2} & \hbox{si}\ x<1.
				      \end{array}\right.$
		\end{enumerate}
	\end{exercice}}


%-------------------------------------------------------------
%----------------------------------------------------
%-----------------------------------------------------
%-------------------------------------------------------
%-------------------------------------------------------------
%----------------------------------------------------
%-----------------------------------------------------
%-------------------------------------------------------
% 
\section{Th\'eor\`emes sur les fonctions continues d\'efinies sur un intervalle}

%-------------------------------------------
\subsection{Composition entre une fonction et une suite, application aux suites r\'ecurrentes:}

\noindent On rappelle le r\'esultat vu lors du chapitre sur les suites.\\

{\noindent

\begin{prop}
	Soient $f$ une fonction num\'erique d\'efinie sur $\mathcal{D}_f$ et une suite $\suiteu$. Si \vsec
	\begin{itemize}
		\item[$\bullet$] \dotfill\vsec
		\item[$\bullet$] \dotfill \vsec
	\end{itemize}
	Alors \dotfill\vsec
\end{prop}

}

{\footnotesize \begin{exercice}
	Calculer les limites \'eventuelles de la suite $\suiteu$ d\'efinie par: $u_0>0$ et pour tout $n\in\N$: $u_{n+1}=f(u_n)$ avec $f(x)=\ln{(2+x)}$.% \'Etudier le comportement asymptotique de la suite.
\end{exercice}}


%-------------------------------------------------------
%------------------------------------------------------------
\subsection{Th\'eor\`eme des valeurs interm\'ediaires}

\noindent\ {Le th\'eor\`{e}me}\\

{\noindent

\begin{theorem}
	Soient $f: I\rightarrow \R$ et $(a,b)\in I^2$ avec $a<b$. Si on a :\vsec
	\begin{itemize}
		\item[$\bullet$] \dotfill \vsec
		\item[$\bullet$] \dotfill \vsec
	\end{itemize}
	alors pour tout $y$ compris entre $f(a)$ et $f(b)$: \dotfill \vsec
\end{theorem}


{\footnotesize \begin{exercice}
	\begin{enumerate}
		\item Montrer que toute fonction polyn\^ome de degr\'e 3 a au moins une racine r\'eelle sur $\R$. De m\^{e}me montrer que toute fonction polyn\^ome de degr\'e impair a au moins une racine r\'eelle sur $\R$.
		\item Soit $f:\ \lbrack a,b\rbrack\rightarrow \lbrack a,b\rbrack $ une fonction continue. Montrer qu'elle admet un point fixe.
		\item Soient deux fonctions continues $f:\ \lbrack 0,1\rbrack\rightarrow \R$ et $g:\ \lbrack 0,1\rbrack\rightarrow \R$ v\'erifiant $f(0)>g(0)$ et $f(1)<g(1)$. Montrer qu'il existe $c\in \; \rbrack 0,1\lbrack$ tel que $f(c)=g(c)$.
	\end{enumerate}
\end{exercice}}


%---------------------------------------------------------
%------------------------------------------------------------
%---------------------------------------------------------
%------------------------------------------------------------
%---------------------------------------------------------
%------------------------------------------------------------
\subsection{Th\'eor\`eme de la bijection}

%---------------------------------------------------------
%------------------------------------------------------------
\noindent\ {Th\'eor\`{e}me de la bijection et fonction r\'eciproque}\\

{\noindent

\begin{theorem}
	Soient $f: I\rightarrow \R$. Si on a : \vsec
	\begin{itemize}
		\item[$\bullet$] \dotfill\vsec
		\item[$\bullet$] \dotfill\vsec
	\end{itemize}
	alors :\vsec
	\begin{itemize}
		\item[$\bullet$] \dotfill\vsec
		\item[$\bullet$] \dotfill\vsec
		\item[$\bullet$] \dotfill\vsec
	\end{itemize}
\end{theorem}
}



{\footnotesize \begin{exercice}
	\begin{enumerate}
		\item \'Etude de la bijectivit\'e de la fonction $f: x\mapsto x^3+2^x$.
		\item \'Etude de la bijectivit\'e de la fonction $f:\ x\mapsto \ddp\frac{x+2}{x-2}$. Donner ensuite l'expression de $f^{-1}$.
	\end{enumerate}
\end{exercice}}

%---------------------------------------------------------
%------------------------------------------------------------
{Fonction arctangente}\\



\begin{prop}
	La fonction tangente est bijective de \ldots \ldots \ldots dans  \ldots \ldots \ldots. \\
	Sa fonction r\'eciproque est la fonction arctangente, d\'efinie de  \ldots \ldots \ldots dans  \ldots \ldots \ldots, et not\'ee \ldots \ldots \ldots \ldots. On a de plus :\vsec
	\begin{itemize}
		\item[$\bullet$] Monotonie : la fonction arctangente est \dotfill\vsec
		\item[$\bullet$] Parit\'e :  la fonction arctangente est \dotfill\vsec
	\end{itemize}
\end{prop}



%---------------------------------------------------------
%------------------------------------------------------------
\noindent\ {Fonctions arccosinus et arcsinus}\\

\noindent Ces fonctions ne font pas partie des fonctions usuelles du programme : il faut refaire la d\'emonstration pour d\'emontrer leur existence et leurs propri\'et\'es.

{\footnotesize \begin{exercice}
	\begin{itemize}
		\item[$\bullet$] Montrer que la fonction cosinus est bijective de $[0,\pi]$ dans $[-1,1]$. Sa bijection r\'eciproque est appel\'ee fonction arccosinus, et est not\'ee $\arccos$. Faire sa repr\'esentation graphique.
		\item[$\bullet$] Montrer que la fonction sinus est bijective de $\left[-\frac{\pi}{2},\frac{\pi}{2}\right]$ dans $[-1,1]$. Sa bijection r\'eciproque est appel\'ee fonction arcsinus, et est not\'ee $\arcsin$. Faire sa repr\'esentation graphique.
	\end{itemize}
\end{exercice}}


{\footnotesize \begin{exercice}
	\begin{enumerate}
		\item \'Etude des points fixes de la fonction tangente sur $\left\rbrack \ddp\frac{\pi}{2},\ddp\frac{3\pi}{2}\right\lbrack$.\\
		\item Soit $n\in\N^{\star}$ fix\'e. Montrer qu'il existe un unique $u_n\in\left\rbrack 2n\pi,2n\pi+\ddp\frac{\pi}{2}\right\lbrack$ tel que $u_n\sin{(u_n)}=1$.
	\end{enumerate}
\end{exercice}}

%-------------------------------------------------------------
%----------------------------------------------------
%-------------------------------------------------------
%------------------------------------------------------------
\subsection{Fonction continue sur un segment}

\begin{defi} D\'efinition d'un segment de $\R$:\vsec\\
	Un segment de $\R$ est \dotfill\vsec
\end{defi}



\begin{exemples}
	\begin{itemize}
		\item[$\bullet$] Exemples de segment de $\R$: \dotfill \vsec
		\item[$\bullet$] Exemples d'intervalles de $\R$ non segment: \dotfill \vsec
	\end{itemize}
\end{exemples}\vsec\vsec

%---------------------------

\begin{theorem} Th\'eor\`{e}me d'une fonction continue sur un segment:\vsec
	\begin{itemize}
		\item[$\bullet$] \dotfill\vsec\\
		      \hspace*{0cm}\dotfill \vsec
		\item[$\bullet$] Plus pr\'ecisement, si $f$ est continue sur $\lbrack a,b\rbrack$ alors:
		      \vspace{2cm}
	\end{itemize}
\end{theorem}

{\footnotesize \begin{exercice}
	Montrer que la fonction $f$ d\'efinie par $f(x)=x^{12}-2x^5+3x^3-1$ admet un minimum sur $\R$.
\end{exercice}}


\end{document}