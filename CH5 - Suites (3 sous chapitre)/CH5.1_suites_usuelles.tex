\documentclass[a4paper, 11pt]{article}
\usepackage[utf8]{inputenc}
\usepackage{amssymb,amsmath,amsthm}
\usepackage{geometry}
\usepackage[T1]{fontenc}
\usepackage[french]{babel}
\usepackage{fontawesome}
\usepackage{pifont}
\usepackage{tcolorbox}
\usepackage{fancybox}
\usepackage{bbold}
\usepackage{tkz-tab}
\usepackage{tikz}
\usepackage{fancyhdr}
\usepackage{sectsty}
\usepackage[framemethod=TikZ]{mdframed}
\usepackage{stackengine}
\usepackage{scalerel}
\usepackage{xcolor}
\usepackage{hyperref}
\usepackage{listings}
\usepackage{enumitem}
\usepackage{stmaryrd} 
\usepackage{comment}


\hypersetup{
    colorlinks=true,
    urlcolor=blue,
    linkcolor=blue,
    breaklinks=true
}





\theoremstyle{definition}
\newtheorem{probleme}{Problème}
\theoremstyle{definition}


%%%%% box environement 
\newenvironment{fminipage}%
     {\begin{Sbox}\begin{minipage}}%
     {\end{minipage}\end{Sbox}\fbox{\TheSbox}}

\newenvironment{dboxminipage}%
     {\begin{Sbox}\begin{minipage}}%
     {\end{minipage}\end{Sbox}\doublebox{\TheSbox}}


%\fancyhead[R]{Chapitre 1 : Nombres}


\newenvironment{remarques}{ 
\paragraph{Remarques :}
	\begin{list}{$\bullet$}{}
}{
	\end{list}
}




\newtcolorbox{tcbdoublebox}[1][]{%
  sharp corners,
  colback=white,
  fontupper={\setlength{\parindent}{20pt}},
  #1
}







%Section
% \pretocmd{\section}{%
%   \ifnum\value{section}=0 \else\clearpage\fi
% }{}{}



\sectionfont{\normalfont\Large \bfseries \underline }
\subsectionfont{\normalfont\Large\itshape\underline}
\subsubsectionfont{\normalfont\large\itshape\underline}



%% Format théoreme, defintion, proposition.. 
\newmdtheoremenv[roundcorner = 5px,
leftmargin=15px,
rightmargin=30px,
innertopmargin=0px,
nobreak=true
]{theorem}{Théorème}

\newmdtheoremenv[roundcorner = 5px,
leftmargin=15px,
rightmargin=30px,
innertopmargin=0px,
]{theorem_break}[theorem]{Théorème}

\newmdtheoremenv[roundcorner = 5px,
leftmargin=15px,
rightmargin=30px,
innertopmargin=0px,
nobreak=true
]{corollaire}[theorem]{Corollaire}
\newcounter{defiCounter}
\usepackage{mdframed}
\newmdtheoremenv[%
roundcorner=5px,
innertopmargin=0px,
leftmargin=15px,
rightmargin=30px,
nobreak=true
]{defi}[defiCounter]{Définition}

\newmdtheoremenv[roundcorner = 5px,
leftmargin=15px,
rightmargin=30px,
innertopmargin=0px,
nobreak=true
]{prop}[theorem]{Proposition}

\newmdtheoremenv[roundcorner = 5px,
leftmargin=15px,
rightmargin=30px,
innertopmargin=0px,
]{prop_break}[theorem]{Proposition}

\newmdtheoremenv[roundcorner = 5px,
leftmargin=15px,
rightmargin=30px,
innertopmargin=0px,
nobreak=true
]{regles}[theorem]{Règles de calculs}


\newtheorem*{exemples}{Exemples}
\newtheorem{exemple}{Exemple}
\newtheorem*{rem}{Remarque}
\newtheorem*{rems}{Remarques}
% Warning sign

\newcommand\warning[1][4ex]{%
  \renewcommand\stacktype{L}%
  \scaleto{\stackon[1.3pt]{\color{red}$\triangle$}{\tiny\bfseries !}}{#1}%
}


\newtheorem{exo}{Exercice}
\newcounter{ExoCounter}
\newtheorem{exercice}[ExoCounter]{Exercice}

\newcounter{counterCorrection}
\newtheorem{correction}[counterCorrection]{\color{red}{Correction}}


\theoremstyle{definition}

%\newtheorem{prop}[theorem]{Proposition}
%\newtheorem{\defi}[1]{
%\begin{tcolorbox}[width=14cm]
%#1
%\end{tcolorbox}
%}


%--------------------------------------- 
% Document
%--------------------------------------- 






\lstset{numbers=left, numberstyle=\tiny, stepnumber=1, numbersep=5pt}




% Header et footer

\pagestyle{fancy}
\fancyhead{}
\fancyfoot{}
\renewcommand{\headwidth}{\textwidth}
\renewcommand{\footrulewidth}{0.4pt}
\renewcommand{\headrulewidth}{0pt}
\renewcommand{\footruleskip}{5px}

\fancyfoot[R]{Olivier Glorieux}
%\fancyfoot[R]{Jules Glorieux}

\fancyfoot[C]{ Page \thepage }
\fancyfoot[L]{1BIOA - Lycée Chaptal}
%\fancyfoot[L]{MP*-Lycée Chaptal}
%\fancyfoot[L]{Famille Lapin}



\newcommand{\Hyp}{\mathbb{H}}
\newcommand{\C}{\mathcal{C}}
\newcommand{\U}{\mathcal{U}}
\newcommand{\R}{\mathbb{R}}
\newcommand{\T}{\mathbb{T}}
\newcommand{\D}{\mathbb{D}}
\newcommand{\N}{\mathbb{N}}
\newcommand{\Z}{\mathbb{Z}}
\newcommand{\F}{\mathcal{F}}




\newcommand{\bA}{\mathbb{A}}
\newcommand{\bB}{\mathbb{B}}
\newcommand{\bC}{\mathbb{C}}
\newcommand{\bD}{\mathbb{D}}
\newcommand{\bE}{\mathbb{E}}
\newcommand{\bF}{\mathbb{F}}
\newcommand{\bG}{\mathbb{G}}
\newcommand{\bH}{\mathbb{H}}
\newcommand{\bI}{\mathbb{I}}
\newcommand{\bJ}{\mathbb{J}}
\newcommand{\bK}{\mathbb{K}}
\newcommand{\bL}{\mathbb{L}}
\newcommand{\bM}{\mathbb{M}}
\newcommand{\bN}{\mathbb{N}}
\newcommand{\bO}{\mathbb{O}}
\newcommand{\bP}{\mathbb{P}}
\newcommand{\bQ}{\mathbb{Q}}
\newcommand{\bR}{\mathbb{R}}
\newcommand{\bS}{\mathbb{S}}
\newcommand{\bT}{\mathbb{T}}
\newcommand{\bU}{\mathbb{U}}
\newcommand{\bV}{\mathbb{V}}
\newcommand{\bW}{\mathbb{W}}
\newcommand{\bX}{\mathbb{X}}
\newcommand{\bY}{\mathbb{Y}}
\newcommand{\bZ}{\mathbb{Z}}



\newcommand{\cA}{\mathcal{A}}
\newcommand{\cB}{\mathcal{B}}
\newcommand{\cC}{\mathcal{C}}
\newcommand{\cD}{\mathcal{D}}
\newcommand{\cE}{\mathcal{E}}
\newcommand{\cF}{\mathcal{F}}
\newcommand{\cG}{\mathcal{G}}
\newcommand{\cH}{\mathcal{H}}
\newcommand{\cI}{\mathcal{I}}
\newcommand{\cJ}{\mathcal{J}}
\newcommand{\cK}{\mathcal{K}}
\newcommand{\cL}{\mathcal{L}}
\newcommand{\cM}{\mathcal{M}}
\newcommand{\cN}{\mathcal{N}}
\newcommand{\cO}{\mathcal{O}}
\newcommand{\cP}{\mathcal{P}}
\newcommand{\cQ}{\mathcal{Q}}
\newcommand{\cR}{\mathcal{R}}
\newcommand{\cS}{\mathcal{S}}
\newcommand{\cT}{\mathcal{T}}
\newcommand{\cU}{\mathcal{U}}
\newcommand{\cV}{\mathcal{V}}
\newcommand{\cW}{\mathcal{W}}
\newcommand{\cX}{\mathcal{X}}
\newcommand{\cY}{\mathcal{Y}}
\newcommand{\cZ}{\mathcal{Z}}







\renewcommand{\phi}{\varphi}
\newcommand{\ddp}{\displaystyle}


\newcommand{\G}{\Gamma}
\newcommand{\g}{\gamma}

\newcommand{\tv}{\rightarrow}
\newcommand{\wt}{\widetilde}
\newcommand{\ssi}{\Leftrightarrow}

\newcommand{\floor}[1]{\left \lfloor #1\right \rfloor}
\newcommand{\rg}{ \mathrm{rg}}
\newcommand{\quadou}{ \quad \text{ ou } \quad}
\newcommand{\quadet}{ \quad \text{ et } \quad}
\newcommand\fillin[1][3cm]{\makebox[#1]{\dotfill}}
\newcommand\cadre[1]{[#1]}
\newcommand{\vsec}{\vspace{0.3cm}}

\DeclareMathOperator{\im}{Im}
\DeclareMathOperator{\cov}{Cov}
\DeclareMathOperator{\vect}{Vect}
\DeclareMathOperator{\Vect}{Vect}
\DeclareMathOperator{\card}{Card}
\DeclareMathOperator{\Card}{Card}
\DeclareMathOperator{\Id}{Id}
\DeclareMathOperator{\PSL}{PSL}
\DeclareMathOperator{\PGL}{PGL}
\DeclareMathOperator{\SL}{SL}
\DeclareMathOperator{\GL}{GL}
\DeclareMathOperator{\SO}{SO}
\DeclareMathOperator{\SU}{SU}
\DeclareMathOperator{\Sp}{Sp}


\DeclareMathOperator{\sh}{sh}
\DeclareMathOperator{\ch}{ch}
\DeclareMathOperator{\argch}{argch}
\DeclareMathOperator{\argsh}{argsh}
\DeclareMathOperator{\imag}{Im}
\DeclareMathOperator{\reel}{Re}



\renewcommand{\Re}{ \mathfrak{Re}}
\renewcommand{\Im}{ \mathfrak{Im}}
\renewcommand{\bar}[1]{ \overline{#1}}
\newcommand{\implique}{\Longrightarrow}
\newcommand{\equivaut}{\Longleftrightarrow}

\renewcommand{\fg}{\fg \,}
\newcommand{\intent}[1]{\llbracket #1\rrbracket }
\newcommand{\cor}[1]{{\color{red} Correction }#1}

\newcommand{\conclusion}[1]{\begin{center} \fbox{#1}\end{center}}


\renewcommand{\title}[1]{\begin{center}
    \begin{tcolorbox}[width=14cm]
    \begin{center}\huge{\textbf{#1 }}
    \end{center}
    \end{tcolorbox}
    \end{center}
    }

    % \renewcommand{\subtitle}[1]{\begin{center}
    % \begin{tcolorbox}[width=10cm]
    % \begin{center}\Large{\textbf{#1 }}
    % \end{center}
    % \end{tcolorbox}
    % \end{center}
    % }

\renewcommand{\thesection}{\Roman{section}} 
\renewcommand{\thesubsection}{\thesection.  \arabic{subsection}}
\renewcommand{\thesubsubsection}{\thesubsection. \alph{subsubsection}} 

\newcommand{\suiteu}{(u_n)_{n\in \N}}
\newcommand{\suitev}{(v_n)_{n\in \N}}
\newcommand{\suite}[1]{(#1_n)_{n\in \N}}
%\newcommand{\suite1}[1]{(#1_n)_{n\in \N}}
\newcommand{\suiteun}[1]{(#1_n)_{n\geq 1}}
\newcommand{\equivalent}[1]{\underset{#1}{\sim}}

\newcommand{\demi}{\frac{1}{2}}
\geometry{hmargin=2.0cm, vmargin=1.5cm}

\author{Olivier Glorieux}
\usetikzlibrary{matrix,arrows,decorations.pathmorphing}

\begin{document}


% \tableofcontents
\title{Chapitre 5.1 - Suites réelles 4 exemples}


\section{Suite arithm\'etique}

%-----------------------------------------------------------
%-----------------------------------
%\subsubsection{D\'efinition par r\'ecurrence:}

\begin{defi} D\'efinition d'une suite arithm\'etique:\\
Une suite $(u_n)_{n\geq 0}$ est dite arithm\'etique de raison $r$ si pour tout $n\in \N$:
$$ u_{n+1}=.....................$$
\end{defi}



%-----------------------------------------------------------
%-----------------------------------
%\subsubsection{Expression explicite:}
{  

\begin{prop} Soit $\suite{u}$ une suite arithm\'etique de raison $r$ et de premier terme $u_0$.\vspace{0.3cm}
\begin{itemize}
\item[$\bullet$] Expression explicite : $u_n =..................$ 
\item[$\bullet$] Limite : $\lim\limits_{n\to +\infty} u_n = \left\{ \begin{array}{c} \vspace*{2cm} \end{array} \right.$
\item[$\bullet$] Somme des termes : $\ddp \sum\limits_{k=0}^n u_k = $
\end{itemize}
\end{prop}

}

% \begin{rem}
% Une suite $(u_n)_{n\geq p}$ est arithm\'etique de raison $r$ et de premier terme $u_{p}$ si\dotfill\\
% On a alors $u_n = \dotfill$ et $\ddp \sum_{k=p}^n u_k = \dotfill$.\\
% \end{rem}

%-----------------------------------------------------------
%-----------------------------------
%-----------------------------------------------------------
%----------------------------------------------------------
\section{Suite g\'eom\'etrique}

%-----------------------------------------------------------
%-----------------------------------
%\subsubsection{D\'efinition par r\'ecurrence:}\vspace{0.3cm}

{  

\begin{defi} D\'efinition d'une suite g\'eom\'etrique:
Une suite $(u_n)_{n\geq 0}$ est dite g\'eom\'etrique de raison $q$
$$ u_{n+1}=.....................$$
\end{defi}

}\vspace{0.3cm}

{  

\begin{prop} Soit $\suite{u}$ une suite g\'eom\'etrique de raison $q$ et de premier terme $u_0$.\vspace{0.3cm}
\begin{itemize}
\item[$\bullet$] Expression explicite : $u_n =$ 
\item[$\bullet$] Limite (pour $u_0>0$) : $\lim\limits_{n\to +\infty} u_n = \left\{ \begin{array}{c} \vspace*{3cm} \end{array} \right.$
\item[$\bullet$] Somme des termes : $\ddp \sum\limits_{k=0}^n u_k =  \left\{ \begin{array}{c} \vspace*{2cm} \end{array} \right.$
\end{itemize}
\end{prop}

}


% \begin{rem}
% Une suite $(u_n)_{n\geq p}$ est g\'eom\'etrique de raison $q$ et de premier terme $u_{p}$ si\dotfill\\
% On a alors $u_n = \dotfill$ et $\ddp \sum_{k=p}^n u_k = \dotfill$.
% \end{rem}

%-----------------------------------------------------------
\newpage
%----------------------------------------------------------
\section{Suite arithm\'etico-g\'eom\'etrique}

%-----------------------------------------------------------
%-----------------------------------

{  

\begin{defi} Soit $\suite{u}$ une suite r\'eelle. On dit qu'elle est arithm\'etico-g\'eom\'etrique s'il existe deux r\'eels $a$ et $b$ ($a\not=1$ et $b\not=0$ sinon on est dans les deux cas pr\'ec\'edents) tels que pour tout $n\in\N$
$$ u_{n+1}=.....................$$
\vspace{0.1cm}
\end{defi}

}\vspace{0.3cm}


%-----------------------------------------------------------
%-----------------------------------
%\subsection{Expression explicite: M\'ethode}\vspace{0.3cm}

\setlength\fboxrule{1pt}
 {
\begin{minipage}[t]{0.9\textwidth}
\begin{itemize}
\item[$\bullet$] \'Etude d'une suite auxiliaire $\suite{v}$ d\'efinie par: $\forall n\in\N,\ v_n=u_n-\alpha$.
\begin{itemize}
\item[$\star$]Chercher $\alpha$  tel que la suite $\suite{v}$ soit g\'eom\'etrique de raison $a$.
\item[$\star$] En d\'eduire son expression explicite de $v_n$
\end{itemize}
\item[$\bullet$] Expression explicite de $\suite{u}$ en utilisant: $\forall n\in\N,\ u_n=v_n+\alpha$. 
\end{itemize}
\end{minipage}}
\setlength\fboxrule{0.5pt}
\begin{rem}
\warning \, Les r\'eels $a$ et $b$ ne doivent pas d\'ependre de $n$. La suite $u_{n+1}= nu_n+3$ n'est pas arithmético-géométrique. La méthode présentée ensuite ne fonctionne pas. 
\end{rem}

\vspace{0.3cm}
\begin{exemple}
Soit $\suite{u}$ la suite d\'efinie par: $u_0=2$ et pour tout $n\in\N$: $u_{n+1}=3u_n+4$. Calculer $u_n$.



\begin{enumerate}
\item \textbf{Chercher $\alpha$  tel que la suite $\suite{v}$ soit g\'eom\'etrique de raison $a$.}
\vspace*{9cm}


\item \textbf{Expression de  $\mathbf{\suite{v}}$}
\vspace*{3cm}

\item \textbf{Retour à la suite $\mathbf{\suite{u}}$}
\vspace*{3cm}

\end{enumerate}
\end{exemple}

% {\footnotesize 
% \begin{exercice} 
% On d\'efinit la suite $\suite{u}$ par: $u_0=4$ et pour tout $n\in\N$: $u_{n+1}=-\ddp\demi u_n+1$. Donner son expression explicite, sa limite et la somme $\ddp \sum_{k=0}^n u_k$.
% %\begin{enumerate}
% %\item On d\'efinit la suite $\suite{u}$ par: $u_0=4$ et pour tout $n\in\N$: $u_{n+1}=-\ddp\demi u_n+1$. Donner son expression explicite, sa limite et la somme de ses $n$ premiers termes.
% %\item On d\'efinit la suite $\suite{u}$ par: $u_0=0$ et pour tout $n\in\N$: $u_{n+1}=\ddp\frac{8}{9}u_n+\ddp\frac{1}{9}$. Donner son expression explicite, sa limite et la somme de ses $n$ premiers termes.
% %\end{enumerate}
% \end{exercice}}



% {\footnotesize 
% \begin{exercice} 
% Trouver le terme g\'en\'eral des suites d\'efinies par $u_{n+1} = n u_n, u_{n+1}= \ddp \frac{2}{n} u_n$, et $u_{n+1}=u_n^2$.
% \end{exercice}}

%-----------------------------------------------------------
\newpage
%----------------------------------------------------------
\section{Suite r\'ecurrente lin\'eaire d'ordre deux}

%-----------------------------------------------------------
%-----------------------------------


\begin{defi} 
Soient $(a,b)\in\bR^2$ avec $b\not=0$. On appelle suite r\'ecurrente lin\'eaire d'ordre deux toute suite $\suite{u}$ v\'erifiant la relation de r\'ecurrence tels que pour tout $n\in\N$
$$ u_{n+2}=......................................$$
avec deux conditions initiales donn\'ees ($u_0$ et $u_1$).
\end{defi}

%-----------------------------------------------------------
%-----------------------------------

\setlength\fboxrule{1pt}

\begin{itemize}
\item[$\bullet$] R\'esolution de l'\'equation caract\'eristique associ\'ee \`{a} la suite:
$$(E)\quad..................................$$
\item[$\bullet$] Expression explicite de la suite selon le signe du discriminant de l'\'equation caract\'eristique:
\begin{itemize}
\item[$\star$] Si $\Delta>0$: $(E)$ a deux solutions r\'eelles distinctes $r_1$ et $r_2$, et l'expression explicite de la suite est :

\framebox(400,40){$ \exists (\alpha,\beta)\in\bR^2,\forall n\in\N, u_n=\hspace{9cm}$ }

\vspace{0.1cm}

\item[$\star$] Si $\Delta=0$, $(E)$ a une solution r\'eelle double $r_0$, et l'expression explicite de la suite est :

\framebox(400,40){$ \exists (\alpha,\beta)\in\bR^2,\forall n\in\N, u_n=\hspace{9cm}$ }
\vspace{0.1cm}
\item[$\star$] Si $\Delta<0$:  $(E)$ a deux solutions complexes conjugu\'ees que l'on \'ecrit sous forme exponentielle $\rho e^{i\theta}$ et $\rho e^{-i\theta}$ (avec $\rho>0$ et $\theta\in\bR$). L'expression explicite de la suite est alors:

\framebox(400,40){$ \exists (\alpha,\beta)\in\bR^2,\forall n\in\N, u_n=\hspace{9cm}$ }
\vspace{0.1cm}
\end{itemize}
\item[$\bullet$] Calcul des constantes $\alpha$ et $\beta$ \`{a} l'aide des valeurs des conditions initiales $u_0$ et $u_1$ en r\'esolvant un syst\`eme lin\'eaire. 
\end{itemize}




\begin{exemple}
\'Etudier la suite $\suite{u}$ d\'efinie par: $u_0=1$ et $u_1=2$ et $\forall n\in\N,  u_{n+2}=-2u_{n+1}+3u_n$.

\begin{enumerate}
\item \textbf{Résolution de l'équation caractéristique}
\vspace*{4cm}


\item \textbf{Expression explicite de $\suite{u}$ avec constantes à déterminer }
\vspace*{3cm}

\item \textbf{Calcul des constantes}
\vspace*{7cm}

\end{enumerate}
\end{exemple}







\end{document}
