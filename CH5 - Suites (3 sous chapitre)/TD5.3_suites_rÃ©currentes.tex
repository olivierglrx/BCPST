\documentclass[a4paper, 11pt]{article}
\usepackage[utf8]{inputenc}
\usepackage{amssymb,amsmath,amsthm}
\usepackage{geometry}
\usepackage[T1]{fontenc}
\usepackage[french]{babel}
\usepackage{fontawesome}
\usepackage{pifont}
\usepackage{tcolorbox}
\usepackage{fancybox}
\usepackage{bbold}
\usepackage{tkz-tab}
\usepackage{tikz}
\usepackage{fancyhdr}
\usepackage{sectsty}
\usepackage[framemethod=TikZ]{mdframed}
\usepackage{stackengine}
\usepackage{scalerel}
\usepackage{xcolor}
\usepackage{hyperref}
\usepackage{listings}
\usepackage{enumitem}
\usepackage{stmaryrd} 
\usepackage{comment}


\hypersetup{
    colorlinks=true,
    urlcolor=blue,
    linkcolor=blue,
    breaklinks=true
}





\theoremstyle{definition}
\newtheorem{probleme}{Problème}
\theoremstyle{definition}


%%%%% box environement 
\newenvironment{fminipage}%
     {\begin{Sbox}\begin{minipage}}%
     {\end{minipage}\end{Sbox}\fbox{\TheSbox}}

\newenvironment{dboxminipage}%
     {\begin{Sbox}\begin{minipage}}%
     {\end{minipage}\end{Sbox}\doublebox{\TheSbox}}


%\fancyhead[R]{Chapitre 1 : Nombres}


\newenvironment{remarques}{ 
\paragraph{Remarques :}
	\begin{list}{$\bullet$}{}
}{
	\end{list}
}




\newtcolorbox{tcbdoublebox}[1][]{%
  sharp corners,
  colback=white,
  fontupper={\setlength{\parindent}{20pt}},
  #1
}







%Section
% \pretocmd{\section}{%
%   \ifnum\value{section}=0 \else\clearpage\fi
% }{}{}



\sectionfont{\normalfont\Large \bfseries \underline }
\subsectionfont{\normalfont\Large\itshape\underline}
\subsubsectionfont{\normalfont\large\itshape\underline}



%% Format théoreme, defintion, proposition.. 
\newmdtheoremenv[roundcorner = 5px,
leftmargin=15px,
rightmargin=30px,
innertopmargin=0px,
nobreak=true
]{theorem}{Théorème}

\newmdtheoremenv[roundcorner = 5px,
leftmargin=15px,
rightmargin=30px,
innertopmargin=0px,
]{theorem_break}[theorem]{Théorème}

\newmdtheoremenv[roundcorner = 5px,
leftmargin=15px,
rightmargin=30px,
innertopmargin=0px,
nobreak=true
]{corollaire}[theorem]{Corollaire}
\newcounter{defiCounter}
\usepackage{mdframed}
\newmdtheoremenv[%
roundcorner=5px,
innertopmargin=0px,
leftmargin=15px,
rightmargin=30px,
nobreak=true
]{defi}[defiCounter]{Définition}

\newmdtheoremenv[roundcorner = 5px,
leftmargin=15px,
rightmargin=30px,
innertopmargin=0px,
nobreak=true
]{prop}[theorem]{Proposition}

\newmdtheoremenv[roundcorner = 5px,
leftmargin=15px,
rightmargin=30px,
innertopmargin=0px,
]{prop_break}[theorem]{Proposition}

\newmdtheoremenv[roundcorner = 5px,
leftmargin=15px,
rightmargin=30px,
innertopmargin=0px,
nobreak=true
]{regles}[theorem]{Règles de calculs}


\newtheorem*{exemples}{Exemples}
\newtheorem{exemple}{Exemple}
\newtheorem*{rem}{Remarque}
\newtheorem*{rems}{Remarques}
% Warning sign

\newcommand\warning[1][4ex]{%
  \renewcommand\stacktype{L}%
  \scaleto{\stackon[1.3pt]{\color{red}$\triangle$}{\tiny\bfseries !}}{#1}%
}


\newtheorem{exo}{Exercice}
\newcounter{ExoCounter}
\newtheorem{exercice}[ExoCounter]{Exercice}

\newcounter{counterCorrection}
\newtheorem{correction}[counterCorrection]{\color{red}{Correction}}


\theoremstyle{definition}

%\newtheorem{prop}[theorem]{Proposition}
%\newtheorem{\defi}[1]{
%\begin{tcolorbox}[width=14cm]
%#1
%\end{tcolorbox}
%}


%--------------------------------------- 
% Document
%--------------------------------------- 






\lstset{numbers=left, numberstyle=\tiny, stepnumber=1, numbersep=5pt}




% Header et footer

\pagestyle{fancy}
\fancyhead{}
\fancyfoot{}
\renewcommand{\headwidth}{\textwidth}
\renewcommand{\footrulewidth}{0.4pt}
\renewcommand{\headrulewidth}{0pt}
\renewcommand{\footruleskip}{5px}

\fancyfoot[R]{Olivier Glorieux}
%\fancyfoot[R]{Jules Glorieux}

\fancyfoot[C]{ Page \thepage }
\fancyfoot[L]{1BIOA - Lycée Chaptal}
%\fancyfoot[L]{MP*-Lycée Chaptal}
%\fancyfoot[L]{Famille Lapin}



\newcommand{\Hyp}{\mathbb{H}}
\newcommand{\C}{\mathcal{C}}
\newcommand{\U}{\mathcal{U}}
\newcommand{\R}{\mathbb{R}}
\newcommand{\T}{\mathbb{T}}
\newcommand{\D}{\mathbb{D}}
\newcommand{\N}{\mathbb{N}}
\newcommand{\Z}{\mathbb{Z}}
\newcommand{\F}{\mathcal{F}}




\newcommand{\bA}{\mathbb{A}}
\newcommand{\bB}{\mathbb{B}}
\newcommand{\bC}{\mathbb{C}}
\newcommand{\bD}{\mathbb{D}}
\newcommand{\bE}{\mathbb{E}}
\newcommand{\bF}{\mathbb{F}}
\newcommand{\bG}{\mathbb{G}}
\newcommand{\bH}{\mathbb{H}}
\newcommand{\bI}{\mathbb{I}}
\newcommand{\bJ}{\mathbb{J}}
\newcommand{\bK}{\mathbb{K}}
\newcommand{\bL}{\mathbb{L}}
\newcommand{\bM}{\mathbb{M}}
\newcommand{\bN}{\mathbb{N}}
\newcommand{\bO}{\mathbb{O}}
\newcommand{\bP}{\mathbb{P}}
\newcommand{\bQ}{\mathbb{Q}}
\newcommand{\bR}{\mathbb{R}}
\newcommand{\bS}{\mathbb{S}}
\newcommand{\bT}{\mathbb{T}}
\newcommand{\bU}{\mathbb{U}}
\newcommand{\bV}{\mathbb{V}}
\newcommand{\bW}{\mathbb{W}}
\newcommand{\bX}{\mathbb{X}}
\newcommand{\bY}{\mathbb{Y}}
\newcommand{\bZ}{\mathbb{Z}}



\newcommand{\cA}{\mathcal{A}}
\newcommand{\cB}{\mathcal{B}}
\newcommand{\cC}{\mathcal{C}}
\newcommand{\cD}{\mathcal{D}}
\newcommand{\cE}{\mathcal{E}}
\newcommand{\cF}{\mathcal{F}}
\newcommand{\cG}{\mathcal{G}}
\newcommand{\cH}{\mathcal{H}}
\newcommand{\cI}{\mathcal{I}}
\newcommand{\cJ}{\mathcal{J}}
\newcommand{\cK}{\mathcal{K}}
\newcommand{\cL}{\mathcal{L}}
\newcommand{\cM}{\mathcal{M}}
\newcommand{\cN}{\mathcal{N}}
\newcommand{\cO}{\mathcal{O}}
\newcommand{\cP}{\mathcal{P}}
\newcommand{\cQ}{\mathcal{Q}}
\newcommand{\cR}{\mathcal{R}}
\newcommand{\cS}{\mathcal{S}}
\newcommand{\cT}{\mathcal{T}}
\newcommand{\cU}{\mathcal{U}}
\newcommand{\cV}{\mathcal{V}}
\newcommand{\cW}{\mathcal{W}}
\newcommand{\cX}{\mathcal{X}}
\newcommand{\cY}{\mathcal{Y}}
\newcommand{\cZ}{\mathcal{Z}}







\renewcommand{\phi}{\varphi}
\newcommand{\ddp}{\displaystyle}


\newcommand{\G}{\Gamma}
\newcommand{\g}{\gamma}

\newcommand{\tv}{\rightarrow}
\newcommand{\wt}{\widetilde}
\newcommand{\ssi}{\Leftrightarrow}

\newcommand{\floor}[1]{\left \lfloor #1\right \rfloor}
\newcommand{\rg}{ \mathrm{rg}}
\newcommand{\quadou}{ \quad \text{ ou } \quad}
\newcommand{\quadet}{ \quad \text{ et } \quad}
\newcommand\fillin[1][3cm]{\makebox[#1]{\dotfill}}
\newcommand\cadre[1]{[#1]}
\newcommand{\vsec}{\vspace{0.3cm}}

\DeclareMathOperator{\im}{Im}
\DeclareMathOperator{\cov}{Cov}
\DeclareMathOperator{\vect}{Vect}
\DeclareMathOperator{\Vect}{Vect}
\DeclareMathOperator{\card}{Card}
\DeclareMathOperator{\Card}{Card}
\DeclareMathOperator{\Id}{Id}
\DeclareMathOperator{\PSL}{PSL}
\DeclareMathOperator{\PGL}{PGL}
\DeclareMathOperator{\SL}{SL}
\DeclareMathOperator{\GL}{GL}
\DeclareMathOperator{\SO}{SO}
\DeclareMathOperator{\SU}{SU}
\DeclareMathOperator{\Sp}{Sp}


\DeclareMathOperator{\sh}{sh}
\DeclareMathOperator{\ch}{ch}
\DeclareMathOperator{\argch}{argch}
\DeclareMathOperator{\argsh}{argsh}
\DeclareMathOperator{\imag}{Im}
\DeclareMathOperator{\reel}{Re}



\renewcommand{\Re}{ \mathfrak{Re}}
\renewcommand{\Im}{ \mathfrak{Im}}
\renewcommand{\bar}[1]{ \overline{#1}}
\newcommand{\implique}{\Longrightarrow}
\newcommand{\equivaut}{\Longleftrightarrow}

\renewcommand{\fg}{\fg \,}
\newcommand{\intent}[1]{\llbracket #1\rrbracket }
\newcommand{\cor}[1]{{\color{red} Correction }#1}

\newcommand{\conclusion}[1]{\begin{center} \fbox{#1}\end{center}}


\renewcommand{\title}[1]{\begin{center}
    \begin{tcolorbox}[width=14cm]
    \begin{center}\huge{\textbf{#1 }}
    \end{center}
    \end{tcolorbox}
    \end{center}
    }

    % \renewcommand{\subtitle}[1]{\begin{center}
    % \begin{tcolorbox}[width=10cm]
    % \begin{center}\Large{\textbf{#1 }}
    % \end{center}
    % \end{tcolorbox}
    % \end{center}
    % }

\renewcommand{\thesection}{\Roman{section}} 
\renewcommand{\thesubsection}{\thesection.  \arabic{subsection}}
\renewcommand{\thesubsubsection}{\thesubsection. \alph{subsubsection}} 

\newcommand{\suiteu}{(u_n)_{n\in \N}}
\newcommand{\suitev}{(v_n)_{n\in \N}}
\newcommand{\suite}[1]{(#1_n)_{n\in \N}}
%\newcommand{\suite1}[1]{(#1_n)_{n\in \N}}
\newcommand{\suiteun}[1]{(#1_n)_{n\geq 1}}
\newcommand{\equivalent}[1]{\underset{#1}{\sim}}

\newcommand{\demi}{\frac{1}{2}}
\geometry{hmargin=2.0cm, vmargin=1.5cm}

\author{Olivier Glorieux}


\newcommand{\type}{TD }
\excludecomment{correction}
%\renewcommand{\type}{Correction TD }

\begin{document}
\title{\type  5.3 - Suites récurrentes $u_{n+1}=f(u_n)$}


\section*{Entraînements}
\begin{exercice} \;
\'Etudier la suite $\suiteu$ d\'efinie par $u_0=0$ et $\forall n \geq 1$, $u_{n+1}=\ddp\frac{(1+u_n)^2}{4}.$
%$\left\lbrace\begin{array}{l}
%u_0=0\vsec\\
%u_{n+1}=\ddp\frac{(1+u_n)^2}{4}.
%\end{array}\right.$
\end{exercice}

\begin{correction} \;
C'est une suite de type $u_{n+1}=f(u_n)$, on donne les id\'ees de l'\'etude. Ainsi la r\'edaction dans une copie doit \^{e}tre beaucoup plus d\'etaill\'ee qu'ici.
\begin{enumerate}
 \item \'Etude de la fonction $f$ associ\'ee: $x\mapsto \ddp\frac{(1+x)^2}{4}$\\
\noindent La fonction $f$ est d\'efinie, continue et d\'erivable sur $\R$ et 
$$\forall x\in\R,\ f^{\prime}(x)=\ddp\frac{1+x}{2}.$$
On obtient ainsi le tableau de variation suivant:
\begin{center}
\begin{tikzpicture}
 \tkzTabInit{ $x$          /1,%
       $f'(x)$      /1,%
       $f$       /2}%
     { $-\infty$, $-1$ ,$+\infty$ }%
  \tkzTabLine {,-,0,+,}%
  \tkzTabVar{
       +/ $+\infty$        /,
        -/$0$           /,%
       +/$+\infty$           /,
                      }
 \tkzTabVal[draw]{2}{3}{0.3}{$0$}{$\frac{1}{4}$}
  \tkzTabVal[draw]{2}{3}{0.6}{$1$}{$1$}
\end{tikzpicture}
\end{center}
\item Calcul des limites \'eventuelles:\\
\noindent La fonction $f$ est continue sur $\R$, ainsi, si la suite $\suiteu$ converge, elle ne peut converger que vers $l$ v\'erifiant 
$$l=f(l)\Leftrightarrow l=1.$$
Ainsi, 1 est la seule limite \'eventuelle de la suite.
\item La suite est bien d\'efinie et elle appartient \`{a} $I$ intervalle stable par $f$:\\
\noindent On remarque que $\lbrack 0,1\rbrack$ est un intervalle stable par $f$ et que $u_0=0\in\lbrack 0,1\rbrack$. Un raisonnement par r\'ecurrence permet alors de v\'erifier que la suite est bien d\'efinie et que:
$$\forall n\in\N,\quad u_n\in\lbrack 0,1\rbrack.$$
\noindent 
\item \'Etude de la monotonie de la suite:\\
La fonction $f$ est croissante sur $\lbrack 0,1\rbrack$ et la suite $\suiteu$ est bien \`a valeurs dans $\lbrack 0,1\rbrack$, ainsi, la suite $\suiteu$ est monotone. Il suffit alors de comparer $u_1$ et $u_0$ et on obtient
$$u_1=\ddp\frac{1}{4}>u_0.$$
Ainsi, un raisonnement par r\'ecurrence permet de montrer que la suite $\suiteu$ est croissante (voir les exemples du cours, le raisonnement par r\'ecurrence est obligatoire).
\noindent 
\item \'Etude de la convergence de la suite:\\
\noindent La suite $\suiteu$ est ainsi croissante et major\'ee par 1, elle converge donc vers une limite finie $l\in\R$ d'apr\`es le th\'eor\`eme sur les suites monotones. De plus, comme la seule limite \'eventuelle est 1, on sait que la suite $\suiteu$ converge vers 1.
\end{enumerate}
\end{correction}










%--------------------------------------------------------------------------------------
%--------------------------------------------------------------------------------------
%\begin{exercice}
%\'Etudier la suite $\suiteu$ d\'efinie par 
%$\left\lbrace\begin{array}{l}
%u_0=1\vsec\\
%u_{n+1}=u_n^2+u_n.
%\end{array}\right.$
%\end{exercice}
%\begin{correction}
%\textbf{\'Etudions la suite $\mathbf{\suiteu}$ d\'efinie par $\mathbf{\left\lbrace\begin{array}{l}
%u_0=1\vsec\\
%u_{n+1}=u_n^2+u_n.
%\end{array}\right.}$}\\
%\noindent C'est une suite de type $u_{n+1}=f(u_n)$, on donne les id\'ees de l'\'etude. Ainsi la r\'edaction dans une copie doit \^{e}tre beaucoup plus d\'etaill\'ee qu'ici.
%\begin{enumerate}
% \item \textbf{\'Etude des variations de la fonction $\mathbf{f}$ associ\'ee: $\mathbf{x\mapsto x^2+x}$:}\\
%\noindent La fonction $f$ est d\'efinie, continue et d\'erivable sur $\R$ et 
%$$\forall x\in\R,\ f^{\prime}(x)=2x+1.$$
%On obtient ainsi le tableau de variation suivant:
%\begin{center}
%\begin{tikzpicture}
% \tkzTabInit{ $x$          /,%
%	%$\sin{(3x)}$             /,
%       %$\cos{(5x)}$     /,%
%       $f'(x)$      /,%
%       $f$       /2}%
%     { $-\infty$, $-\ddp\demi$ ,$+\infty$ }%
%  %\tkzTabLine {0,$+$,t,$+$,t,$+$,0,$-$,t}%
%  %\tkzTabLine {t,$+$,0,$-$,0,$+$,t,$+$,0}%
%  \tkzTabLine {,$-$,0,$+$,}%
%  \tkzTabVar{
%     % {-/ $1$       /,%
%       +/ $+\infty$        /,
%        -/$-\ddp\frac{1}{4}$           /,%
%       +/$+\infty$           /,
%      % +/           /,%
%       %-/ $-1$          /,
%       %R/           /,%
%       %-/ $-\infty$ /}
%                      }
% \tkzTabVal[draw]{2}{3}{0.3}{$0$}{$0$}
% \tkzTabVal[draw]{2}{3}{0.6}{$1$}{$2$}
%\end{tikzpicture}
%\end{center}
%\item \textbf{\'Etude du signe de la fonction $\mathbf{g: x\mapsto f(x)-x=x^2}$:}\\
%\noindent \fbox{Cette fonction est toujours positive sur $\R$ et elle ne s'annule qu'en 0.}
%\item \textbf{Calcul des limites \'eventuelles:}\\
%\noindent La fonction $f$ est continue sur $\R$, ainsi, si la suite $\suiteu$ converge, elle ne peut converger que vers $l$ v\'erifiant 
%$$l=f(l)\Leftrightarrow g(l)=0\Leftrightarrow l=0.$$
%Ainsi, \fbox{0 est la seule limite \'eventuelle de la suite.}
%\item \textbf{Montrons que la suite est bien d\'efinie et elle appartient \`{a} $\mathbf{I}$ intervalle stable par $\mathbf{f}$}:\\
%\noindent On remarque que $\lbrack 1,+\infty\brack$ est un intervalle stable par $f$ et que $u_0=1\in\lbrack 1,+\infty\lbrack$. Un raisonnement par r\'ecurrence permet alors de v\'erifier que \fbox{la suite est bien d\'efinie et que $\forall n\in\N,\quad u_n\geq 1.$}
%\noindent 
%\item \textbf{\'Etude de la monotonie de la suite:}\\
%Comme $g$ est positive sur $\R$, \fbox{la suite est croissante.}
%\noindent 
%\item \textbf{\'Etude de la convergence de la suite:}\\
%\begin{itemize}
%\item[$\star$] La suite $\suiteu$ est croissante donc d'apr\`{e}s le th\'eor\`{e}me sur les suites monotones, elle converge ou elle diverge vers $+\infty$.
%\item[$\star$] On suppose par l'absurde que la suite $\suiteu$ converge vers un r\'eel $l$. On a alors:
%\begin{itemize}
%\item[$\circ$] La suite $\suiteu$ converge vers $l$.
%\item[$\circ$] Pour tout $n\in\N$: $u_n\geq 1$.
%\end{itemize}
%D'apr\`{e}s le th\'eor\`{e}me de passage \`{a} la limite, on obtient donc que: $l\geq 1$. Absurde car la seule limite \'eventuelle de la suite $\suiteu$ est 0. Ainsi \fbox{la suite $\suiteu$ diverge vers $+\infty$.}
%\end{itemize}
%\end{enumerate}
%
%
%
%\end{correction}

%--------------------------------------------------------------------------------------
%--------------------------------------------------------------------------------------
\begin{exercice} \;
\'Etudier la suite $\suiteu$ d\'efinie par $u_0=\ddp\demi$ et $\forall n \geq 1$, $u_{n+1}=\ddp\sqrt{1+u_n^2}.$
%$\left\lbrace\begin{array}{l}
%u_0=\ddp\demi\vsec\\
%u_{n+1}=\ddp\sqrt{1+u_n^2}.
%\end{array}\right.$
\end{exercice}
%--------------------------------------------------------------------------------------

\begin{correction} \;
C'est une suite de type $u_{n+1}=f(u_n)$, on donne les id\'ees de l'\'etude. Ainsi la r\'edaction dans une copie doit \^{e}tre beaucoup plus d\'etaill\'ee qu'ici.
\begin{enumerate}
 \item \'Etude de la fonction $f$ associ\'ee: $x\mapsto \sqrt{1+x^2}$\\
\noindent La fonction $f$ est d\'efinie, continue et d\'erivable sur $\R$ et 
$$\forall x\in\R,\ f^{\prime}(x)=\ddp\frac{x}{\sqrt{1+x^2}}.$$
On obtient ainsi le tableau de variation suivant:
\begin{center}
\begin{tikzpicture}
 \tkzTabInit{ $x$          /1,%
       $f'(x)$      /1,%
       $f(x)$       /2}%
     { $-\infty$, $0$ ,$+\infty$ }%
  \tkzTabLine {,$-$,0,$+$,}%
  \tkzTabVar{
       +/ $+\infty$        /,
        -/$1$           /,%
       +/$+\infty$           /,
                      } 
\end{tikzpicture}
\end{center}
\item Calcul des limites \'eventuelles:\\
\noindent La fonction $f$ est continue sur $\R$, ainsi, si la suite $\suiteu$ converge, elle ne peut converger que vers $l$ v\'erifiant 
$l=f(l)$. Un calcul rapide montre qu'il n'y a pas de limite \'eventuelle.
\item La suite est bien d\'efinie et elle appartient \`{a} $I$ intervalle stable par $f$:\\
\noindent On remarque que $\lbrack 0,+\infty\lbrack$ est un intervalle stable par $f$ et que $u_0=\ddp\demi\in\lbrack 0,+\infty\lbrack$. Un raisonnement par r\'ecurrence permet alors de v\'erifier que la suite est bien d\'efinie et que:
$$\forall n\in\N,\quad u_n\geq 0.$$
\item \'Etude de la monotonie de la suite:\\
La fonction $f$ est croissante sur $\lbrack 0,+\infty\lbrack$ et la suite $\suiteu$ est bien \`a valeurs dans $\lbrack 0,+\infty\lbrack$, ainsi, la suite $\suiteu$ est monotone. Il suffit alors de comparer $u_1$ et $u_0$ et on obtient
$$u_1=\ddp\frac{\sqrt{5}}{2}>u_0.$$
Ainsi, un raisonnemnt par r\'ecurrence permet de montrer que la suite $\suiteu$ est croissante (voir les exemples du cours, le raisonnement par r\'ecurrence est obligatoire).
\item \'Etude de la convergence de la suite:\\
\noindent La suite $\suiteu$ est croissante donc d'apr\`es le th\'eor\`eme sur les suites monotones, elle tend vers une limite finie ou $+\infty$. Comme il n'y a pas de limite \'eventuelle possible, on montre 
par un raisonnement rapide par l'absurde que la suite $\suiteu$ tend vers $+\infty$
\end{enumerate}
\end{correction}















%--------------------------------------------------------------------------------------
\begin{exercice} \;
\'Etudier la suite $\suiteu$ d\'efinie par  $u_0\in \R$ et $\forall n \geq 1$, $u_{n+1}=e^{u_n}.$
%
%$\left\lbrace\begin{array}{l}
%u_0\in\R\vsec\\
%u_{n+1}=e^{u_n}.
%\end{array}\right.$
\end{exercice}

\begin{correction} \;
C'est une suite de type $u_{n+1}=f(u_n)$, on donne les id\'ees de l'\'etude.
\begin{enumerate}
 \item \'Etude de la fonction $f$ associ\'ee: $x\mapsto e^x$\\
\noindent  La fonction $f$ est d\'efinie, continue et d\'erivable sur $\R$ et 
$$\forall x\in\R,\ f^{\prime}(x)=e^x.$$
On obtient ainsi le tableau de variation suivant:
\begin{center}
\begin{tikzpicture}
 \tkzTabInit{ $x$          /1,%
       %$f'(x)$      /,%
       $f(x)$       /2}%
     { $-\infty$ ,$+\infty$ }%
  \tkzTabVar{
       -/ $0$        /,
       +/$+\infty$           /,
                      } 
 \tkzTabVal[draw]{1}{2}{0.5}{$0$}{$1$}
\end{tikzpicture}
\end{center}
\item Calcul des limites \'eventuelles:\\
\noindent La fonction $f$ est continue sur $\R$, ainsi, si la suite $\suiteu$ converge, elle ne peut converger que vers $l$ v\'erifiant 
$l=f(l)$. 
\'Etudions alors la fonction $g:\ x\mapsto g(x)=f(x)-x$. L'\'etude d'une telle fonction donne que:
$$\forall x\in\R,\quad g(x)\geq 1.$$
En particulier, il n'y a pas de valeur d'annulation de $g$ et donc il n'y a pas de limite \'eventuelle pour la suite $\suiteu$.
\item La suite est bien d\'efinie et elle appartient \`{a} $I$ intervalle stable par $f$:\\
\noindent $\R^+$ est un intervalle stable par $f$ et $u_1>0$.
\noindent Ainsi, on montre par r\'ecurrence que la suite est bien d\'efinie et que : $\forall n\in\N^{\star},\quad u_n \geq 0$. (On ne commence pas au rang 0 car $u_0\in\R$).
\item \'Etude de la monotonie de la suite:\\
\noindent Soit $n\in\N$, on a: $u_{n+1}-u_n=g(u_n)\geq 1>0$. Donc la suite $\suiteu$ est croissante.
\item \'Etude de la convergence de la suite:\\
\noindent La suite $\suiteu$ est croissante donc d'apr\`es le th\'eor\`eme sur les suites monotones, elle tend vers une limite finie ou $+\infty$. Comme il n'y a pas de limite \'eventuelle possible, par un raisonnement par l'absurde, on obtient que la suite $\suiteu$ tend vers $+\infty$
\end{enumerate}
\end{correction}




\section*{Type DS}


\begin{exercice} \;
On d\'efinit la suite $\suiteu$ par $u_0\in\R$ et $\forall n \geq 1, u_{n+1}=\ddp\frac{3}{4}u_n^2-2u_n+3$.
%$\left\lbrace\begin{array}{l}
%u_0\in\R\vsec\\
%u_{n+1}=\ddp\frac{3}{4}u_n^2-2u_n+3
%\end{array}\right.$
\begin{enumerate}
\item \'Etudier la fonction $f$ associ\'ee.
\item \'Etudier le signe de $g: x\mapsto f(x)-x$.
\item Calculer les limites \'eventuelles de la suite $\suiteu$.
\item On suppose que $u_0>2$.
\begin{enumerate}
\item Montrer que la suite est bien d\'efinie et que pour tout $n\in\N$: $u_n>2$.
\item \'Etudier la monotonie de la suite $\suiteu$.
\item \'Etudier le comportement \`{a} l'infini de la suite $\suiteu$.
\end{enumerate}
\item On suppose que $u_0\in\left\rbrack \ddp\frac{2}{3},2\right\lbrack $.
\begin{enumerate}
\item Montrer que la suite est bien d\'efinie et que pour tout $n\in\N$: $u_n\in\left\rbrack \ddp\frac{2}{3},2\right\lbrack$.
\item \'Etudier la monotonie de la suite $\suiteu$.
\item \'Etudier le comportement \`{a} l'infini de la suite $\suiteu$.
\end{enumerate} 
\end{enumerate}
\end{exercice}
%--------------------------------------------------------------------------------------
\begin{correction} \;
\begin{enumerate}
\item \textbf{\'Etudier les variations de la fonction $f: x\mapsto \ddp\frac{3}{4}x^2-2x+3$ associ\'ee:}
\begin{itemize}
\item[$\bullet$] La fonction $f$ est bien d\'efinie sur $\R$ comme fonction polynomiale.
\item[$\bullet$] La fonction $f$ est d\'erivable sur $\R$ comme fonction polynomiale et pour tout $x\in\R$: 
$f^{\prime}(x)=\ddp\frac{3}{2}x-2$.
\item[$\bullet$] On obtient ainsi les variations suivantes:
\begin{center}
\begin{tikzpicture}
 \tkzTabInit{ $x$          /1,%
       $f'(x)$      /1,%
       $f$       /2}%
     { $-\infty$, $\ddp\frac{4}{3}$ ,$+\infty$ }%
  \tkzTabLine {,-,0,+,}%
  \tkzTabVar{
       +/ $+\infty$        /,
        -/$\ddp\frac{5}{3}$           /,%
       +/$+\infty$           /,
                      }
 \tkzTabVal[draw]{2}{3}{0.3}{$2$}{$2$}
\end{tikzpicture}
\end{center}
\item[$\bullet$] Les limites en $\pm\infty$ s'obtiennent avec le th\'eor\`{e}me du mon\^{o}me de plus haut degr\'e.
\end{itemize}
\item \textbf{\'Etudier le signe de la fonction $g: x\mapsto f(x)-x=\ddp\frac{3}{4}x^2-3x+3$:}\\
\noindent Le discriminant vaut $\Delta=0$ et l'unique racine est 2. Ainsi

 \fbox{la fonction $g$ est positive sur $\R$ et ne s'annule qu'en 2.}
\item \textbf{Calculer les limites \'eventuelles de la suite $\suiteu$:}\\
\noindent On suppose que la suite $\suiteu$ converge vers un r\'eel $l\in\mathcal{D}_f=\R$.
\begin{itemize}
\item[$\star$] On a donc:
\begin{itemize}
\item[$\circ$] La suite converge vers $l$.
\item[$\circ$] La fonction $f$ est continue sur $\R$ comme fonction polynomiale donc elle est en particulier continue en $l$.
\end{itemize}
Donc d'apr\`{e}s le th\'eor\`{e}me sur les suite et fonction, on obtient que: $\lim\limits_{n\to +\infty} f(u_n)=f(l)$.
\item[$\star$] De plus on a: $\lim\limits_{n\to +\infty} u_{n+1}=l$.
\item[$\star$] On peut donc passer \`{a} la limite dans l'\'egalit\'e: $u_{n+1}=f(u_n)$ et on obtient que: $l=f(l)$/
\item[$\star$] On a donc: $l=f(l)\Leftrightarrow g(l)=0\Leftrightarrow l=2$. \\
\noindent \fbox{La seule limite \'eventuelle est 2.}
\end{itemize}
\item \textbf{On suppose que $u_0>2$:}
\begin{enumerate}
\item \textbf{Montrer que la suite est bien d\'efinie et que pour tout $n\in\N$: $u_n>2$:}\\
\noindent On peut commencer par montrer que l'intervalle $\rbrack 2,+\infty\lbrack$ est stable par $f$. 
%On a:
%\begin{itemize}
%\item[$\circ$] La fonction $f$ est continue sur $\rbrack 2,+\infty\lbrack$.
%\item[$\circ$] La fonction $f$ est strictement croissante sur $\rbrack 2,+\infty\lbrack$.
%\item[$\circ$] $f(2)=2$ et $\lim\limits_{x\to +\infty} f(x)=+\infty$.
%\end{itemize}
%Ainsi d'apr\`{e}s le th\'eor\`{e}me de la bijection, on a en particulier que $f(\rbrack 2,+\infty\lbrack)=\rbrack 2,+\infty\lbrack$. Et comme $\rbrack 2,+\infty\lbrack \subset \rbrack 2,+\infty\lbrack$, \fbox{l'intervalle $\rbrack 2,+\infty\lbrack$ est stable par $f$.} 
On a $f$ strictement croissante sur $[2,+\infty[$, et $f(2) =2$. Donc pour tout $x \in [2,+\infty[, f(x) > 2$ et l'intervalle  $\rbrack 2,+\infty\lbrack$ est stable par $f$.\\
On montre par r\'ecurrence sur $n\in\N$ la propri\'et\'e : $\mathcal{P}(n):\ u_n\ \hbox{existe et}\ u_n>2.$
\begin{itemize}
\item[$\star$] Initialisation: pour $n=0$: par d\'efinition de la suite, $u_0$ existe et $u_0>2$. Donc $\mathcal{P}(0)$ est vraie.
\item[$\star$] H\'er\'edit\'e: soit $n\in\N$ fix\'e, on suppose que la propri\'et\'e vraie \`{a} l'ordre $n$, montrons que $\mathcal{P}(n+1)$ est vraie. Par hypoth\`{e}se de r\'ecurrence, on sait que $u_n$ existe et que $u_n>2$. Donc $f(u_n)$ existe c'est-\`{a}-dire $u_{n+1}$ existe. De plus, l'intervalle $\rbrack 2,+\infty\lbrack$ est stable par $f$. Donc $f(u_n)>2$ c'est-\`{a}-dire $u_{n+1}>2$.
Donc $\mathcal{P}(n+1)$ est vraie.
\item[$\star$] Conclusion: il r\'esulte du principe de r\'ecurrence que

 \fbox{la suite $\suiteu$ est bien d\'efinie et que pour tout $n\in\N,\ u_n>2.$}
\end{itemize}
\item \textbf{\'Etudier la monotonie de la suite $\suiteu$:}\\
\noindent Soit $n\in\N$, on a: $u_{n+1}-u_n=f(u_n)-u_n=g(u_n)$. Ainsi comme le signe de $g$ est positif sur $\R$, on obtient que pour tout $n\in\N$: $u_{n+1}-u_n\geq 0$. Ainsi \fbox{la suite $\suiteu$ est croissante.}
\item \textbf{\'Etudier le comportement \`{a} l'infini de la suite $\suiteu$:}
\begin{itemize}
\item[$\star$] La suite $\suiteu$ est croissante donc d'apr\`{e}s le th\'eor\`{e}me sur les suites monotones, elle converge ou elle diverge vers $+\infty$.
\item[$\star$] On suppose par l'absurde que la suite $\suiteu$ converge vers un r\'eel $l$. On a alors:
\begin{itemize}
\item[$\circ$] La suite $\suiteu$ converge vers $l$.
\item[$\circ$] Comme la suite $\suiteu$ est croissante, on a pour tout $n\in\N$: $u_n\geq u_0$.
\end{itemize}
D'apr\`{e}s le th\'eor\`{e}me de passage \`{a} la limite, on obtient donc que: $l\geq u_0$. Or par hypoth\`{e}se, on sait que $u_0>2$. Ainsi on obtient que: $l>2$. Absurde car la seule limite \'eventuelle de la suite $\suiteu$ est 2. Ainsi \fbox{la suite $\suiteu$ diverge vers $+\infty$.}
\end{itemize}
\end{enumerate}
\item \textbf{On suppose que $u_0\in\left\rbrack \ddp\frac{2}{3},2\right\lbrack $:}
\begin{enumerate}
\item \textbf{Montrer que la suite est bien d\'efinie et que pour tout $n\in\N$: $u_n\left\rbrack \ddp\frac{2}{3},2\right\lbrack$:}\\
\noindent On peut commencer par montrer que l'intervalle $\left\rbrack \ddp\frac{2}{3},2\right\lbrack$ est stable par $f$. Attention, ici $f$ n'est pas monotone sur $\left\rbrack \ddp\frac{2}{3},2\right\lbrack$, il faut donc traiter les deux intervalles $\left\rbrack \ddp\frac{2}{3},\ddp\frac{4}{3}\right\rbrack$ et $\left\rbrack \ddp\frac{4}{3}, 2\right\rbrack$ s\'eparemment.\\
Sur $\left\rbrack \ddp\frac{2}{3},\ddp\frac{4}{3}\right\rbrack$, $f$ est strictement d\'ecroissante et $f\left(\ddp\frac{2}{3}\right) = 2$, $f\left(\ddp\frac{4}{3}\right)=\ddp\frac{5}{3}$. Donc pour tout $x \in \left\rbrack \ddp\frac{2}{3},\ddp\frac{4}{3}\right\rbrack$, $f(x) \in \left\rbrack \ddp\frac{5}{3},2\right\rbrack$, donc $f(x) \in \left\rbrack \ddp\frac{2}{3},2\right\lbrack$.\\
Sur $\left\rbrack\ddp\frac{4}{3},2\right\rbrack$, $f$ est strictement croissante et $f\left(2\right) = 2$, $f\left(\ddp\frac{4}{3}\right)=\ddp\frac{5}{3}$. Donc pour tout $x \in \left\rbrack \ddp\frac{4}{3},2\right\rbrack$, $f(x) \in \left\rbrack \ddp\frac{5}{3},2\right\rbrack$, donc $f(x) \in \left\rbrack \ddp\frac{2}{3},2\right\lbrack$.\\
En en d\'eduit que pour tout $x \in \left\rbrack \ddp\frac{2}{3},2\right\lbrack$, on a bien $f(x) \in \left\rbrack \ddp\frac{2}{3},2\right\lbrack$ : l'intervalle $\left\rbrack \ddp\frac{2}{3},2\right\lbrack$ est stable par $f$.\\
%On a:
%\begin{itemize}
%\item[$\circ$] La fonction $f$ est continue sur $\left\rbrack \ddp\frac{2}{3},\ddp\frac{4}{3}\right\rbrack$.
%\item[$\circ$] La fonction $f$ est strictement d\'ecroissante sur $\left\rbrack \ddp\frac{2}{3},\ddp\frac{4}{3}\right\rbrack$.
%\item[$\circ$] $f\left( \ddp\frac{2}{3}\right)=2$ et $f\left( \ddp\frac{4}{3}\right)=\ddp\frac{5}{3}$.
%\end{itemize}
%Ainsi d'apr\`{e}s le th\'eor\`{e}me de la bijection, on a en particulier que $f(\left\rbrack \ddp\frac{2}{3},\ddp\frac{4}{3}\right\rbrack)=\left\lbrack \ddp\frac{5}{3},2\right\lbrack$. \\
%\noindent On a aussi:
%\begin{itemize}
%\item[$\circ$] La fonction $f$ est continue sur $\left\lbrack \ddp\frac{4}{3},2\right\lbrack$.
%\item[$\circ$] La fonction $f$ est strictement croissante sur $\left\lbrack \ddp\frac{4}{3},2\right\lbrack$.
%\item[$\circ$] $f\left(2\right)=2$ et $f\left( \ddp\frac{4}{3}\right)=\ddp\frac{5}{3}$.
%\end{itemize}
%Ainsi d'apr\`{e}s le th\'eor\`{e}me de la bijection, on a en particulier que $f(\left\lbrack \ddp\frac{4}{3},2\right\lbrack)=\left\lbrack \ddp\frac{5}{3},2\right\lbrack$. \\
%\noindent Au final on obtient donc que: 
%$$f(\left\rbrack \ddp\frac{2}{3},2\right\lbrack)=f(\left\rbrack \ddp\frac{2}{3},\ddp\frac{4}{3}\right\rbrack)\cup f(\left\lbrack \ddp\frac{4}{3},2\right\lbrack)=\left\lbrack \ddp\frac{5}{3},2\right\lbrack\cup\left\lbrack \ddp\frac{5}{3},2\right\lbrack=\left\lbrack \ddp\frac{5}{3},2\right\lbrack.$$
%Et comme $\left\lbrack \ddp\frac{5}{3},2\right\lbrack \subset \left\rbrack \ddp\frac{2}{3},2\right\lbrack$, \fbox{l'intervalle $\left\rbrack \ddp\frac{2}{3},2\right\lbrack$ est stable par $f$.} 
On montre par r\'ecurrence sur $n\in\N$ la propri\'et\'e $\mathcal{P}(n):\ u_n\ \hbox{existe et}\ u_n\in\left\rbrack \ddp\frac{2}{3},2\right\lbrack.$
\begin{itemize}
\item[$\star$] Initialisation: pour $n=0$: par d\'efinition de la suite, $u_0$ existe et $u_0\in\left\rbrack \ddp\frac{2}{3},2\right\lbrack$. Donc $\mathcal{P}(0)$ est vraie.
\item[$\star$] H\'er\'edit\'e: soit $n\in\N$ fix\'e, on suppose que la propri\'et\'e vraie \`{a} l'ordre $n$, montrons que $\mathcal{P}(n+1)$ est vraie.
Par hypoth\`{e}se de r\'ecurrence, on sait que $u_n$ existe et que $u_n\in\left\rbrack \ddp\frac{2}{3},2\right\lbrack$. Donc $f(u_n)$ existe c'est-\`{a}-dire $u_{n+1}$ existe.\\
De plus, $u_n\in\left\rbrack \ddp\frac{2}{3},2\right\lbrack$. Or l'intervalle $\left\rbrack \ddp\frac{2}{3},2\right\lbrack$ est stable par $f$. Donc $f(u_n)\in\left\rbrack \ddp\frac{2}{3},2\right\lbrack$ c'est-\`{a}-dire $u_{n+1}\in\left\rbrack \ddp\frac{2}{3},2\right\lbrack$. Donc $\mathcal{P}(n+1)$ est vraie.
\item[$\star$] Conclusion: il r\'esulte du principe de r\'ecurrence que

 \fbox{la suite $\suiteu$ est bien d\'efinie et que pour tout $n\in\N,\ u_n\in\left\rbrack \ddp\frac{2}{3},2\right\lbrack.$}
\end{itemize}
\item \textbf{\'Etudier la monotonie de la suite $\suiteu$:}\\
\noindent Soit $n\in\N$, on a: $u_{n+1}-u_n=f(u_n)-u_n=g(u_n)$. Ainsi comme le signe de $g$ est positif sur $\R$, on obtient que pour tout $n\in\N$: $u_{n+1}-u_n\geq 0$. Ainsi \fbox{la suite $\suiteu$ est croissante.}
\item \textbf{\'Etudier le comportement \`{a} l'infini de la suite $\suiteu$:}
\begin{itemize}
\item[$\star$] La suite $\suiteu$ est croissante et major\'ee par 2 donc d'apr\`{e}s le th\'eor\`{e}me sur les suites monotones, elle converge.
\item[$\star$] Comme la seule limite \'eventuelle est 2, \fbox{la suite $\suiteu$ converge vers 2.}
\end{itemize}
\end{enumerate} 
\end{enumerate}
\end{correction}








%--------------------------------------------------------------------------------------
\begin{exercice} \;
On d\'efinit la suite $\suiteu$ par 
$\left\lbrace\begin{array}{l}
u_0\in\R\vsec\\
u_{n+1}=\ddp\frac{1}{3}u_n^2-u_n+3
\end{array}\right.$
\begin{enumerate}
\item \'Etudier la fonction $f$ associ\'ee.
\item \'Etudier le signe de $g: x\mapsto f(x)-x$.
\item Calculer les limites \'eventuelles de la suite $\suiteu$.
\item Que peut-on dire de la suite $\suiteu$ lorsque $u_0=3$ ou $u_0=0$ ?
\item On suppose que $u_0\in\rbrack 0,3\lbrack$.
\begin{enumerate}
\item Montrer que la suite est bien d\'efinie et que pour tout $n\in\N$: $u_n\in\rbrack 0,3\lbrack$.
\item \'Etudier la monotonie de la suite $\suiteu$.
\item \'Etudier le comportement \`{a} l'infini de la suite $\suiteu$.
\end{enumerate}
\item On suppose que $u_0>3$.
\begin{enumerate}
\item Montrer que la suite est bien d\'efinie et que pour tout $n\in\N$: $u_n>3$.
\item \'Etudier la monotonie de la suite $\suiteu$.
\item \'Etudier le comportement \`{a} l'infini de la suite $\suiteu$.
\end{enumerate} 
\item On suppose que $u_0<0$.
\begin{enumerate}
\item Montrer que $u_1>3$.
\item En d\'eduire le comportement \`{a} l'infini de la suite $\suiteu$.
\end{enumerate} 
\end{enumerate}
\end{exercice}
\begin{correction} \;
\begin{enumerate}
\item \textbf{\'Etudier les variations de la fonction $f: x\mapsto \ddp\frac{1}{3}x^2-x+3$ associ\'ee:}
\begin{itemize}
\item[$\bullet$] La fonction $f$ est bien d\'efinie sur $\R$ comme fonction polynomiale.
\item[$\bullet$] La fonction $f$ est d\'erivable sur $\R$ comme fonction polynomiale et pour tout $x\in\R$: 
$f^{\prime}(x)=\ddp\frac{2}{3}x-1$.
\item[$\bullet$] On obtient ainsi les variations suivantes:
\begin{center}
\begin{tikzpicture}
 \tkzTabInit{ $x$          /1,%
       $f'(x)$      /1,%
       $f$       /2}%
     { $-\infty$, $\ddp\frac{3}{2}$ ,$+\infty$ }%
  \tkzTabLine {,-,0,+,}%
  \tkzTabVar{
       +/ $+\infty$        /,
        -/$\ddp\frac{9}{4}$           /,%
       +/$+\infty$           /,
                      }
 \tkzTabVal[draw]{2}{3}{0.3}{$3$}{$3$}
 \tkzTabVal[draw]{1}{2}{0.6}{$0$}{$3$}
\end{tikzpicture}
\end{center}
\item[$\bullet$] Les limites en $\pm\infty$ s'obtiennent avec le th\'eor\`{e}me du mon\^{o}me de plus haut degr\'e.
\end{itemize}
\item \textbf{\'Etudier le signe de la fonction $g: x\mapsto f(x)-x=\ddp\frac{1}{3}x^2-2x+3$:}\\
\noindent Le discriminant vaut $\Delta=0$ et l'unique racine est 3. Ainsi

 \fbox{la fonction $g$ est positive sur $\R$ et ne s'annule qu'en 3.}
\item \textbf{Calculer les limites \'eventuelles de la suite $\suiteu$:}\\
\noindent On suppose que la suite $\suiteu$ converge vers un r\'eel $l\in\mathcal{D}_f=\R$.
\begin{itemize}
\item[$\star$] On a donc:
\begin{itemize}
\item[$\circ$] La suite converge vers $l$.
\item[$\circ$] La fonction $f$ est continue sur $\R$ comme fonction polynomiale donc elle est en particulier continue en $l$.
\end{itemize}
Donc d'apr\`{e}s le th\'eor\`{e}me sur les suite et fonction, on obtient que: $\lim\limits_{n\to +\infty} f(u_n)=f(l)$.
\item[$\star$] De plus on a: $\lim\limits_{n\to +\infty} u_{n+1}=l$.
\item[$\star$] On peut donc passer \`{a} la limite dans l'\'egalit\'e: $u_{n+1}=f(u_n)$ et on obtient que: $l=f(l)$/
\item[$\star$] On a donc: $l=f(l)\Leftrightarrow g(l)=0\Leftrightarrow l=3$. \\
\noindent \fbox{La seule limite \'eventuelle est 3.}
\end{itemize}
\item \textbf{Que peut-on dire de la suite $\suiteu$ lorsque $u_0=3$ ou $u_0=0$ ?:}
\begin{itemize}
\item[$\bullet$] Cas 1: si $u_0=3$:\\
\noindent Comme 3 est le point fixe de $f$, on a: $u_1=f(u_0)=f(3)=3$ puis $u_2=f(u_1)=f(3)=3$... On montre alors par r\'ecurrence que

 \fbox{la suite $\suiteu$ est constante \'egale \`{a} 3 et donc qu'elle converge vers 3.}
\item[$\bullet$] Cas 2: si $u_0=0$:\\
\noindent On a par d\'efinition de la suite: $u_1=f(u_0)=f(0)=3$. Mais comme 3 est le point fixe de la fonction $f$, on a alors $u_2=f(u_1)=f(3)=3$ puis $u_3=f(u_2)=f(3)=3$... On montre alors par r\'ecurrence sur $n\geq 1$ que 

\fbox{la suite $\suiteu$ est stationnaire \'egale \`{a} 3 et donc qu'elle converge vers 3.} 
\end{itemize}
\item \textbf{On suppose que $u_0\in\rbrack 0,3\lbrack$.}
\begin{enumerate}
\item \textbf{Montrer que la suite est bien d\'efinie et que pour tout $n\in\N$: $u_n\in\rbrack 0,3\lbrack$:}\\
\noindent 
On peut commencer par montrer que l'intervalle $\rbrack 0,3\lbrack$ est stable par $f$. On traite ici les intervalles $\left\rbrack 0,\ddp\frac{3}{2}\right\rbrack$ et  $\left\lbrack \ddp\frac{3}{2},3\right\lbrack$ s\'eparemment.\\
La fonction $f$ est strictement d\'ecroissante sur $\left\rbrack 0,\ddp\frac{3}{2}\right\rbrack$, et $f( 0)=3$, $f\left( \ddp\frac{3}{2}\right)=\ddp\frac{9}{4}$. Donc pour tout $x \in \left\rbrack 0,\ddp\frac{3}{2}\right\rbrack, f(x) \in \left\rbrack \ddp \frac{9}{4},\ddp\frac{3}{2}\right\rbrack$, donc $f(x) \in \rbrack 0,3\lbrack$.\\
La fonction $f$ est strictement croissante sur $\left\lbrack \ddp\frac{3}{2},3\right\lbrack$, et $f\left(3\right)=3$, $f\left( \ddp\frac{3}{2}\right)=\ddp\frac{9}{4}$. Donc pour tout $x \in \left\lbrack \ddp\frac{3}{2},3\right\lbrack, f(x) \in \left\rbrack \ddp \frac{9}{4},3\right\rbrack$, donc $f(x) \in \rbrack 0,3\lbrack$.\\
Ainsi, pour tout $x \in \rbrack 0,3\lbrack, f(x) \in \rbrack 0,3\lbrack$, et donc l'intervalle $\rbrack 0,3\lbrack$ est stable par $f$.\\
%\begin{itemize}
%\item[$\circ$] La fonction $f$ est continue sur $\left\rbrack 0,\ddp\frac{3}{2}\right\rbrack$.
%\item[$\circ$] La fonction $f$ est strictement d\'ecroissante sur $\left\rbrack 0,\ddp\frac{3}{2}\right\rbrack$.
%\item[$\circ$] $f( 0)=3$ et $f\left( \ddp\frac{3}{2}\right)=\ddp\frac{9}{4}$.
%\end{itemize}
%Ainsi d'apr\`{e}s le th\'eor\`{e}me de la bijection, on a en particulier que $f(\left\rbrack 0,\ddp\frac{3}{2}\right\rbrack)=\left\lbrack \ddp\frac{9}{4},3\right\lbrack$. \\
%\noindent On a aussi:
%\begin{itemize}
%\item[$\circ$] La fonction $f$ est continue sur $\left\lbrack \ddp\frac{3}{2},3\right\lbrack$.
%\item[$\circ$] La fonction $f$ est strictement croissante sur $\left\lbrack \ddp\frac{3}{2},3\right\lbrack$.
%\item[$\circ$] $f\left(3\right)=3$ et $f\left( \ddp\frac{3}{2}\right)=\ddp\frac{9}{4}$.
%\end{itemize}
%Ainsi d'apr\`{e}s le th\'eor\`{e}me de la bijection, on a en particulier que $f(\left\lbrack \ddp\frac{3}{2},3\right\lbrack)=\left\lbrack \ddp\frac{9}{4},3\right\lbrack$. \\
%\noindent Au final on obtient donc que: 
%$$f(\rbrack 0,3\lbrack)=f(\left\rbrack 0,\ddp\frac{3}{2}\right\rbrack)\cup f(\left\lbrack \ddp\frac{3}{2},3\right\lbrack)=\left\lbrack \ddp\frac{9}{4},3\right\lbrack\cup\left\lbrack \ddp\frac{9}{4},3\right\lbrack=\left\lbrack \ddp\frac{9}{4},3\right\lbrack.$$
%Et comme $\left\lbrack \ddp\frac{9}{4},3\right\lbrack \subset \rbrack 0,3\lbrack$, \fbox{l'intervalle $\rbrack 0,3\lbrack$ est stable par $f$.} 
On montre par r\'ecurrence sur $n\in\N$ la propri\'et\'e : $\mathcal{P}(n):\ u_n\ \hbox{existe et}\ u_n\in\rbrack 0,3\lbrack.$
\begin{itemize}
\item[$\star$] Initialisation: pour $n=0$:\\
\noindent Par d\'efinition de la suite, on a bien que $u_0$ existe et $u_0\in\rbrack 0,3\lbrack$. Donc $\mathcal{P}(0)$ est vraie.
\item[$\star$] H\'er\'edit\'e: soit $n\in\N$ fix\'e, on suppose que la propri\'et\'e vraie \`{a} l'ordre $n$, montrons que $\mathcal{P}(n+1)$ est vraie.
\begin{itemize}
\item[$\circ$] Par hypoth\`{e}se de r\'ecurrence, on sait que $u_n$ existe et que $u_n\in\rbrack 0,3\lbrack$. En particulier $u_n$ existe et $u_n\in\mathcal{D}_f$. Donc $f(u_n)$ existe c'est-\`{a}-dire $u_{n+1}$ existe.
\item[$\circ$] Par hypoth\`{e}se de r\'ecurrence, on sait que $u_n$ existe et que $u_n\in\rbrack 0,3\lbrack$. Or l'intervalle $\rbrack 0,3\lbrack$ est stable par $f$. Donc $f(u_n)\in\rbrack 0,3\lbrack$ c'est-\`{a}-dire $u_{n+1}\in\rbrack 0,3\lbrack$.
\end{itemize}
Donc $\mathcal{P}(n+1)$ est vraie.
\item[$\star$] Conclusion: il r\'esulte du principe de r\'ecurrence que 


\fbox{la suite $\suiteu$ est bien d\'efinie et que pour tout $n\in\N,\ u_n\in\rbrack 0,3\lbrack.$}
\end{itemize}
\item \textbf{\'Etudier la monotonie de la suite $\suiteu$:}\\
\noindent Soit $n\in\N$, on a: $u_{n+1}-u_n=f(u_n)-u_n=g(u_n)$. Ainsi comme le signe de $g$ est positif sur $\R$, on obtient que pour tout $n\in\N$: $u_{n+1}-u_n\geq 0$. Ainsi \fbox{la suite $\suiteu$ est croissante.}
\item \textbf{\'Etudier le comportement \`{a} l'infini de la suite $\suiteu$:}
\begin{itemize}
\item[$\star$] La suite $\suiteu$ est croissante et major\'ee par 3 donc d'apr\`{e}s le th\'eor\`{e}me sur les suites monotones, elle converge.
\item[$\star$] Comme la seule limite \'eventuelle est 3, \fbox{la suite $\suiteu$ converge vers 3.}
\end{itemize}
\end{enumerate}
\item \textbf{On suppose que $u_0>3$.}
\begin{enumerate}
\item \textbf{Montrer que la suite est bien d\'efinie et que pour tout $n\in\N$: $u_n>3$:}\\
\noindent On peut commencer par montrer que l'intervalle $\rbrack 3,+\infty\lbrack$ est stable par $f$. On a $f$ strictement croissante sur $\rbrack 3,+\infty\lbrack$, et $f(3) = 3$, donc pour tout $x \in \rbrack 3,+\infty\lbrack$, $f(x) >3$ et l'intervalle $\rbrack 3,+\infty\lbrack$ est stable par $f$.
%\begin{itemize}
%\item[$\circ$] La fonction $f$ est continue sur $\rbrack 3,+\infty\lbrack$.
%\item[$\circ$] La fonction $f$ est strictement croissante sur $\rbrack 3,+\infty\lbrack$.
%\item[$\circ$] $f(3)=3$ et $\lim\limits_{x\to +\infty} f(x)=+\infty$.
%\end{itemize}
%Ainsi d'apr\`{e}s le th\'eor\`{e}me de la bijection, on a en particulier que $f(\rbrack 3,+\infty\lbrack)=\rbrack 3,+\infty\lbrack$. Et comme $\rbrack 3,+\infty\lbrack \subset \rbrack 3,+\infty\lbrack$, \fbox{l'intervalle $\rbrack 3,+\infty\lbrack$ est stable par $f$.} 
\begin{itemize}
\item[$\star$] On montre par r\'ecurrence sur $n\in\N$ la propri\'et\'e
$$\mathcal{P}(n):\ u_n\ \hbox{existe et}\ u_n>3.$$
\item[$\star$] Initialisation: pour $n=0$:\\
\noindent Par d\'efinition de la suite, on a bien que $u_0$ existe et $u_0>3$. Donc $\mathcal{P}(0)$ est vraie.
\item[$\star$] H\'er\'edit\'e: soit $n\in\N$ fix\'e, on suppose que la propri\'et\'e vraie \`{a} l'ordre $n$, montrons que $\mathcal{P}(n+1)$ est vraie.
\begin{itemize}
\item[$\circ$] Par hypoth\`{e}se de r\'ecurrence, on sait que $u_n$ existe et que $u_n>3$. En particulier $u_n$ existe et $u_n\in\mathcal{D}_f$. Donc $f(u_n)$ existe c'est-\`{a}-dire $u_{n+1}$ existe.
\item[$\circ$] Par hypoth\`{e}se de r\'ecurrence, on sait que $u_n$ existe et que $u_n>3$. Or l'intervalle $\rbrack 3,+\infty\lbrack$ est stable par $f$. Donc $f(u_n)>3$ c'est-\`{a}-dire $u_{n+1}>3$.
\end{itemize}
Donc $\mathcal{P}(n+1)$ est vraie.
\item[$\star$] Conclusion: il r\'esulte du principe de r\'ecurrence que

 \fbox{la suite $\suiteu$ est bien d\'efinie et que pour tout $n\in\N,\ u_n>3.$}
\end{itemize}
\item \textbf{\'Etudier la monotonie de la suite $\suiteu$:}\\
\noindent Soit $n\in\N$, on a: $u_{n+1}-u_n=f(u_n)-u_n=g(u_n)$. Ainsi comme le signe de $g$ est positif sur $\R$, on obtient que pour tout $n\in\N$: $u_{n+1}-u_n\geq 0$. Ainsi \fbox{la suite $\suiteu$ est croissante.}
\item \textbf{\'Etudier le comportement \`{a} l'infini de la suite $\suiteu$:}
\begin{itemize}
\item[$\star$] La suite $\suiteu$ est croissante donc d'apr\`{e}s le th\'eor\`{e}me sur les suites monotones, elle converge ou elle diverge vers $+\infty$.
\item[$\star$] On suppose par l'absurde que la suite $\suiteu$ converge vers un r\'eel $l$. On a alors:
\begin{itemize}
\item[$\circ$] La suite $\suiteu$ converge vers $l$.
\item[$\circ$] Comme la suite $\suiteu$ est croissante, on a pour tout $n\in\N$: $u_n\geq u_0$.
\end{itemize}
D'apr\`{e}s le th\'eor\`{e}me de passage \`{a} la limite, on obtient donc que: $l\geq u_0$. Or par hypoth\`{e}se, on sait que $u_0>3$. Ainsi on obtient que: $l>3$. Absurde car la seule limite \'eventuelle de la suite $\suiteu$ est 3. Ainsi \fbox{la suite $\suiteu$ diverge vers $+\infty$.}
\end{itemize}
\end{enumerate} 
\item \textbf{On suppose que $u_0<0$.}
\begin{enumerate}
\item \textbf{Montrer que $u_1>3$:}\\
\noindent On a:
\begin{itemize}
\item[$\star$] La fonction $f$ est continue sur $\rbrack -\infty,0\lbrack$ comme fonction polynomiale.
\item[$\star$] La fonction $f$ est strictement d\'ecroissante sur $\rbrack -\infty,0\lbrack$.
\item[$\star$] $\lim\limits_{x\to -\infty}f(x)=+\infty$ et $f(0)=3$
\end{itemize}
Ainsi d'apr\`{e}s le th\'eor\`{e}me de la bijection, on a en particulier que $f(\rbrack -\infty,0\lbrack)=\left\rbrack 3,+\infty\right\lbrack$. Or on a suppos\'e que $u_0\in\rbrack -\infty,0\lbrack$ donc $f(u_0)\in \left\rbrack 3,+\infty\right\lbrack$, \`{a} savoir $u_1\in \left\rbrack 3,+\infty\right\lbrack$. Donc on a bien \fbox{$u_1>3.$}
\item \textbf{En d\'eduire le comportement \`{a} l'infini de la suite $\suiteu$:}\\
\noindent La suite $(u_n)_{n\geq 1}$ a donc un terme initial $u_1>3$. Ainsi la suite $(u_n)_{n\geq 1}$ se comporte comme la suite de la question 5 et en particulier elle diverge vers $+\infty$. Mais le comportement \`{a} l'infini d'une suite ne d\'epend pas de ses premiers termes donc \fbox{la suite $\suiteu$ diverge vers $+\infty$.}
\end{enumerate} 
\end{enumerate}
\end{correction}



%--------------------------------------------------------------------------------------
%--------------------------------------------------------------------------------------


\end{document}