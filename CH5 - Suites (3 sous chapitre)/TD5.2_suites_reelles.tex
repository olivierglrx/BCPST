\documentclass[a4paper, 11pt]{article}
\input{macro/package.tex}
\input{macro/environement}
% Header et footer

\pagestyle{fancy}
\fancyhead{}
\fancyfoot{}
\renewcommand{\headwidth}{\textwidth}
\renewcommand{\footrulewidth}{0.4pt}
\renewcommand{\headrulewidth}{0pt}
\renewcommand{\footruleskip}{5px}

\fancyfoot[R]{Olivier Glorieux}
%\fancyfoot[R]{Jules Glorieux}

\fancyfoot[C]{ Page \thepage }
\fancyfoot[L]{1BIOA - Lycée Chaptal}
%\fancyfoot[L]{MP*-Lycée Chaptal}
%\fancyfoot[L]{Famille Lapin}

\input{macro/newcommand.tex}
\geometry{hmargin=2.0cm, vmargin=1.5cm}

\author{Olivier Glorieux}


\newcommand{\type}{TD }
\excludecomment{correction}
%\renewcommand{\type}{Correction TD }

\begin{document}
\title{\type  5.2 - Suites réelles}

%----------------------s---------------------------
\section*{Entraînements}
\begin{exercice} \; \'Etudier la monotonie des suites d\'efinies par
\begin{enumerate}
\begin{minipage}[t]{0.4\textwidth}
\item
$\forall n\in\N,\ u_n=\left(\ddp \sum\limits_{k=0}^n \ddp\frac{1}{2^k}\right)-n$ 
\item 
$\forall n\in\N,\ u_n=\ddp\frac{n!}{2^{n+1}}$ 
\item 
$\forall n\in\N^{\star},\ u_n=\ddp\frac{\ln{(n)}}{n}$ 
\end{minipage}
\begin{minipage}[t]{0.4\textwidth}
\item 
$\forall n\in\N,\ u_n=\ddp \sum\limits_{k=0}^{2n}\ddp\frac{(-1)^k}{k+1}$ 
\item 
$\forall n\in\N,\ u_n=n+2(-1)^n$ 
\item $\forall n\in\N,\ u_n=\ddp \sum\limits_{k=2}^{n}\ddp\frac{1}{k\ln{(k)}}$ 
%\item $\forall n\in\N,\ u_n=\ddp\frac{5^n}{n!}$ 
\end{minipage}
\end{enumerate}
\end{exercice}


%----------------------------------------------------
\begin{correction} \;
\'Etude de la monotonie des suites suivantes.
\begin{enumerate}
\item La suite $\suiteu$ est d\'efinie par une somme, on \'etudie donc le signe de $u_{n+1}-u_n$.
$$u_{n+1}-u_n=\sum\limits_{k=0}^{n+1}\ddp\frac{1}{2^k}-(n+1)-\sum\limits_{k=0}^{n}\ddp\frac{1}{2^k}+n=\ddp\frac{1}{2^{n+1}}-1.$$
Or, pour tout $n\in\N$, on a: $\ddp\frac{1}{2^{n+1}}<1$. Ainsi la suite $\suiteu$ est d\'ecroissante.
\item La suite $\suiteu$ est plut\^ot de type produit. Comme tous ses termes sont strictement positifs, on compare $\ddp\frac{u_{n+1}}{u_n}$ \`a 1. On obtient
$$\ddp\frac{u_{n+1}}{u_n}=\ddp\frac{(n+1)!}{2^{n+2}} \times \ddp\frac{2^{n+1}}{n!}=\ddp\frac{n+1}{2}.$$
Un calcul rapide donne
$$\ddp\frac{n+1}{2}<1\Leftrightarrow n+1<2\Leftrightarrow n<1.$$
Ainsi, la suite $(u_n)_{n\in\N^{\star}}$ est croissante ou encorrectione la suite $\suiteu$ est croissante \`{a} partir du rang 1.
\item 
La suite $(u_n)_{n\in\N^{\star}}$ est une suite d\'efinie explicitement et $u_n=f(n)$ avec
$$f:\ x\mapsto f(x)=\ddp\frac{\ln{x}}{x}.$$
L'\'etude de la monotonie de la fonction $f$ sur $\lbrack 1,+\infty\lbrack$ permet d'en d\'eduire directement la monotonie de la suite.\\
\noindent La fonction $f$ est d\'erivable sur $\R^{+\star}$ comme quotient dont le d\'enominateur ne s'annule pas de fonctions d\'erivables. On obtient
$$\forall x\in\R^{+\star},\ f^{\prime}(x)=\ddp\frac{1-\ln{x}}{x^2}.$$
\'Etudions le signe de $1-\ln{x}$ ($x^2\geq 0$ donc le signe de la d\'eriv\'ee est bien le signe de $1-\ln{x}$):\\
\noindent $1-\ln{x}>0  \Leftrightarrow  \ln{x}<1
\Leftrightarrow x<e $ car la fonction exponentielle est strictement croissante.
Ainsi, la fonction $f$ est strictement d\'ecroissante sur $\lbrack e,+\infty\lbrack$. Ainsi, \`a partir du rang 3, la suite $(u_n)_{n\geq 3}$ est d\'ecroissante.
\item La suite $\suiteu$ est d\'efinie par une somme, on \'etudie donc le signe de $u_{n+1}-u_n$.
$$\begin{array}{lll}
u_{n+1}-u_n&=&\sum\limits_{k=0}^{2(n+1)}\ddp\frac{(-1)^k}{k+1}-\sum\limits_{k=0}^{2n}\ddp\frac{(-1)^k}{k+1}\vsec\\
&=& \sum\limits_{k=0}^{2n+2}\ddp\frac{(-1)^k}{k+1}-\sum\limits_{k=0}^{2n}\ddp\frac{(-1)^k}{k+1}\vsec\\
&=& \ddp\frac{1}{2n+3}-\ddp\frac{1}{2n+2}\vsec\\
&=& \ddp\frac{-1}{(2n+3)(2n+2)}.
\end{array}$$
Ainsi, la suite $\suiteu$ est d\'ecroissante.
\item 
On remarque que
$$u_{n+1}-u_n=n+1+2(-1)^{n+1}-n-2(-1)^n=1+2(-1)^{n+1}+2(-1)^{n+1}=1+4(-1)^{n+1}.$$
Ainsi, si $n=2p$ pair, on obtient: $u_{2p+1}-u_{2p}=5>0$ et si $n=2p+1$ impair, on obtient: $u_{2p+2}-u_{2p+1}=-3<0$. Ainsi la suite $\suiteu$ n'est pas monotone.
\item La suite $\suiteu$ est d\'efinie par une somme, on \'etudie donc le signe de $u_{n+1}-u_n$.
$$u_{n+1}-u_n = \sum\limits_{k=2}^{n+1}\ddp\frac{1}{k \ln k}-\sum\limits_{k=2}^{n}\ddp\frac{1}{k \ln k} = \ddp\frac{1}{(n+1) \ln (n+1)} > 0.$$
Ainsi, la suite $\suiteu$ est croissante.
\end{enumerate}
\end{correction}



%





%--------------------------------------------------------------------------------------
%--------------------------------------------------------------------------------------
\begin{exercice} 
\noindent \'Etudier le comportement en $+\infty$ des suites suivantes:
\begin{enumerate}
\begin{minipage}[t]{0.3\textwidth}
\item
$u_n=\ddp\frac{n}{\cos{\left(\ddp\frac{1}{n}  \right)}}$
\item
$u_n=\ddp\sqrt{n+1}-\sqrt{n}$
\item 
$u_n=\ln{(n+1)}-\ln{(n^2)}$
\item  
$u_n=\left(1+\ddp\frac{2}{n}\right)^n$
\item  
$u_n=\ddp\frac{2^n+n}{2^n}$
\item  $u_n=\ddp\frac{n+(-1)^n}{n-\ln{(n^3)}}$
\end{minipage}
\begin{minipage}[t]{0.3\textwidth}
\item  
$u_n=\ddp\frac{1}{n^2}\ddp \sum\limits_{k=1}^n k$
\item 
$u_n=\ddp\frac{3^n-4^n}{3^n+4^n}$
\item  
$u_n=\ddp\frac{\sin{n}}{n}$
\item  
$u_n=\ddp\frac{1+(-1)^n}{n}$
\item  
$u_n=n^2-n\cos{n}+2$
%\item  $u_n=n^2\left( \ddp\demi \right)^n \sin{(n!)}$
\item  $u_n=\ddp\frac{n!+(n+1)!}{(n+2)!}$
%\item  $u_n=\ddp\frac{1}{n^3}\ddp \sum\limits_{k=1}^n k^2$
\end{minipage}
\begin{minipage}[t]{0.3\textwidth}
\item 
$u_n=\ln{(2^n+n)}$
%\item $u_n=\ln{(2^n-n)}$  
\item $u_n=n^{\frac{1}{n}}$  
\item   $u_n=(\ln{n})^n$
\item  $u_n=\ddp\frac{n^3+2^n}{3^n}$ 
%\item   $t_n=(a^n+b^n)^{\frac{1}{n}}$
\item $u_n=(n^2+n+1)^{\frac{1}{n}}$  
\item $u_n=\ddp\frac{1}{a^n}\ddp \sum\limits_{k=1}^n b^k$  
\item $u_n = \ddp n^2 \left(\cos\left(\frac{1}{n^2}\right)-1\right)$
\end{minipage}
\end{enumerate}
\end{exercice}




%----------------------------------------------------
\begin{correction} \;
Je ne donne ici que les r\'eponses et quelques indications pour trouver les limites demand\'ees. Une telle r\'edaction dans une copie serait tr\`es insuffisante.
\begin{enumerate}
\item
$\lim \limits_{n\to +\infty}\ddp\frac{n}{\cos{\left(\ddp\frac{1}{n}  \right)}}=+\infty$ par compos\'ee et produit de limite car $\cos{(0)}=1$.
\item On a ici une forme ind\'etermin\'ee avec une diff\'erence de racines. L'id\'ee est d'utiliser la quantit\'ee conjugu\'ee :
$$\sqrt{n+1}-\sqrt{n} = \ddp \frac{(\sqrt{n+1}-\sqrt{n})(\sqrt{n+1}+\sqrt{n})}{\sqrt{n+1}+\sqrt{n}} = \frac{n+1-n}{\sqrt{n+1}+\sqrt{n}} = \frac{1}{\sqrt{n+1}+\sqrt{n}}.$$
Par quotient de limites, on obtient donc : $\lim \limits_{n\to +\infty}\ddp\sqrt{n+1}-\sqrt{n}=0$.
\item
$\lim \limits_{n\to +\infty}\ln{(n+1)}-\ln{(n^2)}=-\infty$ en utilisant $\ln{\left( \ddp\frac{n+1}{n^2} \right)}$ et le th\'eor\`{e}me des mon\^{o}mes de plus haut degr\'e.
\item
$\lim\limits_{n\to +\infty}\left(1+\ddp\frac{2}{n}\right)^n=e^2$ en utilisant le fait que $\ln{\left( 1+\ddp\frac{2}{n} \right)} \underset{+\infty}{\thicksim} \ddp\frac{2}{n}$ (limite tr\`{e}s classique fait en cours).
\item
$\lim \limits_{n\to +\infty}\ddp\frac{2^n+n}{2^n}=1$ en mettant en facteur en haut et en bas le terme dominant, \`a savoir $2^n$ et en utilisant une croissance compar\'ee car $2^n=e^{n\ln{2}}$. 
\item
$\lim \limits_{n\to +\infty}\ddp\frac{n+(-1)^n}{n-\ln{(n^3)}}=1$ en mettant en facteur en haut et en bas $n$ et en remarquant que $\lim\limits_{n\to +\infty} \ddp\frac{(-1)^n}{n}=0$ par le th\'eor\`eme des gendarmes et que $\lim\limits_{n\to +\infty} \ddp\frac{\ln{(n^3)}}{n}=\lim\limits_{n\to +\infty} \ddp\frac{3\ln{(n)}}{n}=0 $ par croissance compar\'ee.
\item
$\lim \limits_{n\to +\infty}\ddp\frac{1}{n^2}\sum\limits_{k=1}^n k= \ddp\demi$ en \'ecrivant que $\sum\limits_{k=1}^n k=\ddp\frac{n(n+1)}{2}$ et d'apr\`es le th\'eor\`eme sur les mon\^omes de plus haut degr\'e.
\item
$\lim \limits_{n\to +\infty}\ddp\frac{3^n-4^n}{3^n+4^n}=-1$ en mettant en facteur en haut et en bas $4^n$ le terme dominant et appliquant le th\'eor\`eme sur les suites g\'eom\'etriques.
\item
$\lim \limits_{n\to +\infty}\ddp\frac{\sin{n}}{n}=0$ en utilisant un correctionollaire du th\'eor\`eme des gendarmes car $\left| \ddp\frac{\sin{n}}{n} \right|\leq \ddp\frac{1}{n}$ ou le th\'eor\`{e}me des gendarmes.
\item 
$\lim \limits_{n\to +\infty}\ddp\frac{1+(-1)^n}{n}=0$ en utilisant le th\'eor\`eme des gendarmes car: $0\leq \ddp\frac{1+(-1)^n}{n}\leq \ddp\frac{2}{n}$.
\item
$\lim \limits_{n\to +\infty}n^2-n\cos{n}+2=+\infty$ en mettant en facteur le terme dominant $n^2$ et en utilisant le correctionollaire du th\'eor\`eme des gendarmes avec $\left| \ddp\frac{\cos{n}}{n} \right|\leq\ddp\frac{1}{n}$.
%\item 
%$\lim \limits_{n\to +\infty}n^2\left( \ddp\demi \right)^n \sin{(n!)}=0$ en utilisant toujours le correctionollaire du th\'eor\`eme des gendarmes car $\left| n^2\left( \ddp\demi \right)^n\sin{n!} \right|\leq n^2\left( \ddp\demi \right)^n$ ainsi que les croissances compar\'ees car $n^2\left( \ddp\demi \right)^n=n^2 e^{-n\ln{2}}$. 
\item
$\lim \limits_{n\to +\infty}\ddp\frac{n!+(n+1)!}{(n+2)!}=0$ en utilisant la d\'efinition des factorielles.
%\item
%$\lim \limits_{n\to +\infty}\ddp\frac{1}{n^3}\sum\limits_{k=1}^n k^2=\ddp\frac{1}{3}$ en \'ecrivant que $\sum\limits_{k=1}^n k^2=\ddp\frac{n(n+1)(2n+1)}{6}$ et en utilisant le th\'eor\`eme sur les mon\^omes de plus haut degr\'e.
\item
$\lim \limits_{n\to +\infty}\ln{(2^n+n)}=+\infty$ par propri\'et\'e sur les somme et compos\'ee de limites.
%\item
%$\lim \limits_{n\to +\infty}\ln{(2^n-n)}=+\infty$ car $\ln{(2^n-n)}=n\ln{2}+\ln{\left( 1-\ddp\frac{n}{2^n} \right)}$ et en utilisant les croissances compar\'ees, on a: $\lim\limits_{n\to +\infty} 1-\ddp\frac{n}{2^n}=1$.
\item
$\lim \limits_{n\to +\infty}n^{\frac{1}{n}}=1$ car $n^{1/n}=e^{1/n\ln{n}}$ puis par croissance compar\'ee, on a: $\lim\limits_{n\to +\infty} \ddp\frac{\ln{n}}{n}=0$.  
\item
$\lim \limits_{n\to +\infty}(\ln{n})^n=+\infty$. Il n'y a pas de forme ind\'etermin\'ee ici, il suffit d'\'ecrire que $\left( \ln{n}\right)^n=e^{n\ln{(\ln{n})}}$.
\item
$\lim \limits_{n\to +\infty}\ddp\frac{n^3+2^n}{3^n}=0$ en mettant $2^n$ en facteur au num\'erateur et en utilisant ensuite le th\'eor\`eme sur la convergence des suites g\'eom\'etriques et les croissances compar\'ees car $\lim\limits_{n\to +\infty} \ddp\frac{n^3}{2^n}=\lim\limits_{n\to +\infty} \ddp\frac{n^3}{e^{n\ln{2}}}=0$. 
%\item 
%$\lim \limits_{n\to +\infty}(a^n+b^n)^{\frac{1}{n}}$ avec $a>0$ et $b>0$.\\
%\noindent Pour cette limite, il faut \'etudier des cas selon que $a>b$, $a=b$ ou $a<b$. Faisons le par exemple pour $a>b$. On commence par \'ecrire que: $(a^n+b^n)^{1/n}=e^{1/n\ln{(a^n+b^n)}}$. Puis, comme $a>b$, on met le terme dominant en facteur, \`a savoir $a^n$. On obtient pour l'exposant qui est dans l'exponentielle: 
%$$\ddp\frac{\ln{(a^n)}}{n}+\ddp\frac{\ln{\left( 1+(\frac{b}{a})^n \right)}}{n}=\ln{a}+\ddp\frac{\ln{\left( 1+(\frac{b}{a})^n \right)}}{n}.$$
%Le deuxi\`eme terme tend vers 0 (pas de forme ind\'etermin\'ee) donc, par composition de limite, on obtient, si $a>b$,
%$$\lim \limits_{n\to +\infty}(a^n+b^n)^{\frac{1}{n}}=e^{\ln{a}}=a.$$
%Il suffit alors de faire un raisonnement analogue pour les deux autres cas.
\item
$\lim \limits_{n\to +\infty}(n^2+n+1)^{\frac{1}{n}}=1$ en transformant l'expression en mettant le terme dominant $n^2$ en facteur:
$$(n^2+n+1)^{\frac{1}{n}}=e^{1/n\ln{(n^2+n+1)}}.$$
Le terme en exposant dans l'exponentielle est alors
$$\ddp\frac{\ln{(n^2+n+1)}}{n}=\ddp\frac{\ln{(n^2)}}{n}+\ddp\frac{\ln{(1+\frac{1}{n}+\frac{1}{n^2})}}{n}.$$
On obtient alors la limite voulue en utilisant les croissances compar\'ees.
\item
$\lim \limits_{n\to +\infty}\ddp\frac{1}{a^n}\sum\limits_{k=1}^n b^k$.\\
\noindent On suppose ici que $a>0$ et $b>0$. Commen\c{c}ons par calculer l'expression dont on cherche la limite. On obtient, si $b\not= 1$
$$\ddp\frac{1}{a^n}\sum\limits_{k=1}^n b^k=\ddp\frac{b}{1-b}\ddp\frac{1-b^n}{a^n}.$$
Et si $b=1$, on obtient $\ddp\frac{1}{a^n}\sum\limits_{k=1}^n b^k=\ddp\frac{n}{a^n}$.
Etudions alors des cas:
\begin{itemize}
\item[$\star$] Si $b>1$:\\
\noindent On a alors: $u_n\underset{+\infty}{\thicksim} \ddp\frac{-b}{1-b} \left( \ddp\frac{b}{a} \right)^n$ car $1-b^n\underset{+\infty}{\thicksim} -b^n$ et en utilisant ensuite les propri\'et\'es sur le produit et le quotient d'\'equivalent.
Ainsi, on obtient les cas suivants:\\
\noindent Si $b<a$, alors $\lim\limits_{n\to +\infty} u_n=0$\\
\noindent Si $b=a$, alors $\lim\limits_{n\to +\infty} u_n=\ddp\frac{-b}{1-b}$\\
\noindent Si $b>a$, alors $\lim\limits_{n\to +\infty} u_n=+\infty$\\
\item[$\star$] Si $0<b<1$:\\
\noindent On a alors: $u_n\underset{+\infty}{\thicksim} \ddp\frac{b}{1-b} \left( \ddp\frac{1}{a} \right)^n$ car $1-b^n\underset{+\infty}{\thicksim} 1$ et en utilisant ensuite les propri\'et\'es sur le produit et le quotient d'\'equivalent..
Ainsi, on obtient les cas suivants:\\
\noindent Si $a>1$, alors $\lim\limits_{n\to +\infty} u_n=0$\\
\noindent Si $a=1$, alors $\lim\limits_{n\to +\infty} u_n=\ddp\frac{b}{1-b}$\\
\noindent Si $a<1$, alors $\lim\limits_{n\to +\infty} u_n=+\infty$\\
\item[$\star$] Si $b=1$:\\
\noindent On a alors: $u_n\underset{+\infty}{\thicksim} \ddp\frac{n}{a^n}$.
Ainsi, on obtient les cas suivants:\\
\noindent Si $a>1$, alors $\lim\limits_{n\to +\infty} u_n=0$ par croissance compar\'ee\\
\noindent Si $a=1$, alors $\lim\limits_{n\to +\infty} u_n=+\infty$\\
\noindent Si $a<1$, alors $\lim\limits_{n\to +\infty} u_n=+\infty$\\
\end{itemize}
\item $ \lim\limits_{n\to+\infty}  \ddp n^2 \left(\cos\left(\frac{1}{n^2}\right)-1\right)$ : on utilise ici les \'equivalents usuels. On a : $\ddp u_n \underset{+\infty}{\thicksim} n^2 \times \left(-\frac{1}{2n^2}\right) = -\frac{1}{2}$, donc $\lim\limits_{n\to +\infty} u_n=\ddp -\demi$.
\end{enumerate} 
\end{correction}
%--------------------------------------------------
%------------------------------------------------












\begin{exercice}  \;
Calculer les limites des suites suivantes. 
\begin{enumerate}
\noindent \begin{minipage}[t]{0.3\textwidth}
\item $u_n=e^{n^2+n+1}$
\item $u_n=e^{2n}-e^n$
\item $u_n=\ddp\frac{e^n+n^2+n+1}{e^{2n}+1}$
\item $u_n=\ddp\frac{n}{n-1}e^{\frac{1}{n}}$
\item $u_n=e^{n^2}-e^{n+1}$
\item $u_n=\ln{\left(\ddp\frac{e^n+1}{e^n-1}\right)}$
\end{minipage}
\begin{minipage}[t]{0.3\textwidth}

\item $u_n=\ln{\left(\ddp\frac{e^n+n^2}{2n+1}\right)}$
\item $u_n=\ln{\left(\ddp\frac{2-n}{n+4}\right)}$
\item $u_n=\ddp\frac{2^n}{n^2+1}$
\item $u_n=\left( \ddp\demi \right)^n \ln{n}$
\end{minipage}
\begin{minipage}[t]{0.3\textwidth}
\item $u_n=\ddp\frac{e^{\sqrt{n}}}{n^2}$
\item $u_n=e^n-n^{\frac{2}{3}}$
\item $u_n=e^{\frac{1}{n-2}}$
\item $u_n=(2n-1)e^{\frac{1}{n-2}}$
\item $u_n=\ddp\frac{\ln{(n^2+1)}}{n}$
\end{minipage}
\end{enumerate}
\end{exercice}
%--------------------------------------------------
%------------------------------------------------
\begin{correction}  \;
Je ne d\'etaille pas tous les calculs.
\begin{enumerate}
\item  $\mathbf{u_n=e^{n^2+n+1}}$:
\begin{itemize}
\item[$\bullet$]  Par propri\'et\'e sur les sommes et compos\'ee de limites, on obtient que $\lim\limits_{n\to +\infty} u_n=+\infty$.
\end{itemize} 
\item $\mathbf{u_n=e^{2n}-e^n}$
\begin{itemize}
\item[$\bullet$]  FI donc on met en facteur le terme dominant \`{a} savoir $e^{2n}$. On obtient que: $u_n=e^{2n}(1-e^{-n})$. Puis par propri\'et\'e sur les composition, somme et produit de limites, on obtient que: $\lim\limits_{n\to +\infty} u_n=+\infty$.
\end{itemize} 
\item $\mathbf{u_n=\ddp\frac{e^n+n^2+n+1}{e^{2n}+1}}$
\begin{itemize}

\item[$\bullet$]  FI donc on met en facteur le terme dominant au num\'erateur ($e^n$) et au d\'enominateur $e^{2n}$. On obtient alors que par propri\'et\'e sur les compos\'ee, sommes et quotient de limites $\lim\limits_{n\to +\infty} u_n=0$. 
\end{itemize} 
\item $\mathbf{u_n=\ddp\frac{n}{n-1}e^{\frac{1}{n}}}$:
\begin{itemize}
\item[$\bullet$]  Par le th\'eor\`{e}me du mon\^{o}me de plus haut degr\'e, on a: $\lim\limits_{n\to +\infty} \ddp\frac{n}{n-1}=1$. Donc par propri\'et\'e sur les compos\'ee et produit de limites: $\lim\limits_{n\to +\infty} u_n=1$. 
\end{itemize}



\item $\mathbf{u_n=e^{n^2}-e^{n+1}}$:
\begin{itemize}
\item[$\bullet$]  FI donc on met en facteur le terme dominant \`{a} savoir $e^{n^2}$. On obtient que $u_n=e^{n^2}(1-e^{-n^2+n+1})$. Par le th\'eor\`{e}me du mon\^{o}me de plus haut degr\'e, on a: $\lim\limits_{n\to +\infty} -n^2+n+1=-\infty$. Ainsi par propri\'et\'e sur les sommes, compos\'ees etb produit de limites, on obtient que: $\lim\limits_{n\to +\infty} u_n=+\infty$.
\end{itemize} 
\item $\mathbf{u_n=\ln{\left(\ddp\frac{e^n+1}{e^n-1}\right)}}$
\begin{itemize}
\item[$\bullet$]  FI donc on met en facteur le terme dominant au num\'erateur et au d\'enominateur \`{a} savoir $e^n$. On obtient alors $u_n=\ln{\left( \ddp\frac{1+e^{-n}}{1-e^{-n}}  \right)}$. Puis par propri\'et\'es sur les compos\'ees, sommes, quotient de limites, on obtient que: $\lim\limits_{n\to +\infty} u_n=0$. 
\end{itemize} 
\item $\mathbf{u_n=\ln{\left(\ddp\frac{e^n+n^2}{2n+1}\right)}}$
\begin{itemize}

\item[$\bullet$]  FI donc on met en facteur le terme dominant au num\'erateur et au d\'enominateur \`{a} savoir $e^n$ au num\'erateur et $n$ au d\'enominateur. On obtient que: $u_n=\ln{\left( \ddp\frac{e^n}{n}\times \ddp\frac{1+\frac{n^2}{e^n}}{2+\frac{1}{n}}  \right)}$. Par croissance compar\'ee, on a: $\lim\limits_{n\to +\infty}  \ddp\frac{e^n}{n}=+\infty$ et  
$\lim\limits_{n\to +\infty}  \ddp\frac{n^2}{e^n}=0$. Puis par propri\'et\'e sur les sommes, quotients, produit et compos\'ee de limites, on obtient que $\lim\limits_{n\to +\infty} u_n=+\infty$.
\end{itemize} 
\item  $\mathbf{u_n=\ln{\left(\ddp\frac{2-n}{n+4}\right)}}$:
Pas définie pour $n>2 !$

\item $\mathbf{u_n=\ddp\frac{2^n}{n^2+1}}$:
\begin{itemize}
\item[$\bullet$]  FI donc on fait appara\^{i}tre une croissance compar\'ee en mettant en facteur $n^2$ terme dominant au d\'enominateur. On obtient que $u_n=\ddp\frac{e^{\ln{2}n}}{n^2}\times \ddp\frac{1}{1+\frac{1}{n^2}}$. Par croissance compar\'ee, on a: $\lim\limits_{n\to +\infty} \ddp\frac{e^{\ln{2}n}}{n^2}=+\infty$. Puis par propri\'et\'e sur les quotients, somme et produit de limites, on obtient que: $\lim\limits_{n\to +\infty} u_n=+\infty$.
\end{itemize} 
\item $\mathbf{u_n=\left( \ddp\demi \right)^n \ln{n}}$
\begin{itemize}
\item[$\bullet$]  FI car $u_n=\ln{(n)}e^{-n\ln{2}}$. On va faire appara\^{i}tre une croissance compar\'ee en multipliant et divisant par $n$. On obtient que: $u_n=\ddp\frac{n}{e^{\ln{2}n}}\times \ddp\frac{\ln{n}}{n}$. Par croissances compar\'ees, on a: $\lim\limits_{n\to +\infty} \ddp\frac{n}{e^{\ln{2}n}}=0=\lim\limits_{n\to +\infty} \ddp\frac{\ln{n}}{n}$. Ainsi par propri\'et\'e sur le produit de limite, on a: $\lim\limits_{n\to +\infty} u_n=0$. 
\end{itemize} 
\item  $\mathbf{u_n=\ddp\frac{e^{\sqrt{n}}}{n^2}}$
\begin{itemize}
\item[$\bullet$]  FI donc on transforme l'enpression afin de faire appara\^{i}tre une croissance compar\'ee. On pose par enemple $n=\sqrt{n}$ et on obtient que $u_n=u_n=\ddp\frac{e^n}{n^4}$. Ainsi par croissance compar\'ee: $\lim\limits_{n\to +\infty} u_n=+\infty$. Puis par propri\'et\'e sur la composition de limites: $\lim\limits_{n\to +\infty} u_n=+\infty$. 
\end{itemize} 
\item $\mathbf{u_n=e^n-n^{\frac{2}{3}}}$:
\begin{itemize}
\item[$\bullet$]  FI donc on met en facteur le terme dominant \`{a} savoir $e^n$. On obtient que: $u_n=e^n\left( 1-\ddp\frac{n^{\frac{2}{3}}}{e^n}  \right)$. Par croissance compar\'ee, on a: $\lim\limits_{n\to +\infty} \ddp\frac{n^{\frac{2}{3}}}{e^n} =0$. Puis par propri\'et\'e sur les somme et produit de limites, on obtient que: $\lim\limits_{n\to +\infty} u_n=+\infty$.
\end{itemize} 
\item $\mathbf{u_n=e^{\frac{1}{n-2}}}$
\begin{itemize}
\item[$\bullet$] $\lim\limits_{n\to +\infty} u_n=1$
\end{itemize} 
\item $\mathbf{u_n=(2n-1)e^{\frac{1}{n-2}}}$:
\begin{itemize}
\item[$\bullet$] $\lim\limits_{n\to +\infty} u_n=+\infty$
\end{itemize} 
\item$\mathbf{u_n=\ddp\frac{\ln{(n^2+1)}}{n}}$
\begin{itemize}

\item[$\bullet$]  FI donc on met en facteur le terme dominant $n^2$ dans le logarithme afin de faire appara\^{i}tre une croissance compar\'ee. On obtient que: $u_n=2\ddp\frac{\ln{n}}{n}+\ddp\frac{\ln{(1+\frac{1}{n^2})}}{n}$. Par croissance compar\'ee, on a: $\lim\limits_{n\to +\infty} \ddp\frac{\ln{n}}{n}=0$ et par propri\'et\'e sur les quotients, somme et compos\'ee de limites: $\lim\limits_{n\to +\infty} \ddp\frac{\ln{(1+\frac{1}{n^2})}}{n}=0$. Donc par propri\'et\'e sur les sommes de limites, on a: $\lim\limits_{n\to +\infty} u_n=0$. 
\end{itemize} 
\end{enumerate}
\end{correction}





\section*{Type DS}
%--------------------------------------------------------------

%-----------------------------------------------------------------
\begin{exercice}  \; Suites homographiques.\\
\noindent On consid\`ere les suites $\suiteu$ et $\suitev$ d\'efinies par
$$u_0=0\ \hbox{et}\ \forall n\in\N,\ u_{n+1}=\ddp\frac{5u_n-2}{u_n+2} \quad \textmd{ et } \quad \forall \, n \in \N, \ v_n = \ddp\frac{u_n-2}{u_n-1}.$$
\begin{enumerate}
 \item
Montrer que la suite $\suiteu$ est bien d\'efinie et que pour tout $n\geq 3$, $u_n>1$. 
\item 
En d\'eduire que la suite $\suitev$  est bien d\'efinie sur $\N$.
\item 
Montrer que $\suitev$ est g\'eom\'etrique.
\item 
En d\'eduire l'expression explicite de $\suitev$ puis de $\suiteu$.
\item Etudier la convergence de la suite $\suiteu$.
\end{enumerate}
\end{exercice}

%----------------------------------------------------
\begin{correction} \;
\begin{enumerate}
 \item Comme toujours pour ce genre de question, on fait une r\'ecurrence.
\begin{itemize}
\item[$\bullet$] On montre par r\'ecurrence sur $n\geq 3$ la propri\'et\'e $\mathcal{P}(n):\quad u_n \textmd{ d\'efini et  } u_n> 1.$
\item[$\bullet$]  Initialisation: pour $n=3$:\\
\noindent On a: $u_1=-1$ puis $u_2=-7$ et $u_3=\ddp\frac{37}{5}>1$. Ainsi, $\mathcal{P}(3)$ est vraie.
\item[$\bullet$]  H\'er\'edit\'e: soit $n\geq 3$, on suppose la propri\'et\'e vraie \`a l'ordre $n$, montrons que $\mathcal{P}(n+1)$ est vraie. Par hypoth\`ese de r\'ecurrence, on sait que $u_n>1$, donc $u_n-1\not=0$ et $u_{n+1}$ est bien d\'efini. De plus, on a
$$u_{n+1}>1 \; \Leftrightarrow \; \ddp\frac{5u_n-2}{u_n+2}>1 \; \Leftrightarrow \; 5u_n -2>u_n+2 \; \Leftrightarrow \; u_n>1.$$
Ici on a utilis\'e le fait que $u_n>1$ d'apr\`es $\mathcal{P}(n)$, et donc que $u_n+2>0$. On arrive $u_n>1$ qui est bien vrai, donc par \'equivalences, $u_{n+1}>1$ est vrai aussi. Ainsi, $\mathcal{P}(n+1)$ est vraie.
\item[$\bullet$]  Conclusion: il r\'esulte du principe de r\'ecurrence que $\forall n\geq 3,\ u_n>1.$
\end{itemize}
%--
\item La suite $(v_n)_{n\in\N}$ est bien d\'efinie  car $u_0,\ u_1,\ u_2$ ne sont pas \'egaux \`a 1 et ensuite on a $\forall n \geq 3, u_n > 1$. Ainsi pour tout $n\in\N$, on a bien $u_n-1\not= 0$ et $v_n$ bien d\'efini. 
%--
\item Soit $n\in\N$:
$$\begin{array}{lll}
v_{n+1}&=& \ddp\frac{u_{n+1}-2}{u_{n+1}-1}
= \ddp\frac{\frac{5u_n-2-2u_n-4}{u_n+2}}{\frac{5u_n-2-u_n-2}{u_n+2}}
= \ddp\frac{3u_n-6}{4u_n-4}
= \ddp\frac{3}{4}\ddp\frac{u_n-2}{u_n-1}
= \ddp\frac{3}{4}v_n.
\end{array}$$
Ainsi la suite $(v_n)_{n\in\N}$ est une suite g\'eom\'etrique de raison $\ddp\frac{3}{4}$ et de premier terme $2$.
\item On en d\'eduit la formule explicite de $v_n$:
$$\forall n\in\N,\quad v_n=2\left( \ddp\frac{3}{4} \right)^n.$$
En remarquant que: $u_n(v_n-1)=v_n-2$ et que la suite $(v_n)_{n\in\N}$ \'etait toujours diff\'erente de 1, on obtient que
$$\forall n\in\N,\ u_n=\ddp\frac{v_n-2}{v_n-1}\Rightarrow u_n=\ddp\frac{2(\frac{3}{4})^n-2}{2(\frac{3}{4})^n-1}.$$
\item Comme $-1<\ddp\frac{3}{4}<1$, la suite $(v_n)_{n\in\N}$ converge vers 0 et ainsi, on a: $\lim\limits_{n\to\N} u_n=2$.
\end{enumerate}
\end{correction}
%----------------------------------------------------------------





\vspace{0.5cm}

\begin{exercice} \;
Soit une suite $\suiteu$ qui v\'erifie la relation de r\'ecurrence
$$\left\lbrace\begin{array}{l}
u_0\in\R\vsec\\
\forall n\in\N,\ u_{n+1}=-u_n^2+2u_n
\end{array}\right.$$
\begin{enumerate}
 \item 
Calculer $1-u_{n+1}$ en fonction de $1-u_n$.
\item 
D\'eterminer la limite de la suite $\suiteu$, si elle existe, en fonction du premier terme $u_0$.
\end{enumerate}
\vspace{0.5cm}
\end{exercice}




%--------------------------------------------------
%------------------------------------------------
%----------------------------------------------------
\begin{correction} \;
\begin{enumerate}
 \item Soit $n\in\N$, on a: $1-u_{n+1}=1+u_n^2-2u_n=(1-u_n)^2.$
 \item On pose $v_n = 1-u_n$. On a alors $v_{n+1}=v_n^2$. Essayons de calculer $v_n$ : on a $v_1=v_0^2$, $v_2=v_0^4$, $v_3=v_0^8$. On conjecture donc : $\forall n\in\N,\ v_n=v_0^{2^n}$. \\
Montrons par r\'ecurrence sur $n\in\N$ la propri\'et\'e : $\mathcal{P}(n):\quad v_n=v_0^{2^n}.$
\begin{itemize}
\item[$\bullet$]  Initialisation: pour $n=0$:\\
\noindent On a: $v_0^{2^0}=v_0$. Donc $\mathcal{P}(0)$ est vraie.
\item[$\bullet$]  H\'er\'edit\'e: Soit $n\in\N$. On suppose la propri\'et\'e vraie \`a l'ordre $n$, montrons qu'elle est vraie \`a l'ordre $n+1$. On a vu que: $v_{n+1}=v_n^2$. On utilise alors l'hypoth\`ese de r\'ecurrence et on obtient
$$v_{n+1}=\left(v_0^{2^n}  \right)^2=v_0^{2^{n+1}}.$$
Donc $\mathcal{P}(n+1)$ est vraie.
\item[$\bullet$] Conclusion: il r\'esulte du principe de r\'ecurrence que
$$\forall n\in\N,\quad v_n=v_0^{2^n}.$$
\end{itemize}
On obtient donc pour tout $n\in\N$: $u_n=1-(1-u_0)^{2^n}$.
\begin{itemize}
\item[$\bullet$] Si $1-u_0>1 \; \Leftrightarrow \; u_0<0$, alors : $\lim\limits_{n\to +\infty} (1-u_0)^{2^n}=+\infty$, donc $\lim\limits_{n\to +\infty} u_n=-\infty$.
\item[$\bullet$] Si $u_0=0$, alors $1-u_0=1$ et ainsi: $\forall n\in\N,\quad u_n=0$ et donc $\lim\limits_{n\to +\infty} u_n=0$.
\item[$\bullet$] Si $-1<1-u_0<1 \; \Leftrightarrow \; 0<u_0<2$, alors  : $\lim\limits_{n\to +\infty} u_n=1$.
\item[$\bullet$] Si $u_0=2$, alors $1-u_0=-1$ et $(1-u_0)^{2^n} = 1$, et ainsi: $\forall n\in\N,\quad u_n=0$ et donc $\lim\limits_{n\to +\infty} u_n=0$.
\item[$\bullet$] Si $1-u_0 < -1 \; \Leftrightarrow \; u_0 > 2$, alors $(1-u_0)^2 > 1$, et donc $\lim\limits_{n\to +\infty} (1-u_0)^{2^n}=+\infty$, soit $\lim\limits_{n\to +\infty} u_n=-\infty$.
\end{itemize}
\end{enumerate}
\end{correction}













%----------------------------------------------------------------------------------
\begin{exercice} \;
On d\'efinit deux suites $(u_n)_{n\in\N^{\star}}$ et $(v_n)_{n\in\N^{\star}}$ par 
$$u_1=1\quad v_1=12\quad \forall n\in\N^{\star},\ u_{n+1}=\ddp\frac{u_n+2v_n}{3}\quad v_{n+1}=\ddp\frac{u_n+3v_n}{4}.$$
\begin{enumerate}
 \item
On pose, pour tout $n\in\N^{\star}$, $w_n=v_n-u_n$. Donner l'expression de $(w_n)_{n\in\N^{\star}}$. 
\item 
Montrer que $(u_n)_{n\in\N^{\star}}$ et $(v_n)_{n\in\N^{\star}}$ sont adjacentes.
\item 
On pose pour tout $n\in\N^{\star}$, $t_n=3u_n+8v_n$. \\
Donner l'expression de $(t_n)_{n\in\N^{\star}}$ et en d\'eduire la limite de $(u_n)_{n\in\N^{\star}}$ et $(v_n)_{n\in\N^{\star}}$.
\end{enumerate}
\end{exercice}
\vspace{0.5cm}

\begin{correction} \;
\textbf{On d\'efinit deux suites $\mathbf{(u_n)_{n\in\N^{\star}}}$ et $\mathbf{(v_n)_{n\in\N^{\star}}}$ par }
$$\mathbf{u_1=1\quad v_1=12\quad \forall n\in\N^{\star},\ u_{n+1}=\ddp\frac{u_n+2v_n}{3}\quad v_{n+1}=\ddp\frac{u_n+3v_n}{4}.}$$
\begin{enumerate}
\item \textbf{On pose, pour tout $\mathbf{n\in\N^{\star}}$, $\mathbf{w_n=v_n-u_n}$. Donner l'expression de $\mathbf{(w_n)_{n\in\N^{\star}}}$:}\\
\noindent Soit $n\in\N^{\star}$, on a: 
$$w_{n+1}=\ddp\frac{u_n+3v_n}{4}-\ddp\frac{u_n+2v_n}{3}=\ddp\frac{v_n-u_n}{12}=\ddp\frac{1}{12}w_n.$$
Ainsi la suite $(w_n)_{n\in\N^{\star}}$ est une suite g\'eom\'etrique de raison $\ddp\frac{1}{12}$ et de premier terme $w_1=v_1-u_1=11$. On en d\'eduit donc l'expression explicite de $w_n$:
$$\fbox{$\forall n\geq 1,\ w_n=w_1\left( \ddp\frac{1}{12}  \right)^{n-1}=11\left( \ddp\frac{1}{12}  \right)^{n-1}$.}$$
\item \textbf{Montrer que $\mathbf{(u_n)_{n\in\N^{\star}}}$ et $\mathbf{(v_n)_{n\in\N^{\star}}}$ sont adjacentes:}
\begin{itemize}
\item[$\bullet$] \'Etude de la monotonie de la suite $(u_n)_{n\in\N^{\star}}$:\\
\noindent Soit $n\geq 1$, on a:
$$u_{n+1}-u_n=\ddp\frac{u_n+2v_n}{3}-u_n=\ddp\frac{2(v_n-u_n)}{3}=\ddp\frac{2}{3}w_n.$$
Or on conna\^{i}t l'expression de $w_n$, on obtient donc:
$$\forall n\in\N,\ u_{n+1}-u_n=\ddp\frac{2}{3} \times 11\left( \ddp\frac{1}{12}  \right)^{n-1}  \geq 0.$$
Ainsi, la suite $(u_n)_{n\in\N^{\star}}$ est croissante.
\item[$\bullet$] \'Etude de la monotonie de la suite $(v_n)_{n\in\N^{\star}}$:\\
\noindent Soit $n\geq 1$, on a:
$$v_{n+1}-v_n=\ddp\frac{u_n+3v_n}{4}-v_n=\ddp\frac{u_n-v_n}{4}=\ddp\frac{-1}{4}w_n.$$
Or on conna\^{i}t l'expression de $w_n$, on obtient donc:
$$\forall n\in\N,\ v_{n+1}-v_n=\ddp\frac{-11}{4} \left( \ddp\frac{1}{12}  \right)^{n-1}  \leq 0.$$
Ainsi, la suite $(v_n)_{n\in\N^{\star}}$ est d\'ecroissante.
\item[$\bullet$] Montrons que $\lim\limits_{n\to +\infty} v_n-u_n=0$:\\
\noindent On a montr\'e \`{a} la question pr\'ec\'edente que pour tout $n\in\N^{\star}$: $v_n-u_n=11\left( \ddp\frac{1}{12}  \right)^{n-1}$. Comme: $-1<\ddp\frac{1}{12}<1$, on a: $\lim\limits_{n\to +\infty } \left( \ddp\frac{1}{12} \right)^{n-1} =0$. Puis par propri\'et\'e sur le produit de limites, on obtient que: $\lim\limits_{n\to +\infty} v_n-u_n=0$.
\end{itemize}
Ainsi, on a donc montr\'e que les deux suites $(u_n)_{n\in\N^{\star}}$ et $(v_n)_{n\in\N^{\star}}$ sont adjacentes. D'apr\`{e}s le th\'eor\`{e}me sur les suites adjacentes, \fbox{elles convergent donc vers la m\^{e}me limite.}
\item \textbf{On pose pour tout $\mathbf{n\in\N^{\star}}$, $\mathbf{t_n=3u_n+8v_n}$.
Donner l'expression de $\mathbf{(t_n)_{n\in\N^{\star}}}$ et en d\'eduire la limite de $\mathbf{(u_n)_{n\in\N^{\star}}}$ et $\mathbf{(v_n)_{n\in\N^{\star}}}$:}
\begin{itemize}
\item[$\bullet$] Expression de $t_n$ pour tout $n\geq 1$:\\
\noindent Soit $n\geq 1$, on a: $t_{n+1}=3\ddp\frac{u_n+2v_n}{3}+8\ddp\frac{u_n+3v_n}{4}=3u_n+8v_n=t_n$. Ainsi 

\fbox{la suite $(t_n)_{n\in\N^{\star}}$ est constante \'egale \`{a} $t_1=3u_1+8v_1 = 99$.}
\item[$\bullet$] Calcul de la valeur de la limite $l$:\\
\noindent Comme la suite $(t_n)_{n\in\N^{\star}}$ est constante, on a: 
$\forall n\in\N^{\star},\ 3u_n+8v_n=99$. De plus on a d\'emontr\'e \`{a} la question 2 que les deux suites $(u_n)_{n\in\N^{\star}}$ et $(v_n)_{n\in\N^{\star}}$ convergent vers la m\^{e}me limite $l$ et ainsi par propri\'et\'e sur les produits et somme de limites, on obtient que: $\lim\limits_{n\to +\infty} 3u_n+8v_n=11l$. Par passage \`{a} la limite dans l'\'egalit\'e: $3u_n+8v_n=99$, on obtient donc que
$$11l=99 \Leftrightarrow \fbox{$l=9.$}$$
\end{itemize}
\end{enumerate}
\end{correction}









%--------------------------------------------------------------------------------
%----------------------------------------------------------------------------------
\begin{exercice} \;
Soient $(a_n)_{n\in\N}$ et $(b_n)_{n\in\N}$ deux suites telles que $a_0=0$, $b_0=1$ et pour tout $n\in\N$
$$a_{n+1}=-2a_n+b_n\qquad \hbox{et}\qquad b_{n+1}=3a_n.$$
\begin{enumerate}
\item D\'emontrer que la suite $(a_n+b_n)_{n\in\N}$ est constante.
\item Pour tout $n\in\N$, exprimer $a_n$ en fonction de $n$.
\item Pour tout $n\in\N$, d\'eterminer $b_n$ en fonction de $n$.
\end{enumerate}
\end{exercice}


%--------------------------------------------------
%------------------------------------------------
\begin{correction} \;
\textbf{Soient $\mathbf{(a_n)_{n\in\N}}$ et $\mathbf{(b_n)_{n\in\N}}$ deux suites telles que $\mathbf{a_0=0}$, $\mathbf{b_0=1}$ et pour tout $\mathbf{n\in\N}$}
$$\mathbf{a_{n+1}=-2a_n+b_n\qquad \hbox{et}\qquad b_{n+1}=3a_n.}$$
\begin{enumerate}
\item \textbf{D\'emontrer que la suite $\mathbf{(a_n+b_n)_{n\in\N}}$ est constante:}\\
\noindent Soit $n\in\N$, on a:
$$a_{n+1}+b_{n+1}=-2a_n+b_n+3a_n=a_n+b_n.$$
Ainsi \fbox{la suite $(a_n+b_n)_{n\in\N}$ est constante} et donc pour tout $n\in\N$: $a_n+b_n=a_0+b_0=1$. Donc \fbox{$\forall n\in\N,\ a_n+b_n=1$.}
\item \textbf{Pour tout $\mathbf{n\in\N}$, exprimer $\mathbf{a_n}$ en fonction de $\mathbf{n}$:}\\
\noindent Soit $n\in\N$. On a, en utilisant le fait que pour tout $n\in\N,\ a_n+b_n=1$, que pour tout $n\in\N$: $b_n=1-a_n$. Ainsi on obtient que pour tout $n\in\N$:
$$a_{n+1}=-2a_n+b_n \Leftrightarrow a_{n+1}=1-3a_n.$$
On reconna\^{i}t une suite arithm\'etico-g\'eom\'etrique .
\begin{itemize}
\item[$\bullet$] Calcul de la limite \'eventuelle: on r\'esout: $l=1-3l\Leftrightarrow l=\ddp\frac{1}{4}$.
\item[$\bullet$] \'Etude d'une suite auxiliaire: pour tout $n\in\N$, on pose $v_n=a_n-\ddp\frac{1}{4}$. Montrons que $\suitev$ est une suite g\'eom\'etrique de raison $-3$. Soit $n\in\N$, on a:
$$v_{n+1}=a_{n+1}-\ddp\frac{1}{4}=1-3a_n-\ddp\frac{1}{4}=-3\left(a_n-\ddp\frac{1}{4} \right)=-3v_n.$$
Ainsi la suite $\suitev$ est bien une suite g\'eom\'etrique de raison $\ddp\frac{1}{4}$ et de premier terme $v_0=a_0-\ddp\frac{1}{4}=-\ddp\frac{1}{4}$.
On en d\'eduit l'expression explicite de la suite $\suitev$: pour tout $n\in\N$, on a: $v_n=-\ddp\frac{1}{4}(-3)^n$.
\item[$\bullet$] Expression explicite de $a_n$ pour tout $n\in\N$:\\
\noindent Pour tout $n\in\N$, on a: $a_n=v_n+\ddp\frac{1}{4}=-\ddp\frac{1}{4}(-3)^n+\ddp\frac{1}{4}$.On a donc: \fbox{$\forall n\in\N,\ a_n=\ddp\frac{1}{4} \left( 1-(-3)^n  \right)$.}
\end{itemize}
\item \textbf{Pour tout $\mathbf{n\in\N}$, d\'eterminer $\mathbf{b_n}$ en fonction de $\mathbf{n}$:}\\
\noindent Comme pour tout $n\in\N$, on a: $b_{n+1}=3a_n$, on a: $b_n=3a_{n-1}$. Puis en utilisant le r\'esultat de la question pr\'ec\'edente, on obtient que \fbox{$\forall n\in\N,\ b_n=\ddp\frac{3}{4} \left( 1-(-3)^{n-1}  \right)$.}
\end{enumerate}
\end{correction}



\begin{exercice}
Soit $(a, b) \in \R^2$ tels que $0<a<b.$ On pose $u_0=a, v_0=b$ et pour tout $n\in \N$:
$$ u_{n+1} =\sqrt{u_n v_n}, \quad v_{n+1} = \frac{u_n +v_n}{2}.$$
\begin{enumerate}
\item INFO Ecrire une fonction Python qui prend en argument un entier $n$ et deux flottants $(a,b)$ et retourne la valeur de $u_n$.
\item Montrer que pour tout $(x,y)\in (\R_+)^2$ on a 
$$\sqrt{xy} \leq \frac{x+y}{2}$$
\item Montrer que :  $\forall n\in \N, \, 0\leq u_n\leq v_n.$
\item Montrer que $\suiteu$ est croissante et $\suite{v}$ et décroissante. 
\item Montrer que pour tout $(x,y)\in (\R_+)^2$ tel que  $x\geq y>0$ on a $$\frac{x+y}{2}-\sqrt{xy}\leq \frac{1}{2}(x-y)$$
\item Montrer que :  $\forall n\in \N, \, v_n-u_n\leq \frac{1}{2^n}(v_0-u_0).$
\item En déduire que $\suiteu$ et $\suitev$ convergent vers la même limite.

\item INFO On note $\ell$ la limite commune des deux suites. Ecrire une fonction Python qui prend en argument un flottant \texttt{eps} et retourne la valeur de $\ell$ à \texttt{eps} prés. 

\end{enumerate}
\end{exercice}


\begin{correction}
\begin{enumerate}
\item 
\begin{lstlisting}
    from math import sqrt
    def suite_u(n,a,b):
      u=a
      v=b
      for i in range(n):
        u,v=sqrt(u*v), (u+v)/2 #affectation simultanee
      return(u)    
\end{lstlisting}
\item On va procéder par équivalence : 

\begin{align*}
    &\sqrt{xy} \leq \frac{x+y}{2}\\
    \equivaut & xy \leq \left(  \frac{x+y}{2}\right)^2 \quad \text{car} x+y>0\\
    \equivaut & xy \leq \frac{x^2+y^2+2xy}{4}\\
    \equivaut & 4xy \leq x^2+y^2+2xy\\
    \equivaut & 0\leq x^2+y^2-2xy\\
    \equivaut & 0\leq (x-y)^2
\end{align*}

Cette dernière inégalité étant vérifiée pour tout $(x,y)\in \R^2$ et comme on a procédé par équivalence on a bien 
pour tout $(x,y)\in (\R_+)^2$ 
\conclusion{$\sqrt{xy} \leq \frac{x+y}{2}$}

\item Montrons par récurrence la propriété $\cP(n)$ définie pour tout $n$ par : \og $  0\leq u_n\leq v_n$ \fg. 
\textbf{Initialisation:}  Pour $n=0$, la propriété est vraie, d'après l'hypothèse faite dans l'énoncé  $0<a<b.$ 

 \textbf{H\'er\'edit\'e:}\\
Soit $n\geq 0$ fix\'e. On suppose la propri\'et\'e vraie \`a l'ordre $n$. Montrons qu'alors $\mathcal{P}(n+1)$ est vraie.\\
On a $u_{n+1} = \sqrt{u_n v_n}$ qui est bien défini car $u_n $ et $v_n$ sont positifs par hypothèse de récurrence. Cette expression assure aussi que $u_{n+1}$ est positif. 

De plus, 
\begin{align*}
v_{n+1}-u_{n+1} &= \frac{u_n +v_n}{2} - \sqrt{u_n v_n}&  \text{Par définition. }\\
                &\geq 0 &  \text{d'après la question précédente. }\\
\end{align*} 
Ainsi $v_{n+1} \geq  u_{n+1}$
La propriété $\cP$ est donc vraie au rang $n+1$.

\textbf{Conclusion:}\\
Il r\'esulte du principe de r\'ecurrence que pour tout $ n\geq 0$:
\begin{center}
\fbox{$  0\leq u_n\leq v_n$}
\end{center}

\item 
On a $u_{n+1}-u_n = \sqrt{u_nv_n}-u_n = \sqrt{u_n}( \sqrt{v_n} -\sqrt{u_n})$
Or comme $u_n\leq v_n$ et que la fonction racine est croissante on a 
$$u_{n+1}-u_n \geq 0$$

\conclusion{Autrement dit $\suite{u}$ est croissante}

On a $v_n+1-v_n= \frac{u_n+v_n}{2} - v_n  = \frac{u_n-v_n}{2}$
Or comme $u_n\leq v_n$ on a 
$$v_{n+1}-v_n\leq 0$$
\conclusion{Autrement dit $\suite{v}$ est décroissante}

\item 

On va procéder par équivalence : 

\begin{align*}
    &\frac{x+y}{2}-\sqrt{xy}\leq \frac{1}{2}(x-y)\\
    \equivaut &y \leq \sqrt{xy}\\
    \equivaut & y^2 \leq xy & \text{car $y\geq 0$}\\
    \equivaut & y \leq x & \text{car $y> 0$}\\
\end{align*}

Cette dernière inégalité étant vérifiée par hypothèse  et comme on a procédé par équivalence on a bien 
pour tout $(x,y)\in (\R_+)^2$  tel que $0\leq y\leq x$
\conclusion{$\frac{x+y}{2}-\sqrt{xy}\leq \frac{1}{2}(x-y)$}

 



\item 
Montrons par récurrence la propriété définie $\cP(n)$ définie pour tout $n$ par : \og $  v_n-u_n\leq \frac{1}{2^n}(v_0-u_0).$\fg. 
\textbf{Initialisation:}  Pour $n=0$, la propriété est vraie car le terme de gauche vaut $v_0-u_0$ et le terme de droite vaut $\frac{1}{1}(v_0-u_0)$. 

 \textbf{H\'er\'edit\'e:}\\
Soit $n\geq 0$ fix\'e. On suppose la propri\'et\'e vraie \`a l'ordre $n$. Montrons qu'alors $\mathcal{P}(n+1)$ est vraie.\\
%Montrons tout d'abord que $v_{n+1}-u_{n+1} \leq \frac{1}{2} (v_n -u_n)$. 
On  a 
\begin{align*}
v_{n+1}-u_{n+1} &= \frac{u_n +v_n}{2} - \sqrt{u_n v_n} \\
			     &\leq \frac{1}{2} (v_n-u_n) & \text{d'après la question précédente}
														
\end{align*}
On applique  l'hypothèse de récurrence, on a alors 
\begin{align*}
v_{n+1}-u_{n+1} & \leq \frac{1}{2} \times \frac{1}{2^n}(v_0-u_0)\\
						 & \leq \frac{1}{2^{n+1}}(v_0-u_0)				
\end{align*}

La propriété $P$ est donc vraie au rang $n+1$.

\textbf{Conclusion:}\\
Il r\'esulte du principe de r\'ecurrence que pour tout $ n\geq 0$:
\begin{center}
\fbox{$  v_n-u_n\leq \frac{1}{2^n}(v_0-u_0).$}
\end{center}
\item 

\begin{lstlisting}
from math import sqrt, abs
def limite(eps,a,b):
  u=a
  v=b
  while abs(u-v)>eps:
     u,v=sqrt(u*v), (u+v)/2 
  return(u)
\end{lstlisting}




\end{enumerate}
\end{correction}



\end{document}