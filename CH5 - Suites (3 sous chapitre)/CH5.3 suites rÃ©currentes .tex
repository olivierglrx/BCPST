
\documentclass[a4paper, 11pt]{article}
\usepackage[utf8]{inputenc}
\usepackage{amssymb,amsmath,amsthm}
\usepackage{geometry}
\usepackage[T1]{fontenc}
\usepackage[french]{babel}
\usepackage{fontawesome}
\usepackage{pifont}
\usepackage{tcolorbox}
\usepackage{fancybox}
\usepackage{bbold}
\usepackage{tkz-tab}
\usepackage{tikz}
\usepackage{fancyhdr}
\usepackage{sectsty}
\usepackage[framemethod=TikZ]{mdframed}
\usepackage{stackengine}
\usepackage{scalerel}
\usepackage{xcolor}
\usepackage{hyperref}
\usepackage{listings}
\usepackage{enumitem}
\usepackage{stmaryrd} 
\usepackage{comment}


\hypersetup{
    colorlinks=true,
    urlcolor=blue,
    linkcolor=blue,
    breaklinks=true
}





\theoremstyle{definition}
\newtheorem{probleme}{Problème}
\theoremstyle{definition}


%%%%% box environement 
\newenvironment{fminipage}%
     {\begin{Sbox}\begin{minipage}}%
     {\end{minipage}\end{Sbox}\fbox{\TheSbox}}

\newenvironment{dboxminipage}%
     {\begin{Sbox}\begin{minipage}}%
     {\end{minipage}\end{Sbox}\doublebox{\TheSbox}}


%\fancyhead[R]{Chapitre 1 : Nombres}


\newenvironment{remarques}{ 
\paragraph{Remarques :}
	\begin{list}{$\bullet$}{}
}{
	\end{list}
}




\newtcolorbox{tcbdoublebox}[1][]{%
  sharp corners,
  colback=white,
  fontupper={\setlength{\parindent}{20pt}},
  #1
}







%Section
% \pretocmd{\section}{%
%   \ifnum\value{section}=0 \else\clearpage\fi
% }{}{}



\sectionfont{\normalfont\Large \bfseries \underline }
\subsectionfont{\normalfont\Large\itshape\underline}
\subsubsectionfont{\normalfont\large\itshape\underline}



%% Format théoreme, defintion, proposition.. 
\newmdtheoremenv[roundcorner = 5px,
leftmargin=15px,
rightmargin=30px,
innertopmargin=0px,
nobreak=true
]{theorem}{Théorème}

\newmdtheoremenv[roundcorner = 5px,
leftmargin=15px,
rightmargin=30px,
innertopmargin=0px,
]{theorem_break}[theorem]{Théorème}

\newmdtheoremenv[roundcorner = 5px,
leftmargin=15px,
rightmargin=30px,
innertopmargin=0px,
nobreak=true
]{corollaire}[theorem]{Corollaire}
\newcounter{defiCounter}
\usepackage{mdframed}
\newmdtheoremenv[%
roundcorner=5px,
innertopmargin=0px,
leftmargin=15px,
rightmargin=30px,
nobreak=true
]{defi}[defiCounter]{Définition}

\newmdtheoremenv[roundcorner = 5px,
leftmargin=15px,
rightmargin=30px,
innertopmargin=0px,
nobreak=true
]{prop}[theorem]{Proposition}

\newmdtheoremenv[roundcorner = 5px,
leftmargin=15px,
rightmargin=30px,
innertopmargin=0px,
]{prop_break}[theorem]{Proposition}

\newmdtheoremenv[roundcorner = 5px,
leftmargin=15px,
rightmargin=30px,
innertopmargin=0px,
nobreak=true
]{regles}[theorem]{Règles de calculs}


\newtheorem*{exemples}{Exemples}
\newtheorem{exemple}{Exemple}
\newtheorem*{rem}{Remarque}
\newtheorem*{rems}{Remarques}
% Warning sign

\newcommand\warning[1][4ex]{%
  \renewcommand\stacktype{L}%
  \scaleto{\stackon[1.3pt]{\color{red}$\triangle$}{\tiny\bfseries !}}{#1}%
}


\newtheorem{exo}{Exercice}
\newcounter{ExoCounter}
\newtheorem{exercice}[ExoCounter]{Exercice}

\newcounter{counterCorrection}
\newtheorem{correction}[counterCorrection]{\color{red}{Correction}}


\theoremstyle{definition}

%\newtheorem{prop}[theorem]{Proposition}
%\newtheorem{\defi}[1]{
%\begin{tcolorbox}[width=14cm]
%#1
%\end{tcolorbox}
%}


%--------------------------------------- 
% Document
%--------------------------------------- 






\lstset{numbers=left, numberstyle=\tiny, stepnumber=1, numbersep=5pt}




% Header et footer

\pagestyle{fancy}
\fancyhead{}
\fancyfoot{}
\renewcommand{\headwidth}{\textwidth}
\renewcommand{\footrulewidth}{0.4pt}
\renewcommand{\headrulewidth}{0pt}
\renewcommand{\footruleskip}{5px}

\fancyfoot[R]{Olivier Glorieux}
%\fancyfoot[R]{Jules Glorieux}

\fancyfoot[C]{ Page \thepage }
\fancyfoot[L]{1BIOA - Lycée Chaptal}
%\fancyfoot[L]{MP*-Lycée Chaptal}
%\fancyfoot[L]{Famille Lapin}



\newcommand{\Hyp}{\mathbb{H}}
\newcommand{\C}{\mathcal{C}}
\newcommand{\U}{\mathcal{U}}
\newcommand{\R}{\mathbb{R}}
\newcommand{\T}{\mathbb{T}}
\newcommand{\D}{\mathbb{D}}
\newcommand{\N}{\mathbb{N}}
\newcommand{\Z}{\mathbb{Z}}
\newcommand{\F}{\mathcal{F}}




\newcommand{\bA}{\mathbb{A}}
\newcommand{\bB}{\mathbb{B}}
\newcommand{\bC}{\mathbb{C}}
\newcommand{\bD}{\mathbb{D}}
\newcommand{\bE}{\mathbb{E}}
\newcommand{\bF}{\mathbb{F}}
\newcommand{\bG}{\mathbb{G}}
\newcommand{\bH}{\mathbb{H}}
\newcommand{\bI}{\mathbb{I}}
\newcommand{\bJ}{\mathbb{J}}
\newcommand{\bK}{\mathbb{K}}
\newcommand{\bL}{\mathbb{L}}
\newcommand{\bM}{\mathbb{M}}
\newcommand{\bN}{\mathbb{N}}
\newcommand{\bO}{\mathbb{O}}
\newcommand{\bP}{\mathbb{P}}
\newcommand{\bQ}{\mathbb{Q}}
\newcommand{\bR}{\mathbb{R}}
\newcommand{\bS}{\mathbb{S}}
\newcommand{\bT}{\mathbb{T}}
\newcommand{\bU}{\mathbb{U}}
\newcommand{\bV}{\mathbb{V}}
\newcommand{\bW}{\mathbb{W}}
\newcommand{\bX}{\mathbb{X}}
\newcommand{\bY}{\mathbb{Y}}
\newcommand{\bZ}{\mathbb{Z}}



\newcommand{\cA}{\mathcal{A}}
\newcommand{\cB}{\mathcal{B}}
\newcommand{\cC}{\mathcal{C}}
\newcommand{\cD}{\mathcal{D}}
\newcommand{\cE}{\mathcal{E}}
\newcommand{\cF}{\mathcal{F}}
\newcommand{\cG}{\mathcal{G}}
\newcommand{\cH}{\mathcal{H}}
\newcommand{\cI}{\mathcal{I}}
\newcommand{\cJ}{\mathcal{J}}
\newcommand{\cK}{\mathcal{K}}
\newcommand{\cL}{\mathcal{L}}
\newcommand{\cM}{\mathcal{M}}
\newcommand{\cN}{\mathcal{N}}
\newcommand{\cO}{\mathcal{O}}
\newcommand{\cP}{\mathcal{P}}
\newcommand{\cQ}{\mathcal{Q}}
\newcommand{\cR}{\mathcal{R}}
\newcommand{\cS}{\mathcal{S}}
\newcommand{\cT}{\mathcal{T}}
\newcommand{\cU}{\mathcal{U}}
\newcommand{\cV}{\mathcal{V}}
\newcommand{\cW}{\mathcal{W}}
\newcommand{\cX}{\mathcal{X}}
\newcommand{\cY}{\mathcal{Y}}
\newcommand{\cZ}{\mathcal{Z}}







\renewcommand{\phi}{\varphi}
\newcommand{\ddp}{\displaystyle}


\newcommand{\G}{\Gamma}
\newcommand{\g}{\gamma}

\newcommand{\tv}{\rightarrow}
\newcommand{\wt}{\widetilde}
\newcommand{\ssi}{\Leftrightarrow}

\newcommand{\floor}[1]{\left \lfloor #1\right \rfloor}
\newcommand{\rg}{ \mathrm{rg}}
\newcommand{\quadou}{ \quad \text{ ou } \quad}
\newcommand{\quadet}{ \quad \text{ et } \quad}
\newcommand\fillin[1][3cm]{\makebox[#1]{\dotfill}}
\newcommand\cadre[1]{[#1]}
\newcommand{\vsec}{\vspace{0.3cm}}

\DeclareMathOperator{\im}{Im}
\DeclareMathOperator{\cov}{Cov}
\DeclareMathOperator{\vect}{Vect}
\DeclareMathOperator{\Vect}{Vect}
\DeclareMathOperator{\card}{Card}
\DeclareMathOperator{\Card}{Card}
\DeclareMathOperator{\Id}{Id}
\DeclareMathOperator{\PSL}{PSL}
\DeclareMathOperator{\PGL}{PGL}
\DeclareMathOperator{\SL}{SL}
\DeclareMathOperator{\GL}{GL}
\DeclareMathOperator{\SO}{SO}
\DeclareMathOperator{\SU}{SU}
\DeclareMathOperator{\Sp}{Sp}


\DeclareMathOperator{\sh}{sh}
\DeclareMathOperator{\ch}{ch}
\DeclareMathOperator{\argch}{argch}
\DeclareMathOperator{\argsh}{argsh}
\DeclareMathOperator{\imag}{Im}
\DeclareMathOperator{\reel}{Re}



\renewcommand{\Re}{ \mathfrak{Re}}
\renewcommand{\Im}{ \mathfrak{Im}}
\renewcommand{\bar}[1]{ \overline{#1}}
\newcommand{\implique}{\Longrightarrow}
\newcommand{\equivaut}{\Longleftrightarrow}

\renewcommand{\fg}{\fg \,}
\newcommand{\intent}[1]{\llbracket #1\rrbracket }
\newcommand{\cor}[1]{{\color{red} Correction }#1}

\newcommand{\conclusion}[1]{\begin{center} \fbox{#1}\end{center}}


\renewcommand{\title}[1]{\begin{center}
    \begin{tcolorbox}[width=14cm]
    \begin{center}\huge{\textbf{#1 }}
    \end{center}
    \end{tcolorbox}
    \end{center}
    }

    % \renewcommand{\subtitle}[1]{\begin{center}
    % \begin{tcolorbox}[width=10cm]
    % \begin{center}\Large{\textbf{#1 }}
    % \end{center}
    % \end{tcolorbox}
    % \end{center}
    % }

\renewcommand{\thesection}{\Roman{section}} 
\renewcommand{\thesubsection}{\thesection.  \arabic{subsection}}
\renewcommand{\thesubsubsection}{\thesubsection. \alph{subsubsection}} 

\newcommand{\suiteu}{(u_n)_{n\in \N}}
\newcommand{\suitev}{(v_n)_{n\in \N}}
\newcommand{\suite}[1]{(#1_n)_{n\in \N}}
%\newcommand{\suite1}[1]{(#1_n)_{n\in \N}}
\newcommand{\suiteun}[1]{(#1_n)_{n\geq 1}}
\newcommand{\equivalent}[1]{\underset{#1}{\sim}}

\newcommand{\demi}{\frac{1}{2}}
\geometry{hmargin=2.0cm, vmargin=3.5cm}

\author{Olivier Glorieux}
\usetikzlibrary{matrix,arrows,decorations.pathmorphing}

\begin{document}



\title{Chapitre 5.3 - Suites récurrentes $u_{n+1}=f(u_n)$ }






%--------------------------------------------------------------------------------------



Ce type de suite est très souvent étudié en biologie (dynamique des populations, mutation de l'ADN...), en physique (mécanique céleste) en math évidemment, mais aussi en finance,  en météorologie... 

L'étude est assez similaire pour toutes ces suites et nous allons regarder en détail l'exemple suivant : 

Sur l'exemple suivant : 
\begin{exercice} \;
On d\'efinit la suite $\suiteu$ par 
$\left\lbrace\begin{array}{l}
u_0\in\R\vsec\\
u_{n+1}=\ddp\frac{1}{3}u_n^2+\frac{2}{3}
\end{array}\right.$
\begin{enumerate}
\item \'Etudier la fonction $f$ associ\'ee.
\item Montrer que $I_1=[-1,1]$, $I_2= [1,2]$ et $I_3=[2,+\infty[$ sont des intervalles stables par $f$. 
\item \'Etudier le signe de $g: x\mapsto f(x)-x$.
\item Calculer les limites \'eventuelles de la suite $\suiteu$.
\item Que peut-on dire de la suite $\suiteu$ lorsque $u_0=1$ ou $u_0=2$ ?
\item On suppose que $u_0\in\rbrack -1,1\lbrack$.
\begin{enumerate}
\item Montrer que la suite est bien d\'efinie et que pour tout $n\in\N$: $u_n\in\rbrack -1,1\lbrack$.
\item \'Etudier la monotonie de la suite $\suiteu$.
\item \'Etudier le comportement \`{a} l'infini de la suite $\suiteu$.
\end{enumerate}
\item On suppose que $u_1\in[1,2]$.
\begin{enumerate}
\item Montrer que la suite est bien d\'efinie et que pour tout $n\in\N$: $u_n\in [1,2]$.
\item \'Etudier la monotonie de la suite $\suiteu$.
\item \'Etudier le comportement \`{a} l'infini de la suite $\suiteu$.
\end{enumerate} 
\item On suppose que $u_0>2$.
\begin{enumerate}
\item Montrer que $u_n>2$.
\item En d\'eduire le comportement \`{a} l'infini de la suite $\suiteu$.
\end{enumerate} 
\end{enumerate}
\end{exercice}

\warning Les suites sont de la forme $u_{n+1}=f(u_n)$ il n'y a pas $n$ comme variable dans $f$ : par exemple la suite $u_{n+1}= u^2_n +n$ ne rentre pas dans le cadre de cette étude. 

\warning Attention à la confusion entre les suites récurrentes de la forme $u_{n+1}=f(u_n)$ et les suites sous forme $u_n=f(n)$. Ici on s'intéresse au premier cas. Le cas $u_n=f(n)$ est "simple", la suite $\suite{u}$ se comporte comme la fonction $f$ et il suffit d'étudier $f$. 

\newpage

%--------------------------------------------------------------------------------------
%--------------------------------------------------------------------------------------



On d\'efinit la suite $\suiteu$ par 
$\left\lbrace\begin{array}{l}
u_0\in\R\vsec\\
u_{n+1}=\ddp\frac{1}{3}u_n^2+\frac{2}{3}
\end{array}\right.$

\begin{enumerate}
\item \'Etudier la fonction $f$ associ\'ee, c'est-à-dire la fonction $f(x)=\frac{1}{3}x^2+\frac{2}{3}$. \\
\warning Cette étude ne nous dit pas énormément de choses sur $\suite{u}$, le sens de variation de $f$ et celui de $\suite{u}$ \underline{ne sont pas reliés}.
\newpage

\begin{defi}
Soit $f$ une fonction réelle, soit $I$ un intervalle. On dit que $I$ est stable par $f$ si pour tout $x\in I$ on a $f(x)\in I$
\end{defi}
\item Montrer que $I_1=[-1,1]$, $I_2= [1,2]$ et $I_3=[2,+\infty[$ sont des intervalles stables par $f$. 
\vspace{5cm}
\item \'Etudier le signe de $g: x\mapsto f(x)-x$.
\vspace*{-0.5cm}
\paragraph{Remarque} Cette fonction $g$ est particulièrment intéressante, car elle nous permettra de donner le sens de variation de $\suite{u}$. 
\newpage

\item Calculer les limites \'eventuelles de la suite $\suiteu$.
\vspace*{-0.5cm}
\paragraph{Remarque} Question ultra classique. \warning Ici on NE dit PAS que la suite converge, on dit " \underline{SI elle converge} alors sa limite peut valoir ***"  Pour trouver les limites possibles il faut passer à la limite dans l'équation définissant $\suite{u}$. Les limites possibles correspondent alors aux valeurs $\ell$ pour lesquelles $f(\ell) =\ell$ c'est-à-dire $g(\ell)=0$.
\vspace{9cm}


Souvent la valeur de $u_0$ est donnée et  il suffit de faire l'étude dans ce cas. Ici on va voir qu'en fonction de la valeur de $u_0$ il se passe des choses différentes. 
\item Que peut-on dire de la suite $\suiteu$ lorsque $u_0=1$ ou $u_0=2$ ?
\newpage

\item On suppose que $u_0\in\rbrack -1,1\lbrack$.
\paragraph{Remarque} Avec la question 4, les questions suivantes forment le coeur de l'étude des 
\begin{enumerate}
\item Montrer que pour tout $n\in\N$: $u_n\in\rbrack -1,1\lbrack$.
\item \'Etudier la monotonie de la suite $\suiteu$.
\item \'Etudier le comportement \`{a} l'infini de la suite $\suiteu$.
\end{enumerate}
\
\newpage
\item On suppose que $u_1\in[1,2]$.
\begin{enumerate}
\item Montrer que la suite est bien d\'efinie et que pour tout $n\in\N$: $u_n\in [1,2]$.
\item \'Etudier la monotonie de la suite $\suiteu$.
\item \'Etudier le comportement \`{a} l'infini de la suite $\suiteu$.
\end{enumerate} 
\item On suppose que $u_0>2$.
\begin{enumerate}
\item Montrer que $u_n>2$.
\item En d\'eduire le comportement \`{a} l'infini de la suite $\suiteu$.
\end{enumerate} 
\end{enumerate}


\end{document}

\end{document}