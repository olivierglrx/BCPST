\documentclass[a4paper, 11pt,reqno]{article}
\input{macro/package.tex}
\input{macro/environement}
% Header et footer

\pagestyle{fancy}
\fancyhead{}
\fancyfoot{}
\renewcommand{\headwidth}{\textwidth}
\renewcommand{\footrulewidth}{0.4pt}
\renewcommand{\headrulewidth}{0pt}
\renewcommand{\footruleskip}{5px}

\fancyfoot[R]{Olivier Glorieux}
%\fancyfoot[R]{Jules Glorieux}

\fancyfoot[C]{ Page \thepage }
\fancyfoot[L]{1BIOA - Lycée Chaptal}
%\fancyfoot[L]{MP*-Lycée Chaptal}
%\fancyfoot[L]{Famille Lapin}

\input{macro/newcommand.tex}
\geometry{hmargin=2.0cm, vmargin=3.5cm}



\author{Olivier Glorieux}


\begin{document}
\tableofcontents
\title{Chapitre 3 : Entiers, sommes et produits}





\section{Notations $\sum$ et $\prod$ }
\subsection{Définitions}
\begin{defi} 
Soit $n$ un entier naturel et $a_1,a_2,\dots,a_n$ des nombres r\'eels ou complexes.\\
Leur somme est not\'ee  $\ddp \sum\limits_{k=1}^n a_k$
\noindent Cela correspond \`a $\ddp \sum\limits_{k=1}^n a_k=a_1+a_2+\dots +a_n$ 
\end{defi}
%\noindent L'indice de sommation est \dotfill \phantom{\hspace{12cm}}\\
\begin{center}
\fbox{\warning La somme $\ddp \sum\limits_{k=1}^n a_k $ NE DEPEND PAS DE $k$. \warning  }
\end{center}


%\noindent \phantom{\hspace{0cm}}\dotfill\\
On a ainsi $\ddp \sum\limits_{k=1}^n a_k=\ddp \sum\limits_{j=1}^n a_j$


\begin{rem}
\noindent Si le symbole $\ddp\sum$ vous semble complexe dans l'expression $\ddp \sum\limits_{k=1}^n a_k$, ne pas h\'esiter \`a \'ecrire la somme en extension $a_1+a_2+\dots+a_n$ pour mieux la comprendre. 
\end{rem}


{\footnotesize 
\begin{exercice} 
\begin{enumerate}
\item Calculer les quatre sommes suivantes: $\ddp\sum\limits_{k=1}^8 1$, $\ddp\sum\limits_{j=0}^4 \ddp\demi$, $\ddp\sum\limits_{i=1}^n 1$, $\ddp\sum\limits_{k=0}^n (-4)$.
%\item \'Ecrire d'une autre mani\`{e}re $\sum\limits_{k=1}^n a_k+\sum\limits_{k=1}^n b_k$.
\item Comparer $\ddp\sum\limits_{k=2}^5 a_k$ et $\ddp\sum\limits_{k=0}^3 a_{5-k}$.
\item \'Ecrire en extension $\ddp\sum\limits_{k=0}^5 a_{3k+1}$.
\end{enumerate}
\end{exercice}}
 

{\footnotesize 
\begin{exercice} \'Ecrire les sommes suivantes \`{a} l'aide du symbole $\sum$:
\begin{enumerate}
\item $5a_{1}+5a_2+5a_3+5a_4$.
\item $a_n+a_{n+1}+\dots+a_{2n}$.
\item $a_1+a_3+a_5+a_7$.
\item $a_2+a_4+a_6+\dots+a_{2n}$.
\end{enumerate}
\end{exercice}}



\begin{defi} 
Soit $n$ un entier naturel et $a_1,a_2,\dots,a_n$ des nombres r\'eels ou complexes. \\
Leur produit est not\'e \dotfill \phantom{\hspace{8cm}}\\
\noindent Cela correspond \`a $\ddp \prod\limits_{k=1}^n a_k=$\dotfill \phantom{\hspace{5cm}}
\end{defi}


\begin{exemples}
Calculer les produits suivants: $P_1=\ddp\prod\limits_{k=1}^n x$ et $P_2=\ddp\prod\limits_{j=1}^n (2j)$.
\end{exemples}

{\footnotesize 
\begin{exercice} 
Calculer $\ddp P_1=\prod_{k=0}^n 2^k$ et $P_2=\ddp \prod_{k=1}^n \frac{k}{k+1}$.
\end{exercice}}



\subsection{Propriétés}

%--------------------------------------------------
\subsubsection{La lin\'earit\'e de la somme}
%\vspace{0.3cm}
\begin{prop} 
Soient $n$ un entier naturel. 
Soient $a_1,a_2,\dots,a_n,\ b_1,b_2,\dots, b_n$ des r\'eels ou complexes, et soit $\lambda$ un nombre r\'eel ou complexe.
$$\ddp \sum\limits_{k=1}^n (\lambda a_k+b_k)=\hspace*{4cm}$$
\end{prop}





%\setlength\fboxrule{0.5pt}
%\begin{proof} 
%\vspace{2cm}
%\end{proof}


\noindent\warning\, C'est faux avec la multiplication et la division: $\ddp\sum\limits_{k=0}^n a_kb_k \boldmath{\not=} \left(\sum\limits_{k=0}^n a_k\right) \left(\sum\limits_{k=0}^n b_k\right)$ \; et \; 
$\ddp \sum\limits_{k=0}^n \ddp\frac{a_k}{b_k} \boldmath{\not=} \ddp\frac{\sum\limits_{k=0}^n a_k}{\sum\limits_{k=0}^n b_k}\ (b_k\not= 0)$.
%{\footnotesize 
%\begin{exercice} Calculer les sommes suivantes: $\sum\limits_{i=0}^n (1+3^{i})$, $\sum\limits_{j=0}^n \ddp\frac{3}{2^k}$ et $\sum\limits_{k=0}^n (-4+2^{k+2})$.
%\end{exercice}}


\vspace{0.3cm}

%--------------------------------------------------
\subsubsection{La relation de Chasles}



\begin{prop} 
Soient $n$ un entier naturel et $p\in\intent{1,n-1}$.\\
Soient $a_1,a_2,\dots,a_n$ des nombres r\'eels ou complexes.
$$\sum\limits_{k=1}^n a_k=\sum\limits_{k=1}^p a_k +\sum\limits_{k=p+1}^n a_k$$
\end{prop}


%\vspace{0.3cm}

\begin{exemples}
\begin{itemize}
\item[$\bullet$] $\ddp\sum\limits_{j=0}^{n} a_j= a_0+\ddp\sum\limits_{j=1}^{n} a_j $
\item[$\bullet$] $\ddp\sum\limits_{j=0}^{n+1} a_j=\ddp\sum\limits_{j=0}^{n} a_j +a_{n+1}$
\end{itemize}
\end{exemples}

{\footnotesize 
\begin{exercice} 
\begin{itemize}
\item[$\bullet$] \'Ecrire avec une seule somme: $\ddp\sum\limits_{j=1}^n b_j+\sum\limits_{j=n+1}^{2n} b_j = $ %et $b_{n+1}+\sum\limits_{i=1}^n b_i$.
\item[$\bullet$] Simplifier les sommes suivantes : $\ddp\sum\limits_{k=1}^{n} \ddp\frac{1}{k}-\sum\limits_{k=2}^{n+1} \ddp\frac{1}{k}$ \; et \; $\ddp\sum\limits_{i=0}^{n+2} a_i+\sum\limits_{j=1}^{n+1} a_j -2\sum\limits_{k=2}^{n+1} a_k$.
\end{itemize}
\end{exercice}}
\vspace{0.3cm}


%--------------------------------------------------
\subsubsection{Le changement d'indice}


\begin{exemples}
\begin{itemize}
\item[$\bullet$] Transformation de $\ddp\sum\limits_{i=0}^{n} a_{i+1}$:\\
\vspace{1cm}


\item[$\bullet$] Transformation de $\ddp \sum\limits_{i=1}^{n} a_{i-1}:$\\
\vspace{1cm}

\end{itemize}
\end{exemples}



%\colorbox{gristrespale}{

On pose $j=i+1,\ i+2,\ i-1,\ i-2,\dots$.\\ On doit alors:
\begin{itemize}
\item[$\bullet$] Regarder le nouvel ensemble de sommation
\item[$\bullet$]  Transformer $i$ en $j-1 $, $j-2$ $j+1$, $j+2$
\item[$\bullet$] Changer les indices dans toute la somme. 
\end{itemize}


%}



{\footnotesize 
\begin{exercice}
On pose : $\ddp S = \sum\limits_{k=1}^n k$ et $\ddp T = \sum\limits_{k=1}^n (n-k+1)$. Montrer que $S=T$, puis calculer $S+T$. En d\'eduire la valeur de $S$.
\end{exercice}}


\section{Sommes usuelles}

\subsection{Formalisme des récurrences}





\begin{theorem}[Axiome de récurrence]
Si $P(n)$ est une proposition telle que : 
\begin{itemize}
\item $P(0)$ est vraie, (ou plus généralement $P(n_0)$ est vraie pour un certain entier $n_0\in \N$) 
\item $\forall n \in \N, \, P(n) \Rightarrow P(n+1)$, 
\end{itemize}
Alors $P(n)$ est vraie pour tout $n \in \N$. (ou plus généralement $P(n)$ est vraie pour tout entier $n\geq n_0$)
\end{theorem}




\paragraph{Conseil rédactionnel \faHeart :\\}
\setlength\fboxrule{1pt}
\noindent {
\begin{minipage}[t]{0.9\textwidth}
\begin{itemize}
\item[$\bullet$] \textbf{On d\'efinit clairement la propri\'et\'e \`a d\'emontrer:}  \\
Montrons par r\'ecurrence sur l'entier $n\geq n_0$, la propri\'et\'e $\mathcal{P}(n):\ .....$.
\item[$\bullet$] \textbf{Initialisation:} pour $n=n_0$:\\
On v\'erifie que $\mathcal{P}(n_0)$ est vraie.
\item[$\bullet$] \textbf{H\'er\'edit\'e:}\\
Soit $n\geq n_0$ fix\'e. On suppose la propri\'et\'e vraie \`a l'ordre $n$. Montrons qu'alors $\mathcal{P}(n+1)$ est vraie.\\
%\includegraphics[scale=0.08]{../../Fichiers/attention.eps} 
\warning N'oubliez pas de signaler l'endroit o\`u vous utilisez l'hypoth\`ese de r\'ecurrence.
\item[$\bullet$] \textbf{Conclusion:}\\
Il r\'esulte du principe de r\'ecurrence que pour tout $ n\geq n_0,\ \mathcal{P}(n)$.\\
\end{itemize}
\end{minipage}}
\setlength\fboxrule{0.5pt}
\begin{exo} \faHeartO \, 
D\'emontrer que pour tout $n\in\bN:\  \sum\limits_{k=0}^n k^2=\frac{n(n+1)(2n+1)}{6}$.
%\vspace{7cm}
\end{exo}
\begin{proof}
Montrons par r\'ecurrence sur l'entier $n\geq 0$, la propri\'et\'e $\mathcal{P}(n):\sum_{k=0}^n k^2=\frac{n(n+1)(2n+1)}{6}$
\noindent
\textbf{Initialisation:}  Pour $n=0$ on a d'une part :$\sum_{k=0}^0 k^2 =0^2=0$. D'autre part on a : $\frac{0(0+1)(2*0+1)}{6}=0$. La propriété $P$ est donc vraie au rang $0$. 

 \textbf{H\'er\'edit\'e:}\\
Soit $n\geq 0$ fix\'e. On suppose la propri\'et\'e vraie \`a l'ordre $n$. Montrons qu'alors $\mathcal{P}(n+1)$ est vraie.\\

Considérons le membre de gauche de l'égalité de $P(n+1)$, on a : 
\begin{equation*}
\sum_{k=0}^{n+1} k^2 = \sum_{k=0}^{n+1} k^2  = \sum_{k=0}^{n} k^2  +(n+1)^2
\end{equation*}
Par hypothése de récurrence on a $ \sum_{k=0}^{n} k^2  = \frac{n(n+1)(2n+1)}{6}$, donc 
$$\sum_{k=0}^{n+1} k^2 = \frac{n(n+1)(2n+1)}{6}+(n+1)^2.$$
Mettons $(n+1)$ en facteur. On obtient : 
\begin{align*}
\sum_{k=0}^{n+1} k^2 & = \frac{(n+1)[n(2n+1)+6(n+1)}{6}.\\
								& = \frac{(n+1)[2n^2+7n+6]}{6}.
\end{align*}
Or $(2n^2+7n+6)=(2n+3)(n+2)=(2(n+1)+1) ( (n+1)+1)$, donc :
		$$\sum_{k=0}^{n+1} k^2  =\frac{(n+1)((n+1)+1) (2(n+1)+1)			}{6}$$
La propriété $P$ est donc vraie au rang $n+1$.

\textbf{Conclusion:}\\
Il r\'esulte du principe de r\'ecurrence que pour tout $ n\geq 0$:
\begin{center}
\fbox{$\sum_{k=0}^n k^2=\frac{n(n+1)(2n+1)}{6}.$}
\end{center}






\end{proof}


\subsection{$\sum_{k=0}^n k^p$ }

\begin{prop}
$$\sum_{k=0}^n 1= n+1$$
$$\sum_{k=0}^n k^1=\frac{n(n+1)}{2}$$
$$\sum_{k=0}^n k^2= \frac{n(n+1)(2n+1)}{6}$$
$$\sum_{k=0}^n k^3=\left(\frac{n(n+1)}{2}\right)^2$$
\end{prop}
Plus généralement on a 
$$\sum_{k=m}^n k^1=\frac{(n+m)(n-m+1)}{2}$$

Exemple 
Calculer les sommes suivantes : 
$\sum_{k=10}^{100} k$, $\sum_{k=10}^{100} k^2$,  $\sum_{k=1}^{2^n} k^3$


\subsection{Somme géométrique}
\begin{prop}
Soit $q\neq 1$ on a alors : 

$$\sum_{k=0}^n q^k =\frac{1-q^{n+1}}{1-q}$$
\end{prop}


\subsection{Binome de Newton}



\subsubsection{Coefficients binomiaux: d\'efinition et propri\'et\'es}






\begin{defi} Soit $(n,p)$ deux entiers naturels.
\begin{itemize}
\item[$\bullet$] Si $p\in \intent{0,n}$, on appelle coefficient binomial, le nombre
$$\binom{n}{p}=\frac{n! }{p! (n-p)!}$$
\item[$\bullet$] Par convention si $p\notin \intent{0,n}$: $\ddp\binom{n}{p}=0$
\end{itemize}
\end{defi}

\vspace{0.3cm}

\begin{exemples}
\begin{itemize}
\item[$\bullet$] $\ddp\binom{n}{0}=\frac{n! }{0! (n)!}=1$, $\ddp\binom{n}{1}=\frac{n! }{1! (n-1)!}=n$, $\ddp\binom{n}{2}=\frac{n! }{2! (n-2)!}=\frac{n(n-1)}{2}$\\
\vspace{1cm}
\item[$\bullet$] $\ddp\binom{n}{n}=\frac{n! }{n! (n-n)!}=1$, \quad\quad\quad $\ddp\binom{n}{n-1}=\frac{n! }{(n-1)! (n-(n-1))!}=n$, \\
\item []
\vspace{1cm}
$\ddp\binom{n}{n-2}=\frac{n! }{(n-2)! (n-(n-2))!}=\frac{n(n-1)}{2} $
\end{itemize}
\end{exemples}
\vspace{0.4cm}

\begin{rem}
Les coefficients binomiaux seront tr\`es utiles en d\'enombrement : $\ddp \binom{n}{p}$ est le nombre de tirages simultan\'es (sans ordre et sans r\'ep\'etition) de $p$ boules parmi $n$.
\end{rem}




\begin{prop} Soit $(n,k)\in\N^2$.
\begin{itemize}
\item[$\bullet$] Sym\'etrie des coefficients binomiaux: 
$$\binom{n}{k}=\binom{n}{n-k}$$
\item[$\bullet$] Triangle de Pascal:\\
$$\binom{n}{k}+\binom{n}{k-1}=\binom{n+1}{k}$$
%\item[$\bullet$] $\ddp k \binom{n}{k} = $\\
\end{itemize}
\end{prop}

\begin{exercice}
Faire la preuve de la proposition (pas besoin de récurrence)
\end{exercice}


\begin{cor}
$$\binom{n}{k}=\frac{n! }{k! (n-k)!}$$
et 

$$\binom{n}{n-k}=\frac{n! }{(n-k)! (n-(n-k))!}=\frac{n! }{k! (n-k)!}$$



\begin{align*}
\binom{n}{k} +\binom{n}{k-1} &= \frac{n!}{(n-k)!\,k!}+ \frac{n!}{(n-(k-1))!\,(k-1)!}\\
&= \frac{n!}{(n-k)!\,k!}+ \frac{n!}{(n-k+1))!\,(k-1)!}\\
&= \frac{(n-k+1)n! + k  n!}{(n-k+1)!\,k!} \quad \text{ on met au même dénominateur} \\
&= \frac{(n+1)n!}{(n-k+1)!\,k!} \\
&= \frac{(n+1)!}{(n+1-k)!\,k!} \\
&=\binom{n+1}{k}
\end{align*}
\end{cor}



\begin{prop}
Soit $k\in \intent{0,n}$ :
$$k\binom{n}{k}= n\binom{n-1}{k-1}$$
En conséquence 
$$\binom{n}{k}= \frac{n}{k}\binom{n-1}{k-1}$$
\end{prop}
%
%\begin{cor}
% $$\ddp\binom{n}{k}\binom{k}{j} =\frac{n! }{k! (n-k)!} \frac{k! }{j! (k-j)!}=\frac{n! }{ (n-k)!} \frac{1 }{(k-j)!j!} $$
% 
% et 
% $$\ddp\binom{n}{j}\ddp\binom{n-j}{n-k} = \frac{n! }{j! (n-j)!} \frac{(n-j)! }{(n-j-(n-k)! (n-k)!}= \frac{n! }{j!} \frac{1 }{(k-j)! (n-k)!}$$
%\end{cor}



\begin{exercice} Soit $(n,p,k,j)\in\N^4$ avec $k\in\intent{0,n}$ et $j\in\intent{0,k}$. Montrer que $\ddp\binom{n}{k}\binom{k}{j}=\ddp\binom{n}{j}\ddp\binom{n-j}{n-k}$.
\end{exercice}

\subsubsection{Binôme de Newton}

\begin{theorem}[Binôme de Newton]

\noindent\\
\doublebox{\hspace{0.5cm}
$\ddp\forall (a,b)\in \bC^2,\,  \forall n \in \N, \quad  (a+b)^n =\sum_{k=0}^n \binom{n}{k}a^k b^{n-k}.$
\hspace{0.5cm}
}

\end{theorem}









%-----------------------------------------------------------
%----------------------------------------------------------
%-----------------------------------------------------------
%----------------------------------------------------------
%-----------------------------------------------------------
%----------------------------------------------------------
%-----------------------------------------------------------
%----------------------------------------------------------
%-----------------------------------------------------------
%----------------------------------------------------------
%-----------------------------------------------------------
%----------------------------------------------------------



\section{Approfondissement}
\subsection{Récurrence double}

\subsection{Récurrence forte}
Ce type de raisonnement s'utilise quand on a une propriété $P(n)$ qui dépend de tous les $k\leq n$.

\begin{exercice}
Soit $\suite{u}$ la suite définie par 
$$u_0=1 \quadet \forall n\in \N, u_{n+1} =\sum_{k=0}^n u_k$$
Montrer que $\forall n \in \N, u_n\leq 2^n$
\end{exercice}


\begin{exercice}
Soit $\suite{u}$ la suite définie par 
$$u_1=3 \quadet \forall n\in \N, u_{n+1} =\frac{2}{n}\sum_{k=0}^n u_k$$
Montrer que $\forall n \in \N, u_n=3$
\end{exercice}
\subsection{Somme double}



%------------------------------------------------
\subsubsection{D\'efinition et notations}


\noindent On souhaite calculer des sommes dont les termes d\'ependent de deux indices, par exemple : $\ddp \sum_{i=0}^n \sum_{j=0}^p a_{i,j}$.

\begin{exemples}
Calculer les sommes doubles suivantes: $S_1=\ddp\sum\limits_{i=0}^n\sum\limits_{j=0}^n (i+j)$ et $S_2=\ddp \sum\limits_{i=0}^n\sum\limits_{j=0}^i ij$.

\end{exemples}


\begin{rem}
On pourra noter ces sommes doubles \`a l'aide d'un seul symbole $\ddp \sum$, en indiquant en dessous comment varient les deux indices. Par exemple, on a : $S_1=\ddp\sum\limits_{i=0}^n\sum\limits_{j=0}^n (i+j) = \dotfill$ \\
et $S_2=\ddp \sum\limits_{i=0}^n\sum\limits_{j=0}^i ij = \dotfill$\\
De plus, il est toujours conseill\'e de repr\'esenter le domaine des indices pour lequel on effectue la somme.
\end{rem}

\vspace*{0.5cm}

%--------------------------------------------------
%------------------------------------------------
\subsubsection{M\'ethode directe}



\begin{minipage}[t]{0.6\textwidth}
\setlength\fboxrule{1pt}
\noindent {
\begin{minipage}[t]{1\textwidth}
Calcul de $\ddp \sum\limits_{i}\sum\limits_{j} a_{ij}=\sum\limits_{i} \left\lbrack \sum\limits_{j} a_{ij}\right\rbrack:$\phantom{4cm}
\begin{itemize}
\item[$\bullet$] On calcule d'abord la somme la plus intérieure (ici celle d'indice $j$) qui dépond ou non de $i$

\item[$\bullet$] On calcule la deuxième somme. 
\end{itemize}
\end{minipage}}
\end{minipage}



{\footnotesize 
\begin{exercice} 
Calculer les sommes suivantes $S_1= \ddp \sum\limits_{i=0}^n \sum\limits_{j=0}^i x^j$, $S_2=\ddp \sum\limits_{k=0}^{n^2}\sum\limits_{i=k}^{k+2} ki^2$ et $S_3=\ddp\sum\limits_{j=1}^n\sum\limits_{i=0}^j \ddp\frac{x^i}{x^{j}}$.
\end{exercice}}


%--------------------------------------------------
%------------------------------------------------
\subsubsection{Inversion des symboles sommes}

\noindent Lorsque l'on n'arrive pas \`{a} calculer la somme la plus int\'erieure directement, on commence par inverser le sens des symboles somme. Il faut alors distinguer le cas d'indices li\'es ou non li\'es.





\begin{enumerate}
\item Indices non li\'es : les indices ne d\'ependent pas les uns des autres.
On inverse sans pr\'ecaution:\\
$$\sum_{i=0}^n \sum_{j=0}^{n'}  a_{i,j} = \sum_{j=0}^{n'} \sum_{i=0}^n  a_{i,j} $$


\item Indices li\'es : les bornes d'un indice d\'ependent des autres. On fait très attention : 

$$\ddp \sum\limits_{i=0}^n\sum\limits_{j=0}^i a_{ij}= $$
Car: $\left\lbrace \begin{array}{lllll} 0&\leq &i & \leq & n\\  0&\leq &j & \leq & i \end{array}\right. \Longleftrightarrow $


$$\ddp \sum\limits_{i=0}^n\sum\limits_{j=i}^n a_{ij}=$$
Car: $\left\lbrace \begin{array}{lllll} 0&\leq &i & \leq & n\\  i&\leq &j & \leq & n \end{array}\right. \Longleftrightarrow$


\end{enumerate}






\begin{rem}
Dessiner le domaine des indices pour lequel on effectue la somme, ou repr\'esenter les positions relatives des indices sur un axe r\'eel si l'\'echange d'indices li\'es vous para\^it difficile.
\end{rem}



\begin{exemples}
Calculer les sommes doubles suivantes: $S_1= \ddp \sum\limits_{i=1}^n\sum\limits_{j=i}^n \ddp\frac{i}{j}$ et $S_2= \ddp \sum\limits_{j=1}^n\sum\limits_{i=1}^j x^{j}$.
\end{exemples}

{\footnotesize 
\begin{exercice} 
V\'erifier que $\ddp \sum\limits_{k=1}^n k2^k=\sum\limits_{k=1}^n \sum\limits_{l=1}^k 2^k$. Donner alors une expression simple de cette somme en intervertissant l'ordre de sommation.
\end{exercice}}
%------------------------------------------------
%-------------------------------------------------
%-------------------------------------------------
%--------------------------------------------------
%------------------------------------------------



\end{document}
