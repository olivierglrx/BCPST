\documentclass[a4paper, 11pt,reqno]{article}
\usepackage[utf8]{inputenc}
\usepackage{amssymb,amsmath,amsthm}
\usepackage{geometry}
\usepackage[T1]{fontenc}
\usepackage[french]{babel}
\usepackage{fontawesome}
\usepackage{pifont}
\usepackage{tcolorbox}
\usepackage{fancybox}
\usepackage{bbold}
\usepackage{tkz-tab}
\usepackage{tikz}
\usepackage{fancyhdr}
\usepackage{sectsty}
\usepackage[framemethod=TikZ]{mdframed}
\usepackage{stackengine}
\usepackage{scalerel}
\usepackage{xcolor}
\usepackage{hyperref}
\usepackage{listings}
\usepackage{enumitem}
\usepackage{stmaryrd} 
\usepackage{comment}


\hypersetup{
    colorlinks=true,
    urlcolor=blue,
    linkcolor=blue,
    breaklinks=true
}





\theoremstyle{definition}
\newtheorem{probleme}{Problème}
\theoremstyle{definition}


%%%%% box environement 
\newenvironment{fminipage}%
     {\begin{Sbox}\begin{minipage}}%
     {\end{minipage}\end{Sbox}\fbox{\TheSbox}}

\newenvironment{dboxminipage}%
     {\begin{Sbox}\begin{minipage}}%
     {\end{minipage}\end{Sbox}\doublebox{\TheSbox}}


%\fancyhead[R]{Chapitre 1 : Nombres}


\newenvironment{remarques}{ 
\paragraph{Remarques :}
	\begin{list}{$\bullet$}{}
}{
	\end{list}
}




\newtcolorbox{tcbdoublebox}[1][]{%
  sharp corners,
  colback=white,
  fontupper={\setlength{\parindent}{20pt}},
  #1
}







%Section
% \pretocmd{\section}{%
%   \ifnum\value{section}=0 \else\clearpage\fi
% }{}{}



\sectionfont{\normalfont\Large \bfseries \underline }
\subsectionfont{\normalfont\Large\itshape\underline}
\subsubsectionfont{\normalfont\large\itshape\underline}



%% Format théoreme, defintion, proposition.. 
\newmdtheoremenv[roundcorner = 5px,
leftmargin=15px,
rightmargin=30px,
innertopmargin=0px,
nobreak=true
]{theorem}{Théorème}

\newmdtheoremenv[roundcorner = 5px,
leftmargin=15px,
rightmargin=30px,
innertopmargin=0px,
]{theorem_break}[theorem]{Théorème}

\newmdtheoremenv[roundcorner = 5px,
leftmargin=15px,
rightmargin=30px,
innertopmargin=0px,
nobreak=true
]{corollaire}[theorem]{Corollaire}
\newcounter{defiCounter}
\usepackage{mdframed}
\newmdtheoremenv[%
roundcorner=5px,
innertopmargin=0px,
leftmargin=15px,
rightmargin=30px,
nobreak=true
]{defi}[defiCounter]{Définition}

\newmdtheoremenv[roundcorner = 5px,
leftmargin=15px,
rightmargin=30px,
innertopmargin=0px,
nobreak=true
]{prop}[theorem]{Proposition}

\newmdtheoremenv[roundcorner = 5px,
leftmargin=15px,
rightmargin=30px,
innertopmargin=0px,
]{prop_break}[theorem]{Proposition}

\newmdtheoremenv[roundcorner = 5px,
leftmargin=15px,
rightmargin=30px,
innertopmargin=0px,
nobreak=true
]{regles}[theorem]{Règles de calculs}


\newtheorem*{exemples}{Exemples}
\newtheorem{exemple}{Exemple}
\newtheorem*{rem}{Remarque}
\newtheorem*{rems}{Remarques}
% Warning sign

\newcommand\warning[1][4ex]{%
  \renewcommand\stacktype{L}%
  \scaleto{\stackon[1.3pt]{\color{red}$\triangle$}{\tiny\bfseries !}}{#1}%
}


\newtheorem{exo}{Exercice}
\newcounter{ExoCounter}
\newtheorem{exercice}[ExoCounter]{Exercice}

\newcounter{counterCorrection}
\newtheorem{correction}[counterCorrection]{\color{red}{Correction}}


\theoremstyle{definition}

%\newtheorem{prop}[theorem]{Proposition}
%\newtheorem{\defi}[1]{
%\begin{tcolorbox}[width=14cm]
%#1
%\end{tcolorbox}
%}


%--------------------------------------- 
% Document
%--------------------------------------- 






\lstset{numbers=left, numberstyle=\tiny, stepnumber=1, numbersep=5pt}




% Header et footer

\pagestyle{fancy}
\fancyhead{}
\fancyfoot{}
\renewcommand{\headwidth}{\textwidth}
\renewcommand{\footrulewidth}{0.4pt}
\renewcommand{\headrulewidth}{0pt}
\renewcommand{\footruleskip}{5px}

\fancyfoot[R]{Olivier Glorieux}
%\fancyfoot[R]{Jules Glorieux}

\fancyfoot[C]{ Page \thepage }
\fancyfoot[L]{1BIOA - Lycée Chaptal}
%\fancyfoot[L]{MP*-Lycée Chaptal}
%\fancyfoot[L]{Famille Lapin}



\newcommand{\Hyp}{\mathbb{H}}
\newcommand{\C}{\mathcal{C}}
\newcommand{\U}{\mathcal{U}}
\newcommand{\R}{\mathbb{R}}
\newcommand{\T}{\mathbb{T}}
\newcommand{\D}{\mathbb{D}}
\newcommand{\N}{\mathbb{N}}
\newcommand{\Z}{\mathbb{Z}}
\newcommand{\F}{\mathcal{F}}




\newcommand{\bA}{\mathbb{A}}
\newcommand{\bB}{\mathbb{B}}
\newcommand{\bC}{\mathbb{C}}
\newcommand{\bD}{\mathbb{D}}
\newcommand{\bE}{\mathbb{E}}
\newcommand{\bF}{\mathbb{F}}
\newcommand{\bG}{\mathbb{G}}
\newcommand{\bH}{\mathbb{H}}
\newcommand{\bI}{\mathbb{I}}
\newcommand{\bJ}{\mathbb{J}}
\newcommand{\bK}{\mathbb{K}}
\newcommand{\bL}{\mathbb{L}}
\newcommand{\bM}{\mathbb{M}}
\newcommand{\bN}{\mathbb{N}}
\newcommand{\bO}{\mathbb{O}}
\newcommand{\bP}{\mathbb{P}}
\newcommand{\bQ}{\mathbb{Q}}
\newcommand{\bR}{\mathbb{R}}
\newcommand{\bS}{\mathbb{S}}
\newcommand{\bT}{\mathbb{T}}
\newcommand{\bU}{\mathbb{U}}
\newcommand{\bV}{\mathbb{V}}
\newcommand{\bW}{\mathbb{W}}
\newcommand{\bX}{\mathbb{X}}
\newcommand{\bY}{\mathbb{Y}}
\newcommand{\bZ}{\mathbb{Z}}



\newcommand{\cA}{\mathcal{A}}
\newcommand{\cB}{\mathcal{B}}
\newcommand{\cC}{\mathcal{C}}
\newcommand{\cD}{\mathcal{D}}
\newcommand{\cE}{\mathcal{E}}
\newcommand{\cF}{\mathcal{F}}
\newcommand{\cG}{\mathcal{G}}
\newcommand{\cH}{\mathcal{H}}
\newcommand{\cI}{\mathcal{I}}
\newcommand{\cJ}{\mathcal{J}}
\newcommand{\cK}{\mathcal{K}}
\newcommand{\cL}{\mathcal{L}}
\newcommand{\cM}{\mathcal{M}}
\newcommand{\cN}{\mathcal{N}}
\newcommand{\cO}{\mathcal{O}}
\newcommand{\cP}{\mathcal{P}}
\newcommand{\cQ}{\mathcal{Q}}
\newcommand{\cR}{\mathcal{R}}
\newcommand{\cS}{\mathcal{S}}
\newcommand{\cT}{\mathcal{T}}
\newcommand{\cU}{\mathcal{U}}
\newcommand{\cV}{\mathcal{V}}
\newcommand{\cW}{\mathcal{W}}
\newcommand{\cX}{\mathcal{X}}
\newcommand{\cY}{\mathcal{Y}}
\newcommand{\cZ}{\mathcal{Z}}







\renewcommand{\phi}{\varphi}
\newcommand{\ddp}{\displaystyle}


\newcommand{\G}{\Gamma}
\newcommand{\g}{\gamma}

\newcommand{\tv}{\rightarrow}
\newcommand{\wt}{\widetilde}
\newcommand{\ssi}{\Leftrightarrow}

\newcommand{\floor}[1]{\left \lfloor #1\right \rfloor}
\newcommand{\rg}{ \mathrm{rg}}
\newcommand{\quadou}{ \quad \text{ ou } \quad}
\newcommand{\quadet}{ \quad \text{ et } \quad}
\newcommand\fillin[1][3cm]{\makebox[#1]{\dotfill}}
\newcommand\cadre[1]{[#1]}
\newcommand{\vsec}{\vspace{0.3cm}}

\DeclareMathOperator{\im}{Im}
\DeclareMathOperator{\cov}{Cov}
\DeclareMathOperator{\vect}{Vect}
\DeclareMathOperator{\Vect}{Vect}
\DeclareMathOperator{\card}{Card}
\DeclareMathOperator{\Card}{Card}
\DeclareMathOperator{\Id}{Id}
\DeclareMathOperator{\PSL}{PSL}
\DeclareMathOperator{\PGL}{PGL}
\DeclareMathOperator{\SL}{SL}
\DeclareMathOperator{\GL}{GL}
\DeclareMathOperator{\SO}{SO}
\DeclareMathOperator{\SU}{SU}
\DeclareMathOperator{\Sp}{Sp}


\DeclareMathOperator{\sh}{sh}
\DeclareMathOperator{\ch}{ch}
\DeclareMathOperator{\argch}{argch}
\DeclareMathOperator{\argsh}{argsh}
\DeclareMathOperator{\imag}{Im}
\DeclareMathOperator{\reel}{Re}



\renewcommand{\Re}{ \mathfrak{Re}}
\renewcommand{\Im}{ \mathfrak{Im}}
\renewcommand{\bar}[1]{ \overline{#1}}
\newcommand{\implique}{\Longrightarrow}
\newcommand{\equivaut}{\Longleftrightarrow}

\renewcommand{\fg}{\fg \,}
\newcommand{\intent}[1]{\llbracket #1\rrbracket }
\newcommand{\cor}[1]{{\color{red} Correction }#1}

\newcommand{\conclusion}[1]{\begin{center} \fbox{#1}\end{center}}


\renewcommand{\title}[1]{\begin{center}
    \begin{tcolorbox}[width=14cm]
    \begin{center}\huge{\textbf{#1 }}
    \end{center}
    \end{tcolorbox}
    \end{center}
    }

    % \renewcommand{\subtitle}[1]{\begin{center}
    % \begin{tcolorbox}[width=10cm]
    % \begin{center}\Large{\textbf{#1 }}
    % \end{center}
    % \end{tcolorbox}
    % \end{center}
    % }

\renewcommand{\thesection}{\Roman{section}} 
\renewcommand{\thesubsection}{\thesection.  \arabic{subsection}}
\renewcommand{\thesubsubsection}{\thesubsection. \alph{subsubsection}} 

\newcommand{\suiteu}{(u_n)_{n\in \N}}
\newcommand{\suitev}{(v_n)_{n\in \N}}
\newcommand{\suite}[1]{(#1_n)_{n\in \N}}
%\newcommand{\suite1}[1]{(#1_n)_{n\in \N}}
\newcommand{\suiteun}[1]{(#1_n)_{n\geq 1}}
\newcommand{\equivalent}[1]{\underset{#1}{\sim}}

\newcommand{\demi}{\frac{1}{2}}
\geometry{hmargin=2.0cm, vmargin=1.5cm}

\newcommand{\type}{TD }
\excludecomment{correction}
%\renewcommand{\type}{Correction TD }


\begin{document}

\title{\type  3 - Sommes, produits et récurrences}

%--------------------------

%-----------------------------------------------------------------------------------------------
\section*{Entraînements}


\vspace{0.2cm}

%------------------------------------------------
%-------------------------------------------------
\begin{exercice}  \;
\noindent Soit $n\in\N,\ n\geq 3$. Simplifier les nombres suivants:
$$A=\ddp\frac{7!}{6!},\ \ \ \ B=\ddp\frac{3\times 4!}{(3!)^2},\ \ \ \ 
C=\ddp\frac{n!}{(n-1)!},\ \ \ \ D=\ddp\frac{(n+1)!}{(n-3)!}\ \ \hbox{et}\ \ E=\ddp\frac{(n+1)!}{(n-2)!}+\ddp\frac{n!}{(n-1)!}.$$
\end{exercice}
\vspace{0.4cm} 


%------------------------------------------------
%-------------------------------------------------
%---------------------------------------------------
%-------------------------------------------------

%-------------------------------------------------
\begin{correction}  \; \vsec
\begin{itemize}
\item[$\bullet$] $A=\ddp\frac{7!}{6!}=\ddp\frac{7\times 6!}{6!}=\fbox{$7.$}$
\item[$\bullet$] $B=\ddp\frac{3\times 4!}{(3!)^2}=\ddp\frac{3\times 4\times 3!}{3!\times 3\times 2}=\fbox{$2$.}$
\item[$\bullet$] $C=\ddp\frac{n!}{(n-1)!}=\ddp\frac{n\times (n-1)!}{(n-1)!}=\fbox{$n$.}$
\item[$\bullet$] $D=\ddp\frac{(n+1)!}{(n-3)!}=\ddp\frac{(n+1)\times n\times (n-1)\times (n-2)\times (n-3)!}{(n-3)!}=\fbox{$(n+1) n (n-1) (n-2)$.}$
\item[$\bullet$] $E=\ddp\frac{(n+1)!}{(n-2)!}+\ddp\frac{n!}{(n-1)!}=(n+1)n(n-1)+n=n(n^2-1+1)=\fbox{$n^3$.}$
\end{itemize}
\end{correction}
\vspace{0.4cm}





%-------------------------------------------------
%--------------------------------------------------
%----------------------------------------------------------------------------------------------
%-----------------------------------------------------------------------------------------------
%\section{Calculs de sommes}
\vspace{0.2cm}

%------------------------------------------------
%-------------------------------------------------
\begin{exercice} \;
Soit $n\in\N^{\star}$. Calculer les expressions suivantes:\\
\begin{enumerate}
\begin{minipage}[t]{0.3\textwidth}
\item $\ddp \sum\limits_{k=0}^{n} x^{2k}$ \; et \; $\ddp \sum\limits_{k=0}^{n} x^{2k+1}$
\item $\ddp \sum\limits_{k=0}^{n} a^k 2^{3k} x^{-k}$ avec $x\not= 0$
\item $\ddp \sum\limits_{i=0}^{n} (i^2+n+3)$
%\item $\ddp \sum\limits_{j=8}^{21}\ddp\frac{2j-5}{6}$
\item $\ddp \sum\limits_{i=1}^{n} (2i-1)^3$
\end{minipage}
\begin{minipage}[t]{0.25\textwidth}
%\item $\ddp \sum\limits_{k=0}^{n} \ddp\frac{p}{q+1}$
\item $\ddp \sum\limits_{k=2}^{n^2} (1-a^2)^{2k+1}$
\item $\ddp \sum\limits_{k=1}^n (3\times 2^k+1) $ 
\item $\ddp\frac{1}{n}\ddp \sum\limits_{k=0}^{n-1} \exp{\left(\ddp\frac{k}{n}\right)} $ 
\item $\ddp \sum\limits_{k=0}^n (2k-1+2^k) $ 
%\item $\ddp \sum\limits_{k=1}^n 2^{2k+1} $ 
%\item $\ddp \sum\limits_{i=0}^n 3(i+1)i$
\end{minipage}
\begin{minipage}[t]{0.35\textwidth}
\item $\ddp \sum\limits_{j=0}^n\ddp \binom{n}{j}a^j$ \; et \;  $\ddp \sum\limits_{j=1}^{n+1}\ddp \binom{n}{j}a^j$
\item $\ddp \sum\limits_{i=0}^{n}\ddp \binom{n}{i}(-1)^{i}$ \; et \; $\ddp \sum\limits_{i=1}^{n}\ddp \binom{n+1}{i}(-1)^{i}$
\item $\ddp \sum\limits_{j=0}^n\ddp \binom{n}{j}  \ddp\frac{(-1)^{j-1}}{2^{j+1}}$
\item $\ddp \sum\limits_{k=0}^{n-1} \ddp\frac{1}{3^k}\ddp \binom{n}{k}$
%\item $\ddp \sum\limits_{k=0}^n \left(   3k-4+5k^2-(-1)^{k+4}3^{2k-1}+\binom{n}{k} (-2)^{k+1}\ddp\frac{1}{3^{k+2}} \right)$
\end{minipage}
\end{enumerate}
\end{exercice}
%------------------------------------------------
\begin{correction}   \;
\begin{enumerate}
\item \textbf{Calcul de $\mathbf{\ddp \sum\limits_{k=0}^{n} x^{2k}}$:}\\
\noindent On reconna\^{i}t la somme des termes d'une suite g\'eom\'etrique et on obtient donc en utilisant le fait que $x^{2k}=(x^2)^k$:\\
\noindent  
\begin{center}
$\ddp \sum\limits_{k=0}^{n} x^{2k}=\ddp \sum\limits_{k=0}^{n} (x^2)^{k}=\fbox{$\left\lbrace \begin{array}{ll}  \ddp\frac{1-x^{2n+2}}{1-x^2} & \hbox{si}\ x\not= 1\ \hbox{et}\ x\not=-1\vsec\\ n+1 & \hbox{si}\ x= 1\ \hbox{ou}\ x=-1.   \end{array}\right.$}$\\
\end{center}

\textbf{Calcul de $\mathbf{\ddp \sum\limits_{k=0}^{n} x^{2k+1}}$:}\\
\noindent On reconna\^{i}t la somme des termes d'une suite g\'eom\'etrique et on obtient donc en utilisant le fait que $x^{2k+1}=(x^2)^k\times x$:\\
\noindent \begin{center}
$\ddp \sum\limits_{k=0}^{n} x^{2k+1}=x\ddp \sum\limits_{k=0}^{n} (x^2)^{k}=\fbox{$\left\lbrace \begin{array}{ll}  x\times \ddp\frac{1-x^{2n+2}}{1-x^2} & \hbox{si}\ x\not= 1\ \hbox{et}\ x\not=-1\vsec\\ n+1 & \hbox{si}\ x= 1\vsec\\ -(n+1)& \hbox{si}\ x=-1.   \end{array}\right.$}$
\end{center}
\item  \textbf{Calcul de $\mathbf{\ddp \sum\limits_{k=0}^{n} a^k 2^{3k} x^{-k}}$:}\\
\noindent On reconna\^{i}t la somme des termes d'une suite g\'eom\'etrique et on obtient donc en utilisant le fait que $a^k 2^{3k} x^{-k}=a^k(2^3)^k\times \ddp\frac{1}{x^k}=a^k\times 8^k\times \left(\ddp\frac{1}{x} \right)^k=\left(\ddp\frac{8a}{x} \right)^k$:\\

\begin{center}
 $\ddp \sum\limits_{k=0}^{n} a^k 2^{3k} x^{-k}=\ddp \sum\limits_{k=0}^{n} \left(\ddp\frac{8a}{x}\right)^k=\fbox{$\left\lbrace \begin{array}{ll}  \ddp\frac{1-\left( \frac{8a}{x}\right)^{n+1}}{1-\left( \frac{8a}{x}\right)} & \hbox{si}\ x\not= 8a \vsec\\ n+1 & \hbox{si}\ x= 8a.   \end{array}\right.$}$ 
\end{center}

\item  \textbf{Calcul de $\mathbf{\ddp \sum\limits_{i=0}^{n} (i^2+n+3)}$:}\\
\noindent Par lin\'earit\'e de la somme, on obtient: 
\begin{align*}
\ddp \sum\limits_{i=0}^{n} (i^2+n+3)&=\ddp \sum\limits_{i=0}^{n} i^2+(n+3)\ddp \sum\limits_{i=0}^{n} 1\\
&= \ddp\frac{n(n+1)(2n+1)}{6}+(n+3)(n+1)\\
&=\fbox{$ \ddp\frac{n+1}{6}\left( 2n^2+7n+18 \right)$}
\end{align*}

%\item  \textbf{Calcul de $\mathbf{\ddp \sum\limits_{j=8}^{21}\ddp\frac{2j-5}{6}}$:}\\
%\noindent Par lin\'earit\'e de la somme, on obtient: $\ddp \sum\limits_{j=8}^{21} \ddp\frac{2j-5}{6}=
%\ddp\frac{1}{6}\left\lbrack    2\ddp \sum\limits_{j=8}^{21} j-5\ddp \sum\limits_{j=8}^{21}1  \right\rbrack=
%\ddp\frac{1}{6}\left\lbrack    2\times \ddp\frac{(21+8)(21-8+1)}{2}-5(21-8+1)  \right\rbrack=
%\fbox{$\ddp\frac{28}{3}.$}$
\item  \textbf{Calcul de $\mathbf{\ddp \sum\limits_{i=1}^{n} (2i-1)^3}$:}\\
\noindent On commence par d\'evelopper la puissance cube \`{a} l'int\'erieur de la somme puis on utilise la lin\'earit\'e de la somme. On obtient donc:
\begin{align*}
\ddp \sum\limits_{i=1}^{n} (2i-1)^3&=\ddp \sum\limits_{i=1}^{n} \left( 8i^3-12i^2+6i-1 \right)\\
&=8 \ddp \sum\limits_{i=1}^{n} i^3-12\ddp \sum\limits_{i=1}^{n}i^2+6\ddp \sum\limits_{i=1}^{n} i-\ddp \sum\limits_{i=1}^{n}1. 
\end{align*}
On utilise ensuite le formulaire sur les sommes et on obtient alors:
\begin{align*}
 \ddp \sum\limits_{i=1}^{n} (2i-1)^3&=8\left( \ddp\frac{n(n+1)}{2} \right)^2-12\ddp\frac{n(n+1)(2n+1)}{6}+6\ddp\frac{n(n+1)}{2}-n\\
 	&= \fbox{$n^2(4n^2+4n+1)$} .
\end{align*}

Une autre solution consiste à faire la somme des paires entre $1$ et $2n$ puis simplifier l'expression avec la somme de tous les entiers au cube. 
%\item  \textbf{Calcul de $\mathbf{\ddp \sum\limits_{k=0}^{n} \ddp\frac{p}{q+1}}$:}\\
%\noindent Comme ni $p$, ni $q$ ne d\'ependent de l'indice de sommation $k$, par lin\'earit\'e de la somme, on obtient: $\ddp \sum\limits_{k=0}^{n} \ddp\frac{p}{q+1}=\ddp\frac{p}{q+1}\ddp \sum\limits_{k=0}^{n} 1=\fbox{$ \ddp\frac{p(n+1)}{q+1}$.}$
%-------------------
\item  \textbf{Calcul de $\mathbf{\ddp \sum\limits_{k=2}^{n^2} (1-a^2)^{2k+1}}$:}\\
\noindent On commence par utiliser les propri\'et\'es sur les puissances et on obtient que: $(1-a^2)^{2k+1}=\lbrack(1-a^2)^2\rbrack^k \times (1-a^2)^1$. Par lin\'earit\'e de la somme et en reconnaissant de plus la somme d'une suite g\'eom\'etrique, on a: $\ddp \sum\limits_{k=2}^{n^2} (1-a^2)^{2k+1}=(1-a^2)\ddp \sum\limits_{k=2}^{n^2} \lbrack (1-a^2)^2 \rbrack^k$. On doit donc \'etudier deux cas selon que $(1-a^2)^2\not= 1$ ou que $(1-a^2)^2= 1$.
\begin{itemize}
\item[$\bullet$] Cas 1: si $(1-a^2)^2\not= 1$:\\
\noindent On obtient alors: $\ddp \sum\limits_{k=2}^{n^2} (1-a^2)^{2k+1}=(1-a^2)\times ((1-a^2)^2)^2\times \ddp\frac{1-\lbrack (1-a^2)^2\rbrack^{n^2-1}}{1-(1-a^2)^2}=(1-a^2)^5\times \ddp\frac{1- (1-a^2)^{2n^2-2}}{2a^2-a^4}=\fbox{$(1-a^2)^5\times \ddp\frac{1- (1-a^2)^{2n^2-2}}{a^2(2-a^2)}$.}$
\item[$\bullet$] Cas 2: si $(1-a^2)^2= 1$:\\
\noindent Regardons \`{a} quels $a$ cela correctionrespond: $(1-a^2)^2= 1\Leftrightarrow 1-a^2=1\ \hbox{ou}\ 1-a^2=-1\Leftrightarrow a^2=0\ \hbox{ou}\ a^2=2\Leftrightarrow a=-\sqrt{2}\ \hbox{ou}\ a=0\ \hbox{ou}\ a=\sqrt{2}$. Calculons alors la somme pour ces $a$: $\ddp \sum\limits_{k=2}^{n^2} (1-a^2)^{2k+1}=(1-a^2)\ddp \sum\limits_{k=2}^{n^2} 1=(1-a^2)\times (n^2-1).$ Il faut alors distinguer encorrectione deux cas: 
\begin{itemize}
\item[$\star$] Si $a=0$ alors $1-a^2=1$ et \fbox{$\ddp \sum\limits_{k=2}^{n^2} (1-a^2)^{2k+1}=n^2-1.$}
\item[$\star$] Si $a=-\sqrt{2}$ ou $a=\sqrt{2}$ alors $1-a^2=-1$ et \fbox{$\ddp \sum\limits_{k=2}^{n^2} (1-a^2)^{2k+1}=-n^2+1.$}
\end{itemize}
\end{itemize}
\item  \textbf{Calcul de $\mathbf{\ddp \sum\limits_{k=1}^n (3\times 2^k+1) }$:}\\
\noindent $\ddp \sum\limits_{k=1}^n (3\times 2^k+1)=3 \ddp \sum\limits_{k=1}^n 2^k+\ddp \sum\limits_{k=1}^n 1=3\times 2\times \ddp\frac{1-2^n}{1-2} +n $ par lin\'earit\'e et car $2\not=1$. Donc 
\begin{center}
\fbox{$\ddp \sum\limits_{k=1}^n (3\times 2^k+1) =6(2^n-1)+n.$}
\end{center}

\item  \textbf{Calcul de $\mathbf{\ddp\frac{1}{n}\ddp \sum\limits_{k=0}^{n-1} \exp{\left(\ddp\frac{k}{n}\right)} }$:}\\
\noindent $\ddp\frac{1}{n}\ddp \sum\limits_{k=0}^{n-1} \exp{\left(\ddp\frac{k}{n}\right)} = \ddp\frac{1}{n}\ddp \sum\limits_{k=0}^{n-1} \left( e^{\frac{1}{n}} \right)^{k}=\ddp\frac{1}{n} \ddp\frac{ 1-\left( e^{\frac{1}{n}} \right)^{n}  }{1-e^{\frac{1}{n}} }$ car $e^{\frac{1}{n}} \not= 1$. Ainsi \fbox{$\ddp\frac{1}{n}\ddp \sum\limits_{k=0}^{n-1} \exp{\left(\ddp\frac{k}{n}\right)} =\ddp\frac{1}{n} \ddp\frac{ 1-e  }{1-e^{\frac{1}{n}} }$.} 


\item  \textbf{Calcul de $\mathbf{\ddp \sum\limits_{k=0}^n (2k-1+2^k) }$:}\\
\noindent $\ddp \sum\limits_{k=0}^n (2k-1+2^k)=2\ddp \sum\limits_{k=0}^n k-\ddp \sum\limits_{k=0}^n 1+\ddp \sum\limits_{k=0}^n 2^k=2\ddp\frac{n(n+1)}{2} -(n+1)+ \ddp\frac{1-2^{n+1}}{1-2}$ par  lin\'earit\'e et car $2\not=1$. Ainsi \fbox{$\ddp \sum\limits_{k=0}^n (2k-1+2^k)=n^2+2^{n+1}-2$.}
%\item  \textbf{Calcul de $\mathbf{\ddp \sum\limits_{k=1}^n 2^{2k+1} }$:}\\
%\noindent $\ddp \sum\limits_{k=1}^n 2^{2k+1} =2\ddp \sum\limits_{k=1}^n (2^2)^k=2\ddp \sum\limits_{k=1}^n 4^k=2\times 4\times \ddp\frac{1-4^n}{1-4} $ par  lin\'earit\'e et car $4\not=1$. Ainsi \fbox{$\ddp \sum\limits_{k=1}^n 2^{2k+1} =\ddp\frac{8}{3}(4^n-1)$.}
%\item  \textbf{Calcul de $\mathbf{\ddp \sum\limits_{i=0}^n 3(i+1)i}$:}\\
%\noindent $\ddp \sum\limits_{i=0}^n 3(i+1)i=3\ddp \sum\limits_{i=0}^n i^2+3\ddp \sum\limits_{i=0}^n i=3\ddp\frac{n(n+1)(2n+1)}{6}+3\ddp\frac{n(n+1)}{2}=\fbox{$ n(n+1)(n+2)$.}$ On a utilis\'e la lin\'earit\'e de la somme et le formulaire sur les sommes usuelles.
\item  \textbf{Calcul de $\mathbf{\ddp \sum\limits_{j=0}^n\ddp \binom{n}{j}a^j}$:}\\
\noindent \fbox{$\ddp \sum\limits_{j=0}^n\binom{n}{j}a^j=(1+a)^n$} en reconnaissant un bin\^{o}me de Newton car $\ddp \sum\limits_{j=0}^n\binom{n}{j}a^j=\ddp \sum\limits_{j=0}^n\binom{n}{j}a^j 1^{n-j}$\\
\noindent  \textbf{Calcul de $\mathbf{\ddp \sum\limits_{j=1}^{n+1}\ddp \binom{n}{j}a^j}$:}\\
\noindent On se ram\`{e}ne \`{a} la formule du bin\^{o}me de Newton en utilisant la relation de Chasles: 
$\ddp \sum\limits_{j=1}^{n+1}\ddp \binom{n}{j}a^j=\ddp \sum\limits_{j=0}^{n}\ddp \binom{n}{j}a^j-\binom{n}{0}a^0+\binom{n}{n+1}a^{n+1}$. Par convention, on a: $\binom{n}{n+1}=0$ et ainsi on obtient en utilisant le bin\^{o}me de Newton: \fbox{$\ddp \sum\limits_{j=1}^{n+1}\ddp \binom{n}{j}a^j=(1+a)^n-1 .$}
\item  \textbf{Calcul de $\mathbf{\ddp \sum\limits_{i=0}^{n}\ddp \binom{n}{i}(-1)^{i}}$:}\\
\noindent \noindent \fbox{$\ddp \sum\limits_{j=0}^n\binom{n}{j}(-1)^j=0$} gr\^ace au bin\^{o}me de Newton car $\ddp \sum\limits_{j=0}^n\binom{n}{j}(-1)^j=\ddp \sum\limits_{j=0}^n\binom{n}{j}(-1)^j 1^{n-j}=(1-1)^n$.
\item  \textbf{Calcul de $\mathbf{\ddp \sum\limits_{i=1}^{n}\ddp \binom{n+1}{i}(-1)^{i}}$:}\\
\noindent On se ram\`{e}ne \`{a} la formule du bin\^{o}me de Newton en utilisant la relation de Chasles: 
$\ddp \sum\limits_{i=1}^{n}\ddp \binom{n+1}{i}(-1)^{i}=\ddp \sum\limits_{i=0}^{n+1}\ddp \binom{n+1}{i}(-1)^{i}-\ddp \binom{n+1}{0}(-1)^{0}-\ddp \binom{n+1}{n+1}(-1)^{n+1}=(1-1)^{n+1}-1-(-1)^{n+1}=-1+(-1)^{n+2}=-1+(-1)^n=(-1)^n-1.$ Ainsi on obtient que: \fbox{$\ddp \sum\limits_{i=1}^{n}\ddp \binom{n+1}{i}(-1)^{i}=(-1)^n-1$.}
\item  \textbf{Calcul de $\mathbf{\ddp \sum\limits_{j=0}^n\ddp \binom{n}{j}  \ddp\frac{(-1)^{j-1}}{2^{j+1}}}$:}\\
\noindent On se ram\`{e}ne \`{a} la formule du bin\^{o}me de Newton en utilisant les propri\'et\'es sur les puissances. On obtient
$\ddp \sum\limits_{j=0}^n\ddp \binom{n}{j}  \ddp\frac{(-1)^{j-1}}{2^{j+1}}=\ddp\frac{-1}{2}\ddp \sum\limits_{j=0}^n\ddp \binom{n}{j} \left( \ddp\frac{-1}{2}\right)^j=\ddp\frac{-1}{2} \left(  1-\ddp\demi \right)^n=\fbox{$ \ddp\frac{-1}{2^{n+1}}  $.}$
\item  \textbf{Calcul de $\mathbf{\ddp \sum\limits_{k=0}^{n-1} \ddp\frac{1}{3^k}\ddp \binom{n}{k}}$:}\\
\noindent $\ddp \sum\limits_{k=0}^{n-1} \ddp\frac{1}{3^k}\binom{n}{k}=\ddp \sum\limits_{k=0}^{n-1}\binom{n}{k}\left( \ddp\frac{1}{3}\right)^k1^{n-k}$. Afin de pouvoir utiliser la formule du bin\^{o}me de Newton, on utilise la relation de Chasles pour obtenir:
$\ddp \sum\limits_{k=0}^{n-1} \ddp\frac{1}{3^k}\binom{n}{k}=\ddp \sum\limits_{k=0}^{n}\binom{n}{k}\left( \ddp\frac{1}{3}\right)^k1^{n-k}-\binom{n}{n}\left( \ddp\frac{1}{3}\right)^n=\left( 1+\ddp\frac{1}{3} \right)^n- \ddp\frac{1}{3^n}=\left(\ddp\frac{4}{3} \right)^n- \ddp\frac{1}{3^n}=\fbox{$\ddp\frac{4^n-1}{3^n}.$}$
\end{enumerate}
\end{correction}

%------------------------------------------------
























%-------------------------------------------------
\begin{exercice} \; \textbf{Coefficients binomiaux} \\
Calculer les sommes suivantes: 
\begin{enumerate}
\item $S_1 = \ddp \sum\limits_{j=0}^n j\ddp \binom{n}{j}$
\item $T = \ddp \sum\limits_{k=1}^n k(k-1)\ddp \binom{n}{k}$, puis $S_2 = \ddp \sum\limits_{k=1}^n k^2\ddp \binom{n}{k}$ (on pourra \'ecrire que $k^2=k(k-1)+k$).
\item $S_3= \ddp \sum\limits_{i=0}^n \ddp\frac{1}{i+1}\ddp \binom{n}{i}$.
\end{enumerate}
\end{exercice}
%------------------------------------------------



%-------------------------------------------------
\begin{correction}   \;
\begin{enumerate}
\item \textbf{Calcul de $\mathbf{S_1=\ddp \sum\limits_{j=0}^n j\ddp \binom{n}{j}}$:}\\
\noindent On peut d\'ej\`{a} remarquer que: $\ddp S_1= \sum\limits_{j=0}^n j\binom{n}{j}=0\times \binom{n}{0}+\ddp \sum\limits_{j=1}^n j\binom{n}{j}=\ddp \sum\limits_{j=1}^n j\binom{n}{j}$. \\
Ici on ne sait pas calculer la somme sans transformation car il y a le $j$. On utilise d'abord une propri\'et\'e des coefficients binomiaux, et on obtient: 
$$S_1=\ddp \sum\limits_{j=0}^n j\ddp \binom{n}{j}=\ddp \sum\limits_{j=1}^n n\binom{n-1}{j-1}=n\ddp \sum\limits_{j=1}^n \binom{n-1}{j-1}$$ 
car $n$ est alors ind\'ependant de l'indice de sommation donc on peut le sortir de la somme. Pour se ramener \`{a} du bin\^{o}me de Newton, on commence par poser le changement de variable: $i=j-1$ et on obtient $\ddp S_1=\sum\limits_{j=0}^n j\binom{n}{j}=n \ddp \sum\limits_{i=0}^{n-1} \binom{n-1}{i}$ (c'est ici qu'il est mieux d'\^{e}tre pass\'e au d\'ebut d'une somme allant de 0 \`{a} $n$ \`{a} une somme allant de 1 \`{a} $n$ car sinon on aurait un indice commencant \`{a} -1. Si on n'a pas chang\'e la somme au d\'ebut, une autre m\'ethode est alors de faire ici une relation de Chasles afin d'isoler l'indice -1). On reconna\^{i}t alors un bin\^{o}me de Newton et on obtient \fbox{$S_1=\ddp \sum\limits_{j=0}^n j\binom{n}{j}=n2^{n-1}$.}
\item   \textbf{Calcul de $\mathbf{T=\ddp \sum\limits_{k=1}^n k(k-1)\ddp \binom{n}{k}}$:}\\
\noindent Il s'agit ici d'appliquer deux fois de suite la propri\'et\'e sur les coefficients binomiaux : $T=n\ddp \sum\limits_{k=2}^{n} (k-1)\binom{n-1}{k-1}$ en reprenant les calculs faits au-dessus. On pourra aussi remarquer que la somme $T$ peut \^{e}tre commenc\'ee \`{a} 2. Puis en r\'eappliquant la propri\'et\'e sur les coefficients binomiaux : $(k-1)\binom{n-1}{k-1}=(n-1)\binom{n-2}{k-2}$, on obtient que: $T=n(n-1)\ddp \sum\limits_{k=2}^{n} \binom{n-2}{k-2}$. On effectue alors le changement de variable $j=k-2$ et on obtient $T=n(n-1)\ddp \sum\limits_{j=0}^{n-2} \binom{n-2}{j}$. Donc en utilisant le bin\^{o}me de Newton, on a: \fbox{$T=n(n-1)2^{n-2}$.}\\
\noindent \textbf{Calcul de $S_2=\mathbf{\ddp \sum\limits_{k=1}^n k^2\ddp \binom{n}{k}}$:}\\
\noindent Comme $k^2=k(k-1)+k$ et par lin\'earit\'e de la somme, on obtient que: $S_2=\ddp \sum\limits_{k=1}^{n} k^2\binom{n}{k}=\ddp \sum\limits_{k=1}^{n} k(k-1)\binom{n}{k}+\ddp \sum\limits_{k=1}^{n} k\binom{n}{k}=\ddp \sum\limits_{k=2}^{n} k(k-1)\binom{n}{k}+\ddp \sum\limits_{k=1}^{n} k\binom{n}{k}=T+S_1=\fbox{$n(n+1)2^{n-2}$.}$
\item  \textbf{Calcul de $\mathbf{\ddp S_3=\sum\limits_{i=0}^n \ddp\frac{1}{i+1}\ddp \binom{n}{i}}$:}\\
\noindent L\`{a} encorrectione, il faut commencer par utiliser la propri\'et\'e sur les coefficients binomiaux. 
Comme $\ddp (i+1)\binom{n+1}{i+1}=(n+1)\binom{n}{i}$, on obtient que: $\ddp\frac{1}{i+1}\binom{n}{i}= \ddp\frac{1}{n+1}\binom{n+1}{i+1}$. Ainsi, la somme devient: 
$S_3=\ddp \sum\limits_{i=0}^n \ddp\frac{1}{i+1}\ddp \binom{n}{i}=\ddp \sum\limits_{i=0}^n \ddp\frac{1}{n+1}\binom{n+1}{i+1}= \ddp\frac{1}{n+1} \ddp \sum\limits_{i=0}^n \binom{n+1}{i+1}$ car $\ddp\frac{1}{n+1} $ ne d\'epend pas de l'indice de sommation $i$. On fait le changement d'indice $j=i+1$ et on utilise aussi la relation de Chasles pour faire appara\^{i}tre le bin\^{o}me de Newton. On obtient $S_3=\ddp \sum\limits_{i=0}^n \ddp\frac{1}{i+1}\ddp \binom{n}{i}= \ddp\frac{1}{n+1} \ddp \sum\limits_{j=1}^{n+1} \binom{n+1}{j}= \ddp\frac{1}{n+1} \left\lbrack \ddp \sum\limits_{j=0}^{n+1} \binom{n+1}{j}- \binom{n+1}{0} \right\rbrack$. Ainsi, on obtient 
\fbox{$\ddp S_3= \sum\limits_{i=0}^n \ddp\frac{1}{i+1}\ddp \binom{n}{i}= \ddp\frac{1}{n+1} \left\lbrack 2^{n+1}-1 \right\rbrack$.}
\end{enumerate}
\end{correction}
%------------------------------------------------























%-------------------------------------------------
\begin{exercice} \; \textbf{Sommes t\'elescopiques}
\begin{enumerate}
\item Soit $x_0,x_1,\dots,x_n$ des nombres r\'eels avec $n\in\N$. Calculer : \; $\ddp \sum\limits_{i=0}^{n} (x_{i+1}-x_i) $ \; et \; $\ddp \sum\limits_{i=1}^{n} (x_{i+1}-x_{i-1})$.
\item Calculer : $\ddp \sum\limits_{k=3}^{n} \ln{\left\lbrack  \ddp\frac{k^2}{(k+1)(k-2)} \right\rbrack}$
\end{enumerate}
\end{exercice}
%\newpage

\begin{correction}   \;
\begin{enumerate}
\item D\`{e}s que l'on a une soustraction entre deux sommes de m\^{e}me type avec juste un d\'ecalage d'indice, il faut reconna\^{i}tre une somme t\'elescopique et savoir la calculer. Le calcul utilise un ou plusieurs changements d'indice puis la relation de Chasles.
\begin{itemize}
\item[$\bullet$] \textbf{Calcul de $\mathbf{S=\ddp \sum\limits_{i=0}^{n} (x_{i+1}-x_i)}$:}\\
\noindent $S=\ddp \sum\limits_{i=0}^{n} (x_{i+1}-x_i)=\ddp \sum\limits_{i=0}^{n} x_{i+1}-\ddp \sum\limits_{i=0}^{n} x_i $ par lin\'earit\'e. On pose alors le changement d'indice: $j=i+1$ dans la premi\`{e}re somme et on obtient: $S=\ddp \sum\limits_{j=1}^{n+1} x_{j}-\ddp \sum\limits_{i=0}^{n} x_i $. Comme l'indice de sommation est muet, on a: $S=\ddp \sum\limits_{i=1}^{n+1} x_{i}-\ddp \sum\limits_{i=0}^{n} x_i$. La relation de Chasles donne: $S=\ddp \sum\limits_{i=1}^{n} x_{i}+x_{n+1}-\ddp \sum\limits_{i=1}^{n} x_i-x_0=\fbox{$x_{n+1}-x_0$.}$
\item[$\bullet$] \textbf{Calcul de $S^{\prime}=\mathbf{\ddp \sum\limits_{i=1}^{n} (x_{i+1}-x_{i-1})}$:}\\
\noindent $S^{\prime}=\ddp \sum\limits_{i=1}^{n} (x_{i+1}-x_{i-1})=\ddp \sum\limits_{i=1}^{n} x_{i+1}-\ddp \sum\limits_{i=1}^{n} x_{i-1} $ par lin\'earit\'e. On pose alors le changement d'indice: $j=i+1$ dans la premi\`{e}re somme et le changement $k=i-1$ dans la deuxi\`{e}me somme et on obtient: $S^{\prime}=\ddp \sum\limits_{j=2}^{n+1} x_{j}-\ddp \sum\limits_{k=0}^{n-1} x_k $. Comme l'indice de sommation est muet, on a: $S^{\prime}=\ddp \sum\limits_{i=2}^{n+1} x_{i}-\ddp \sum\limits_{i=0}^{n-1} x_i$. La relation de Chasles donne: $S^{\prime}=\ddp \sum\limits_{i=2}^{n-1} x_{i}+x_{n}+x_{n+1}-\ddp \sum\limits_{i=2}^{n-1} x_i-x_0-x_1= \fbox{$ x_{n+1}+x_n-x_0-x_1$.}$
\end{itemize}
%\item Il s'agit ici de faire appara\^{i}tre une somme t\'elescopique puis d'appliquer la m\'ethode pr\'ec\'edente.
%\item \textbf{Calcul de $\mathbf{S=\ddp \sum\limits_{k=1}^{n} \ddp\frac{1}{k(k+1)}}$: }
%\begin{itemize}
%\item[$\bullet$]  Classiquement pour ce type de somme, on commence par montrer qu'il existe deux r\'eels $a$ et $b$ tels que pour tout $k\in\N^{\star}$: $\ddp\frac{1}{k(k+1)}=\ddp\frac{a}{k}+\ddp\frac{b}{k+1}$. En mettant au m\^{e}me d\'enominateur, on obtient que: $\forall k\in\N^{\star},\quad \ddp\frac{1}{k(k+1)}=\ddp\frac{  (a+b)k+a }{k(k+1)}.$
%Cette relation doit \^{e}tre vraie pour tout $k\in\N^{\star}$ donc, par identification, on obtient que: $\left\lbrace \begin{array}{lll}  a+b&=&0\vsec\\ a&=&1  \end{array}\right.$ donc $a=1$ et $b=-1$. Ainsi, on obtient, par lin\'earit\'e, que: $S=\ddp \sum\limits_{k=1}^{n}  \ddp\frac{1}{k}-\ddp \sum\limits_{k=1}^{n} \ddp\frac{1}{k+1}.$
%\item[$\bullet$] Il s'agit alors bien d'une somme t\'elescopique. On pose le changement d'indice: $j=k+1$ dans la deuxi\`{e}me somme et on obtient: $S=\ddp \sum\limits_{k=1}^{n}  \ddp\frac{1}{k}-\ddp \sum\limits_{j=2}^{n+1} \ddp\frac{1}{j}=\ddp \sum\limits_{k=1}^{n}  \ddp\frac{1}{k}-\ddp \sum\limits_{k=2}^{n+1} \ddp\frac{1}{k}=\fbox{$1-\ddp\frac{1}{n+1}$}$ en utilisation le fait que l'indice de sommation est muet et la relation de Chasles.
%\end{itemize}
%\item \textbf{Calcul de $\mathbf{S=\ddp \sum\limits_{k=1}^{n} \ln{\left\lbrack  \ddp\frac{k}{k+1} \right\rbrack}}$:}
%\begin{itemize}
%\item[$\bullet$]  On transforme cette somme en utilisant les propri\'et\'es du logarithme n\'ep\'erien et on obtient: $\ln{\left(  \ddp\frac{k}{k+1}\right)}=\ln{(k)}-\ln{(k+1)}.$
%\item[$\bullet$]  Ainsi transform\'ee, la somme $S$ est bien de type t\'elescopique car on a bien une soustraction de 2 sommes de m\^{e}me type avec juste un d\'ecalage d'indice. En effet, par lin\'earit\'e, on obtient:
%$S=\ddp \sum\limits_{k=1}^{n} \ln{(k)}-\ddp \sum\limits_{k=1}^{n} \ln{(k+1)}.$ On pose le changement d'indice $j=k+1$ dans la deuxi\`{e}me somme et on obtient $S=\ddp \sum\limits_{k=1}^{n} \ln{(k)}-\ddp \sum\limits_{j=2}^{n+1} \ln{(j)}=\ln{(1)}-\ln{(n+1)}=\fbox{$-\ln{(n+1)}$.}$
%\end{itemize}
\item \textbf{Calcul de $\mathbf{S=\ddp \sum\limits_{k=3}^{n} \ln{\left\lbrack  \ddp\frac{k^2}{(k+1)(k-2)} \right\rbrack}}$:}\\ 
On transforme cette somme en utilisant les propri\'et\'es du logarithme n\'ep\'erien et on obtient: $\ln{\left(  \ddp\frac{k^2}{(k+1)(k-2)}\right)}=2\ln{(k)}-\ln{(k+1)}-\ln{(k-2)}.$\\
Ainsi transform\'ee, la somme $S$ est bien de type t\'elescopique car on a bien une soustraction de 3 sommes de m\^{e}me type avec juste des d\'ecalages d'indice. En effet, par lin\'earit\'e, on obtient:
$S=2\ddp \sum\limits_{k=3}^{n} \ln{(k)}-\ddp \sum\limits_{k=3}^{n} \ln{(k+1)}-\ddp \sum\limits_{k=3}^{n} \ln{(k-2)}.$ On pose le changement d'indice $j=k+1$ dans la deuxi\`{e}me somme et le changement $i=k-2$ dans la troisi\`{e}me somme et on obtient 
$$\begin{array}{lll}
S&=&2\ddp \sum\limits_{k=3}^{n} \ln{(k)}-\ddp \sum\limits_{j=4}^{n+1} \ln{(j)}-\ddp \sum\limits_{j=1}^{n-2} \ln{(j)}\vsec\\
&=& 2\ln{(3)}+2\ln{(n-1)}+2\ln{(n)}-\ln{(n-1)}-\ln{(n)}-\ln{(n+1)}-\ln{(1)}-\ln{(2)}-\ln{(3)}\vsec\\
&=& \fbox{$\ln{\left( \ddp\frac{ 3n(n-1) }{ 2(n+1) }  \right)}$.}\end{array}$$
\end{enumerate}
\end{correction}
%------------------------------------------------
%-------------------------------------------------




%------------------------------------------------
%-------------------------------------------------
\begin{exercice}  \; \textbf{Sommes t\'elescopiques}
\begin{enumerate}
\item D\'eterminer $(a,b)\in\bR^2$ tels que $\forall k\in\N^{\star}, \ \ddp\frac{1}{(k+1)(k+2)}=\ddp\frac{a}{k+1}+\ddp\frac{b}{k+2}$. En d\'eduire : $\ddp \sum\limits_{k=1}^n \ddp\frac{1}{(k+1)(k+2)} $. 
\item D\'eterminer trois r\'eels $a$, $b$ et $c$ tels que : $\forall k\in\N^{\star},\ \ddp\frac{k-1}{k(k+1)(k+3)}=\ddp\frac{a}{k}+\ddp\frac{b}{k+1}+\ddp\frac{c}{k+3}.$ En d\'eduire la valeur de $\ddp \sum\limits_{k=1}^{n}  \ddp\frac{k-1}{k(k+1)(k+3)}$.
\item D\'eterminer trois r\'eels $a$, $b$ et $c$ tels que : $\forall k\in\N^{\star},\ \ddp\frac{1}{k(k+1)(k+2)}=\ddp\frac{a}{k}+\ddp\frac{b}{k+1}+\ddp\frac{c}{k+2}.$ En d\'eduire la valeur de $\ddp \sum\limits_{k=1}^{n}  \ddp\frac{1}{k(k+1)(k+2)}$.\\
Retrouver ce r\'esultat par r\'ecurrence : montrer que $\forall n\geq 1$: $\ddp \sum\limits_{k=1}^{n} \ddp\frac{1}{k(k+1)(k+2)}=\ddp\frac{n(n+3)}{4(n+1)(n+2)}$.
\end{enumerate}
\end{exercice}
%\newpage
%------------------------------------------------
%-------------------------------------------------


\begin{correction}   \;
\begin{enumerate}
\item \textbf{Calcul de $\mathbf{S=\ddp \sum\limits_{k=1}^{n}   \ddp\frac{1}{(k+1)(k+2)}}$:}
\begin{itemize}
\item[$\bullet$] On commence par montrer qu'il existe deux r\'eels $a$ et $b$ tels que pour tout $k\in\N^{\star}$: $\ddp\frac{1}{(k+1)(k+2)}=\ddp\frac{a}{k+1}+\ddp\frac{b}{k+2}$. En mettant au m\^{e}me d\'enominateur, on obtient que: $\forall k\in\N^{\star},\quad \ddp\frac{1}{(k+1)(k+2)}=\ddp\frac{  (a+b)k+2a+b }{(k+1)(k+2)}.$
Cette relation doit \^{e}tre vraie pour tout $k\in\N^{\star}$ donc, par identification, on obtient que: $\left\lbrace \begin{array}{lll}  a+b&=&0\vsec\\ 2a+b&=&1  \end{array}\right.$ donc $a=1$ et $b=-1$. Ainsi, on obtient, par lin\'earit\'e, que: $S=\ddp \sum\limits_{k=1}^{n}  \ddp\frac{1}{k+1}-\ddp \sum\limits_{k=1}^{n} \ddp\frac{1}{k+2}.$
\item[$\bullet$] Il s'agit alors bien d'une somme t\'elescopique. On pose le changement d'indice: $j=k+1$ dans la premi\`{e}re somme et le changement d'indice: $i=k+2$ dans la deuxi\`{e}me somme et on obtient: $S=\ddp \sum\limits_{j=2}^{n+1}  \ddp\frac{1}{j}-\ddp \sum\limits_{i=3}^{n+2} \ddp\frac{1}{i}=\ddp \sum\limits_{k=2}^{n+1}  \ddp\frac{1}{k}-\ddp \sum\limits_{k=3}^{n+2} \ddp\frac{1}{k}=\fbox{$\ddp\demi-\ddp\frac{1}{n+2}$}$ en utilisation le fait que l'indice de sommation est muet et la relation de Chasles.
\end{itemize}
\item
\begin{itemize}
\item[$\bullet$] On cherche \`{a} d\'eterminer trois r\'eels $a$, $b$ et $c$ tels que $\forall k\in\N^{\star},\quad \ddp\frac{k-1}{k(k+1)(k+3)}=\ddp\frac{a}{k}+\ddp\frac{b}{k+1}+\ddp\frac{c}{k+3}.$ On met sur le m\^{e}me d\'enominateur puis on identifie car la relation doit \^{e}tre vraie pour tout $k\in\N^{\star}$. On obtient:  $\forall k\in\N^{\star},\quad \ddp\frac{k-1}{k(k+1)(k+3)}=\ddp\frac{ k^2(a+b+c)+k(4a+3b+c)+3a  }{k(k+1)(k+3)}$. Ainsi, par identification, on doit r\'esoudre le syst\`{e}me suivant: 
$\left\lbrace\begin{array}{lll} a+b+c&=&0\vsec\\ 4a+3b+c&=& 1\vsec\\ 3a&=&-1\end{array}\right.$. La r\'esolution du syst\`{e}me donne: \fbox{$a=-\ddp\frac{1}{3},\ b=1\ \hbox{et}\ c=-\ddp\frac{2}{3}$.}
\item[$\bullet$] En d\'eduire la valeur de $S=\ddp \sum\limits_{k=1}^{n}  \ddp\frac{k-1}{k(k+1)(k+3)}$. On obtient donc par lin\'earit\'e:
$S=-\ddp\frac{1}{3} \ddp \sum\limits_{k=1}^{n}  \ddp\frac{1}{k}+\ddp \sum\limits_{k=1}^{n}  \ddp\frac{1}{k+1}-\ddp\frac{2}{3}\ddp \sum\limits_{k=1}^{n}  \ddp\frac{1}{k+3}.$ On pose les changements de variable suivant: $j=k+1$ et $i=k+3$ et on obtient: 
$S=-\ddp\frac{1}{3} \ddp \sum\limits_{k=1}^{n}  \ddp\frac{1}{k}+\ddp \sum\limits_{j=2}^{n+1}  \ddp\frac{1}{j}-\ddp\frac{2}{3}\ddp \sum\limits_{i=4}^{n+3}  \ddp\frac{1}{i}=-\ddp\frac{1}{3} \ddp \sum\limits_{k=1}^{n}  \ddp\frac{1}{k}+\ddp \sum\limits_{k=2}^{n+1}  \ddp\frac{1}{k}-\ddp\frac{2}{3}\ddp \sum\limits_{k=4}^{n+3}  \ddp\frac{1}{k}$ car l'indice de sommation est muet. D'apr\`{e}s la relation de Chasles, on obtient: $S=-\ddp\frac{1}{3}  \left( 1+\ddp\frac{1}{2}+\ddp\frac{1}{3} \right)+\ddp\frac{1}{2}+\ddp\frac{1}{3}+\ddp\frac{1}{n+1}-\ddp\frac{2}{3}\left( \ddp\frac{1}{n+1}+\ddp\frac{1}{n+2}+\ddp\frac{1}{n+3}\right)=\fbox{$\ddp\frac{2}{9}+\ddp\frac{1}{3}\left( \ddp\frac{1}{n+1}-\ddp\frac{2}{n+2}-\ddp\frac{2}{n+3}\right)$.}$
\end{itemize}
\item  \textbf{Calcul de $\mathbf{\ddp \sum\limits_{k=1}^{n}  \ddp\frac{1}{k(k+1)(k+2)}}$:}\\
\begin{itemize}
\item[$\star$] M\'ethode 1 : calcul direct.
\begin{itemize}
\item[$\bullet$] On commence par montrer qu'il existe trois r\'eels $a,\ b$ et $c$ tels que pour tout $k\in\N^{\star}$: $\ddp\frac{1}{k(k+1)(k+2)}=\ddp\frac{a}{k}+\ddp\frac{b}{k+1}+\ddp\frac{c}{k+2}$. En mettant au m\^{e}me d\'enominateur, on obtient que: $\forall k\in\N^{\star},\quad \ddp\frac{1}{k(k+1)(k+2)}=\ddp\frac{  (a+b+c)k^2+(3a+2b+c)k+2a }{k(k+1)(k+2)}.$
Cette relation doit \^{e}tre vraie pour tout $k\in\N^{\star}$ donc, par identification, on obtient que: $\left\lbrace \begin{array}{lll}  a+b+c&=&0\vsec\\ 3a+2b+c&=&0\vsec\\ 2a&=&1  \end{array}\right.$ donc $a=c=\ddp\demi$ et $b=-1$. Ainsi, on obtient, par lin\'earit\'e, que: $\ddp \sum\limits_{k=1}^{n}  \ddp\frac{1}{k(k+1)(k+2)}=\ddp\demi\ddp \sum\limits_{k=1}^{n}  \ddp\frac{1}{k}-\ddp \sum\limits_{k=1}^{n} \ddp\frac{1}{k+1}+\ddp\demi\ddp \sum\limits_{k=1}^{n}  \ddp\frac{1}{k+2}.$
\item[$\bullet$]  Il s'agit alors bien d'une somme t\'elescopique. On pose le changement d'indice: $j=k+1$ dans la deuxi\`{e}me somme et le changement d'indice: $i=k+2$ dans la troisi\`{e}me somme et on obtient:
$$
\begin{array}{lll}
\ddp \sum\limits_{k=1}^{n}  \ddp\frac{1}{k(k+1)(k+2)}
&=&\ddp\demi\ddp \sum\limits_{k=1}^{n}  \ddp\frac{1}{k}-\ddp \sum\limits_{j=2}^{n+1}  \ddp\frac{1}{j}+\ddp\demi\ddp \sum\limits_{i=3}^{n+2} \ddp\frac{1}{i}=\ddp\demi\ddp \sum\limits_{k=1}^{n}  \ddp\frac{1}{k}-\ddp \sum\limits_{k=2}^{n+1}  \ddp\frac{1}{k}+\ddp\demi\ddp \sum\limits_{k=3}^{n+2} \ddp\frac{1}{k}\vsec\\
&=&
\ddp\demi\left(  1+\ddp\demi \right)-\left(  1+\ddp\frac{1}{n+1} \right)+\ddp\demi\left( \ddp\frac{1}{n+1} +\ddp\frac{1}{n+2}    \right)
=\fbox{$\ddp\frac{n(n+3)}{4(n+1)(n+2)}$}\end{array}$$ en utilisation le fait que l'indice de sommation est muet, la relation de Chasles et en mettant tout au m\^{e}me d\'enominateur.
\end{itemize}

\item[$\star$] M\'ethode 2 : par r\'ecurrence.
\begin{itemize}
\item[$\bullet$] On montre par r\'ecurrence sur $n\in\N^{\star}$ la propri\'et\'e 
$$\mathcal{P}(n):\  \ddp \sum\limits_{k=1}^{n} \ddp\frac{1}{k(k+1)(k+2)}=\ddp\frac{n(n+3)}{4(n+1)(n+2)} .$$
\item[$\bullet$] Initialisation: pour $n=1$:
\begin{itemize}
\item[$\star$] D'un c\^{o}t\'e, on a: $\ddp \sum\limits_{k=1}^{1} \ddp\frac{1}{k(k+1)(k+2)}=\ddp\frac{1}{1(1+1)(1+2)}=\ddp\frac{1}{6}$.
\item[$\star$] De l'autre c\^{o}t\'e, on a: $\ddp\frac{n(n+3)}{4(n+1)(n+2)}=\ddp\frac{1(1+3)}{4(1+1)(1+2)}=\ddp\frac{4}{4\times 6}=\ddp\frac{1}{6}$.
\end{itemize}
Donc $\mathcal{P}(1)$ est vraie.
\item[$\bullet$] H\'er\'edit\'e: soit $n\in\N^{\star}$ fix\'e. On suppose la propri\'et\'e vraie au rang $n$, montrons qu'elle est vraie au rang $n+1$.\\
\noindent $\ddp \sum\limits_{k=1}^{n+1} \ddp\frac{1}{k(k+1)(k+2)}=\ddp \sum\limits_{k=1}^{n} \ddp\frac{1}{k(k+1)(k+2)}+\ddp\frac{1}{(n+1)(n+2)(n+3)}$ d'apr\`{e}s la relation de Chasles. Puis par hypoth\`{e}se de r\'ecurrence, on obtient que:
$\ddp \sum\limits_{k=1}^{n+1} \ddp\frac{1}{k(k+1)(k+2)}=\ddp\frac{n(n+3)}{4(n+1)(n+2)} +\ddp\frac{1}{(n+1)(n+2)(n+3)}=
\ddp\frac{n^3+6n^2+9n+4}{4(n+1)(n+2)(n+3)}$ en mettant au m\^{e}me d\'enominateur. Pour le num\'erateur on remarque que -1 est racine \'evidente et ainsi en factorisant par $n+1$ on obtient par identification des coefficients que: 
$n^3+6n^2+9n+4=(n+1)(n^2+5n+4)$. Puis le calcul du discriminant donne que $n^3+6n^2+9n+4=(n+1)(n^2+5n+4)=(n+1)(n+1)(n+4)$. Ainsi on obtient que: $\ddp \sum\limits_{k=1}^{n+1} \ddp\frac{1}{k(k+1)(k+2)}=\ddp\frac{(n+1)(n+4)}{4(n+2)(n+3)}$. Donc $\mathcal{P}(n+1)$ est vraie.  
\item[$\bullet$] Conclusion: il r\'esulte du principe de r\'ecurrence que pour tout $n\in\N^{\star}$: $\ddp \sum\limits_{k=1}^{n} \ddp\frac{1}{k(k+1)(k+2)}=\ddp\frac{n(n+3)}{4(n+1)(n+2)}$.
\end{itemize}
\end{itemize}
\end{enumerate}
\end{correction}





%------------------------------------------------
%-------------------------------------------------
\begin{exercice}  \;  \textbf{Sommes et d\'erivation:} 
Soit $n\in\N^{\star}$ et $S=\ddp \sum\limits_{k=1}^n k\ddp \binom{n}{k}$. %Calculons $S$ en utilisant une m\'ethode par d\'erivation terme \`a terme.
%\begin{enumerate}
% \item M\'ethode 1: Avec la formule des chefs.\\
%\noindent Calculer $S$ directement en utilisant une propri\'et\'e des coefficients bin\^omiaux. \\
%\noindent De la m\^eme fa\c{c}on, calculer alors $T=\ddp \sum\limits_{k=1}^n k(k-1)\ddp \binom{n}{k}$ puis $\ddp \sum\limits_{k=1}^n k^2\ddp \binom{n}{k}$ (on pourra \'ecrire que $k^2=k(k-1)+k$).
%\item M\'ethode 2: En d\'erivant.
\begin{enumerate}
\item
On pose, pour tout $x$ dans $\bR$, $f(x)=\ddp \sum\limits_{k=0}^n \ddp \binom{n}{k}x^k$. Calculer $f(x)$.
\item 
En d\'eduire, pour tout $x$ dans $\bR$, la valeur de $g(x)=\ddp \sum\limits_{k=1}^n k\ddp \binom{n}{k}x^{k-1}$, puis en d\'eduire $S$.
\end{enumerate}
%\end{enumerate}
\end{exercice}
%------------------------------------------------


%------------------------------------------------
%-------------------------------------------------

%------------------------------------------------
%-------------------------------------------------
\begin{correction} \;
La d\'erivation d'une somme finie est une m\'ethode tr\`{e}s classique qui permet d'obtenir plein de nouvelles sommes. Il s'agit juste d'utiliser le fait que $(f+g)^{\prime}=f^{\prime}+g^{\prime}$ et ainsi la d\'eriv\'ee d'une somme est \'egale \`{a} la somme des d\'eriv\'ees.
\begin{enumerate}
\item D'apr\`{e}s le bin\^{o}me de Newton, on sait que: \fbox{$f(x)=(1+x)^n$.}
\item La fonction $f$ est ainsi d\'erivable sur $\R$ comme compos\'ee de fonctions d\'erivables. La fonction $f$ est d\'efinie par deux expressions diff\'erentes que l'on peut d\'eriver:
\begin{itemize}
\item[$\bullet$] D'un c\^{o}t\'e, la fonction $f$ vaut: $f(x)=(1+x)^n$. Ainsi, en d\'erivant, on obtient que: \begin{center}
\fbox{$\forall x\in\R,\ f^{\prime}(x)=n(1+x)^{n-1}$.}
\end{center}
\item[$\bullet$] De l'autre c\^{o}t\'e, la fonction $f$ vaut $f(x)=\ddp \sum\limits_{k=0}^{n} \binom{n}{k}x^k=1+nx+\dots+nx^{n-1}+x^n=1+\ddp \sum\limits_{k=1}^{n} \binom{n}{k}x^k$. La d\'eriv\'ee d'une somme \'etant \'egale \`{a} la somme des d\'eriv\'ees, on obtient que: \begin{center}
\fbox{$\forall x\in\R,\ f^{\prime}(x)=0+\ddp \sum\limits_{k=1}^{n} k\binom{n}{k}x^{k-1}$}
\end{center} 
\vsec
car le premier terme pour $k=0$ est constant donc sa d\'eriv\'ee est nulle. 

\vsec
%ATTENTION, la somme commence bien \`{a} $k=1$ car le terme pour $k=0$ dans $f(x)$ est le terme constant $1$ qui est nul lorsqu'on d\'erive.
\end{itemize}
On obtient donc que: \fbox{$\forall x\in\R,\ g(x)=\ddp \sum\limits_{k=1}^{n} k\binom{n}{k}x^{k-1}=n(1+x)^{n-1}$.}
\item Il s'agit de remarquer que $S=g(1)=f^{\prime}(1)$ et ainsi, on obtient que: \fbox{$S=n2^{n-1}$.} On retrouve bien le m\^{e}me r\'esultat.
\end{enumerate}
%\end{enumerate}
\end{correction}


















%-------------------------------------------------
\begin{exercice}  \; \textbf{Sommes et d\'erivation:} Soit $n\in\N^{\star}$. Pour tout $x\in\bR\setminus\lbrace 1\rbrace$, on pose $f(x)=\ddp \sum\limits_{k=0}^n x^k$.
\begin{enumerate}
\item Calculer $f(x)$.
\item En d\'erivant, calculer $\ddp \sum\limits_{k=1}^n kx^{k-1}$, et en d\'eduire $\ddp \sum\limits_{k=1}^n kx^k$.
\item Calculer de la m\^eme fa\c con : $\ddp \sum\limits_{k=2}^n k(k-1)x^{k-2}$.
\end{enumerate}

\end{exercice}
%------------------------------------------------

%------------------------------------------------
%-------------------------------------------------
\begin{correction}  \; Il s'agit ici du m\^{e}me type de m\'ethode que pour l'exercice pr\'ec\'edent sauf que cette fois ci, on l'applique \`{a} la somme des termes d'une suite g\'eom\'etrique et plus au bin\^{o}me de Newton.
\begin{enumerate}
\item On reconna\^{i}t la somme des termes d'une suite g\'eom\'etrique et ainsi, on obtient, comme $x\not= 1$:\\
\noindent  \fbox{$\forall x\in\R\setminus\lbrace 1\rbrace,\ f(x)=\ddp\frac{1-x^{n+1}}{1-x}$.}
\item La fonction $f$ est d\'erivable sur $\R\setminus\lbrace 1\rbrace$ comme produit, somme et quotient dont le d\'enominateur ne s'annule pas de fonctions d\'erivables. 
\begin{itemize}
\item[$\bullet$] D'un c\^{o}t\'e, la fonction $f$ vaut: $f(x)=\ddp\frac{1-x^{n+1}}{1-x}$. Ainsi, en d\'erivant, on obtient que:\\
\noindent \fbox{$\forall x\in\R\setminus\lbrace 1 \rbrace,\ f^{\prime}(x)=\ddp\frac{  1+nx^{n+1} -(n+1)x^n }{(1-x)^2}$.}
\item[$\bullet$] De l'autre c\^{o}t\'e, la fonction $f$ vaut $f(x)=\ddp \sum\limits_{k=0}^{n} x^k=1+\ddp \sum\limits_{k=1}^{n} x^k$. La d\'eriv\'ee d'une somme \'etant \'egale \`{a} la somme des d\'eriv\'ees, on obtient que:\\
\noindent \fbox{$\forall x\in\R\setminus\lbrace 1 \rbrace,\  f^{\prime}(x)=\ddp \sum\limits_{k=1}^{n} k x^{k-1}$.} \\
\noindent \warning La somme commence bien \`{a} $k=1$ car le terme pour $k=0$ dans $f(x)$ est le terme constant $1$ qui est nul lorsqu'on d\'erive.
\end{itemize} 
On obtient donc que: \fbox{$\forall x\in\R\setminus\lbrace 1\rbrace,\ \ddp \sum\limits_{k=1}^{n} k x^{k-1}=\ddp\frac{  1+nx^{n+1} -(n+1)x^n }{(1-x)^2}$.}\\
On a : $\ddp \sum\limits_{k=1}^{n} k x^{k}=\ddp \sum\limits_{k=1}^{n} k x\times x^{k-1}=x\ddp \sum\limits_{k=1}^{n} k x^{k-1}$. D'apr\`{e}s la question pr\'ec\'edente, on obtient donc:\begin{center}
 \fbox{$\forall x\in\R\setminus\lbrace 1\rbrace,\ \ddp \sum\limits_{k=1}^{n} k x^{k}=x\times \ddp\frac{  1+nx^{n+1} -(n+1)x^n }{(1-x)^2}$.}
\end{center}
\item Il faut ici remarquer que la somme correctionrespond \`{a} d\'eriver deux fois la somme $f(x)=\ddp \sum\limits_{k=0}^{n} x^k=1+x+\ddp \sum\limits_{k=2}^{n} x^k$. La fonction $f$ est bien deux fois d\'erivables comme fonction polynomiale.
Et en d\'erivant deux fois, on obtient bien: $\forall x\in\R\setminus\lbrace 1\rbrace,\ f^{\prime\prime}(x)=\ddp \sum\limits_{k=2}^{n} k(k-1)x^{k-2}$. Cette somme commence bien \`{a} $k=2$ car quand on d\'erive deux fois les termes $1$ et $x$, ils deviennent nuls. En d\'erivant deux fois l'autre expression de $f$, on obtient la valeur de la somme:\\ 
\vsec
 \begin{center}
\fbox{$\forall x\in\R\setminus\lbrace 1\rbrace,\ \ddp \sum\limits_{k=2}^{n} k(k-1)x^{k-2}=\ddp\frac{  2-n(n+1)x^{n-1} +2(n^2-1)x^n-n(n-1)x^{n+1}    }{(1-x)^3}$. }
\end{center}
\end{enumerate}
\end{correction}
%------------------------------------------------













%-------------------------------------------------
\begin{exercice}   \; \textbf{Sommes d'indices pairs et impairs}\\
\noindent Soit $n$ un entier naturel non nul. On d\'efinit les sommes suivantes: $S_n=\ddp \sum\limits_{k=0}^{n} \binom{2n}{2k}\quad \hbox{et}\quad T_n=\ddp \sum\limits_{k=0}^{n-1} \binom{2n}{2k+1}.$
\begin{enumerate}
\item Montrer que $S_n+T_n=2^{2n}$ et $S_n-T_n=0$.
\item En d\'eduire une expression de $S_n$ et de $T_n$ en fonction de $n$.
\end{enumerate} 
\end{exercice}
%------------------------------------------------

%-------------------------------------------------
\begin{correction}   
\begin{enumerate}
\item 
\begin{itemize}
\item[$\bullet$] \textbf{Calcul de $\mathbf{S_n+T_n}$:}\\
\noindent Si on ne voit pas comment d\'ebuter, on commence par \'ecrire la somme $S_n+T_n$ sous forme d\'evelopp\'ee. On obtient alors que: $S_n+T_n=\ddp \sum\limits_{k=0}^{2n} \binom{2n}{k}$ car on se rend compte en \'ecrivant les sommes sous forme d\'evelopp\'ees que l'on obtient au final la somme de tous les coefficients binomiaux: $S_n$ correctionrespond en effet \`{a} la somme des coefficients binomiaux $\binom{2n}{k}$ avec $k$ pair et $T_n$ correctionrespond \`{a} la somme des coefficients binomiaux $\binom{2n}{k}$ avec $k$ impair donc en sommant les deux on a bien la somme de tous les coefficients binomiaux pour $k$ allant de 0 \`{a} $2n$. Ainsi, d'apr\`{e}s le bin\^{o}me de Newton, on obtient que: \fbox{$S_n+T_n=2^{2n}=4^n$.}
\item[$\bullet$] \textbf{Calcul de $\mathbf{S_n-T_n}$:}\\
\noindent De m\^{e}me, on peut commencer par \'ecrire la somme $S_n-T_n$ sous forme d\'evelopp\'ee. On obtient alors que: $S_n-T_n=\ddp \sum\limits_{k=0}^{2n} \binom{2n}{k}(-1)^k$ car on se rend compte en \'ecrivant les sommes sous forme d\'evelopp\'ees que l'on obtient au final la somme de tous les coefficients binomiaux coefficient\'es par 1 ou par -1: les coefficients binomiaux $\binom{2n}{k}$ avec $k$ pair sont coefficient\'e par 1 et les coefficients binomiaux $\binom{2n}{k}$ avec $k$ impair sont coefficient\'e par -1. Ainsi cela revient bien \`{a} sommer tous les nombres $\binom{2n}{k}(-1)^k$ pour $k$ allant de 0 \`{a} 
$2n$. 
Ainsi, d'apr\`{e}s le bin\^{o}me de Newton, on obtient que: \fbox{$S_n+T_n=(1-1)^n=0$. }
\end{itemize}
%\begin{itemize}
%\item[$\bullet$] On montre par r\'ecurrence sur $n\in\N^{\star}$ la propri\'et\'e 
%$$\mathcal{P}(n):\quad S_n+T_n=2^{2n}\quad \hbox{et}\quad S_n-T_n=0.$$
%\item[$\bullet$] Initialisation: pour $n=1$:
%\begin{itemize}
%\item[$\star$] $S_1+T_1=\ddp \sum\limits_{k=0}^{1} \binom{2}{2k}+\ddp \sum\limits_{k=0}^{0} \binom{2}{2k+1}=\binom{2}{0}+\binom{2}{2}+\binom{2}{1}=1+1+2=4$. Or $2^{2}=4$ donc $S_1+T_1=2^{2}$.
%\item[$\star$] $S_1-T_1=\ddp \sum\limits_{k=0}^{1} \binom{2}{2k}-\ddp \sum\limits_{k=0}^{0} \binom{2}{2k+1}=\binom{2}{0}+\binom{2}{2}-\binom{2}{1}=1+1-2=0$.
%\end{itemize}
%Ainsi $\mathcal{P}(1)$ est vraie.
%\item[$\bullet$] H\'er\'edit\'e: soit $n\in\N^{\star}$ fix\'e, on suppose la propri\'et\'e vraie \`{a} l'ordre $n$, montrons qu'elle est vraie \`{a} l'ordre $n+1$.
%\begin{itemize}
%\item[$\star$] $S_{n+1}+T_{n+1}=\ddp \sum\limits_{k=0}^{n+1} \binom{2(n+1)}{2k}+\ddp \sum\limits_{k=0}^{n} \binom{2(n+1)}{2k+1}=\ddp \sum\limits_{k=0}^{n} \binom{2n+2}{2k}+\ddp \sum\limits_{k=0}^{n} \binom{2(n+1)}{2k+1}$
%\item[$\star$]
%\end{itemize}

%\item[$\bullet$]
%\end{itemize}
\item Il s'agit alors juste de r\'esoudre le syst\`{e}me $\left\lbrace \begin{array}{lll}  S_n+T_n&=&2^{2n}\vsec \\ S_n-T_n&=& 0. \end{array}\right.$. On obtient alors: $2S_n=2^{2n}$  donc $S_n= 2^{2n-1}$ et $T_n=S_n=2^{2n-1}$.
\end{enumerate}
\end{correction}

%-------------------------------------------------
%\vspace{1cm} 


%------------------------------------------------
%-------------------------------------------------
%---------------------------------------------------
%-------------------------------------------------
%-------------------------------------------------
%--------------------------------------------------
%----------------------------------------------------------------------------------------------
%-----------------------------------------------------------------------------------------------

%------------------------------------------------
%-------------------------------------------------
%\vspace{1cm} 
%\newpage

%------------------------------------------------
%-------------------------------------------------
%---------------------------------------------------
%-------------------------------------------------
%-------------------------------------------------
%--------------------------------------------------
%----------------------------------------------------------------------------------------------
%-----------------------------------------------------------------------------------------------
%\section{Calculs de produits}
\vspace{0.2cm}

%------------------------------------------------
%-------------------------------------------------
\begin{exercice}  \; 
Soit $(n,p,i)\in\N^2$ non nuls. Calculer les produits suivants:
\begin{enumerate}
\begin{minipage}[t]{0.3\textwidth}
\item $\ddp \prod\limits_{k=1}^n k$ \; et \; $\ddp \prod\limits_{k=i}^{i+n} k$
%\item $\ddp \prod\limits_{k=i}^{i+n} k$
%\item $\ddp \prod\limits_{k=1}^n (3k+1)$
\item $\ddp \prod\limits_{k=1}^n \exp{\left(\ddp\frac{k}{n}\right)}$ 
\item $\ddp \prod\limits_{k=1}^n \frac{2k}{2k+1}$ 
\end{minipage}
\begin{minipage}[t]{0.6\textwidth}
\item $\ddp \prod\limits_{k=1}^n (4k-2)$ 
\item $\ddp \prod\limits_{k=2}^n \left(  1-\ddp\frac{1}{k^2} \right)$
\item $\ddp \prod\limits_{k=0}^{p-1} \ddp\frac{n-k}{p-k}$. On exprimera le r\'esultat \`{a} l'aide de factorielles.
\end{minipage}
\end{enumerate}
\end{exercice}
%------------------------------------------------
%-------------------------------------------------

%------------------------------------------------
%-------------------------------------------------
\begin{correction} \; 
%Avec la seule m\'ethode est de les \'ecrire sous forme d\'evelopp\'ee avec les ....
\begin{enumerate}
\item \textbf{Calcul de $\mathbf{\ddp \prod\limits_{k=1}^n k}$} $ =\ddp 1\times 2\times 3\times \dots\times (n-1)\times n=\fbox{$n!$}$\\
\noindent \textbf{Calcul de $\mathbf{\ddp \prod\limits_{k=i}^{i+n} k}$:}
$$\begin{array}{lll}
\ddp \prod\limits_{k=i}^{i+n} k&=&i\times (i+1)\times (i+2)\times \dots\times  (i+n)\vsec\\
&=&\ddp\frac{\left\lbrack1\times 2\times 3\times \dots\times (i-1)\right\rbrack\times\left\lbrack i\times (i+1)\times (i+2)\times \dots\times  (i+n)\right\rbrack}{\left\lbrack1\times 2\times 3\times \dots\times (i-1)\right\rbrack}\vsec\\
&=& \ddp\frac{1\times 2\times 3\times \dots\times (i-1)\times i\times (i+1)\times (i+2)\times \dots\times  (i+n)}{1\times 2\times 3\times \dots\times (i-1)} =\fbox{$\ddp\frac{(i+n)!}{(i-1)!}$.}\end{array}$$
%\item  \textbf{Calcul de $\mathbf{\ddp \prod\limits_{k=1}^n (3k+1)}$:} \`{A} ne pas faire.
%---------
\item  \textbf{Calcul de $\mathbf{\ddp \prod\limits_{k=1}^n \exp{\left(\ddp\frac{k}{n}\right)}}$:}\\
\noindent $\ddp \prod\limits_{k=1}^n \exp{\left(\ddp\frac{k}{n}\right)}=
e^{\frac{1}{n}}\times e^{\frac{2}{n}}\times e^{\frac{3}{n}}\times \dots e^{\frac{n-1}{n}}\times e^{\frac{n}{n}}
=e^{\frac{1+2+3+\dots+(n-1)+n}{n}}=
e^{\tiny{ \ddp \sum\limits_{\tiny{k=1}}^n \frac{k}{n}}}=e^{\tiny{  \frac{1}{n} \ddp \sum\limits_{k=1}^n k  }}=\fbox{$e^{\tiny{  \frac{n+1}{2}}}$. }$
%---------
\item  \textbf{Calcul de $\mathbf{\ddp \prod\limits_{k=1}^n \frac{2k}{2k+1}}:$} \\
$$\begin{array}{lll}
\ddp \prod\limits_{k=1}^n \frac{2k}{2k+1} &= & \ddp  \frac{2n \times (2n-2) \times \ldots \times 4\times 2}{(2n+1) \times (2n-1) \times \ldots \times 3 \times 1} \;= \; \frac{(2n \times (2n-2) \times \ldots \times 4\times 2)^2}{(2n+1) \times 2n \times \ldots \times 3\times 2 \times 1} \vsec\\
& =&\ddp  \frac{(2^n n \times (n-1) \times \ldots \times 2 \times 1))^2}{(2n+1)!} \; = \; \fbox{$\ddp \frac{2^{2n} (n!)^2}{(2n+1)!}$}
\end{array}$$
%---------
\item  \textbf{Calcul de $\mathbf{\ddp \prod\limits_{k=1}^n (4k-2)}:$} \\
$$\begin{array}{lll}
\hspace*{-1cm}\ddp \prod\limits_{k=1}^n (4k-2) &= & \ddp \prod\limits_{k=1}^n 2(2k-1) \; = \; 2^n \times  (2n-1)\times(2n-3) \times \ldots \times 3\times1 \vsec\\
& = & \ddp \frac{2^n \times2n \times (2n-1) \times (2n-2) \times \ldots \times 3 \times 2 \times 1}{2n \times (2n-2) \times \ldots \times 4 \times 2} \;= \;  \frac{2^n \times (2n)!}{2^n\times n \times (n-1) \times \ldots \times 2 \times 1} \; = \; \fbox{$\ddp \frac{(2n)!}{n!}$}
\end{array}$$
%---------
\item  \textbf{Calcul de $\mathbf{\ddp \prod\limits_{k=2}^n \left(  1-\ddp\frac{1}{k^2} \right)}$:}\\
\noindent $\ddp \prod\limits_{k=2}^n \left(  1-\ddp\frac{1}{k^2} \right)=\ddp \prod\limits_{k=2}^n \left(  \ddp\frac{k^2-1}{k^2} \right)=\ddp\frac{ \ddp \prod\limits_{k=2}^n (k^2-1)  }{\ddp \prod\limits_{k=2}^n k^2}=\ddp\frac{\ddp \prod\limits_{k=2}^n (k-1)}{\ddp \prod\limits_{k=2}^n k }\times \ddp\frac{\ddp \prod\limits_{k=2}^n (k+1)}{\ddp \prod\limits_{k=2}^n k }=\ddp\frac{\ddp \prod\limits_{k=1}^{n-1} k}{\ddp \prod\limits_{k=2}^n k }\times \ddp\frac{\ddp \prod\limits_{k=3}^{n+1} k}{\ddp \prod\limits_{k=2}^n k }=\fbox{$\ddp\frac{n+1}{2n}$.}$
%---------
\item  \textbf{Calcul de $\mathbf{\ddp \prod\limits_{k=0}^{p-1} \ddp\frac{n-k}{p-k}}$:}\\
\noindent 
$\ddp \prod\limits_{k=0}^{p-1} \ddp\frac{n-k}{p-k}=\ddp\frac{ \ddp \prod\limits_{k=0}^{p-1} (n-k)  }{\ddp \prod\limits_{k=0}^{p-1} (p-k)}
=\ddp\frac{n(n-1)(n-2)\times \dots \times (n-p+1)}{p(p-1)(p-2)\times \dots \times 2\times 1}=
\ddp\frac{n(n-1)(n-2)\times \dots \times (n-p+1)}{p!}.$ \\
On essaye alors d'\'ecrire le num\'erateur avec des factorielles. On obtient :
$n(n-1)(n-2)\times \dots \times (n-p+1)=\ddp\frac{ \left\lbrack n(n-1)(n-2)\times \dots \times (n-p+1) \right\rbrack \times \lbrack     (n-p)\times \dots \times 2\times 1\rbrack}{(n-p)\times \dots \times 2\times 1}=\ddp\frac{n!}{(n-p)!}$. Ainsi on obtient au final que: 
$\ddp \prod\limits_{k=0}^{p-1} \ddp\frac{n-k}{p-k}=\fbox{$\ddp\frac{n!}{p!(n-p)!}=\binom{n}{p}$.}$
\end{enumerate}
\end{correction}
%------------------------------------------------



%------------------------------------------------
%-------------------------------------------------
%\begin{exercice}
%Soit $n\in\N^{\star}$. Calculer $\ddp \sum\limits_{k=1}^n \ln{\left( 1+\ddp\frac{1}{k}  \right)}$.
%\end{exercice}
%------------------------------------------------
%-------------------------------------------------
\begin{exercice} \; 
Soit $n\in\N^{\star}$. Calculer $\ddp \sum\limits_{k=1}^n \ln{\left( \ddp\frac{k}{2}  \right)}$.
\end{exercice}
%------------------------------------------------

%-------------------------------------------------
\begin{correction}  \; 
On cherche \`{a} calculer $S=\ddp \sum\limits_{k=1}^n \ln{\left( \ddp\frac{k}{2}  \right)}$.  D'apr\`{e}s les propri\'et\'es du logarithme n\'ep\'erien, on a: 
$$S=\ln{\left( \ddp \prod\limits_{k=1}^n \frac{k}{2} \right)} = \ln{\left( \ddp \frac{\prod\limits_{k=1}^n k}{\prod\limits_{k=1}^n 2}  \right)} =  \ln{\left( \ddp \frac{n!}{2^n}\right)}.$$
Ainsi on obtient que: \fbox{$S=\ddp \ln{\left( \ddp \frac{n!}{2^n}\right)}$.}
\end{correction}




%-------------------------------------------------

%\section{Calculs de sommes doubles}
\vspace{0.2cm}

%------------------------------------------------
%-------------------------------------------------
\begin{exercice}  \; 
Dans cet exercice, $n, m$ et $p$ sont deux entiers naturels non nuls et $x$ un nombre complexe. Calculer les sommes doubles suivantes:
\begin{enumerate}
\begin{minipage}[t]{0.3\textwidth}
\item $\ddp \sum\limits_{p=0}^{n} \ddp \sum\limits_{q=0}^{m} p(q^2+1)$
\item $\ddp \sum\limits_{i=1}^n\ddp \sum\limits_{j=1}^n 1$ \; et \; $\ddp \sum\limits_{i=1}^n\ddp \sum\limits_{j=1}^i 1$
\item $\ddp \sum\limits_{i=1}^n\ddp \sum\limits_{j=1}^n i2^j$
\end{minipage}
\begin{minipage}[t]{0.3\textwidth}
%\item $\ddp \sum\limits_{i=1}^n\ddp \sum\limits_{j=1}^n \ddp\frac{i^2}{j}$
\item $\ddp \sum\limits_{k=0}^n\ddp \sum\limits_{l=k}^n \ddp\frac{k}{l+1}$
%\item $\ddp \sum\limits_{j=0}^n\ddp \sum\limits_{i=j}^{j+p} (i-j)^2$
%\item $\ddp \sum\limits_{i=1}^n\ddp \sum\limits_{j=1}^i 1$
\item  $\ddp \sum\limits_{i=1}^n\ddp \sum\limits_{j=1}^i x^j$
\item  $\ddp \sum\limits_{k=0}^{n^2}\ddp \sum\limits_{i=k}^{k+2} ki^2$
\end{minipage}
\begin{minipage}[t]{0.3\textwidth}
\item  $\ddp \sum\limits_{j=1}^n\ddp \sum\limits_{i=0}^j \ddp\frac{ x^{i} }{  x^j }$
\item  $\ddp \sum\limits_{i=1}^n\ddp \sum\limits_{j=i}^n \binom{j}{i}$
\end{minipage}
 \end{enumerate}
\end{exercice}

%-------------------------------------------------
\begin{correction}  \; 
\begin{enumerate}
\item \textbf{Calcul de $\mathbf{\ddp \sum\limits_{p=0}^{n} \ddp \sum\limits_{q=0}^{m} p(q^2+1)}$:}
$$\begin{array}{lll} \ddp \sum\limits_{p=0}^{n} \ddp \sum\limits_{q=0}^{m} p(q^2+1)&=&\ddp \sum\limits_{p=0}^{n} \left\lbrack p\ddp \sum\limits_{q=0}^{m} q^2+p\ddp \sum\limits_{q=0}^{m} 1    \right\rbrack\vsec\\
&=& \ddp \sum\limits_{p=0}^{n} \left\lbrack p\ddp\frac{m(m+1)(2m+1)}{6}+p(m+1)    \right\rbrack
=\ddp\frac{m(m+1)(2m+1)}{6}\ddp \sum\limits_{p=0}^{n} p+(m+1)\ddp \sum\limits_{p=0}^{n} p\vsec\\
&=&  \fbox{$\ddp\frac{m(m+1)(2m+1)}{6}\ddp\frac{n(n+1)}{2}+(m+1)\ddp\frac{n(n+1)}{2}.$}\end{array}$$
%--------
\item  \textbf{Calcul de $\mathbf{\ddp \sum\limits_{i=1}^n\ddp \sum\limits_{j=1}^n 1}$ et de $\mathbf{\ddp \sum\limits_{i=1}^n\ddp \sum\limits_{j=1}^i 1}$:}\\
\noindent $\ddp \sum\limits_{i=1}^n\ddp \sum\limits_{j=1}^n 1=\ddp \sum\limits_{i=1}^n\left\lbrack \ddp \sum\limits_{j=1}^n 1   \right\rbrack= \ddp \sum\limits_{i=1}^n\left\lbrack n \right\rbrack=n\ddp \sum\limits_{i=1}^n 1=\fbox{$n^2$.}$\vsec\\
\noindent $\ddp \sum\limits_{i=1}^n\ddp \sum\limits_{j=1}^i 1=\ddp \sum\limits_{i=1}^n\left\lbrack \ddp \sum\limits_{j=1}^i 1\right\rbrack=\ddp \sum\limits_{i=1}^n\left\lbrack i \right\rbrack=\ddp \sum\limits_{i=1}^n i=\fbox{$\ddp\frac{n(n+1)}{2}$.}$
%--------
\item  \textbf{Calcul de $\mathbf{\ddp \sum\limits_{i=1}^n\ddp \sum\limits_{j=1}^n i2^j}$:}\\
\noindent $\ddp \sum\limits_{i=1}^n\ddp \sum\limits_{j=1}^n i2^j=\ddp \sum\limits_{i=1}^n\left\lbrack \ddp \sum\limits_{j=1}^n i2^j \right\rbrack=\ddp \sum\limits_{i=1}^n  \left\lbrack i\ddp \sum\limits_{j=1}^n 2^j \right\rbrack
 =\ddp \sum\limits_{i=1}^n \left\lbrack i\times 2\ddp\frac{1-2^n}{1-2} \right\rbrack =2(2^n-1) \ddp \sum\limits_{i=1}^n i=\fbox{$(2^n-1)n(n+1) $.}$
 %--------
\item  \textbf{Calcul de $\mathbf{\ddp \sum\limits_{k=0}^n\ddp \sum\limits_{l=k}^n \ddp\frac{k}{l+1}}$:}\\
\noindent On commence par essayer de calculer la somme la plus int\'erieure. On n'y arrive pas car on ne conna\^{i}t pas la somme des inverses. Ainsi on va donc commencer par inverser le sens des symboles sommes. On a :
$$\ddp \sum\limits_{k=0}^n\ddp \sum\limits_{l=k}^n \ddp\frac{k}{l+1} \; = \; \sum_{0\leq k \leq l \leq n} \ddp\frac{k}{l+1}  \; = \;  \ddp \sum\limits_{l=0}^n\ddp \sum\limits_{k=0}^l \ddp\frac{k}{l+1}$$
On peut \'egalement d\'etailler les calculs : $\left\lbrace \begin{array}{lllll}
0 & \leq & k & \leq & n\\
k & \leq & l & \leq & n
\end{array}\right.
\Longleftrightarrow
\left\lbrace \begin{array}{lllll}
0 & \leq &l & \leq & n\\
0 & \leq & k & \leq & l.
\end{array}\right.
$
Ainsi on obtient que: $$\begin{array}{lll}
\ddp \sum\limits_{l=0}^n\ddp \sum\limits_{k=0}^l \ddp\frac{k}{l+1}&=& \ddp \sum\limits_{l=0}^n\left\lbrack \ddp\frac{1}{l+1}
 \ddp \sum\limits_{k=0}^l k\right\rbrack
=  \ddp \sum\limits_{l=0}^n\left\lbrack   \ddp\frac{1}{l+1}  \times \ddp\frac{l(l+1)}{2} \right\rbrack
=\ddp\demi  \ddp \sum\limits_{l=0}^n l=\fbox{ $\ddp\frac{n(n+1)}{4}$.}
\end{array}$$
%--------
%\item  \textbf{Calcul de $\mathbf{\ddp \sum\limits_{j=0}^n\ddp \sum\limits_{i=j}^{j+p} (i-j)^2}$:}\\
%\noindent \`{A} ne pas faire, calculs horribles!
%\item  \textbf{Calcul de $\mathbf{\ddp \sum\limits_{i=1}^n\ddp \sum\limits_{j=1}^i 1}$:}\\
%\noindent $\ddp \sum\limits_{i=1}^n\ddp \sum\limits_{j=1}^i 1=\ddp \sum\limits_{i=1}^n\left\lbrack \ddp \sum\limits_{j=1}^i 1\right\rbrack=\ddp \sum\limits_{i=1}^n\left\lbrack i \right\rbrack=\ddp \sum\limits_{i=1}^n i=\fbox{$\ddp\frac{n(n+1)}{2}$.}$
%--------
\item  \textbf{Calcul de $\mathbf{\ddp \sum\limits_{i=1}^n\ddp \sum\limits_{j=1}^i x^j}$:} 
\begin{itemize}
\item[$\bullet$] Si $x=1$, on a : $\ddp \sum\limits_{i=1}^n\ddp \sum\limits_{j=1}^i x^j = \ddp \sum\limits_{i=1}^n\ddp \sum\limits_{j=1}^i 1 = \ddp \sum\limits_{i=1}^n i = \frac{n(n+1)}{2}.$ 
\item[$\bullet$] Si $x\not= 1$:
\noindent  $\ddp \sum\limits_{i=1}^n\ddp \sum\limits_{j=1}^i x^j=\ddp \sum\limits_{i=1}^n\left\lbrack \ddp \sum\limits_{j=1}^i x^j \right\rbrack=\ddp \sum\limits_{i=1}^n\left\lbrack x\ddp\frac{1-x^{i}  }{1-x} \right\rbrack
= \ddp\frac{x}{1-x}  \ddp \sum\limits_{i=1}^n (1-x^{i}) =\ddp\frac{x}{1-x}  \left\lbrack \ddp \sum\limits_{i=1}^n  1 -  \ddp \sum\limits_{i=1}^n  x^i   \right\rbrack$. \\
Ainsi on obtient que: \fbox{$\ddp \sum\limits_{i=1}^n\ddp \sum\limits_{j=1}^i x^j= \ddp\frac{x}{1-x}  \left\lbrack  n-x\ddp\frac{1-x^n}{1-x}    \right\rbrack $.}
\end{itemize}
%----
\item  \textbf{Calcul de $\mathbf{\ddp \sum\limits_{k=0}^{n^2}\ddp \sum\limits_{i=k}^{k+2} ki^2}$:}
$$\begin{array}{lll}
\ddp \sum\limits_{k=0}^{n^2}\ddp \sum\limits_{i=k}^{k+2} ki^2 &=& \ddp \sum\limits_{k=0}^{n^2} \left\lbrack k \ddp \sum\limits_{i=k}^{k+2} i^2 \right\rbrack
= \ddp \sum\limits_{k=0}^{n^2} \left\lbrack k \left(  k^2+(k+1)^2+(k+2)^2  \right) \right\rbrack
=\ddp \sum\limits_{k=0}^{n^2}  k (3k^2+6k+5)\vsec\\
&=& 3\ddp \sum\limits_{k=0}^{n^2} k^3  +6\ddp \sum\limits_{k=0}^{n^2} k^2+5\ddp \sum\limits_{k=0}^{n^2} k
=\fbox{$3\left(  \ddp\frac{ n^2(n^2+1) }{ 2 }  \right)^2+n^2(n^2+1)(2n^2+1)+5\ddp\frac{n^2(n^2+1)}{2} $.}\end{array}$$
%----
\item  \textbf{Calcul de $\mathbf{\ddp \sum\limits_{j=1}^n\ddp \sum\limits_{i=0}^j \ddp\frac{ x^{i} }{  x^j }}$:}
\begin{itemize}
\item[$\bullet$] Si $x=1$, on a : $\ddp \sum\limits_{j=1}^n\ddp \sum\limits_{i=0}^j \ddp\frac{ x^{i} }{  x^j } = \sum\limits_{j=1}^n\ddp \sum\limits_{i=0}^j 1 = \sum\limits_{j=1}^n j = \frac{n(n+1)}{2}.$ 
\item[$\bullet$] Si $x\not= 1$ :
\noindent $\ddp \sum\limits_{j=1}^n\ddp \sum\limits_{i=0}^j \ddp\frac{ x^{i} }{  x^j }=\ddp \sum\limits_{j=1}^n\left\lbrack x^{-j} \ddp \sum\limits_{i=0}^j x^{i}  \right\rbrack=
\ddp \sum\limits_{j=1}^n\left\lbrack x^{-j}  \ddp\frac{1-x^{j+1}}{1-x}  \right\rbrack= \ddp\frac{1}{1-x}\ddp \sum\limits_{j=1}^n \left\lbrack \left( \ddp\frac{1}{x}\right)^j-x  \right\rbrack$
\begin{center}
\fbox{$\ddp \sum\limits_{j=1}^n\ddp \sum\limits_{i=0}^j \ddp\frac{ x^{i} }{  x^j } =  \ddp\frac{1}{1-x}  \left\lbrack \ddp\frac{1-\left( \frac{1}{x}\right)^n}{x-1}-xn   \right\rbrack.$}
\end{center}

\end{itemize}
%--------
\item  \textbf{Calcul de $\mathbf{\ddp \sum\limits_{i=1}^n\ddp \sum\limits_{j=i}^n \binom{j}{i}}$:}\\
\noindent On commence par essayer de calculer la somme la plus int\'erieure. On n'y arrive pas. Ainsi on va donc commencer par inverser le sens des symboles sommes. On a :
$$\sum\limits_{i=1}^n\ddp \sum\limits_{j=i}^n \binom{j}{i} \;=\; \sum\limits_{1\leq i \leq j \leq n} \binom{j}{i} \;=\;\ddp \sum\limits_{j=1}^n\ddp \sum\limits_{i=1}^j \binom{j}{i}$$
On peut \'egalement d\'etailler les calculs :
$\left\lbrace \begin{array}{lllll}
1 & \leq & i & \leq & n\\
i & \leq & j & \leq & n
\end{array}\right.
\Longleftrightarrow
\left\lbrace \begin{array}{lllll}
1 & \leq &j & \leq & n\\
1 & \leq & i & \leq & j.
\end{array}\right.
$
Ainsi on obtient que: 
$\ddp \sum\limits_{j=1}^n\ddp \sum\limits_{i=1}^j \binom{j}{i}=
\ddp \sum\limits_{j=1}^n \left\lbrack  \ddp \sum\limits_{i=1}^j \binom{j}{i}  \right\rbrack=
\ddp \sum\limits_{j=1}^n \left\lbrack 2^j-1 \right\rbrack=\fbox{$2(2^n-1)-n.$}
$
\end{enumerate}
\end{correction}


% \section{Récurrences}
% \begin{exercice}

% Comparer (avec une inégalité large) pour tout $n\in \N$, les nombres $n$ et $ 3^n$. (Prouver cette inégalité) 
% \end{exercice}
% \begin{correction}



% On  va montrer par récurrence que $\mathcal{P}(n) : n\leq 3^n$ pour tout $n\in \N$. 



% \textbf{Initialisation:}  Pour $n=0$, on a bien $0\leq 3^0=1$. La propriété $\cP$ est donc vraie au rang $0$. 

%  \textbf{H\'er\'edit\'e:}\\
% Soit $n\geq 0$ fix\'e. On suppose la propri\'et\'e vraie \`a l'ordre $n$. Montrons qu'alors $\mathcal{P}(n+1)$ est vraie.\\

% On a par hypothèse de récurrence: 
% \begin{equation*}
% n+1\leq 3^n +1 \\
% \end{equation*}
% Or  pour tout $n\in \N$ , $1 \leq 2\times 3^n$ donc 
% $$3^n+1 \leq 3 \times 3^n =3^{n+1}$$

% La propriété $\cP$ est donc vraie au rang $n+1$.

% \textbf{Conclusion:}\\
% Il r\'esulte du principe de r\'ecurrence que pour tout $ n\geq 0$:
% \begin{center}
% \fbox{$\mathcal{P}(n):n\leq 3^n$}
% \end{center}
% \end{correction}
% \begin{correction}


















% \begin{exercice}
% On considère la suite $\suite{u}$ définie par $u_0=1$, $u_1= 3$  et pour tout $n\in \N$:
% $$u_{n+2} =3u_{n+1} -2u_n$$
% \begin{enumerate}
% \item Enoncer l'inégalité triangulaire.
% \item Montrer que  : $\forall n\in \N,\, |u_n| \leq 4^n$. (On pourra faire apparaitre une hypothèse de récurrence sur $u_n$ et $u_{n+1}$.)
% \end{enumerate}
% \end{exercice}
% \vspace{1cm}




% \begin{enumerate}
% \item Cf cours $\forall x, y \in \R^2$:
% $$|x+y|\leq |x|+|y|.$$
% \item On  va montrer par récurrence que $\mathcal{P}(n) : |u_n| \leq 4^n$  et $|u_{n+1}| \leq 4^{n+1}$. 



% \textbf{Initialisation:}  Pour $n=0$, on a $|u_0| = 1\leq 4^0=1$ et $|u_1|=3 \leq 4^1$. 
% La propriété est vraie au rang $0$. 
 
%  \textbf{H\'er\'edit\'e:}\\
% Soit $n\geq 0$ fix\'e. On suppose la propri\'et\'e vraie \`a l'ordre $n$. Montrons qu'alors $\mathcal{P}(n+1)  :$ \og  $|u_{n+1}| \leq 4^{n+1}$  et $|u_{n+2}| \leq 4^{n+2}$ \fg{}  est vraie.\\

% $u_{n+1}\leq 4^{n+1}$ par hypothèse de récurrence. Il suffit donc de montrer que $|u_{n+2}| \leq 4^{n+2}$. On a 
% \begin{align*}
% |u_{n+2}|&= |3u_{n+1} -2u_n| \quad \text{ par définition de $\suite{u}$}\\
% 				&= |3u_{n+1}|+|2u_n| \quad \text{ par l'inégalité triangulaire}\\
% 								&\leq  3*4^{n+1}+2*4^n \quad \text{ par hypothèse de récurrence}\\
% 								&\leq  4^{n}(3*4+2) \\
% 								&\leq  4^{n}(14) \\
% 								&\leq  4^{n}(4^2) \\								
% 								&\leq  4^{n+2} \\																
% \end{align*}

% La propriété $\cP$ est donc vraie au rang $n+1$.

% \textbf{Conclusion:}\\
% Il r\'esulte du principe de r\'ecurrence que pour tout $ n\geq 0$:
% \begin{center}
% \fbox{$\mathcal{P}(n): |u_n| \leq 4^n$ }
% \end{center}




% \end{enumerate}




% \end{correction}







\section*{Type DS}

\begin{exercice}
%\paragraph{Exercice 1 : Suite de Fibonacci}
Soit $\suite{F}$ la suite définie par $F_0 =0, \, F_1=1 $ \text{ et pour tout $n \geq 0 $ \,} 
$$ F_{n+2} = F_{n+1} +F_n.$$

\begin{enumerate}
\item Montrer que pour tout $n\in \N$ on a : $\ddp \sum_{k=0}^n F_{2k+1} =F_{2n+2}$
et $\ddp \sum_{k=0}^n F_{2k} =F_{2n+1}-1$.
\item Montrer que pout tout $n\in \N$ on a $\ddp \sum_{k=0}^n F_{k}^2 =F_nF_{n+1}$.
\item \begin{enumerate}
\item On note $\varphi = \frac{1+\sqrt{5}}{2}$ et $\psi=\frac{1-\sqrt{5}}{2}$. Montrer que 
$\varphi^2 =\varphi+1$ et $\psi^2 =\psi+1$.
\item Montrer que l'expression explicite de $F_n$ st donnée par $F_n =\frac{1}{\sqrt{5}}(\varphi^n-\psi^n)$.
\item En déduire que $\ddp \lim_{n\tv \infty} \frac{F_{n+1}}{F_n}=\varphi.$
\end{enumerate}
\end{enumerate}

\end{exercice}


\begin{correction}
\begin{enumerate}
\item Nous allons montrer ces propriétés par récurrence sur l'entier $n\in \N$. 
Soit $\cP(n)$ la prorpriété définie pour tout $n\in\N$ par:
$$\cP(n) := \text{ \og }\ddp  \sum_{k=0}^n F_{2k+1} =F_{2n+2} \text{ 
et } \ddp \sum_{k=0}^n F_{2k} =F_{2n+1}-1 \text{ \fg }.$$
Montrons  $\cP(0)$. Vérifions la première égalité : 
$$\sum_{k=0}^0 F_{2k+1} =F_{0+1}=F_1=1$$
et 
$$F_2 =F_1+F_0 =1$$
Donc la première égalité est vraie au rang $0$. 

Vérifions la sedonde égalité : 
$$\sum_{k=0}^0 F_{2k} =F_{0}=0$$
et 
$$F_{2*0+1}-1 =F_1-1=0$$
Donc la seconde égalité est vraie au rang $0$. 
Ainsi $\cP(0)$ est vraie. 


 \textbf{H\'er\'edit\'e:}\\
Soit $n\geq 0$ fix\'e. On suppose la propri\'et\'e vraie \`a l'ordre $n$. Montrons qu'alors $\mathcal{P}(n+1)$ est vraie.\\

Considérons la première égalité de $\cP(n+1)$. Son membre de gauche vaut : 
\begin{equation*}
 \sum_{k=0}^{n+1} F_{2k+1}=  \sum_{k=0}^n F_{2k+1} +F_{2n+3}
\end{equation*}
Par hypothèse de récurrence on a $ \ddp \sum_{k=0}^n F_{2k+1} = F_{2n+2}$, donc 
\begin{align*}
 \sum_{k=0}^{n+1} F_{2k+1} & =F_{2n+2}+F_{2n+3}.\\
								& = F_{2n+4}.\quad \text{ d'après la définition de $\suite{F}$}\\
								& = F_{2(n+1)+2}.\\
\end{align*}
La première égalité est donc héréditaire. 


Considérons la sedonde égalité de $\cP(n+1)$. Son membre de gauche vaut : 
\begin{equation*}
 \sum_{k=0}^{n+1} F_{2k}=  \sum_{k=0}^n F_{2k} +F_{2n+2}
\end{equation*}
Par hypothèse de récurrence on a $ \ddp \sum_{k=0}^n F_{2k} = F_{2n+1}-1$, donc 
\begin{align*}
 \sum_{k=0}^{n+1} F_{2k} & =F_{2n+1}-1+F_{2n+2}.\\
								& = F_{2n+3}-1.\quad \text{ d'après la définition de $\suite{F}$}\\
								& = F_{2(n+1)+1}-1.\\
\end{align*}
La seconde égalité est donc héréditaire. Finalement la propriété $\cP(n+1)$ est vraie. 


\textbf{Conclusion:}\\
Il r\'esulte du principe de r\'ecurrence que pour tout $ n\geq 0$:
\begin{center}
\fbox{$\ddp \sum_{k=0}^n F_{2k+1} =F_{2n+2} \text{ 
et } \ddp \sum_{k=0}^n F_{2k} =F_{2n+1}-1 $}
\end{center}


\item On  va montrer par récurrence que $\mathcal{P}(n) :\ddp \sum_{k=0}^n F_{k}^2 =F_{n}F_{n+1}$. 



\textbf{Initialisation:}  Pour $n=0$, on a $\sum_{k=0}^0 F_{k}^2 = F_0^2=0$ et $F_0F_1=0$. 
La propriété est donc vraie au rang $0$. 
 
 \textbf{H\'er\'edit\'e:}\\
Soit $n\geq 0$ fix\'e. On suppose la propri\'et\'e vraie \`a l'ordre $n$. 

On a $\ddp \sum_{k=0}^{n+1} F_{k}^2 =\sum_{k=0}^{n} F_{k}^2 +F_{n+1}^2$
Par hypothèse de récurrence on a $\sum_{k=0}^{n} F_{k}^2 = F_n F_{n+1}$ donc : 
\begin{align*}
 \sum_{k=0}^{n+1} F_{k}^2 &=  F_n F_{n+1} +F_{n+1}^2\\
				&= F_{n+1} (F_n +F_{n+1}) \\
				&=   F_{n+1}F_{n+2}  \quad \text{ par définition de $\suite{F}$ }													
\end{align*}

La propriété $\cP$ est donc vraie au rang $n+1$.

\textbf{Conclusion:}\\
Il r\'esulte du principe de r\'ecurrence que pour tout $ n\geq 0$:
\begin{center}
\fbox{$\mathcal{P}(n): \ddp \sum_{k=0}^n F_{k}^2 =F_{n}F_{n+1}$}
\end{center}

\item Le polynôme du second degrès $X^2-X-1$ a pour discriminant $\Delta =1+4=5$ les racines sont donc 
$\varphi = \frac{1+\sqrt{5}}{2}$ et $\psi=\frac{1-\sqrt{5}}{2}$. 
En particulier, ces nombres vérifient : $\varphi^2 -\varphi -1 =0$ et $\psi^2 -\psi-1=0$, c'est-à-dire 

\begin{center}
\fbox{$\varphi^2 =\varphi+1$ et $\psi^2 =\psi+1$.}
\end{center}





\item  Notons  :$u_n =\frac{1}{\sqrt{5}}(\varphi^n-\psi^n)$ 
On a 
$$u_0= \frac{1}{\sqrt{5}}(\varphi^0-\psi^0)=0$$
$$u_1= \frac{1}{\sqrt{5}}(\varphi^1-\psi^1)=1$$
et pour tout $n\in\N$ on a 
\begin{align*}
u_{n+2} &= \frac{1}{\sqrt{5}}(\varphi^{n+2}-\psi^{n+2}) \\
			&= \frac{1}{\sqrt{5}}(\varphi^n (\varphi^2)-\psi^n (\psi^2) ) \\
			&= \frac{1}{\sqrt{5}}(\varphi^n (\varphi +1)-\psi^n (\psi +1)  ) \quad \text{ D'après la question précédente} \\			
			&= \frac{1}{\sqrt{5}}(\varphi^{n+1} +\varphi^n-\psi^{n+1} -\psi^n   ) \\
			&= \frac{1}{\sqrt{5}}(\varphi^{n+1} -\psi^{n+1}) +  \frac{1}{\sqrt{5}} \varphi^n-\psi^n   ) \\
			&=u_{n+1}+u_n
\end{align*}
Donc $u_n$ satisfait aussi la relation de récrurrence. 
Ainsi  pour tout $n\in \N$, $u_n=F_n= \frac{1}{\sqrt{5}}(\varphi^n-\psi^n)$. 


\item D'après la question précédente on a pour tout $n\in \N$: 
$$\frac{F_{n+1}}{F_n} =  \frac{\varphi^{n+1}-\psi^{n+1}}{\varphi^n-\psi^n}$$
Donc,
\begin{align*}
\frac{F_{n+1}}{F_n} &=\varphi \frac{\varphi^{n}\left(1-\frac{\psi^{n+1}}{\varphi^{n+1}}\right)}{\varphi^n\left(1-\frac{\psi^n}{\varphi^n}\right)}\\
&=\varphi \frac{1-\left(\frac{\psi}{\varphi}\right)^{n+1}}{1-\left(\frac{\psi}{\varphi}\right)^n}
\end{align*}


Remarquons que $|\varphi| >|\psi|$ en particulier $|\frac{\psi}{\varphi}|<1$ et donc 
$$\lim_{n\tv \infty} \left(\frac{\psi}{\varphi}\right)^{n+1} =0.$$
Finalemetn 
\begin{center}
\fbox{$ \lim_{n\tv \infty} \frac{F_{n+1}}{F_n} =\varphi$.} 
\end{center}

\end{enumerate}

\end{correction}

%%%%%
%%%%%
%%%%%
%%%%%
%%%%%
%%%%%
%%%%%

\vspace{0.5cm}
\begin{exercice}
Dans cet exercice, on considère une suite quelconque de nombres réels $\suite{a}$, et on pose pour tout $n\in \N$:
$$b_n =\sum_{k=0}^n \binom{n}{k} a_k.$$
\begin{center}
\textbf{Partie I : Quelques exemples}
\end{center}
\begin{enumerate}
\item Calculer $b_n$ pour tout $n \in \N$ lorsque la suite $\suite{a}$ est la suite constante égale à $1$.
\item Calculer $b_n$ pour tout $n \in \N$ lorsque la suite $\suite{a}$ est définie par $a_n=\exp(n)$. 

\item 
\begin{enumerate}
\item Démontrer que, pour tout $(n\geq 1,n\geq k\geq 1)$, $$k\binom{n}{k}=n \binom{n-1}{k-1}.$$
\item En déduire que : $\forall n \in \N, \ddp \sum_{k=0}^n \binom{n}{k} k = n2^{n-1}$.
\item Calculer la valeur de $b_n$, pour tout $n \in \N$ lorsque la suite $\suite{a}$ est définie par $a_n=\frac{1}{n+1}$. 
\end{enumerate}
\end{enumerate}
\begin{center}
\textbf{Partie II : Formule d'inversion }
\end{center}
Le  but de cette partie est de montrer que la suite $\suite{a}$ s'exprime en fonction de la suite $\suite{b}$. 
\begin{enumerate}
\item Montrer que pour tout $(k, n , p ) \in \N^3,$ tel que $k\leq p \leq n$ on  a:
$$\binom{n+1}{p}\binom{p}{k}=\binom{n+1}{k}\binom{n+1-k}{p-k}.$$
\item Montrer que, pour tout $(k,n) \in \N^2$, tel que $k\leq n$ on  a :
$$\sum_{i=0}^{n-k} (-1)^i  \binom{n+1-k}{i}=(-1)^{n-k}.$$ 
\item  Montrer que pour tout $n\in \N$ on a $$\sum_{p=0}^{n}\sum_{k=0}^p  \binom{n+1}{k}\binom{n+1-k}{p-k} (-1)^{p-k}  b_k	 = \sum_{k=0}^{n} (-1)^{n-k}  \binom{n+1}{k} b_k $$
\item Donner, pour tout $n\in \N$, l'expression de $a_{n+1}$ en fonction de $b_{n+1}$ et de $a_0, ..., a_n$. 
\item Prouver, par récurrence forte sur $n$ que :
$$\forall n \in \N, a_n =\sum_{k=0}^n (-1)^{n-k} \binom{n}{k} b_k.$$
\item En utilisant le résultat précédent montrer que pour tout $n\in \N$:
$$\sum_{k=0}^{n}   \binom{n}{k}k2^k(-1)^{n-k}=2n.$$ 
\end{enumerate}




\end{exercice}

\begin{correction}
\begin{center}
\textbf{Partie I : Quelques exemples}
\end{center}
\begin{enumerate}
\item Pour $a_n=1$, $b_n=\sum_{k=0}^n \binom{n}{k}  =2^n$.
\item  Pour $a_n=exp(n)$, $b_n=\sum_{k=0}^n \binom{n}{k}e^k  =(1+e^1)^n$.
\item
\begin{enumerate}
\item  $$k\binom{n}{k}= k \frac{n! }{k! (n-k)!} = \frac{n! }{(k-1)! (n-k)!}  = n \frac{(n-1)! }{(k-1)! ((n-1)-(k-1))!}=  n\binom{n-1}{k-1}$$
\item
Comme le premier terme est nul  $\sum_{k=0}^n \binom{n}{k} k = \sum_{k=1}^n n\binom{n-1}{k-1}$
Et d'après la question précédente on a donc $ \sum_{k=0}^n \binom{n}{k} k= n  \sum_{k=1}^n \binom{n-1}{k-1}$
Or en faisant un changement de variable on obtient $\sum_{k=1}^n \binom{n-1}{k-1}= \sum_{k=0}^{n-1} \binom{n-1}{k}$. 
Donc $$ \sum_{k=0}^n \binom{n}{k} k = n 2^{n-1}$$

\item 
D'après la question 3a) on a $ (k+1)\binom{n+1}{k+1} = (n+1)\binom{n}{k}$. Donc 
$$\frac{1}{n+1}\binom{n+1}{k+1} =\frac{1}{k+1}\binom{n}{k}$$

Ainsi $$\sum_{k=0}^n \binom{n}{k} \frac{1}{k+1} = \sum_{k=0}^{n} \frac{1}{n+1}\binom{n+1}{k+1}$$
On fait un changement de variable $k+1=j$ on obtient 
\begin{align*}
\sum_{k=0}^n \binom{n}{k} \frac{1}{k+1} &=  \sum_{j=1}^{n+1} \frac{1}{n+1}\binom{n+1}{j}\\
															&=  \frac{1}{n+1} \left( \sum_{j=0}^{n+1} \binom{n+1}{j} -1\right)\\
															&= \frac{1}{n+1} \left( 2^{n+1}-1\right)
\end{align*}
\end{enumerate}
\end{enumerate}



\begin{center}
\textbf{Partie II : Formule d'inversion }
\end{center}
\begin{enumerate}
\item C'est l'exercice 2 du DM 4.
\item 
$$\sum_{i=0}^{n-k} (-1)^i  \binom{n+1-k}{i}= \sum_{i=0}^{n-k+1} (-1)^i  \binom{n+1-k}{i} -(-1)^{n-k+1} $$ 
Et d'après le BdN : 
$$ \sum_{i=0}^{n-k+1} (-1)^i  \binom{n+1-k}{i} =(1-1)^{n-k} =0$$ 
et 
$$-(-1)^{n-k+1} =(-1)^{n-k}$$
Ce qui donne le résultat. 
\item $b_{n+1} = \ddp \sum_{k=0}^{n+1} \binom{n+1}{k} a_k =\ddp  \sum_{k=0}^{n} \binom{n+1}{k} a_k   +a_{n+1}$. Donc
$$a_{n+1} = b_{n+1}-\sum_{k=0}^{n} \binom{n+1}{k} a_k$$
\item Soit $P(n)$ la propriété : " $\forall p \leq n \, a_p =\ddp \sum_{k=0}^p (-1)^{p-k} \binom{p}{k} b_p.$"\\

Montrons $P(0) : " \forall j \leq 0 \, a_j =\ddp \sum_{k=0}^j (-1)^{j-k} \binom{j}{k} b_k.$" Il suffit  de vérifier $a_0 = \ddp \sum_{k=0}^0 (-1)^{0-k} \binom{0}{k} b_k.$

Et on a  $\ddp \sum_{k=0}^0 (-1)^{0-k} \binom{0}{k} b_k. = b_0$ 
Par ailleurs, par définition $b_0 =  \ddp  \sum_{k=0}^{0} \binom{0}{k} a_k =a_0$. Ainsi $P(0)$ est vraie.  

\underline{Hérédité}


On suppose que $P$ est vraie pour un certain entier naturel $n$ fixé. Montrons $P(n+1)$. 
Pour cela il suffit de vérifier que 
$$a_{n+1} =\sum_{k=0}^{n+1} (-1)^{n+1-k} \binom{n+1}{k} b_k$$
Or on a vu que  
$$a_{n+1}=b_{n+1}-\sum_{p=0}^{n} \binom{n+1}{p}  a_p$$ 
et en utilisant l'hypothése de récurrence on obtient 
\begin{align*}
\sum_{p=0}^{n} \binom{n+1}{p} a_p &=\sum_{p=0}^{n} \binom{n+1}{p} \sum_{k=0}^p (-1)^{p-k} \binom{p}{k} b_k\\
												&=\sum_{p=0}^{n}\sum_{k=0}^p  \binom{n+1}{p} \binom{p}{k} (-1)^{p-k}  b_k\\
												&=\sum_{p=0}^{n}\sum_{k=0}^p  \binom{n+1}{k}\binom{n+1-k}{p-k} (-1)^{p-k}  b_k												
\end{align*}
D'après la question  II . 1. 


On échange les deux symboles sommes on obtient : 
\begin{align*}
\sum_{p=0}^{n}\sum_{k=0}^p  \binom{n+1}{k}\binom{n+1-k}{p-k} (-1)^{p-k}  b_k		&=
\sum_{k=0}^{n}\sum_{p=k}^n  \binom{n+1}{k}\binom{n+1-k}{p-k} (-1)^{p-k}  b_k	\\
&=\sum_{k=0}^{n}   \binom{n+1}{k} b_k \sum_{p=k}^n \binom{n+1-k}{p-k} (-1)^{p-k}  	\\
&= \sum_{k=0}^{n}   \binom{n+1}{k} b_k \sum_{i=0}^{n-k} \binom{n+1-k}{i} (-1)^{i}  	
\end{align*}
En faisant le cahngement d'indice $p-k=i$. 

On obtient finalement  en utilisant la question  II. 2.
\begin{align*}
\sum_{p=0}^{n} \binom{n+1}{p} a_p &=\sum_{k=0}^{n}   \binom{n+1}{k} b_k (-1)^{n-k}
\end{align*}
On conclut en remarquant que $b_{n+1} 	= 	 \binom{n+1}{n+1} b_{n+1} (-1)^{n+1-(n+1)}$
et ainsi 
$$a_{n+1}= 	 (-1)^{n+1-(n+1)}  \binom{n+1}{n+1} b_{n+1} + \sum_{k=0}^{n}  (-1)^{n+1-k} \binom{n+1}{k} b_k = \sum_{k=0}^{n+1}  (-1)^{n+1-k} \binom{n+1}{k} b_k $$
									
											 

\item On a vu dans la partie I que pour $a_n=n$ on a
$b_n=n2^{n-1}$. Donc en appliquant le résultat précédent on a 
$$ \sum_{k=0}^{n}   \binom{n}{k}k2^{k-1}(-1)^{n-k} = n$$ 
Ce qui donne finalement 
$$ \sum_{k=0}^{n}   \binom{n}{k}k2^k(-1)^{n-k} =  2 \sum_{k=0}^{n}   \binom{n}{k}k2^{k-1}(-1)^{n-k}= 2 n$$
\end{enumerate}


\end{correction}







%------------------------------------------------------------------------------------
%------------------------------------------------------------------------------------
%------------------------------------------------------------------------------------
%------------------------------------------------------------------------------------



%\subsection{Somme double $\max$ + info}

\begin{exercice}
\begin{enumerate}
\item Montrer par récurrence que pour tout $n\in \N$, 
$$ \sum_{k=1}^n  k^2= \frac{n(n+1)(2n+1)}{6}$$

\item Soit $i\in \N$ et $n\in \N$ tel que $i\leq n$. Caculer en fonction de $i$ et $n$ :
$$\sum_{j=i+1}^n j$$

\item 
On rappelle que l'on note $\max(i,j) =  \left\{
    \begin{array}{ll}
       i & \text{si\, } i \geq j \\
        j & \mbox{sinon.}
    \end{array}
\right.
$
Montrer que pour tout $n\in \N$, $$\ddp \sum_{i=1}^n \sum_{j=1}^n \max(i,j)  = \sum_{i=1}^n \frac{n^2+i^2 +n-i}{2}$$ 
\item En déduire que 
$$\ddp\sum_{i=1}^n \sum_{j=1}^n  \max(i,j) = \left(   \frac{n(n+1)(4n-1)}{6} \right)	 $$

\item On note $$S_k =  \sum_{i,j\in \intent{1,1000}} \max(i^k,j^k).$$ 
\begin{enumerate}
\item Rappeler ce que renvoie l'instruction Python $\texttt{range(a,b)}$ avec deux entiers $a,b\in \N$ tel que $a\leq b$.
\item Ecrire un script Python qui demande à l'utilsateur la valeur de $k$, calcul $S_k$ et affiche le résultat. 
\end{enumerate}
\end{enumerate}
\end{exercice}



\begin{correction}
Pour $n=0 $ on a d'une part $\sum_{k=1}^0 k^2 =0$ et 
$\frac{0*1*(2*0+1)}{6}=0$ la propriété est vraie au rang 0

Montrons l'hérédité de la formule et supposons qu'il existe $n\in \N$ tel que $ \sum_{k=1}^n  k^2= \frac{n(n+1)(2n+1)}{6}$
On a 
$$\sum_{k=1}^{n+1} k^2 =\sum_{k=1}^{n} k^2 +  (n+1)^2$$
et donc par hypothèse : 
\begin{align*}
\sum_{k=1}^{n+1} k^2 &=\frac{n(n+1)(2n+1)}{6} +  (n+1)^2\\
									&= \frac{(n+1) [ n(2n+1) +6(n+1)}{(n+1)^2}\\
									&= \frac{(n+1) [ 2n^2+ +7n+6)}{(n+1)^2}		\\												&= \frac{(n+1) [ (2n+3)(n+2)}{(n+1)^2}				\\											&= \frac{(n+1) ((n+1)+1) (2(n+1)+1)}{(n+1)^2}			
\end{align*}
La formule est  bien héréditaire elle est donc vraie pour tout $n\in \N$. 



D'après le cours :
$\ddp \sum_{j=i+1}^n j = \frac{(n+i+1)(n-i)}{2}$
\begin{align*}
\sum_{i,j \in \intent{1,n}} \max(i,j) &= \sum_{i=1}^n \sum_{j=1}^n\max(i,j)\\
												&= \sum_{i=1}^n \sum_{j=1}^i\max(i,j) + \sum_{i=1}^n \sum_{j=i+1}^n\max(i,j)\\
												&= \sum_{i=1}^n \sum_{j=1}^i i  + \sum_{i=1}^n \sum_{j=i+1}^n j\\
													&= \sum_{i=1}^n i^2+ \sum_{i=1}^n \frac{(n+i+1)(n-i)}{2} \\
													&= \sum_{i=1}^n i^2 + \frac{n^2-i^2 +n-i}{2}\\		
													&= \sum_{i=1}^n \frac{n^2+i^2 +n-i}{2}\\	
\end{align*}
\begin{align*}
\sum_{i,j \in \intent{1,n}} \max(i,j)	&= \frac{1}{2} \left( n (n^2+n) + \sum_{i=1}^n i^2  -  \sum_{i=1}^n i\right)\\
													&= \frac{1}{2} \left(  n^2 (n+1) +  \frac{n(n+1)(2n+1)}{6} -   \frac{n(n+1)}{2}\right)\\
													&=\frac{1}{2} \left(   \frac{n(n+1)(6n +(2n+1) -3 }{6} \right)\\
													&=\frac{1}{2} \left(   \frac{n(n+1)(8n-2)}{6} \right)\\		
													&= \left(   \frac{n(n+1)(4n-1)}{6} \right)											
\end{align*}

\texttt{range(a,b)} renvoie la suite d'entiers de $a$ à $b-1$. 


\begin{lstlisting}
k = int(input ('quelle est la valeur de k ? ))
S=0
for i in range(1,1001):  #on fait une boucle for pour obtenir la somme sur i 
  for j in range(1,1001):  #et une deuxieme pour la somme sur j
    if i<=j:
      S=S+j**k
    else:
      S=S+i**k
print(S)
\end{lstlisting}


\end{correction}



%------------------------------------------------------------------------------------
%------------------------------------------------------------------------------------
%------------------------------------------------------------------------------------
%------------------------------------------------------------------------------------
%------------------------------------------------------------------------------------


%------------------------------------------------------------------------------------





%------------------------------------------------------------------------------------
%------------------------------------------------------------------------------------
%------------------------------------------------------------------------------------
%------------------------------------------------------------------------------------
%------------------------------------------------------------------------------------


\end{document}