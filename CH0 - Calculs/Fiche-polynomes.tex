\documentclass[a4paper, 11pt]{article}
\input{macro/package.tex}
\input{macro/environement}
% Header et footer

\pagestyle{fancy}
\fancyhead{}
\fancyfoot{}
\renewcommand{\headwidth}{\textwidth}
\renewcommand{\footrulewidth}{0.4pt}
\renewcommand{\headrulewidth}{0pt}
\renewcommand{\footruleskip}{5px}

\fancyfoot[R]{Olivier Glorieux}
%\fancyfoot[R]{Jules Glorieux}

\fancyfoot[C]{ Page \thepage }
\fancyfoot[L]{1BIOA - Lycée Chaptal}
%\fancyfoot[L]{MP*-Lycée Chaptal}
%\fancyfoot[L]{Famille Lapin}

\input{macro/newcommand.tex}
\geometry{hmargin=2.0cm, vmargin=1.5cm}




\begin{document}

\title{Fonctions polynomiales}
\begin{defi}
Soit $a$ un réel et soit $f$ la fonction définie pour tout $x\in \R$ par 
$$f(x)= a$$ 
On dit alors que $f$ est la fonction constante égale à $a$.
\end{defi}

\begin{defi}
Soit $a,b$ deux réels  et $a\neq 0$.  Soit $f$ la fonction définie pour tout $x\in \R$ par 
$$f(x)= ax+b$$ 
On dit alors que $f$ est une fonction affine.
On appelle $a$ le coefficient directeur (ou pente) de $f$ et $b$ son ordonnée à l'origine. 
\end{defi}

\begin{prop}
Soit $f(x)=ax+b$ une fonction affine. On a alors 
\begin{itemize}
\item Si $a>0$, $f(x) \geq 0 \equivaut x\geq\frac{-b}{a}$
\item Si $a<0$, $f(x) \geq 0 \equivaut x\leq\frac{-b}{a}$
\end{itemize}

\end{prop}

\begin{defi}
Soit $a,b,c$ trois réels  et $a\neq 0$.  Soit $f$ la fonction définie pour tout $x\in \R$ par 
$$f(x)= ax^2+bx+c$$ 
On dit alors que $f$ est une fonction polynomiale de degré $2$.
\end{defi}

\begin{defi}
On appelle discriminant d'une fonction polynomiale de degré $2$, $f(x)= ax^2+bx+c$,  le nombre souvent noté $\Delta$: 
$$\Delta = b^2 -4ac$$
\end{defi}
\begin{defi}
On appelle racine de $f$ un nombre $r$ tel que $f(r)=0$
\end{defi}
\begin{prop}
Soit $f$ une fonction polynomiale de degré $2$, $f(x)= ax^2+bx+c$. Soit $\Delta$ son discriminant. On a alors :
\begin{itemize}
\item Si $\Delta>0$, $f$ admet deux racines réelles 
$$r_1 = \frac{-b+\sqrt{\Delta} }{2a} \quad r_2 = \frac{-b-\sqrt{\Delta} }{2a}$$
\item Si $\Delta=0$, $f$ admet une unique racine réelle 
$$r = \frac{-b}{2a} $$
\item Si $\Delta<0$, $f$ n'admet aucune racine réelle (mais des racines complexes...)
\end{itemize}
\end{prop}

% \begin{exercice}
% Ecrire un script Python qui permet de résoudre les équations polynomiales de degré 2. 
% \end{exercice}

% \subsection{Fonctions polynomiales de degré $n$}
% \begin{defi}
% Soit $a_0,a_1,a_2, \dots, a_n$, $n+1$ réels avec $a_n\neq 0$.  Soit $f$ la fonction définie pour tout $x\in \R$ par 
% $$f(x)= a_nx^n+a_{n-1}x^{n-1} + \dots +a_1 x +a_0$$ 
% On dit alors que $f$ est une fonction polynomiale de degré $n$.
% \end{defi}

% \begin{defi}(Généralisation)
% On appelle racine de $f$ un nombre $r$ tel que $f(r)=0$
% \end{defi}


























\end{document}
