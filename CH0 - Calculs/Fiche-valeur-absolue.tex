\documentclass[a4paper, 11pt]{article}
\input{macro/package.tex}
\input{macro/environement}
% Header et footer

\pagestyle{fancy}
\fancyhead{}
\fancyfoot{}
\renewcommand{\headwidth}{\textwidth}
\renewcommand{\footrulewidth}{0.4pt}
\renewcommand{\headrulewidth}{0pt}
\renewcommand{\footruleskip}{5px}

\fancyfoot[R]{Olivier Glorieux}
%\fancyfoot[R]{Jules Glorieux}

\fancyfoot[C]{ Page \thepage }
\fancyfoot[L]{1BIOA - Lycée Chaptal}
%\fancyfoot[L]{MP*-Lycée Chaptal}
%\fancyfoot[L]{Famille Lapin}

\input{macro/newcommand.tex}
\geometry{hmargin=2.0cm, vmargin=1.5cm}




\begin{document}



\title{Valeur absolue}





\begin{defi}
La fonction \emph{valeur absolue} est notée $| \cdot | : \R \mapsto\R$ et définie par:
$$|x| = \left\{ \begin{array}{rl}
 x \quad \text{ si }  \, x \geq 0 \vspace{0.3cm}\\  
-x \quad  \text{ si } \,  x< 0\vspace{0.3cm}
\end{array} \right.$$
\end{defi}
\paragraph{Remarque :}
\begin{description}
\item $\bullet$ En particulier, la fonction valeur absolue est toujours positive. De plus,  $|x| =0$ si et seulement si  $x=0$. 
\item $\bullet$  Soit $(x,y) \in \R^2$ alors, $|x-y|$ correspond à la distance entre $x$ et $y$ sur la droite réelle. 
\item $\bullet$ On a  $x \leq |x|$ et $-x\leq |x|$. 
\item $\bullet$  La fonction valeur absolue est continue sur $\R$ mais pas dérivable en $0$. 
\end{description}

\vbox{
\begin{prop}
Soit, $(x,y) \in \R^2$, on a alors :
\begin{enumerate}
\item $|x|=|-x|$
\item $|x y | = |x | |y |$
\item Si $y\neq 0$ alors, $\left|\frac{x}{y}\right| =\frac{|x |}{|y|}$.
\end{enumerate}
\end{prop}
}

\begin{prop}[Inégalité triangulaire \warning]
Soit, $(x,y) \in \R^2$, on a alors :
$$|x+y|\leq |x|+|y|.$$
\end{prop}


\begin{prop}
Pour tout $x\in \R$, $$\sqrt{(x^2)}=|x|$$ 

Pour tout $x\in \R^+$,

$$(\sqrt{x})^2= x =|x|$$
\end{prop}
\begin{prop}

Pour tout $x\in \R$,
$$|x|^2=|x^2|=x^2$$
\end{prop}


\begin{prop}\, 
Soit $a\in \R$  un réel et $\epsilon >0$ un réel strictement positif. Pour tout $x\in \R$ on a 
$$|x-a| \leq \epsilon \Longleftrightarrow a-\epsilon \leq x \leq a+\epsilon \Longleftrightarrow x\in [a-\epsilon, a+\epsilon].$$
\end{prop}
En particulier, pour tout $\epsilon >0$, l'inégalité $| x | \leq \epsilon $ est équivalente à 
$-\epsilon \leq x\leq \epsilon $ ou a  $x\in [-\epsilon , \epsilon ]$.






\vspace{0.3cm}
\fbox{
\begin{minipage}[t]{0.9\textwidth}
\begin{itemize}
%\item[$\bullet$] M\'ethode avec les valeurs absolues: \dotfill \phantom{\hspace{5cm}}
\item[$\bullet$] Pour enlever une valeur absolue, il faut conna\^{i}tre le signe de ce qui est \`{a} l'int\'erieur.\\
$\Longrightarrow$ \'Etude de cas selon le signe de ce qui est \`{a} l'int\'erieur de la valeur absolue.
\end{itemize}
\end{minipage}}






\end{document}
