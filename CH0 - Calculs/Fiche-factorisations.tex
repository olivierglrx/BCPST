\documentclass[a4paper, 11pt]{article}
\usepackage[utf8]{inputenc}
\usepackage{amssymb,amsmath,amsthm}
\usepackage{geometry}
\usepackage[T1]{fontenc}
\usepackage[french]{babel}
\usepackage{fontawesome}
\usepackage{pifont}
\usepackage{tcolorbox}
\usepackage{fancybox}
\usepackage{bbold}
\usepackage{tkz-tab}
\usepackage{tikz}
\usepackage{fancyhdr}
\usepackage{sectsty}
\usepackage[framemethod=TikZ]{mdframed}
\usepackage{stackengine}
\usepackage{scalerel}
\usepackage{xcolor}
\usepackage{hyperref}
\usepackage{listings}
\usepackage{enumitem}
\usepackage{stmaryrd} 
\usepackage{comment}


\hypersetup{
    colorlinks=true,
    urlcolor=blue,
    linkcolor=blue,
    breaklinks=true
}





\theoremstyle{definition}
\newtheorem{probleme}{Problème}
\theoremstyle{definition}


%%%%% box environement 
\newenvironment{fminipage}%
     {\begin{Sbox}\begin{minipage}}%
     {\end{minipage}\end{Sbox}\fbox{\TheSbox}}

\newenvironment{dboxminipage}%
     {\begin{Sbox}\begin{minipage}}%
     {\end{minipage}\end{Sbox}\doublebox{\TheSbox}}


%\fancyhead[R]{Chapitre 1 : Nombres}


\newenvironment{remarques}{ 
\paragraph{Remarques :}
	\begin{list}{$\bullet$}{}
}{
	\end{list}
}




\newtcolorbox{tcbdoublebox}[1][]{%
  sharp corners,
  colback=white,
  fontupper={\setlength{\parindent}{20pt}},
  #1
}







%Section
% \pretocmd{\section}{%
%   \ifnum\value{section}=0 \else\clearpage\fi
% }{}{}



\sectionfont{\normalfont\Large \bfseries \underline }
\subsectionfont{\normalfont\Large\itshape\underline}
\subsubsectionfont{\normalfont\large\itshape\underline}



%% Format théoreme, defintion, proposition.. 
\newmdtheoremenv[roundcorner = 5px,
leftmargin=15px,
rightmargin=30px,
innertopmargin=0px,
nobreak=true
]{theorem}{Théorème}

\newmdtheoremenv[roundcorner = 5px,
leftmargin=15px,
rightmargin=30px,
innertopmargin=0px,
]{theorem_break}[theorem]{Théorème}

\newmdtheoremenv[roundcorner = 5px,
leftmargin=15px,
rightmargin=30px,
innertopmargin=0px,
nobreak=true
]{corollaire}[theorem]{Corollaire}
\newcounter{defiCounter}
\usepackage{mdframed}
\newmdtheoremenv[%
roundcorner=5px,
innertopmargin=0px,
leftmargin=15px,
rightmargin=30px,
nobreak=true
]{defi}[defiCounter]{Définition}

\newmdtheoremenv[roundcorner = 5px,
leftmargin=15px,
rightmargin=30px,
innertopmargin=0px,
nobreak=true
]{prop}[theorem]{Proposition}

\newmdtheoremenv[roundcorner = 5px,
leftmargin=15px,
rightmargin=30px,
innertopmargin=0px,
]{prop_break}[theorem]{Proposition}

\newmdtheoremenv[roundcorner = 5px,
leftmargin=15px,
rightmargin=30px,
innertopmargin=0px,
nobreak=true
]{regles}[theorem]{Règles de calculs}


\newtheorem*{exemples}{Exemples}
\newtheorem{exemple}{Exemple}
\newtheorem*{rem}{Remarque}
\newtheorem*{rems}{Remarques}
% Warning sign

\newcommand\warning[1][4ex]{%
  \renewcommand\stacktype{L}%
  \scaleto{\stackon[1.3pt]{\color{red}$\triangle$}{\tiny\bfseries !}}{#1}%
}


\newtheorem{exo}{Exercice}
\newcounter{ExoCounter}
\newtheorem{exercice}[ExoCounter]{Exercice}

\newcounter{counterCorrection}
\newtheorem{correction}[counterCorrection]{\color{red}{Correction}}


\theoremstyle{definition}

%\newtheorem{prop}[theorem]{Proposition}
%\newtheorem{\defi}[1]{
%\begin{tcolorbox}[width=14cm]
%#1
%\end{tcolorbox}
%}


%--------------------------------------- 
% Document
%--------------------------------------- 






\lstset{numbers=left, numberstyle=\tiny, stepnumber=1, numbersep=5pt}




% Header et footer

\pagestyle{fancy}
\fancyhead{}
\fancyfoot{}
\renewcommand{\headwidth}{\textwidth}
\renewcommand{\footrulewidth}{0.4pt}
\renewcommand{\headrulewidth}{0pt}
\renewcommand{\footruleskip}{5px}

\fancyfoot[R]{Olivier Glorieux}
%\fancyfoot[R]{Jules Glorieux}

\fancyfoot[C]{ Page \thepage }
\fancyfoot[L]{1BIOA - Lycée Chaptal}
%\fancyfoot[L]{MP*-Lycée Chaptal}
%\fancyfoot[L]{Famille Lapin}



\newcommand{\Hyp}{\mathbb{H}}
\newcommand{\C}{\mathcal{C}}
\newcommand{\U}{\mathcal{U}}
\newcommand{\R}{\mathbb{R}}
\newcommand{\T}{\mathbb{T}}
\newcommand{\D}{\mathbb{D}}
\newcommand{\N}{\mathbb{N}}
\newcommand{\Z}{\mathbb{Z}}
\newcommand{\F}{\mathcal{F}}




\newcommand{\bA}{\mathbb{A}}
\newcommand{\bB}{\mathbb{B}}
\newcommand{\bC}{\mathbb{C}}
\newcommand{\bD}{\mathbb{D}}
\newcommand{\bE}{\mathbb{E}}
\newcommand{\bF}{\mathbb{F}}
\newcommand{\bG}{\mathbb{G}}
\newcommand{\bH}{\mathbb{H}}
\newcommand{\bI}{\mathbb{I}}
\newcommand{\bJ}{\mathbb{J}}
\newcommand{\bK}{\mathbb{K}}
\newcommand{\bL}{\mathbb{L}}
\newcommand{\bM}{\mathbb{M}}
\newcommand{\bN}{\mathbb{N}}
\newcommand{\bO}{\mathbb{O}}
\newcommand{\bP}{\mathbb{P}}
\newcommand{\bQ}{\mathbb{Q}}
\newcommand{\bR}{\mathbb{R}}
\newcommand{\bS}{\mathbb{S}}
\newcommand{\bT}{\mathbb{T}}
\newcommand{\bU}{\mathbb{U}}
\newcommand{\bV}{\mathbb{V}}
\newcommand{\bW}{\mathbb{W}}
\newcommand{\bX}{\mathbb{X}}
\newcommand{\bY}{\mathbb{Y}}
\newcommand{\bZ}{\mathbb{Z}}



\newcommand{\cA}{\mathcal{A}}
\newcommand{\cB}{\mathcal{B}}
\newcommand{\cC}{\mathcal{C}}
\newcommand{\cD}{\mathcal{D}}
\newcommand{\cE}{\mathcal{E}}
\newcommand{\cF}{\mathcal{F}}
\newcommand{\cG}{\mathcal{G}}
\newcommand{\cH}{\mathcal{H}}
\newcommand{\cI}{\mathcal{I}}
\newcommand{\cJ}{\mathcal{J}}
\newcommand{\cK}{\mathcal{K}}
\newcommand{\cL}{\mathcal{L}}
\newcommand{\cM}{\mathcal{M}}
\newcommand{\cN}{\mathcal{N}}
\newcommand{\cO}{\mathcal{O}}
\newcommand{\cP}{\mathcal{P}}
\newcommand{\cQ}{\mathcal{Q}}
\newcommand{\cR}{\mathcal{R}}
\newcommand{\cS}{\mathcal{S}}
\newcommand{\cT}{\mathcal{T}}
\newcommand{\cU}{\mathcal{U}}
\newcommand{\cV}{\mathcal{V}}
\newcommand{\cW}{\mathcal{W}}
\newcommand{\cX}{\mathcal{X}}
\newcommand{\cY}{\mathcal{Y}}
\newcommand{\cZ}{\mathcal{Z}}







\renewcommand{\phi}{\varphi}
\newcommand{\ddp}{\displaystyle}


\newcommand{\G}{\Gamma}
\newcommand{\g}{\gamma}

\newcommand{\tv}{\rightarrow}
\newcommand{\wt}{\widetilde}
\newcommand{\ssi}{\Leftrightarrow}

\newcommand{\floor}[1]{\left \lfloor #1\right \rfloor}
\newcommand{\rg}{ \mathrm{rg}}
\newcommand{\quadou}{ \quad \text{ ou } \quad}
\newcommand{\quadet}{ \quad \text{ et } \quad}
\newcommand\fillin[1][3cm]{\makebox[#1]{\dotfill}}
\newcommand\cadre[1]{[#1]}
\newcommand{\vsec}{\vspace{0.3cm}}

\DeclareMathOperator{\im}{Im}
\DeclareMathOperator{\cov}{Cov}
\DeclareMathOperator{\vect}{Vect}
\DeclareMathOperator{\Vect}{Vect}
\DeclareMathOperator{\card}{Card}
\DeclareMathOperator{\Card}{Card}
\DeclareMathOperator{\Id}{Id}
\DeclareMathOperator{\PSL}{PSL}
\DeclareMathOperator{\PGL}{PGL}
\DeclareMathOperator{\SL}{SL}
\DeclareMathOperator{\GL}{GL}
\DeclareMathOperator{\SO}{SO}
\DeclareMathOperator{\SU}{SU}
\DeclareMathOperator{\Sp}{Sp}


\DeclareMathOperator{\sh}{sh}
\DeclareMathOperator{\ch}{ch}
\DeclareMathOperator{\argch}{argch}
\DeclareMathOperator{\argsh}{argsh}
\DeclareMathOperator{\imag}{Im}
\DeclareMathOperator{\reel}{Re}



\renewcommand{\Re}{ \mathfrak{Re}}
\renewcommand{\Im}{ \mathfrak{Im}}
\renewcommand{\bar}[1]{ \overline{#1}}
\newcommand{\implique}{\Longrightarrow}
\newcommand{\equivaut}{\Longleftrightarrow}

\renewcommand{\fg}{\fg \,}
\newcommand{\intent}[1]{\llbracket #1\rrbracket }
\newcommand{\cor}[1]{{\color{red} Correction }#1}

\newcommand{\conclusion}[1]{\begin{center} \fbox{#1}\end{center}}


\renewcommand{\title}[1]{\begin{center}
    \begin{tcolorbox}[width=14cm]
    \begin{center}\huge{\textbf{#1 }}
    \end{center}
    \end{tcolorbox}
    \end{center}
    }

    % \renewcommand{\subtitle}[1]{\begin{center}
    % \begin{tcolorbox}[width=10cm]
    % \begin{center}\Large{\textbf{#1 }}
    % \end{center}
    % \end{tcolorbox}
    % \end{center}
    % }

\renewcommand{\thesection}{\Roman{section}} 
\renewcommand{\thesubsection}{\thesection.  \arabic{subsection}}
\renewcommand{\thesubsubsection}{\thesubsection. \alph{subsubsection}} 

\newcommand{\suiteu}{(u_n)_{n\in \N}}
\newcommand{\suitev}{(v_n)_{n\in \N}}
\newcommand{\suite}[1]{(#1_n)_{n\in \N}}
%\newcommand{\suite1}[1]{(#1_n)_{n\in \N}}
\newcommand{\suiteun}[1]{(#1_n)_{n\geq 1}}
\newcommand{\equivalent}[1]{\underset{#1}{\sim}}

\newcommand{\demi}{\frac{1}{2}}
\geometry{hmargin=2.0cm, vmargin=1.5cm}




\begin{document}




\title{Règles de calculs}

\section{Fractions et puissances}
\begin{prop}
	Soit $(a,b,c,d)\in (\R^*)^4$.On a
	\begin{center}
		\begin{tabular}{ l c l }
			$\ddp \frac{a+c}{b}=\frac{a}{b}+ \frac{c}{b}$          & et & $ \ddp \frac{a}{b}+\frac{c}{d}=\frac{ad+bc}{bd}$                                       \\  \\

			$\ddp c\frac{a}{b}=\frac{ac}{b} $                      & et & $\ddp \frac{ca}{cb}=\frac{c}{c}\frac{a}{b}  =\frac{a}{b}$ \vsec                        \\


			$\ddp \frac{\frac{a}{b}}{c} = \frac{a}{bc} $           & et & $\ddp \frac{a}{\frac{b}{c}} = \frac{ac}{b}$\vsec                                       \\

			$\ddp \frac{\frac{a}{b}}{\frac{c}{d}} = \frac{ad}{bc}$ & et & $\ddp \frac{a\frac{1}{c}}{b\frac{1}{c}} =\frac{\frac{a}{c}}{\frac{b}{c}} =\frac{a}{b}$
		\end{tabular}
	\end{center}
	--------------------------\\
	\footnotesize{ $\frac{a}{b+c }\neq \frac{a}{b}+\frac{a}{c}$.... c'est très faux...}
\end{prop}

\begin{prop} \label{prop-regle de calcul puissance}
	Pour tout $(x,y) \in \R^2$, non nuls si besoin, pour tout $n,m\in \Z$ on a :
	\begin{itemize}
		\item[$\bullet$] $(xy)^n =x^n y^n$
		\item[$\bullet$] $x^{n} \times x^m = x^{n+m}$ \; et \; $\ddp \frac{x^n}{x^m} =x^{n-m}$
		\item[$\bullet$] $ (x^n)^m = x^{nm}$
	\end{itemize}
\end{prop}



\section{Factorisations}
\begin{prop}
	Soit $(a,b,c,d)\in \R^4$. On a
	$$(a+b)^2=a^2+2ab+b^2$$
	$$(a-b)^2 = a^2 -2ab+b^2$$
	$$a^2-b^2 =(a-b)(a+b)$$
	Ces égalités sont souvent appelées identités remarquables.
\end{prop}


\begin{prop}
	$$(ax+b)  (cx+d) = acx^2 +(bc+ad) x +bd $$
	$$ax+b   (cx+d) = ax +bc x +bd $$
	$$(ax+b)  cx+d = acx^2 +bc x +d $$


	$$c (ax+b) = acx +bc$$
	$$c ax+b = acx +b$$

	$$c-(ax+b) = c-ax-b$$
	$$c-ax+b = c-ax+b$$
\end{prop}

\begin{prop}
	Si $x\mapsto P(x) $ est une fonction polynomiale et $r$ une racine de $P$ (ie. $P(r)=0$) alors on peut factoriser $P(x)$ par $(x-r)$.

	Autrement dit  il existe une fonction polynomiale $Q$ (de degré 1 de moins que celui de $P$) tel que $P(x)=(x-r)Q(x)$
\end{prop}

\section{Exercices}
Les exercices sont tirés du très bon \href{https://colasbd.github.io/cdc/cahier_de_calcul_enonces_v13.pdf}{Cahier de calculs}
\begin{exercice}
	Mettre sous la forme d’une seule fraction, qu’on écrira sous la forme la plus simple possible
	\begin{itemize}
		\item[$\bullet$] $\ddp \frac{1}{(n+1)^2}+\frac{1}{n+1}-\frac{1}{n} \text { pour } n \in \mathbb{N}^* $
		\item[$\bullet$]  $\ddp \frac{a^3-b^3}{(a-b)^2}-\frac{(a+b)^2}{a-b} \text { pour }(a, b) \in \mathbb{Z}^2$
		\item[$\bullet$]  $\ddp \frac{\frac{6(n+1)}{n(n-1)(2 n-2)}}{\frac{2 n+2}{n^2(n-1)^2}} \text { pour } n \in \mathbb{N}^* \backslash\{1\} .  $
	\end{itemize}
\end{exercice}

\begin{exercice}
	Soit $k \in \mathbb{R} \backslash\{1\}$ et $x \in \mathbb{R} \backslash\{2\}$. Écrire les fractions suivantes sous la forme $a+\frac{b}{X}$ avec $a$ et $b$ entiers et $X \in \mathbb{R}$.
	$$A=\frac{29}{6}\quadet B=\frac{k}{k-1} \quadet C=\frac{3 x-1}{x-2}$$
\end{exercice}

\begin{exercice}
	Soit $t \in \mathbb{R} \backslash\{-1\}$. On donne $A=\ddp \frac{1}{1+t^2}-\frac{1}{(1+t)^2}$ et $B=\ddp \left(1+t^2\right)(1+t)^2$.
	Simplifier $A B$ autant que possible.
\end{exercice}

\begin{exercice}
	Développer, réduire et ordonner les expressions polynomiales suivantes selon les puissances croissantes de $x$.
	\begin{itemize}[label=$\bullet$]
		\begin{minipage}[t]{0.55\textwidth}
			\item $(x-2)^2\left(-x^2+3 x-1\right)-(2 x-1)\left(x^3+2\right)$
			\item  $(2 x+3)(5 x-8)-(2 x-4)(5 x-1)$
			\item  $\left((x+1)^2(x-1)\left(x^2-x+1\right)+1\right) x-x^6-x^5+2$
		\end{minipage}
		\begin{minipage}[t]{0.45\textwidth}
			\item  $(x+1)(x-1)^2-2\left(x^2+x+1\right)$
			\item $\left(x^2+\sqrt{2} x+1\right)\left(1-\sqrt{2} x+x^2\right)$
			\item $\left(x^2+x+1\right)^2$
		\end{minipage}
	\end{itemize}
\end{exercice}

\begin{exercice}
	Factoriser les expressions polynomiales de la variable réelle $x$ suivantes.
	\begin{itemize}[label=$\bullet$]
		\begin{minipage}[t]{0.55\textwidth}
			\item$-(6 x+7)(6 x-1)+36 x^2-49$
			\item $25-(10 x+3)^2$
		\end{minipage}
		\begin{minipage}[t]{0.45\textwidth}
			\item  $(6 x-8)(4 x-5)+36 x^2-64$
			\item $(-9 x-8)(8 x+8)+64 x^2-64$
		\end{minipage}
	\end{itemize}

\end{exercice}

\begin{exercice}
	Factoriser sur $\mathbb{R}$ les expressions polynomiales suivantes dont les variables représentent des nombres réels.
	\begin{itemize}[label=$\bullet$]
		\begin{minipage}[t]{0.55\textwidth}
			\item$(x+y)^2-z^2$
			\item  $x y-x-y+1$
			\item $x^2+6 x y+9 y^2-169 x^2$
		\end{minipage}
		\begin{minipage}[t]{0.45\textwidth}
			\item   $x^3+x^2 y+2 x^2+2 x y+x+y$
			\item$y^2\left(a^2+b^2\right)+16 x^4\left(-a^2-b^2\right) $
		\end{minipage}
	\end{itemize}

\end{exercice}

\end{document}
