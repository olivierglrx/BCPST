\documentclass[a4paper, 11pt]{article}
\input{macro/package.tex}
\input{macro/environement}
% Header et footer

\pagestyle{fancy}
\fancyhead{}
\fancyfoot{}
\renewcommand{\headwidth}{\textwidth}
\renewcommand{\footrulewidth}{0.4pt}
\renewcommand{\headrulewidth}{0pt}
\renewcommand{\footruleskip}{5px}

\fancyfoot[R]{Olivier Glorieux}
%\fancyfoot[R]{Jules Glorieux}

\fancyfoot[C]{ Page \thepage }
\fancyfoot[L]{1BIOA - Lycée Chaptal}
%\fancyfoot[L]{MP*-Lycée Chaptal}
%\fancyfoot[L]{Famille Lapin}

\input{macro/newcommand.tex}
\geometry{hmargin=2.0cm, vmargin=1.5cm}




\begin{document}



\title{Fractions, racines, puissances}




\begin{prop}
Soit $(a,b,c,d)\in (\R^*)^3$.On a
$$\frac{a+c}{b}=\frac{a}{b}+ \frac{c}{b}
$$
$$\frac{a}{b}+\frac{c}{d}=\frac{ad+bc}{bd}$$
$$c\frac{a}{b}=\frac{ac}{b}  $$

$$\frac{ca}{cb}=\frac{c}{c}\frac{a}{b}  =\frac{a}{b}$$

$$\frac{\frac{a}{b}}{c} = \frac{a}{bc} \quadet \frac{a}{\frac{b}{c}} = \frac{ac}{b}$$
 $$\frac{\frac{a}{b}}{\frac{c}{d}} = \frac{ad}{bc}$$

 $$\frac{a\frac{1}{c}}{b\frac{1}{c}} =\frac{\frac{a}{c}}{\frac{b}{c}} =\frac{a}{b}$$
 
\end{prop}

\warning $\frac{a}{b+c }\neq \frac{a}{b}+\frac{a}{c}$....  très faux...

\begin{prop} \label{prop-regle de calcul puissance}
Pour tout $(x,y) \in \R^2$, non nuls si besoin, pour tout $n,m\in \Z$ on a :
\begin{itemize}
\item[$\bullet$] $(xy)^n =x^n y^n$
\item[$\bullet$] $x^{n} \times x^m = x^{n+m}$ \; et \; $\ddp \frac{x^n}{x^m} =x^{n-m}$
\item[$\bullet$] $ (x^n)^m = x^{nm}$
\end{itemize}
\end{prop}


\begin{prop}
Soit $(a,b)\in \R^+$:
$$\sqrt{ab} =\sqrt{a}\sqrt{b}$$
$$\sqrt{\frac{a}{b}} =\frac{\sqrt{a}}{\sqrt{b}}$$


$$\sqrt{a^2} =|a| \text{ Donc vaut $a$ si $a>0$} $$


\end{prop}

\warning $\sqrt{a+b} \ne \sqrt{a}+\sqrt{b} $....  Très faux !!

\end{document}
