\documentclass[a4paper, 11pt]{article}
\input{macro/package.tex}
\input{macro/environement}
% Header et footer

\pagestyle{fancy}
\fancyhead{}
\fancyfoot{}
\renewcommand{\headwidth}{\textwidth}
\renewcommand{\footrulewidth}{0.4pt}
\renewcommand{\headrulewidth}{0pt}
\renewcommand{\footruleskip}{5px}

\fancyfoot[R]{Olivier Glorieux}
%\fancyfoot[R]{Jules Glorieux}

\fancyfoot[C]{ Page \thepage }
\fancyfoot[L]{1BIOA - Lycée Chaptal}
%\fancyfoot[L]{MP*-Lycée Chaptal}
%\fancyfoot[L]{Famille Lapin}

\input{macro/newcommand.tex}
\geometry{hmargin=2.0cm, vmargin=1.5cm}




\begin{document}

\title{Exponentielle}
\begin{theorem}
Il existe une unique fonction $f$ définie sur $\R$ vérifiant : 
$$\forall x\in \R,\quad f'(x) =f(x) \quadet f(0)=1$$
\end{theorem}

\begin{defi}
On appelle exponentielle et on note $\exp$ la fonction $f$ du théorème précédent. 


\end{defi}

\begin{center}
    \centering

\includegraphics[scale=0.25]{CH0 - Calculs/images/exp.png}\par\medskip
    \textbf{Graphe de la fonction exponentielle}
\end{center}
\paragraph{Remarques}
\begin{itemize}
    \item Par définition $\exp$ est définie et dérivable sur $\R$ et $\exp'(x)=\exp(x)$ et $\exp(0)=1$
    \item On utilise parfois (souvent) la notation $e^x$ au lieu de $\exp(x)$ on a alors, par exemple : 
    \begin{itemize}
\item[$\bullet$] $e^1 =e$ 
\item[$\bullet$]  $e^0=1$ 
\end{itemize}
    
\end{itemize}


\begin{prop}
Soient ($a, b\in \R$):
\begin{itemize}
\item[$\bullet$] $\ddp e^{a}e^{b}=e^{a+b}$
\item[$\bullet$] $\ddp\frac{e^{a}}{e^{b}}=e^{a-b}$
\item[$\bullet$] $(e^{a})^b=e^{ab}$
\end{itemize}
\end{prop}




\newpage

\title{Logarithme}


\begin{theorem}
Il existe une unique fonction $f$ définie sur $\R_+^*$ vérifiant : 
$$\forall x>0,\quad f'(x) =\frac{1}{x} \quadet f(1)=0$$
\end{theorem}

\begin{defi}
On appelle logarithme népérien et on note $\ln$ la fonction $f$ du théorème précédent. 
\end{defi}



\begin{center}
\includegraphics[scale=0.3]{CH0 - Calculs/images/ln.png}\par\medskip
    \textbf{Graphe de la fonction logarithme}
\end{center}
\paragraph{Remarques}
\begin{itemize}
    \item Par définition $\ln$ est définie et dérivable sur $\R_+^*$ et 
    $\forall x>0\, \ln'(x) =\frac{1}{x}  \quadet \ln(1)=0$.
    \item En physique/chimie le logarithme décimal est souvent utilisé, il est noté $\log$ et définie par $\log(x)  =\frac{\ln(x)}{\ln(10)}$

\end{itemize}


\begin{prop}
Soient ($a>0,\ b>0$):
\begin{itemize}
\item[$\bullet$] $\ln{(ab)}=\ln(a)+\ln(b) $
\item[$\bullet$] $\ln{\left(\ddp\frac{a}{b}\right)}=\ln(a) -\ln(b)$
\item[$\bullet$] $\ln{(a^p)}=p\ln(a)$
\end{itemize}
\end{prop}



\paragraph{Lien entre $\exp$ et $\ln$}
\begin{prop}
On a pour tout \fbox{$x\in \R_+^*$} : 
$$\exp(\ln(x)) =x$$
On a pour tout \fbox{$x\in \R$} : 
$$\ln(\exp(x)) =x$$

Pour tout \fbox{$a\in \R, b\in \R_+^*$} :
$$\exp(a\ln(b)) =b^a$$

\end{prop}
\vspace{0.5cm}

\end{document}