\documentclass[a4paper, 11pt]{article}
\input{macro/package.tex}
\input{macro/environement}
% Header et footer

\pagestyle{fancy}
\fancyhead{}
\fancyfoot{}
\renewcommand{\headwidth}{\textwidth}
\renewcommand{\footrulewidth}{0.4pt}
\renewcommand{\headrulewidth}{0pt}
\renewcommand{\footruleskip}{5px}

\fancyfoot[R]{Olivier Glorieux}
%\fancyfoot[R]{Jules Glorieux}

\fancyfoot[C]{ Page \thepage }
\fancyfoot[L]{1BIOA - Lycée Chaptal}
%\fancyfoot[L]{MP*-Lycée Chaptal}
%\fancyfoot[L]{Famille Lapin}

\input{macro/newcommand.tex}
\geometry{hmargin=2.0cm, vmargin=1.5cm}



\begin{document}

\title{Alphabet Grec}




Voici la plupart des lettres de l'alphabet grec communément utilisées en cours de math/physique.
\begin{center}
\begin{tabular}{|c|c|c|}

\hline
Minuscule  & Majuscule & Nom\\
\hline
$\alpha$ &  & alpha  \\
\hline
$\beta $ &  & beta  \\
\hline
$\gamma $ &$ \Gamma $& gamma  \\
\hline
$\delta$  & $\Delta $& delta \\
\hline
$\epsilon $& & epsilon \\
\hline
$\zeta$ & & zeta \\
\hline
$\eta$ & & eta \\
\hline

$\mu$ & & mu \\
\hline
$\nu$ & & nu \\
\hline
$\theta $&$ \Theta $& théta\\
\hline
$\iota$ & & iota\\
\hline
$\kappa $& & kappa\\
\hline
$\lambda $&$ \Lambda $& lambda \\
\hline
$\xi$  & & xi \\
\hline
$\pi$ & $\Pi$ & pi \\
\hline
$\rho $& & rho\\
\hline
$\sigma $& $\Sigma $& sigma \\
\hline
$\tau$ & & tau \\
\hline
$\phi$  ou $\varphi$ & $\Phi$ & phi \\
\hline
$\psi$  &$ \Psi$ & psi  \\
\hline
$\omega$ & $\Omega$ & omega \\
\hline
$\chi$ &  & chi (prononcé ki) \\
\hline
\end{tabular}

\end{center}

Par ailleurs j'utilise régulièrement les locutions latines suivantes :
\begin{itemize}
    \item \texttt{ie.} \textit{id est} qui signifie "c'est-à-dire"
    \item \texttt{eg.} \textit{exempli gratia} qui signifie "pour l'exemple"
 \end{itemize}

Dans les corrections/démonstrations, j'utilise la notation $\square$ qui signale la fin de la preuve. Cette notation - à l'instar de $\forall, \exists, \implique$ -  a été démocratisée par \href{https://fr.wikipedia.org/wiki/Nicolas_Bourbaki}{Nicolas Bourbaki} célébre groupe de mathématiciens du début du 20éme siécle. Evitez de l'utiliser dans vos copies, cela pourrait paraitre un peu hautain. 



\end{document}