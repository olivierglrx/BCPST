\documentclass[a4paper, 11pt]{article}
\usepackage[utf8]{inputenc}
\usepackage{amssymb,amsmath,amsthm}
\usepackage{geometry}
\usepackage[T1]{fontenc}
\usepackage[french]{babel}
\usepackage{fontawesome}
\usepackage{pifont}
\usepackage{tcolorbox}
\usepackage{fancybox}
\usepackage{bbold}
\usepackage{tkz-tab}
\usepackage{tikz}
\usepackage{fancyhdr}
\usepackage{sectsty}
\usepackage[framemethod=TikZ]{mdframed}
\usepackage{stackengine}
\usepackage{scalerel}
\usepackage{xcolor}
\usepackage{hyperref}
\usepackage{listings}
\usepackage{enumitem}
\usepackage{stmaryrd} 
\usepackage{comment}


\hypersetup{
    colorlinks=true,
    urlcolor=blue,
    linkcolor=blue,
    breaklinks=true
}





\theoremstyle{definition}
\newtheorem{probleme}{Problème}
\theoremstyle{definition}


%%%%% box environement 
\newenvironment{fminipage}%
     {\begin{Sbox}\begin{minipage}}%
     {\end{minipage}\end{Sbox}\fbox{\TheSbox}}

\newenvironment{dboxminipage}%
     {\begin{Sbox}\begin{minipage}}%
     {\end{minipage}\end{Sbox}\doublebox{\TheSbox}}


%\fancyhead[R]{Chapitre 1 : Nombres}


\newenvironment{remarques}{ 
\paragraph{Remarques :}
	\begin{list}{$\bullet$}{}
}{
	\end{list}
}




\newtcolorbox{tcbdoublebox}[1][]{%
  sharp corners,
  colback=white,
  fontupper={\setlength{\parindent}{20pt}},
  #1
}







%Section
% \pretocmd{\section}{%
%   \ifnum\value{section}=0 \else\clearpage\fi
% }{}{}



\sectionfont{\normalfont\Large \bfseries \underline }
\subsectionfont{\normalfont\Large\itshape\underline}
\subsubsectionfont{\normalfont\large\itshape\underline}



%% Format théoreme, defintion, proposition.. 
\newmdtheoremenv[roundcorner = 5px,
leftmargin=15px,
rightmargin=30px,
innertopmargin=0px,
nobreak=true
]{theorem}{Théorème}

\newmdtheoremenv[roundcorner = 5px,
leftmargin=15px,
rightmargin=30px,
innertopmargin=0px,
]{theorem_break}[theorem]{Théorème}

\newmdtheoremenv[roundcorner = 5px,
leftmargin=15px,
rightmargin=30px,
innertopmargin=0px,
nobreak=true
]{corollaire}[theorem]{Corollaire}
\newcounter{defiCounter}
\usepackage{mdframed}
\newmdtheoremenv[%
roundcorner=5px,
innertopmargin=0px,
leftmargin=15px,
rightmargin=30px,
nobreak=true
]{defi}[defiCounter]{Définition}

\newmdtheoremenv[roundcorner = 5px,
leftmargin=15px,
rightmargin=30px,
innertopmargin=0px,
nobreak=true
]{prop}[theorem]{Proposition}

\newmdtheoremenv[roundcorner = 5px,
leftmargin=15px,
rightmargin=30px,
innertopmargin=0px,
]{prop_break}[theorem]{Proposition}

\newmdtheoremenv[roundcorner = 5px,
leftmargin=15px,
rightmargin=30px,
innertopmargin=0px,
nobreak=true
]{regles}[theorem]{Règles de calculs}


\newtheorem*{exemples}{Exemples}
\newtheorem{exemple}{Exemple}
\newtheorem*{rem}{Remarque}
\newtheorem*{rems}{Remarques}
% Warning sign

\newcommand\warning[1][4ex]{%
  \renewcommand\stacktype{L}%
  \scaleto{\stackon[1.3pt]{\color{red}$\triangle$}{\tiny\bfseries !}}{#1}%
}


\newtheorem{exo}{Exercice}
\newcounter{ExoCounter}
\newtheorem{exercice}[ExoCounter]{Exercice}

\newcounter{counterCorrection}
\newtheorem{correction}[counterCorrection]{\color{red}{Correction}}


\theoremstyle{definition}

%\newtheorem{prop}[theorem]{Proposition}
%\newtheorem{\defi}[1]{
%\begin{tcolorbox}[width=14cm]
%#1
%\end{tcolorbox}
%}


%--------------------------------------- 
% Document
%--------------------------------------- 






\lstset{numbers=left, numberstyle=\tiny, stepnumber=1, numbersep=5pt}




% Header et footer

\pagestyle{fancy}
\fancyhead{}
\fancyfoot{}
\renewcommand{\headwidth}{\textwidth}
\renewcommand{\footrulewidth}{0.4pt}
\renewcommand{\headrulewidth}{0pt}
\renewcommand{\footruleskip}{5px}

\fancyfoot[R]{Olivier Glorieux}
%\fancyfoot[R]{Jules Glorieux}

\fancyfoot[C]{ Page \thepage }
\fancyfoot[L]{1BIOA - Lycée Chaptal}
%\fancyfoot[L]{MP*-Lycée Chaptal}
%\fancyfoot[L]{Famille Lapin}



\newcommand{\Hyp}{\mathbb{H}}
\newcommand{\C}{\mathcal{C}}
\newcommand{\U}{\mathcal{U}}
\newcommand{\R}{\mathbb{R}}
\newcommand{\T}{\mathbb{T}}
\newcommand{\D}{\mathbb{D}}
\newcommand{\N}{\mathbb{N}}
\newcommand{\Z}{\mathbb{Z}}
\newcommand{\F}{\mathcal{F}}




\newcommand{\bA}{\mathbb{A}}
\newcommand{\bB}{\mathbb{B}}
\newcommand{\bC}{\mathbb{C}}
\newcommand{\bD}{\mathbb{D}}
\newcommand{\bE}{\mathbb{E}}
\newcommand{\bF}{\mathbb{F}}
\newcommand{\bG}{\mathbb{G}}
\newcommand{\bH}{\mathbb{H}}
\newcommand{\bI}{\mathbb{I}}
\newcommand{\bJ}{\mathbb{J}}
\newcommand{\bK}{\mathbb{K}}
\newcommand{\bL}{\mathbb{L}}
\newcommand{\bM}{\mathbb{M}}
\newcommand{\bN}{\mathbb{N}}
\newcommand{\bO}{\mathbb{O}}
\newcommand{\bP}{\mathbb{P}}
\newcommand{\bQ}{\mathbb{Q}}
\newcommand{\bR}{\mathbb{R}}
\newcommand{\bS}{\mathbb{S}}
\newcommand{\bT}{\mathbb{T}}
\newcommand{\bU}{\mathbb{U}}
\newcommand{\bV}{\mathbb{V}}
\newcommand{\bW}{\mathbb{W}}
\newcommand{\bX}{\mathbb{X}}
\newcommand{\bY}{\mathbb{Y}}
\newcommand{\bZ}{\mathbb{Z}}



\newcommand{\cA}{\mathcal{A}}
\newcommand{\cB}{\mathcal{B}}
\newcommand{\cC}{\mathcal{C}}
\newcommand{\cD}{\mathcal{D}}
\newcommand{\cE}{\mathcal{E}}
\newcommand{\cF}{\mathcal{F}}
\newcommand{\cG}{\mathcal{G}}
\newcommand{\cH}{\mathcal{H}}
\newcommand{\cI}{\mathcal{I}}
\newcommand{\cJ}{\mathcal{J}}
\newcommand{\cK}{\mathcal{K}}
\newcommand{\cL}{\mathcal{L}}
\newcommand{\cM}{\mathcal{M}}
\newcommand{\cN}{\mathcal{N}}
\newcommand{\cO}{\mathcal{O}}
\newcommand{\cP}{\mathcal{P}}
\newcommand{\cQ}{\mathcal{Q}}
\newcommand{\cR}{\mathcal{R}}
\newcommand{\cS}{\mathcal{S}}
\newcommand{\cT}{\mathcal{T}}
\newcommand{\cU}{\mathcal{U}}
\newcommand{\cV}{\mathcal{V}}
\newcommand{\cW}{\mathcal{W}}
\newcommand{\cX}{\mathcal{X}}
\newcommand{\cY}{\mathcal{Y}}
\newcommand{\cZ}{\mathcal{Z}}







\renewcommand{\phi}{\varphi}
\newcommand{\ddp}{\displaystyle}


\newcommand{\G}{\Gamma}
\newcommand{\g}{\gamma}

\newcommand{\tv}{\rightarrow}
\newcommand{\wt}{\widetilde}
\newcommand{\ssi}{\Leftrightarrow}

\newcommand{\floor}[1]{\left \lfloor #1\right \rfloor}
\newcommand{\rg}{ \mathrm{rg}}
\newcommand{\quadou}{ \quad \text{ ou } \quad}
\newcommand{\quadet}{ \quad \text{ et } \quad}
\newcommand\fillin[1][3cm]{\makebox[#1]{\dotfill}}
\newcommand\cadre[1]{[#1]}
\newcommand{\vsec}{\vspace{0.3cm}}

\DeclareMathOperator{\im}{Im}
\DeclareMathOperator{\cov}{Cov}
\DeclareMathOperator{\vect}{Vect}
\DeclareMathOperator{\Vect}{Vect}
\DeclareMathOperator{\card}{Card}
\DeclareMathOperator{\Card}{Card}
\DeclareMathOperator{\Id}{Id}
\DeclareMathOperator{\PSL}{PSL}
\DeclareMathOperator{\PGL}{PGL}
\DeclareMathOperator{\SL}{SL}
\DeclareMathOperator{\GL}{GL}
\DeclareMathOperator{\SO}{SO}
\DeclareMathOperator{\SU}{SU}
\DeclareMathOperator{\Sp}{Sp}


\DeclareMathOperator{\sh}{sh}
\DeclareMathOperator{\ch}{ch}
\DeclareMathOperator{\argch}{argch}
\DeclareMathOperator{\argsh}{argsh}
\DeclareMathOperator{\imag}{Im}
\DeclareMathOperator{\reel}{Re}



\renewcommand{\Re}{ \mathfrak{Re}}
\renewcommand{\Im}{ \mathfrak{Im}}
\renewcommand{\bar}[1]{ \overline{#1}}
\newcommand{\implique}{\Longrightarrow}
\newcommand{\equivaut}{\Longleftrightarrow}

\renewcommand{\fg}{\fg \,}
\newcommand{\intent}[1]{\llbracket #1\rrbracket }
\newcommand{\cor}[1]{{\color{red} Correction }#1}

\newcommand{\conclusion}[1]{\begin{center} \fbox{#1}\end{center}}


\renewcommand{\title}[1]{\begin{center}
    \begin{tcolorbox}[width=14cm]
    \begin{center}\huge{\textbf{#1 }}
    \end{center}
    \end{tcolorbox}
    \end{center}
    }

    % \renewcommand{\subtitle}[1]{\begin{center}
    % \begin{tcolorbox}[width=10cm]
    % \begin{center}\Large{\textbf{#1 }}
    % \end{center}
    % \end{tcolorbox}
    % \end{center}
    % }

\renewcommand{\thesection}{\Roman{section}} 
\renewcommand{\thesubsection}{\thesection.  \arabic{subsection}}
\renewcommand{\thesubsubsection}{\thesubsection. \alph{subsubsection}} 

\newcommand{\suiteu}{(u_n)_{n\in \N}}
\newcommand{\suitev}{(v_n)_{n\in \N}}
\newcommand{\suite}[1]{(#1_n)_{n\in \N}}
%\newcommand{\suite1}[1]{(#1_n)_{n\in \N}}
\newcommand{\suiteun}[1]{(#1_n)_{n\geq 1}}
\newcommand{\equivalent}[1]{\underset{#1}{\sim}}

\newcommand{\demi}{\frac{1}{2}}
\geometry{hmargin=2.0cm, vmargin=2.5cm}

\newcommand{\subscript}[2]{$#1 _ #2$}

\usepackage{tocloft}

\cftsetindents{section}{0em}{3em}


\renewcommand\cfttoctitlefont{\hfill\Large\bfseries}
\renewcommand\cftaftertoctitle{\hfill\mbox{}}



\begin{document}
   \title{Chapitre 0 : Résolution d'équations}
   \tableofcontents

\section{Quantificateurs}

\begin{defi}
Une propri\'et\'e est un \'enonc\'e math\'ematique dont on peut dire sans ambiguit\'e s'il est vrai ou faux.
\end{defi}
Certaines propriétés ne dépendent pas de variables : 
\begin{itemize}
    \item $P_1  :  " 3 > \pi" $, $P_1$ est fausse.
    \item $P_2  : $ "La fonction exponentielle est croissante sur $\R$. $P_2$ est vraie.
\end{itemize}

Mais il est courant qu'elles en dépendent: 
\begin{itemize}
    \item $P_1 (x) :  " 3 > x" $. Ici, $P_1(2)$  est vraie et   $P_1(\pi)$ est fausse.
    \item $P_2(f)  : $ "La fonction $f$ est croissante sur $\R$". Ici, $P_2(\exp)$ est vraie, $P_2(x\mapsto x^2)$ est fausse. 
\end{itemize}

Les (in)-équations sont des propriétés mathématiques contenant une (in)-égalité entre deux expressions mathématiques. Résoudre une (in)-équation c'est donné sous la forme la plus simple possible toutes les valeurs pour lesquelles cette propriété est vérifiée. Dans ce chapitre, on se restreindra au cas où les équations contiennent une seule inconnue réelle. On considérera dans d'autres chapitres le cas où les variables sont plusieurs réelles (systèmes linéaires) ou mêmes des fonctions (équations différentielles).\\


On utilisera dans la suite du cours les quantificateurs suivants : 
\begin{defi} Soit $E$ un ensemble et $P(x)$ une propri\'et\'e dépendant d'une variable $x\in E$.
\begin{itemize}
\item[$\bullet$] $\mathbf{\forall}$ se lit 'quelque soit' ou 'pour tout'. 
Si $P(x)$ est vraie pour tout $x\in E$, on \'ecrit: $\forall x\in E,\ P(x)$
\item[$\bullet$] $\mathbf{\exists}$ se lit 'il existe'.
Si $P(x)$ est vraie pour au moins un $x\in E$, on \'ecrit: $\exists x\in E,\ P(x)$
\end{itemize}
\end{defi}



\section{Equations polynomiales} 

\begin{exercice}
    Résoudre sur $\R$ :  
    \vspace{-0.5cm}
    
\begin{minipage}[t]{0.3\textwidth}
   $$(E_1) \quad  x^2+3x+2=0$$
        $$(E_2) \quad  x^2+2x+1=0$$
\end{minipage}
\begin{minipage}[t]{0.3\textwidth}
        $$(E_3) \quad x^2+3x+2\geq 0$$
        $$(E_4) \quad  x^2+2x+1\geq 0$$
\end{minipage}
\begin{minipage}[t]{0.3\textwidth}
        $$(E_5)\quad  x^2+2x+1\leq 0$$
        $$(E_6)\quad  x^2+2x+2<0$$
\end{minipage}
\end{exercice}


\begin{exercice}
    Résoudre sur $\R$ :
    
    $$    (P_1) :\quad   x^3+3x^2+2x=0, \quad (P_2) \quad   x^3-3x+2=0,\quad 
        (P_3) \quad x^4+2x+1\geq 0$$
    
\end{exercice}


\paragraph{Points à retenir}
\begin{itemize}
    \item  La formule du discriminant et des racines.
    \item La factorisation quand on obtient une racine.
    \item L'écriture des solutions sous forme d'intervalles ou d'ensembles.
\end{itemize}

\section{Equations rationnelles}
\begin{exercice}
    Résoudre sur $\R$ :



   $$ (Q_1): \quad   \frac{-2}{x+3}= x,\, \quad (Q_2): \quad   \frac{-2}{x+3}\leq x,\, \quad  (Q_3): \quad   \frac{x+1}{x-1}<\frac{x-3}{x+2}$$

\end{exercice}
\paragraph{Points à retenir}
\begin{itemize}
    \item La condition pour multiplier une inéquation. (signe)
    \item Les règles de calculs sur les fractions.
    \item La mise au même dénominateur. 
\end{itemize}




\section{Equations avec des radicaux}
\begin{exercice}
    Résoudre sur $\R$ :\\


\begin{enumerate}[label=(\subscript{R}{{\arabic*}}) : $\,$]
\hspace{1cm}
\begin{minipage}{0.33\textwidth}
   \item $ \sqrt{x}=x$\\
        \item $ \sqrt{x+2}=x$\\
        \item $\sqrt{x+1}= -x+1$
\end{minipage}
\begin{minipage}{0.3\textwidth}
        \item $\sqrt{x}< 2x+1$\\
        \item $\sqrt{x-2}\geq x$\\
         \item $ \sqrt{x^2-1}\geq x$
\end{minipage}
\begin{minipage}{0.3\textwidth}
        \item $ \sqrt{x^2+x}< \sqrt{x-1}$\\
        \item $ \sqrt{x+1}-\sqrt{x-1}\leq x$\\
         \item $ \frac{1}{\sqrt{x+1}}> x$
\end{minipage}
\end{enumerate}

\end{exercice}
\paragraph{Points à retenir}
\begin{itemize}
    \item Les implications et les équivalences entre deux propositions.
    \item Les disjonctions de cas. 
    \item La condition pour mettre au carré. (signe)
    \item Les règles de calculs sur les racines.
    \item Les identités remarquables. 
\end{itemize}

\section{Valeurs absolues}
\begin{exercice}
    Résoudre  sur $\R$ : 
         $$(V_1):  \quad |x| = 1,  \quad (V_2) : \quad |x+1| = -1, \quad(V_3) : \quad |x+1| =\sqrt{x}.$$
     $$(V_4):  \quad |x-1| \leq 1-2x,  \quad (V_5) : \quad |x+1| \leq |1-2x|, \quad(V_6) : \quad || x|-5| \geqslant|| 3 x|-3| .$$
\end{exercice}

\paragraph{Points à retenir}
\begin{itemize}
    \item La définition de la valeur absolue, son graphe.
    \item Les disjonctions de cas. 
\end{itemize}





\section{Changement de variables}
\begin{exercice}
Résoudre

\begin{enumerate}[label=(\subscript{CV}{{\arabic*}}) : $\,$]
\hspace{1cm}
\begin{minipage}[t]{0.33\textwidth}
    \item $\frac{\sqrt{x}}{\sqrt{x}-1}=\sqrt{x}+1$\\
    \item $x^4+3x^2+2=0$
\end{minipage}
\begin{minipage}[t]{0.33\textwidth}
    \item $x^4+3x^2+2\geq 0$\\
    \item $\frac{1}{e^x-1}\leq e^x$
\end{minipage}
\begin{minipage}[t]{0.3\textwidth}
    \item$\frac{1}{\ln(x)-1}\leq \ln(x)+1$\\
    \item $(\ln(x))^2+2\ln(x)+1=0$
\end{minipage}
\end{enumerate}

\paragraph{Points à retenir}
\begin{itemize}
    \item Savoir trouver un changement de variable.
    \item Passer des solutions de l'équation originale à celle où l'on a changé la variable. 
\end{itemize}
\end{exercice}



\section{Paramètres}



\begin{exercice}
Résoudre les équations suivantes d'inconnue $x$ et de paramètre $\lambda \in \R$ :\\

\begin{enumerate}[label=(\subscript{P}{{\arabic*}}) : $\,$]
\hspace{1cm}
\begin{minipage}{0.3\textwidth}
        \item $ \lambda x^2 +2x+1 =0$\\
        \item $ \frac{1}{x+\lambda}\geq x-\lambda$
    \end{minipage}
    \begin{minipage}{0.3\textwidth}
        \item $  x -1 =2\lambda x +1$\\
        \item $  x -1 <2\lambda x +1$
    \end{minipage}
    \begin{minipage}{0.3\textwidth}
        \item $|x-\lambda| = \frac{1}{2}x+1 $\\
        \item $ \exp(2x) + \lambda \exp(x) +1=0$
    \end{minipage}
\end{enumerate}
\end{exercice}

\paragraph{Points à retenir}
\begin{itemize}
    \item Résoudre une équation à paramètre c'est résoudre beaucoup d'équations à la fois. On donne pour chaque valeur du paramètre l'ensemble des solutions. 
    \item Ne pas confondre le paramètre avec l'inconnue ! 
\end{itemize}






\section{Par étude de fonctions}
\begin{exercice}

Résoudre les inéquations suivantes à l'aide d'une étude de fonction 
\vspace{0.4cm}
    \begin{enumerate}[label=(\subscript{I}{{\arabic*}}) : $\,$]
\hspace{1cm}
\begin{minipage}{0.4\textwidth}
        \item $ \ln(x+1)\leq x$\\
        \item $ e^{x}-1 \geq x$
    \end{minipage}
    \begin{minipage}{0.4\textwidth}
        \item $ \sin(x)\leq x$\\
        \item $ \sin(x)\geq \frac{\pi x}{2}$
    \end{minipage}
\end{enumerate}
\end{exercice}


\paragraph{Points à retenir}
\begin{itemize}
    \item Montrer une inégalité sur un ensemble $I$ revient à résoudre une inéquation et montrer que l'ensemble des solutions est tout l'ensemble $I$.
    \item Etudier la différence des membres d'une inégalité afin de comparer à $0$
\end{itemize}


\section{Un peu de tout}

\begin{exercice}

Résoudre les équations suivantes d'inconnue $x$
\vspace{0.4cm}
    \begin{enumerate}[label=(\subscript{T}{{\arabic*}}) : $\,$]
\hspace{1cm}
\begin{minipage}{0.5\textwidth}
        \item $ \ddp  (x^2-1)e^{x} - (x^2-1)e^{(x^2)}\geq 0$\\
        \item $\ddp  \frac{2x-1}{x^2-x+1}-1\leq 0$
    \end{minipage}
    \begin{minipage}{0.5\textwidth}
        \item $  \ddp xe^{x} - x \leq 0$\\
        \item $ \ddp \frac{e^{x}(e^{2x}+1) - e^x(2e^{2x})}{(e^{2x}+1)^2}\geq 0$
    \end{minipage}

\end{enumerate}
\end{exercice}
\vspace{0.4cm}
\begin{exercice}
    On admet que pour tout $x\in \R$: $e^{x}\geq x+1$. 
   \begin{itemize}
       \item  Montrer que pour tout $x\in \R$: $$e^{2x}-x\geq 0.$$
    \item  Montrer que pour tout $x\in \R$: $$e^{x}-2x\geq 0.$$
    
   \end{itemize} 
    
\end{exercice}




\end{document}